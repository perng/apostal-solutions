\chapter{The Lebesgue Integral}

\section{Upper functions}

\begin{problembox}[10.1: Properties of max and min functions]
Prove that $\max(f, g) + \min(f, g) = f + g$, and that 
\[ \max(f + h, g + h) = \max(f, g) + h, \quad \min(f + h, g + h) = \min(f, g) + h. \]
\end{problembox}

\begin{problembox}[10.2: Sequences of max and min functions]
Let $\{f_n\}$ and $\{g_n\}$ be increasing sequences of functions on an interval $I$. Let $u_n = \max(f_n, g_n)$ and $v_n = \min(f_n, g_n)$.
\begin{enumerate}[label=(\alph*)]
    \item Prove that $\{u_n\}$ and $\{v_n\}$ are increasing on $I$.
    \item If $f_n \to f$ a.e. on $I$ and if $g_n \to g$ a.e. on $I$, prove that $u_n \to \max(f, g)$ and $v_n \to \min(f, g)$ a.e. on $I$.
\end{enumerate}
\end{problembox}

\begin{problembox}[10.3: Divergence of integral sequence]
Let $\{s_n\}$ be an increasing sequence of step functions which converges pointwise on an interval $I$ to a limit function $f$. If $I$ is unbounded and if $f(x) \geq 1$ almost everywhere on $I$, prove that the sequence $\{\int_I s_n\}$ diverges.
\end{problembox}

\begin{problembox}[10.4: Example of upper function]
This exercise gives an example of an upper function $f$ on the interval $I = [0, 1]$ such that $-f \notin U(I)$. Let $\{r_1, r_2, \ldots\}$ denote the set of rational numbers in $[0, 1]$ and let $I_n = [r_n - 4^{-n}, r_n + 4^{-n}] \cap I$. Let $f(x) = 1$ if $x \in I_n$ for some $n$, and let $f(x) = 0$ otherwise.
\begin{enumerate}[label=(\alph*)]
    \item Let $f_n(x) = 1$ if $x \in I_n$, $f_n(x) = 0$ if $x \notin I_n$, and let $s_n = \max(f_1, \ldots, f_n)$. Show that $\{s_n\}$ is an increasing sequence of step functions which generates $f$. This shows that $f \in U(I)$.
    \item Prove that $\int_I f \leq 2/3$.
    \item If a step function $s$ satisfies $s(x) \leq -f(x)$ on $I$, show that $s(x) \leq -1$ almost everywhere on $I$ and hence $\int_I s \leq -1$.
    \item Assume that $-f \in U(I)$ and use (b) and (c) to obtain a contradiction.
\end{enumerate}
\end{problembox}

\section{Convergence theorems}

\begin{problembox}[10.5: Non-interchangeable limit and integral]
If $f_n(x) = e^{-nx} - 2e^{-2nx}$, show that 
\[\sum_{n=1}^{\infty} \int_{0}^{\infty} f_n(x) \, dx \neq \int_{0}^{\infty} \sum_{n=1}^{\infty} f_n(x) \, dx.\]
\end{problembox}

\begin{problembox}[10.6: Integral evaluations]
Justify the following equations:
\begin{enumerate}[label=(\alph*)]
    \item $\int_{0}^{1} \log \frac{1}{1-x} \, dx = \int_{0}^{1} \sum_{n=1}^{\infty} \frac{x^n}{n} \, dx = \sum_{n=1}^{\infty} \frac{1}{n} \int_{0}^{1} x^n \, dx = 1.$
    \item $\int_{0}^{1} \frac{x^{p-1}}{1-x} \log \left( \frac{1}{x} \right) \, dx = \sum_{n=0}^{\infty} \frac{1}{(n+p)^2} \quad (p > 0).$
\end{enumerate}
\end{problembox}

\begin{problembox}[10.7: Tannery's convergence theorem]
Prove Tannery's convergence theorem for Riemann integrals: Given a sequence of functions $\{f_n\}$ and an increasing sequence $\{p_n\}$ of real numbers such that $p_n \to +\infty$ as $n \to \infty$. Assume that
\begin{enumerate}[label=(\alph*)]
    \item $f_n \to f$ uniformly on $[a,b]$ for every $b \geq a$.
    \item $f_n$ is Riemann-integrable on $[a,b]$ for every $b \geq a$.
    \item $|f_n(x)| \leq g(x)$ almost everywhere on $[a,+\infty)$, where $g$ is nonnegative and improper Riemann-integrable on $[a,+\infty)$.
\end{enumerate}
Then both $f$ and $|f|$ are improper Riemann-integrable on $[a,+\infty)$, the sequence $\{\int_a^{p_n} f_n\}$ converges, and
\[\int_{a}^{+\infty} f(x) \, dx = \lim_{n \to \infty} \int_{a}^{p_n} f_n(x) \, dx.\]

\begin{enumerate}[label=(\alph*),resume]
    \item Use Tannery's theorem to prove that
    \[\lim_{n \to \infty} \int_{0}^{n} \left( 1 - \frac{x}{n} \right)^n x^p \, dx = \int_{0}^{\infty} e^{-x}x^p \, dx, \quad \text{if } p > -1.\]
\end{enumerate}
\end{problembox}

\begin{problembox}[10.8: Fatou's lemma]
Prove Fatou's lemma: Given a sequence $\{f_n\}$ of nonnegative functions in $L(I)$ such that (a) $\{f_n\}$ converges almost everywhere on $I$ to a limit function $f$, and (b) $\int_I f_n \leq A$ for some $A > 0$ and all $n \geq 1$. Then the limit function $f \in L(I)$ and $\int_I f \leq A$.

\textbf{Note.} It is not asserted that $\{f_n\}$ converges. (Compare with Theorem 10.24.)

\textbf{Hint.} Let $g_n(x) = \inf \{f_n(x), f_{n+1}(x), \ldots\}$. Then $g_n \to f$ a.e. on $I$ and $\int_I g_n \leq \int_I f_n \leq A$ so $\lim_{n \to \infty} \int_I g_n$ exists and is $\leq A$. Now apply Theorem 10.24.
\end{problembox}

\section{Improper Riemann Integrals}

\begin{problembox}[10.9: Existence of improper integrals]
\begin{enumerate}[label=(\alph*)]
    \item If $p > 1$, prove that the integral $\int_1^{+\infty} x^{-p} \sin x \, dx$ exists both as an improper Riemann integral and as a Lebesgue integral. \textbf{Hint.} Integration by parts.
    \item If $0 < p \leq 1$, prove that the integral in (a) exists as an improper Riemann integral but not as a Lebesgue integral. \textbf{Hint.} Let
    \[g(x) = 
    \begin{cases} 
    \frac{\sqrt{2}}{2x} & \text{if } m + \frac{\pi}{4} \leq x \leq m + \frac{3\pi}{4} \text{ for } n = 1, 2, \ldots, \\ 
    0 & \text{otherwise},
    \end{cases}\]
    and show that
    \[\int_{1}^{m\pi} x^{-p} |\sin x| \, dx \geq \int_{\pi}^{m\pi} g(x) \, dx \geq \frac{\sqrt{2}}{4} \sum_{k=2}^{n} \frac{1}{k}.\]
\end{enumerate}
\end{problembox}

\begin{problembox}[10.10: Trigonometric integrals]
\begin{enumerate}[label=(\alph*)]
    \item Use the trigonometric identity $\sin 2x = 2 \sin x \cos x$, along with the formula $\int_{0}^{\infty} \sin x/x \, dx = \pi/2$, to show that
    \[\int_{0}^{\infty} \frac{\sin x \cos x}{x} \, dx = \frac{\pi}{4}.\]
    \item Use integration by parts in (a) to derive the formula
    \[\int_{0}^{\infty} \frac{\sin^2 x}{x^2} \, dx = \frac{\pi}{2}.\]
    \item Use the identity $\sin^2 x + \cos^2 x = 1$, along with (b), to obtain
    \[\int_{0}^{\infty} \frac{\sin^4 x}{x^2} \, dx = \frac{\pi}{4}.\]
    \item Use the result of (c) to obtain
    \[\int_{0}^{\infty} \frac{\sin^4 x}{x^4} \, dx = \frac{\pi}{3}.\]
\end{enumerate}
\end{problembox}

\begin{problembox}[10.11: Existence of logarithmic integrals]
If $a > 1$, prove that the integral $\int_{a}^{+\infty} x^p (\log x)^q \, dx$ exists, both as an improper Riemann integral and as a Lebesgue integral for all $q$ if $p < -1$, or for $q < -1$ if $p = -1$.
\end{problembox}

\begin{problembox}[10.12: Existence of integrals]
Prove that each of the following integrals exists, both as an improper Riemann integral and as a Lebesgue integral.
\begin{enumerate}[label=(\alph*)]
    \item $\int_{1}^{\infty} \sin^2 \frac{1}{x} \, dx$,
    \item $\int_{0}^{\infty} x^pe^{-x^q} \, dx \quad (p > 0, q > 0)$.
\end{enumerate}
\end{problembox}

\begin{problembox}[10.13: Determine existence of integrals]
Determine whether or not each of the following integrals exists, either as an improper Riemann integral or as a Lebesgue integral.
\begin{enumerate}[label=(\alph*)]
    \item $\int_{0}^{\infty} e^{-(t^2 + t^{-2})} \, dt$,
    \item $\int_{0}^{\infty} \frac{\cos x}{\sqrt{x}} \, dx$,
    \item $\int_{0}^{\infty} \frac{\log x}{x(x^2 - 1)^{1/2}} \, dx$,
    \item $\int_{0}^{\infty} e^{-x} \sin \frac{1}{x} \, dx$,
    \item $\int_{0}^{1} \log x \sin \frac{1}{x} \, dx$,
    \item $\int_{0}^{\infty} e^{-x} \log (\cos^2 x) \, dx$.
\end{enumerate}
\end{problembox}

\begin{problembox}[10.14: Parameter-dependent integrals]
Determine those values of $p$ and $q$ for which the following Lebesgue integrals exist.
\begin{enumerate}[label=(\alph*)]
    \item $\int_{0}^{1} x^p (1 - x^2)^q \, dx$,
    \item $\int_{0}^{\infty} x^x e^{-x^p} \, dx$,
    \item $\int_{0}^{\infty} \frac{x^{p-1} - x^{q-1}}{1 - x} \, dx$,
    \item $\int_{0}^{\infty} \frac{\sin(x^p)}{x^q} \, dx$,
    \item $\int_{0}^{\infty} \frac{x^{p-1}}{1 + x^q} \, dx$,
    \item $\int_{\pi}^{\infty} (\log x)^p (\sin x)^{-1/3} \, dx$.
\end{enumerate}
\end{problembox}

\begin{problembox}[10.15: Integral evaluations]
Prove that the following improper Riemann integrals have the values indicated ($m$ and $n$ denote positive integers).
\begin{enumerate}[label=(\alph*)]
    \item $\int_{0}^{\infty} \frac{\sin^{2n+1} x}{x} \, dx = \frac{\pi(2n)!}{2^{2n+1}(n!)^2}$,
    \item $\int_{1}^{\infty} \frac{\log x}{x^{n+1}} \, dx = n^{-2}$,
    \item $\int_{0}^{\infty} x^n (1 + x)^{n-m-1} \, dx = \frac{n!(m-1)!}{(m+n)!}$.
\end{enumerate}
\end{problembox}

\begin{problembox}[10.16: Periodic function integral]
Given that $f$ is Riemann-integrable on $[0, 1]$, that $f$ is periodic with period 1, and that $\int_{0}^{1} f(x) \, dx = 0$. Prove that the improper Riemann integral $\int_{1}^{\infty} x^{-s} f(x) \, dx$ exists if $s > 0$. \textbf{Hint.} Let $g(x) = \int_{1}^{x} f(t) \, dt$ and write $\int_{1}^{x} x^{-s} f(x) \, dx = \int_{1}^{x} x^{-s} dg(x)$.
\end{problembox}

\begin{problembox}[10.17: Limit of integral transformations]
Assume that $f \in R$ on $[a, b]$ for every $b > a > 0$. Define $g$ by the equation $xg(x) = \int_{1}^{x} f(t) \, dt$ if $x > 0$, assume that the limit $\lim_{x \to +\infty} g(x)$ exists, and denote this limit by $B$. If $a$ and $b$ are fixed positive numbers, prove that
\begin{enumerate}[label=(\alph*)]
    \item $\int_{a}^{b} \frac{f(x)}{x} \, dx = g(b) - g(a) + \int_{a}^{b} \frac{g(x)}{x} \, dx.$
    \item $\lim_{T \to +\infty} \int_{aT}^{bT} \frac{f(x)}{x} \, dx = B \log \frac{b}{a}.$
    \item $\int_{1}^{\infty} \frac{f(ax) - f(bx)}{x} \, dx = B \log \frac{a}{b} + \int_{a}^{b} \frac{f(t)}{t} \, dt.$
\end{enumerate}
\begin{enumerate}[label=(\alph*),resume]
    \item Assume that the limit $\lim_{x \to 0^+} x \int_{x}^{1} f(t)x^{-2} \, dt$ exists, denote this limit by $A$, and prove that
    \[\int_{0}^{1} \frac{f(ax) - f(bx)}{x} \, dx = A \log \frac{b}{a} - \int_{a}^{b} \frac{f(t)}{t} \, dt.\]
    \item Combine (c) and (d) to deduce
    \[\int_{0}^{\infty} \frac{f(ax) - f(bx)}{x} \, dx = (B - A) \log \frac{a}{b}\]
    and use this result to evaluate the following integrals:
    \[\int_{0}^{\infty} \frac{\cos ax - \cos bx}{x} \, dx, \quad \int_{0}^{\infty} \frac{e^{-ax} - e^{-bx}}{x} \, dx.\]
\end{enumerate}
\end{problembox}

\section{Lebesgue integrals}

\begin{problembox}[10.18: Existence of Lebesgue integrals]
Prove that each of the following exists as a Lebesgue integral.
\begin{enumerate}[label=(\alph*)]
    \item $\int_{0}^{1} \frac{x \log x}{(1 + x)^2} \, dx$,
    \item $\int_{0}^{1} \frac{x^p - 1}{\log x} \, dx \quad (p > -1)$,
    \item $\int_{0}^{1} \log x \log (1 + x) \, dx$,
    \item $\int_{0}^{1} \frac{\log (1 - x)}{(1 - x)^{1/2}} \, dx.$
\end{enumerate}
\end{problembox}

\begin{problembox}[10.19: Existence of singular integral]
Assume that $f$ is continuous on $[0, 1]$, $f(0) = 0$, $f'(0)$ exists. Prove that the Lebesgue integral $\int_{0}^{1} f(x)x^{-3/2} \, dx$ exists.
\end{problembox}

\begin{problembox}[10.20: Existence/non-existence of integrals]
Prove that the integrals in (a) and (c) exist as Lebesgue integrals but that those in (b) and (d) do not.
\begin{enumerate}[label=(\alph*)]
    \item $\int_{0}^{\infty} x^2 e^{-x^8 \sin^2 x} \, dx$,
    \item $\int_{0}^{\infty} x^3 e^{-x^8 \sin^2 x} \, dx$,
    \item $\int_{1}^{\infty} \frac{dx}{1 + x^4 \sin^2 x}$,
    \item $\int_{1}^{\infty} \frac{dx}{1 + x^2 \sin^2 x}.$
\end{enumerate}
\textbf{Hint.} Obtain upper and lower bounds for the integrals over suitably chosen neighborhoods of the points $n\pi$ ($n = 1, 2, 3, \ldots$).
\end{problembox}

\section{Functions defined by integrals}

\begin{problembox}[10.21: Domain of integral functions]
Determine the set $S$ of those real values of $y$ for which each of the following integrals exists as a Lebesgue integral.
\begin{enumerate}[label=(\alph*)]
    \item $\int_{0}^{\infty} \frac{\cos xy}{1 + x^2} \, dx$,
    \item $\int_{0}^{\infty} (x^2 + y^2)^{-1} \, dx$,
    \item $\int_{0}^{\infty} \frac{\sin^2 xy}{x^2} \, dx$,
    \item $\int_{0}^{\infty} e^{-x^2} \cos 2xy \, dx.$
\end{enumerate}
\end{problembox}

\begin{problembox}[10.22: Differential equation for integral]
Let $F(y) = \int_{0}^{\infty} e^{-x^2} \cos 2xy \, dx$ if $y \in \mathbb{R}$. Show that $F$ satisfies the differential equation $F'(y) + 2y F(y) = 0$ and deduce that $F(y) = \frac{1}{2} \sqrt{\pi} e^{-y^2}$. (Use the result $\int_{0}^{\infty} e^{-x^2} \, dx = \frac{1}{2} \sqrt{\pi}$, derived in Exercise 7.19.)
\end{problembox}

\begin{problembox}[10.23: Integral with trigonometric kernel]
Let $F(y) = \int_{0}^{\infty} \frac{\sin xy}{x(x^2 + 1)} \, dx$ if $y > 0$. Show that $F$ satisfies the differential equation $F''(y) - F(y) + \pi / 2 = 0$ and deduce that $F(y) = \frac{1}{2} \pi (1 - e^{-y})$. Use this result to deduce the following equations, valid for $y > 0$ and $a > 0$:
\begin{align*}
\int_{0}^{\infty} \frac{\sin xy}{x(x^2 + a^2)} \, dx &= \frac{\pi}{2a^2} (1 - e^{-ay}), \\
\int_{0}^{\infty} \frac{\cos xy}{x^2 + a^2} \, dx &= \frac{\pi e^{-ay}}{2a}, \\
\int_{0}^{\infty} \frac{x \sin xy}{x^2 + a^2} \, dx &= \frac{\pi}{2} e^{-ay}.
\end{align*}
\textbf{Note.} You may use $\int_{0}^{\infty} \frac{\sin x}{x} \, dx = \frac{\pi}{2}.$
\end{problembox}

\begin{problembox}[10.24: Non-interchangeable iterated integrals]
Show that $\int_{1}^{\infty} \left[ \int_{1}^{\infty} f(x, y) \, dx \right] dy \neq \int_{1}^{\infty} \left[ \int_{1}^{\infty} f(x, y) \, dy \right] dx$ if
\begin{enumerate}[label=(\alph*)]
    \item $f(x, y) = \frac{x - y}{(x + y)^3}$,
    \item $f(x, y) = \frac{x^2 - y^2}{(x^2 + y^2)^2}.$
\end{enumerate}
\end{problembox}

\begin{problembox}[10.25: Non-interchangeable integration order]
Show that the order of integration cannot be interchanged in the following integrals:
\begin{enumerate}[label=(\alph*)]
    \item $\int_{0}^{1} \left[ \int_{0}^{1} \frac{x - y}{(x + y)^{3}} \, dx \right] dy$,
    \item $\int_{0}^{1} \left[ \int_{1}^{\infty} (e^{-xy} - 2e^{-2xy}) \, dy \right] dx.$
\end{enumerate}
\end{problembox}

\begin{problembox}[10.26: Integral evaluation via iterated integral]
Let $f(x, y) = \int_{0}^{\infty} dt / [(1 + x^{2}t^{2})(1 + y^{2}t^{2})]$ if $(x, y) \neq (0, 0)$. Show (by methods of elementary calculus) that $f(x, y) = \frac{1}{2}\pi(x + y)^{-1}$. Evaluate the iterated integral $\int_{0}^{1} \left[ \int_{0}^{1} f(x, y) \, dx \right] dy$ to derive the formula:
\[\int_{0}^{\infty} \frac{(\arctan x)^{2}}{x^{2}} \, dx = \pi \log 2.\]
\end{problembox}

\begin{problembox}[10.27: Trigonometric integral evaluation]
Let $f(y) = \int_{0}^{\infty} \frac{\sin x \cos xy}{x} \, dx$ if $y \geq 0$. Show (by methods of elementary calculus) that $f(y) = \pi/2$ if $0 \leq y < 1$ and that $f(y) = 0$ if $y > 1$. Evaluate the integral $\int_{0}^{1} f(y) \, dy$ to derive the formula
\[\int_{0}^{\infty} \frac{\sin ax \sin x}{x^{2}} \, dx = \begin{cases} 
\frac{\pi a}{2} & \text{if } 0 \leq a \leq 1, \\
\frac{\pi}{2} & \text{if } a \geq 1.
\end{cases}\]
\end{problembox}

\begin{problembox}[10.28: Series of integrals]
\begin{enumerate}[label=(\alph*)]
    \item If $s > 0$ and $a > 0$, show that the series
    \[\sum_{n=1}^{\infty} \frac{1}{n} \int_{a}^{\infty} \frac{\sin 2n\pi x}{x^{s}} \, dx\]
    converges and prove that
    \[\lim_{a \to +\infty} \sum_{n=1}^{\infty} \frac{1}{n} \int_{a}^{\infty} \frac{\sin 2n\pi x}{x^{s}} \, dx = 0.\]
    \item Let $f(x) = \sum_{n=1}^{\infty} \sin (2n\pi x)/n$. Show that
    \[\int_{0}^{\infty} \frac{f(x)}{x^{s}} \, dx = (2\pi)^{s-1} \zeta (2 - s) \int_{0}^{\infty} \frac{\sin t}{t^{s}} \, dt, \quad \text{if } 0 < s < 1,\]
    where $\zeta$ denotes the Riemann zeta function.
\end{enumerate}
\end{problembox}

\begin{problembox}[10.29: Derivatives of Gamma function]
\begin{enumerate}[label=(\alph*)]
    \item Derive the following formula for the nth derivative of the Gamma function:
    \[\Gamma^{(n)}(x) = \int_{0}^{\infty} e^{-t} t^{x-1} (\log t)^{n} \, dt \quad (x > 0).\]
    \item When $x = 1$, show that this can be written as follows:
    \[\Gamma^{(n)}(1) = \int_{0}^{1} (t^{2} + (-1)^{n} e^{t-1/t}) e^{-t} t^{-2} (\log t)^{n} \, dt.\]
    \item Use (b) to show that $\Gamma^{(n)}(1)$ has the same sign as $(-1)^{n}$.
\end{enumerate}
\end{problembox}

\begin{problembox}[10.30: Properties of Gamma function]
Use the result $\int_{0}^{\infty} e^{-x^{2}} \, dx = \frac{1}{2} \sqrt{\pi}$ to prove that $\Gamma(\frac{1}{2}) = \sqrt{\pi}$. Prove that $\Gamma(n + 1) = n!$ and that $\Gamma(n + \frac{1}{2}) = (2n)! \sqrt{\pi}/4^{n}n!$ if $n = 0, 1, 2, \ldots$.
\end{problembox}

\begin{problembox}[10.31: Series representation of Gamma function]
\begin{enumerate}[label=(\alph*)]
    \item Show that for $x > 0$ we have the series representation
    \[\Gamma(x) = \sum_{n=0}^{\infty} \frac{(-1)^n}{n!} \frac{1}{n + x} + \sum_{n=0}^{\infty} c_n x^n,\]
    where $c_n = (1/n!) \int_0^\infty t^{-1} e^{-t} (\log t)^n dt$. \textbf{Hint:} Write $\int_0^\infty = \int_0^1 + \int_1^\infty$ and use an appropriate power series expansion in each integral.
    \item Show that the power series $\sum_{n=0}^{\infty} c_n z^n$ converges for every complex $z$ and that the series $\sum_{n=0}^{\infty} [(-1)^n / n!]/(n + z)$ converges for every complex $z \neq 0, -1, -2, \ldots$.
\end{enumerate}
\end{problembox}

\begin{problembox}[10.32: Limit of Laplace transform]
Assume that $f$ is of bounded variation on $[0, b]$ for every $b > 0$, and that $\lim_{x \to +\infty} f(x)$ exists. Denote this limit by $f(\infty)$ and prove that
\[\lim_{y \to 0+} y \int_0^\infty e^{-xy}f(x) \, dx = f(\infty).\]
\textbf{Hint.} Use integration by parts.
\end{problembox}

\begin{problembox}[10.33: Limit of Mellin transform]
Assume that $f$ is of bounded variation on $[0, 1]$. Prove that
\[\lim_{y \to 0+} y \int_0^1 x^{y-1}f(x) \, dx = f(0+).\]
\end{problembox}

\section{Measurable functions}

\begin{problembox}[10.34: Measurability of derivative]
If $f$ is Lebesgue-integrable on an open interval $I$ and if $f'(x)$ exists almost everywhere on $I$, prove that $f'$ is measurable on $I$.
\end{problembox}

\begin{problembox}[10.35: Measurable functions]
\begin{enumerate}[label=(\alph*)]
    \item Let $\{s_n\}$ be a sequence of step functions such that $s_n \to f$ everywhere on $\mathbb{R}$. Prove that, for every real $a$,
    \[f^{-1}((a, +\infty)) = \bigcup_{n=1}^\infty \bigcap_{k=n}^\infty s_k^{-1} \left( \left( a + \frac{1}{n}, +\infty \right) \right).\]
    \item If $f$ is measurable on $\mathbb{R}$, prove that for every open subset $A$ of $\mathbb{R}$ the set $f^{-1}(A)$ is measurable.
\end{enumerate}
\end{problembox}

\begin{problembox}[10.36: Nonmeasurable set example]
This exercise describes an example of a nonmeasurable set in $\mathbb{R}$. If $x$ and $y$ are real numbers in the interval $[0, 1]$, we say that $x$ and $y$ are equivalent, written $x \sim y$, whenever $x - y$ is rational. The relation $\sim$ is an equivalence relation, and the interval $[0, 1]$ can be expressed as a disjoint union of subsets (called equivalence classes) in each of which no two distinct points are equivalent. Choose a point from each equivalence class and let $E$ be the set of points so chosen. We assume that $E$ is measurable and obtain a contradiction. Let $A = \{r_1, r_2, \ldots \}$ denote the set of rational numbers in $[-1, 1]$ and let $E_n = \{r_n + x : x \in E\}$.
\begin{enumerate}[label=(\alph*)]
    \item Prove that each $E_n$ is measurable and that $\mu(E_n) = \mu(E)$.
    \item Prove that $\{E_1, E_2, \ldots \}$ is a disjoint collection of sets whose union contains $[0, 1]$ and is contained in $[-1, 2]$.
    \item Use parts (a) and (b) along with the countable additivity of Lebesgue measure to obtain a contradiction.
\end{enumerate}
\end{problembox}

\begin{problembox}[10.37: Nonmeasurable function]
Refer to Exercise 10.36 and prove that the characteristic function $\chi_E$ is not measurable. Let $f = \chi_E - \chi_{I-E}$ where $I = [0, 1]$. Prove that $|f| \in L(I)$ but that $f \notin M(I)$. (Compare with Corollary 1 of Theorem 10.35.)
\end{problembox}

\section{Square-integrable functions}

\begin{problembox}[10.38: Norm convergence]
If $\lim_{n \to \infty} \| f_n - f \| = 0$, prove that $\lim_{n \to \infty} \| f_n \| = \| f \|$.
\end{problembox}

\begin{problembox}[10.39: Almost everywhere convergence]
If $\lim_{n \to \infty} \| f_n - f \| = 0$ and if $\lim_{n \to \infty} f_n(x) = g(x)$ almost everywhere on $I$, prove that $f(x) = g(x)$ almost everywhere on $I$.
\end{problembox}

\begin{problembox}[10.40: Uniform convergence]
If $f_n \to f$ uniformly on a compact interval $I$, and if each $f_n$ is continuous on $I$, prove that $\lim_{n \to \infty} \| f_n - f \| = 0$.
\end{problembox}

\begin{problembox}[10.41: Weak convergence]
If $\lim_{n \to \infty} \| f_n - f \| = 0$, prove that $\lim_{n \to \infty} \int_0^x f_n \cdot g = \int_0^x f \cdot g$ for every $g$ in $L^2(I)$.
\end{problembox}

\begin{problembox}[10.42: Product convergence]
If $\lim_{n \to \infty} \| f_n - f \| = 0$ and $\lim_{n \to \infty} \| g_n - g \| = 0$, prove that $\lim_{n \to \infty} \int_0^x f_n \cdot g_n = \int_0^x f \cdot g$.
\end{problembox}