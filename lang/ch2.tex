\chapter{Rings}

\section{Localization and Prime Ideals}

\noindent\textbf{Definitions and Theorems:}
\begin{itemize}
\item A \textbf{multiplicative subset} of a ring $A$ is a subset $S$ such that $1 \in S$ and if $s, t \in S$ then $st \in S$.
\item The \textbf{localization} $S^{-1}A$ is the ring of fractions $a/s$ where $a \in A$ and $s \in S$, with the usual addition and multiplication.
\item A \textbf{local ring} is a commutative ring with exactly one maximal ideal.
\item A \textbf{prime ideal} $\mathfrak{p}$ is an ideal such that if $ab \in \mathfrak{p}$ then either $a \in \mathfrak{p}$ or $b \in \mathfrak{p}$.
\item A \textbf{maximal ideal} is an ideal that is maximal with respect to inclusion among proper ideals.
\end{itemize}

We let $A$ denote a commutative ring.

\begin{problembox}[2.1: Maximal Ideal in Localization]
Suppose that $1 \neq 0$ in $A$. Let $S$ be a multiplicative subset of $A$ not containing $0$. Let $\mathfrak{p}$ be a maximal element in the set of ideals of $A$ whose intersection with $S$ is empty. Show that $\mathfrak{p}$ is prime.
\end{problembox}

\noindent\textbf{Strategy:} Use proof by contradiction. Assume $\mathfrak{p}$ is not prime, so there exist $a, b \in A$ with $ab \in \mathfrak{p}$ but $a, b \notin \mathfrak{p}$. Use the maximality of $\mathfrak{p}$ to show that $\mathfrak{p} + (a)$ and $\mathfrak{p} + (b)$ must intersect $S$, then multiply the resulting equations to reach a contradiction.

\noindent\textbf{Solution:}
Let $\mathfrak{p}$ be a maximal element in the set of ideals of $A$ whose intersection with $S$ is empty. We need to show that $\mathfrak{p}$ is prime.

Suppose for contradiction that $\mathfrak{p}$ is not prime. Then there exist elements $a, b \in A$ such that $ab \in \mathfrak{p}$ but $a \notin \mathfrak{p}$ and $b \notin \mathfrak{p}$.

Since $a \notin \mathfrak{p}$, the ideal $\mathfrak{p} + (a)$ properly contains $\mathfrak{p}$. By maximality of $\mathfrak{p}$, we must have $(\mathfrak{p} + (a)) \cap S \neq \emptyset$. Similarly, $(\mathfrak{p} + (b)) \cap S \neq \emptyset$.

This means there exist $p_1, p_2 \in \mathfrak{p}$, $r_1, r_2 \in A$, and $s_1, s_2 \in S$ such that:
\[ p_1 + r_1a = s_1 \quad \text{and} \quad p_2 + r_2b = s_2 \]

Multiplying these equations:
\[ (p_1 + r_1a)(p_2 + r_2b) = s_1s_2 \]

Expanding the left side:
\[ p_1p_2 + p_1r_2b + p_2r_1a + r_1r_2ab = s_1s_2 \]

Since $p_1, p_2, ab \in \mathfrak{p}$, we have $p_1p_2 + p_1r_2b + p_2r_1a + r_1r_2ab \in \mathfrak{p}$. But $s_1s_2 \in S$ since $S$ is multiplicative. This contradicts the fact that $\mathfrak{p} \cap S = \emptyset$.

Therefore, $\mathfrak{p}$ must be prime.


\qed
\begin{problembox}[2.2: Surjective Homomorphism Preserves Local Property]
Let $f: A \rightarrow A'$ be a surjective homomorphism of rings, and assume that $A$ is local, $A' \neq 0$. Show that $A'$ is local.
\end{problembox}

\noindent\textbf{Strategy:} Show that $f(\mathfrak{m})$ is the unique maximal ideal of $A'$, where $\mathfrak{m}$ is the unique maximal ideal of $A$. Use the surjectivity of $f$ and the correspondence between ideals to establish both maximality and uniqueness.

\noindent\textbf{Solution:}
Let $\mathfrak{m}$ be the unique maximal ideal of $A$. Since $f$ is surjective, $f(\mathfrak{m})$ is an ideal of $A'$.

We claim that $f(\mathfrak{m})$ is the unique maximal ideal of $A'$.

First, $f(\mathfrak{m})$ is maximal: if $I$ is an ideal of $A'$ containing $f(\mathfrak{m})$, then $f^{-1}(I)$ is an ideal of $A$ containing $\mathfrak{m}$. Since $\mathfrak{m}$ is maximal, either $f^{-1}(I) = \mathfrak{m}$ or $f^{-1}(I) = A$. If $f^{-1}(I) = A$, then $I = A'$ since $f$ is surjective. If $f^{-1}(I) = \mathfrak{m}$, then $I = f(\mathfrak{m})$. Thus $f(\mathfrak{m})$ is maximal.

Second, $f(\mathfrak{m})$ is unique: if $I$ is any maximal ideal of $A'$, then $f^{-1}(I)$ is a proper ideal of $A$ (since $f$ is surjective and $A' \neq 0$). Since $\mathfrak{m}$ is the unique maximal ideal, $f^{-1}(I) \subseteq \mathfrak{m}$, which implies $I \subseteq f(\mathfrak{m})$. By maximality of $I$, we have $I = f(\mathfrak{m})$.

Therefore, $A'$ has exactly one maximal ideal and is local.


\qed
\begin{problembox}[2.3: Unique Maximal Ideal in Localization]
Let $\mathfrak{p}$ be a prime ideal of $A$. Show that $A_{\mathfrak{p}}$ has a unique maximal ideal, consisting of all elements $a/s$ with $a \in \mathfrak{p}$ and $s \notin \mathfrak{p}$.
\end{problembox}

\noindent\textbf{Strategy:} Define $\mathfrak{m} = \{a/s : a \in \mathfrak{p}, s \notin \mathfrak{p}\}$ and show it is the unique maximal ideal by proving it's an ideal, maximal (every element not in it is a unit), and unique (any proper ideal is contained in it).

\noindent\textbf{Solution:}
Let $S = A \setminus \mathfrak{p}$. Since $\mathfrak{p}$ is prime, $S$ is a multiplicative subset of $A$.

Let $\mathfrak{m} = \{a/s : a \in \mathfrak{p}, s \notin \mathfrak{p}\}$. We need to show that $\mathfrak{m}$ is the unique maximal ideal of $A_{\mathfrak{p}}$.

First, $\mathfrak{m}$ is an ideal: if $a_1/s_1, a_2/s_2 \in \mathfrak{m}$, then $a_1/s_1 + a_2/s_2 = (a_1s_2 + a_2s_1)/(s_1s_2) \in \mathfrak{m}$ since $a_1s_2 + a_2s_1 \in \mathfrak{p}$ and $s_1s_2 \notin \mathfrak{p}$. If $a/s \in \mathfrak{m}$ and $b/t \in A_{\mathfrak{p}}$, then $(a/s)(b/t) = (ab)/(st) \in \mathfrak{m}$ since $ab \in \mathfrak{p}$ and $st \notin \mathfrak{p}$.

Second, $\mathfrak{m}$ is maximal: if $a/s \in A_{\mathfrak{p}} \setminus \mathfrak{m}$, then $a \notin \mathfrak{p}$, so $a \in S$. Then $s/a \in A_{\mathfrak{p}}$ and $(a/s)(s/a) = 1$, so $a/s$ is a unit. This shows that every element not in $\mathfrak{m}$ is a unit, which means $\mathfrak{m}$ is maximal.

Finally, $\mathfrak{m}$ is unique: any proper ideal $I$ of $A_{\mathfrak{p}}$ must be contained in $\mathfrak{m}$, since if $I$ contains an element $a/s$ with $a \notin \mathfrak{p}$, then $a/s$ is a unit, which would make $I = A_{\mathfrak{p}}$.

Therefore, $A_{\mathfrak{p}}$ is a local ring with unique maximal ideal $\mathfrak{m}$.


\qed
\section{Principal and Factorial Rings}

\noindent\textbf{Definitions and Theorems:}
\begin{itemize}
\item A \textbf{principal ring} (PID) is an integral domain in which every ideal is principal (generated by a single element).
\item A \textbf{factorial ring} (UFD) is an integral domain in which every non-zero non-unit element can be written as a product of irreducible elements, and this factorization is unique up to order and units.
\item A \textbf{prime element} $p$ in a ring $A$ is a non-zero non-unit such that if $p$ divides $ab$ then $p$ divides $a$ or $p$ divides $b$.
\item An \textbf{irreducible element} $p$ is a non-zero non-unit such that if $p = ab$ then either $a$ or $b$ is a unit.
\item A \textbf{greatest common divisor} (GCD) of elements $a_1, \ldots, a_n$ is an element $d$ such that $d$ divides each $a_i$ and if $e$ divides each $a_i$ then $e$ divides $d$.
\end{itemize}

\begin{problembox}[2.4: Localization Preserves Principal Property]
Let $A$ be a principal ring and $S$ a multiplicative subset with $0 \notin S$. Show that $S^{-1}A$ is principal.
\end{problembox}

\noindent\textbf{Strategy:} For any ideal $I$ of $S^{-1}A$, consider the ideal $J = \{a \in A : a/1 \in I\}$ of $A$. Since $A$ is principal, $J = (d)$ for some $d \in A$. Show that $I = (d/1)$.

\noindent\textbf{Solution:}
Let $I$ be an ideal of $S^{-1}A$. We need to show that $I$ is principal.

Let $J = \{a \in A : a/1 \in I\}$. Then $J$ is an ideal of $A$. Since $A$ is principal, $J = (d)$ for some $d \in A$.

We claim that $I = (d/1)$. 

First, if $a/1 \in I$, then $a \in J = (d)$, so $a = rd$ for some $r \in A$. Then $a/1 = (rd)/1 = (r/1)(d/1) \in (d/1)$.

Second, if $a/s \in I$, then $a/1 = (a/s)(s/1) \in I$, so $a \in J = (d)$. Thus $a = rd$ for some $r \in A$, and $a/s = (rd)/s = (r/s)(d/1) \in (d/1)$.

Therefore, $I = (d/1)$ and $S^{-1}A$ is principal.


\qed
\begin{problembox}[2.5: Localization Preserves Factorial Property]
Let $A$ be a factorial ring and $S$ a multiplicative subset with $0 \notin S$. Show that $S^{-1}A$ is factorial, and that the prime elements of $S^{-1}A$ are of the form $up$ with primes $p$ of $A$ such that $(p) \cap S$ is empty, and units $u$ in $S^{-1}A$.
\end{problembox}

\noindent\textbf{Strategy:} Use the factorization of elements in $A$ to show that $S^{-1}A$ is factorial. For the prime elements, show that elements in $S$ become units in $S^{-1}A$, while irreducible elements not in $S$ remain irreducible, and use the unique factorization property of $A$.

\noindent\textbf{Solution:}
First, we show that $S^{-1}A$ is factorial. Let $a/s \in S^{-1}A$ be a non-zero non-unit. Then $a \in A$ is non-zero and not a unit in $S^{-1}A$.

Since $A$ is factorial, $a$ can be written as a product of irreducible elements in $A$: $a = p_1 \cdots p_n$. Then $a/s = (p_1/1) \cdots (p_n/1)(1/s)$.

We need to show that each $p_i/1$ is either irreducible or a unit in $S^{-1}A$. If $p_i \in S$, then $p_i/1$ is a unit. If $p_i \notin S$, then $p_i/1$ is irreducible in $S^{-1}A$ (since if $p_i/1 = (a/s)(b/t)$, then $p_i st = ab$, which would contradict the irreducibility of $p_i$ in $A$ unless one of $a$ or $b$ is a unit).

For uniqueness, suppose $a/s = (p_1/1) \cdots (p_m/1)(1/s_1) = (q_1/1) \cdots (q_n/1)(1/s_2)$ where $p_i, q_j$ are irreducible in $A$ and not in $S$. Then $a s_1 = p_1 \cdots p_m$ and $a s_2 = q_1 \cdots q_n$. Since $A$ is factorial, these factorizations are the same up to units and order.

For the second part, let $p$ be a prime element of $A$ such that $(p) \cap S = \emptyset$. We show that $p/1$ is prime in $S^{-1}A$. If $(p/1)$ divides $(a/s)(b/t)$, then $p$ divides $ab$ in $A$, so $p$ divides $a$ or $p$ divides $b$. Thus $(p/1)$ divides $(a/s)$ or $(b/t)$.

Conversely, if $q$ is a prime element of $S^{-1}A$, then $q = a/s$ where $a \in A$ is irreducible and $a \notin S$. Since $q$ is prime, $a$ must be prime in $A$.


\qed
\begin{problembox}[2.6: Localization at Prime is Principal]
Let $A$ be a factorial ring and $p$ a prime element. Show that the local ring $A_{(p)}$ is principal.
\end{problembox}

\noindent\textbf{Strategy:} Use Problem 2.4 since factorial rings are principal, or show directly that any ideal $I$ of $A_{(p)}$ corresponds to an ideal $J$ of $A$ contained in $(p)$, which must be $(p^n)$ for some $n \geq 0$.

\noindent\textbf{Solution:}
Let $S = A \setminus (p)$. Then $A_{(p)} = S^{-1}A$.

By Problem 2.4, since $A$ is principal (factorial rings are principal), $A_{(p)}$ is principal.

Alternatively, we can show this directly. Let $I$ be an ideal of $A_{(p)}$. Let $J = \{a \in A : a/1 \in I\}$. Then $J$ is an ideal of $A$ contained in $(p)$ (since if $a \notin (p)$, then $a/1$ is a unit in $A_{(p)}$).

Since $A$ is factorial, $J = (p^n)$ for some $n \geq 0$. Then $I = (p^n/1) = (p/1)^n$.


\qed
\begin{problembox}[2.7: GCD in Principal Rings]
Let $A$ be a principal ring and $a_1, \ldots, a_n$ non-zero elements of $A$. Let $(a_1, \ldots, a_n) = (d)$. Show that $d$ is a greatest common divisor for the $a_i$ ($i = 1, \ldots, n$).
\end{problembox}

\noindent\textbf{Strategy:} Show that $d$ divides each $a_i$ (since $(a_1, \ldots, a_n) \subseteq (d)$), and that any common divisor $e$ of the $a_i$ divides $d$ (using the fact that $d$ is a linear combination of the $a_i$).

\noindent\textbf{Solution:}
Since $(a_1, \ldots, a_n) = (d)$, we have $d \in (a_1, \ldots, a_n)$, so $d = r_1a_1 + \cdots + r_na_n$ for some $r_i \in A$. This shows that $d$ is a linear combination of the $a_i$.

Also, since $(a_1, \ldots, a_n) \subseteq (d)$, each $a_i \in (d)$, so $d$ divides each $a_i$.

Now let $e$ be any element that divides each $a_i$. Then $a_i = s_i e$ for some $s_i \in A$. Since $d = r_1a_1 + \cdots + r_na_n = r_1(s_1e) + \cdots + r_n(s_ne) = (r_1s_1 + \cdots + r_ns_n)e$, we have $e$ divides $d$.

Therefore, $d$ is a greatest common divisor of the $a_i$.


\qed
\section{Group of Units}

\noindent\textbf{Definitions and Theorems:}
\begin{itemize}
\item The \textbf{group of units} of a ring $A$ is the set of all invertible elements, denoted $A^*$.
\item A \textbf{cyclic group} is a group generated by a single element.
\item A group is of \textbf{type} $(n_1, n_2, \ldots, n_k)$ if it is isomorphic to $\mathbb{Z}/n_1\mathbb{Z} \times \mathbb{Z}/n_2\mathbb{Z} \times \cdots \times \mathbb{Z}/n_k\mathbb{Z}$.
\item The \textbf{Euler totient function} $\phi(n)$ counts the number of integers between 1 and $n$ that are coprime to $n$.
\end{itemize}

\begin{problembox}[2.8: Structure of Units Modulo $p^r$]
Let $p$ be a prime number, and let $A$ be the ring $\mathbb{Z}/p^r\mathbb{Z}$ ($r =$ integer $\geq 1$). Let $G$ be the group of units in $A$, i.e. the group of integers prime to $p$, modulo $p^r$. Show that $G$ is cyclic, except in the case when
\[ p = 2, \quad r \geq 3, \]
in which case it is of type $(2, 2^{r-2})$. 

[Hint: In the general case, show that $G$ is the product of a cyclic group generated by $1 + p$, and a cyclic group of order $p - 1$. In the exceptional case, show that $G$ is the product of the group $\{\pm 1\}$ with the cyclic group generated by the residue class of $5 \mod 2^r$.]
\end{problembox}

\noindent\textbf{Strategy:} Use induction on $r$ and the structure of the multiplicative group modulo prime powers. For odd primes, use the exact sequence involving $U_1$ and $(\mathbb{Z}/p\mathbb{Z})^*$. For $p = 2$, handle the exceptional case by showing that $G$ is the product of $\{\pm 1\}$ and a cyclic group generated by 5.

\noindent\textbf{Solution:}
We will prove this by induction on $r$. The key insight is to use the structure of the multiplicative group modulo prime powers.

For $r = 1$, $G = (\mathbb{Z}/p\mathbb{Z})^*$ is cyclic of order $p-1$ by the primitive root theorem.

For $r > 1$, we consider the exact sequence:
\[ 1 \rightarrow U_1 \rightarrow G \rightarrow (\mathbb{Z}/p\mathbb{Z})^* \rightarrow 1 \]
where $U_1 = \{1 + ap : a \in \mathbb{Z}/p^{r-1}\mathbb{Z}\}$.

The group $U_1$ is isomorphic to the additive group $\mathbb{Z}/p^{r-1}\mathbb{Z}$ via the map $1 + ap \mapsto a$. This is a cyclic group of order $p^{r-1}$.

For odd primes $p$, $U_1$ is cyclic and $(\mathbb{Z}/p\mathbb{Z})^*$ is cyclic of order $p-1$. Since $\gcd(p^{r-1}, p-1) = 1$, the group $G$ is cyclic.

For $p = 2$, we need to be more careful. For $r = 2$, $G$ is cyclic of order 1. For $r = 3$, $G$ has order 2 and is cyclic.

For $r \geq 3$, the group $U_1$ is cyclic of order $2^{r-1}$, but $(\mathbb{Z}/2\mathbb{Z})^*$ is trivial. However, the group $U_2 = \{1 + 4a : a \in \mathbb{Z}/2^{r-2}\mathbb{Z}\}$ is cyclic of order $2^{r-2}$.

The group $G$ is the product of $\{\pm 1\}$ (which has order 2) and the cyclic group generated by 5 (which has order $2^{r-2}$). Since these groups have coprime orders, $G$ is of type $(2, 2^{r-2})$.

The key fact is that 5 generates a cyclic subgroup of order $2^{r-2}$ in $G$ for $r \geq 3$, and this subgroup together with $\{\pm 1\}$ generates all of $G$.


\qed
\section{Quadratic Rings}

\noindent\textbf{Definitions and Theorems:}
\begin{itemize}
\item A \textbf{quadratic ring} is a subring of $\mathbb{C}$ of the form $\mathbb{Z}[\sqrt{d}]$ where $d$ is a square-free integer.
\item The \textbf{norm} of an element $a + b\sqrt{d}$ is $N(a + b\sqrt{d}) = a^2 - db^2$.
\item A \textbf{unit} in a ring is an element with a multiplicative inverse.
\item An \textbf{irreducible element} is a non-zero non-unit that cannot be written as a product of two non-units.
\item The \textbf{Gaussian integers} are the ring $\mathbb{Z}[i]$ where $i = \sqrt{-1}$.
\end{itemize}

\begin{problembox}[2.9: Principal Ring of Gaussian Integers]
Let $i$ be the complex number $\sqrt{-1}$. Show that the ring $\mathbb{Z}[i]$ is principal, and hence factorial. What are the units?
\end{problembox}

\noindent\textbf{Strategy:} Use the Euclidean algorithm with the norm function $N(a + bi) = a^2 + b^2$. For any ideal $I$, choose an element $\alpha$ with minimal norm and show that $I = (\alpha)$ by using the division algorithm in $\mathbb{C}$ and the fact that the norm decreases in each step.

\noindent\textbf{Solution:}
To show that $\mathbb{Z}[i]$ is principal, we use the Euclidean algorithm with the norm function $N(a + bi) = a^2 + b^2$.

Let $I$ be a non-zero ideal of $\mathbb{Z}[i]$. Let $\alpha$ be a non-zero element of $I$ with minimal norm. We claim that $I = (\alpha)$.

Let $\beta \in I$. We need to show that $\alpha$ divides $\beta$. Consider the complex number $\beta/\alpha = x + yi$ where $x, y \in \mathbb{Q}$. Let $m, n$ be integers such that $|x - m| \leq 1/2$ and $|y - n| \leq 1/2$.

Let $\gamma = \beta - \alpha(m + ni)$. Then $\gamma \in I$ and $N(\gamma) = N(\alpha)N((x-m) + (y-n)i) = N(\alpha)((x-m)^2 + (y-n)^2) < N(\alpha)$ since $(x-m)^2 + (y-n)^2 \leq 1/4 + 1/4 = 1/2 < 1$.

By minimality of $N(\alpha)$, we must have $\gamma = 0$, so $\beta = \alpha(m + ni)$. Therefore, $I = (\alpha)$.

Since $\mathbb{Z}[i]$ is principal, it is also factorial (UFD).

The units of $\mathbb{Z}[i]$ are the elements with norm 1. These are $\pm 1, \pm i$.


\qed
\begin{problembox}[2.10: Non-Factorial Quadratic Ring]
Let $D$ be an integer $\geq 1$, and let $R$ be the set of all elements $a + b\sqrt{-D}$ with $a, b \in \mathbb{Z}$.
\begin{enumerate}[label=(\alph*)]
    \item Show that $R$ is a ring.
    \item Using the fact that complex conjugation is an automorphism of $\mathbb{C}$, show that complex conjugation induces an automorphism of $R$.
    \item Show that if $D \geq 2$ then the only units in $R$ are $\pm 1$.
    \item Show that $3, 2 + \sqrt{-5}, 2 - \sqrt{-5}$ are irreducible elements in $\mathbb{Z}[\sqrt{-5}]$.
\end{enumerate}
\end{problembox}

\noindent\textbf{Strategy:} For (a), verify ring axioms directly. For (b), show that complex conjugation preserves the structure of $R$. For (c), use the norm function to show that units must have norm 1. For (d), use the norm to show that these elements cannot be factored into non-units.

\noindent\textbf{Solution:}
\begin{enumerate}[label=(\alph*)]
    \item We need to show that $R$ is closed under addition and multiplication. Let $\alpha = a + b\sqrt{-D}$ and $\beta = c + d\sqrt{-D}$ be elements of $R$.
    
    Then $\alpha + \beta = (a + c) + (b + d)\sqrt{-D} \in R$ and $\alpha\beta = (ac - bdD) + (ad + bc)\sqrt{-D} \in R$.
    
    Also, $0 = 0 + 0\sqrt{-D} \in R$ and $1 = 1 + 0\sqrt{-D} \in R$. Therefore, $R$ is a ring.
    
    \item Complex conjugation is the map $\sigma: \mathbb{C} \rightarrow \mathbb{C}$ defined by $\sigma(a + bi) = a - bi$. This is an automorphism of $\mathbb{C}$.
    
    For $\alpha = a + b\sqrt{-D} \in R$, we have $\sigma(\alpha) = a - b\sqrt{-D} \in R$. Since $\sigma$ preserves addition and multiplication, it induces an automorphism of $R$.
    
    \item Let $\alpha = a + b\sqrt{-D}$ be a unit in $R$. Then there exists $\beta = c + d\sqrt{-D} \in R$ such that $\alpha\beta = 1$.
    
    Taking norms: $N(\alpha)N(\beta) = N(1) = 1$. Since $N(\alpha) = a^2 + Db^2 \geq 0$ and $N(\beta) = c^2 + Dd^2 \geq 0$, we must have $N(\alpha) = N(\beta) = 1$.
    
    If $D \geq 2$, then $N(\alpha) = a^2 + Db^2 = 1$ implies $b = 0$ and $a^2 = 1$. Therefore, $\alpha = \pm 1$.
    
    \item We show that these elements are irreducible in $\mathbb{Z}[\sqrt{-5}]$.
    
    For 3: If $3 = \alpha\beta$ where $\alpha, \beta \in \mathbb{Z}[\sqrt{-5}]$ are non-units, then $N(3) = 9 = N(\alpha)N(\beta)$. Since $\alpha, \beta$ are non-units, $N(\alpha), N(\beta) > 1$. The only possibility is $N(\alpha) = N(\beta) = 3$. But there are no elements in $\mathbb{Z}[\sqrt{-5}]$ with norm 3 (since $a^2 + 5b^2 = 3$ has no integer solutions). Therefore, 3 is irreducible.
    
    For $2 + \sqrt{-5}$: $N(2 + \sqrt{-5}) = 4 + 5 = 9$. If $2 + \sqrt{-5} = \alpha\beta$ where $\alpha, \beta$ are non-units, then $N(\alpha) = N(\beta) = 3$, which is impossible as above. Therefore, $2 + \sqrt{-5}$ is irreducible.
    
    Similarly, $2 - \sqrt{-5}$ is irreducible.
\end{enumerate}


\qed
\section{Trigonometric Polynomials}

\noindent\textbf{Definitions and Theorems:}
\begin{itemize}
\item A \textbf{trigonometric polynomial} is a finite linear combination of functions $\cos(nx)$ and $\sin(nx)$ for non-negative integers $n$.
\item The \textbf{trigonometric degree} of a trigonometric polynomial is the maximum frequency appearing in its expression.
\item A \textbf{zero divisor} in a ring is a non-zero element $a$ such that there exists a non-zero element $b$ with $ab = 0$.
\item An \textbf{irreducible element} in a ring is a non-zero non-unit that cannot be written as a product of two non-units.
\end{itemize}

\begin{problembox}[2.11: Trigonometric Polynomial Ring]
Let $R$ be the ring of trigonometric polynomials as defined in the text. Show that $R$ consists of all functions $f$ on $\mathbb{R}$ which have an expression of the form
\[ f(x) = a_0 + \sum_{m=1}^n (a_m \cos mx + b_m \sin mx), \]
where $a_0, a_m, b_m$ are real numbers. Define the trigonometric degree $\deg_{tr}(f)$ to be the maximum of the integers $r, s$ such that $a_r, b_s \neq 0$. Prove that
\[ \deg_{tr}(fg) = \deg_{tr}(f) + \deg_{tr}(g). \]
Deduce from this that $R$ has no divisors of $0$, and also deduce that the functions $\sin x$ and $1 - \cos x$ are irreducible elements in that ring.
\end{problembox}

\noindent\textbf{Strategy:} Use trigonometric identities to show that multiplication of trigonometric polynomials adds their degrees. Use this degree property to show that the product of non-zero elements is non-zero (no zero divisors), and that elements of degree 1 cannot be factored into non-constant elements.

\noindent\textbf{Solution:}
First, we show that $R$ consists of all functions of the given form. This follows from the fact that any trigonometric polynomial can be written as a finite linear combination of $\cos(nx)$ and $\sin(nx)$ terms.

Now we prove that $\deg_{tr}(fg) = \deg_{tr}(f) + \deg_{tr}(g)$.

Let $f(x) = a_0 + \sum_{m=1}^n (a_m \cos mx + b_m \sin mx)$ and $g(x) = c_0 + \sum_{k=1}^p (c_k \cos kx + d_k \sin kx)$.

When we multiply $f$ and $g$, we get terms of the form:
\begin{align*}
\cos(mx) \cos(kx) &= \frac{1}{2}(\cos((m+k)x) + \cos((m-k)x)) \\
\cos(mx) \sin(kx) &= \frac{1}{2}(\sin((m+k)x) + \sin((m-k)x)) \\
\sin(mx) \cos(kx) &= \frac{1}{2}(\sin((m+k)x) - \sin((m-k)x)) \\
\sin(mx) \sin(kx) &= \frac{1}{2}(-\cos((m+k)x) + \cos((m-k)x))
\end{align*}

The highest frequency that can appear is $m + k$ where $m$ is the highest frequency in $f$ and $k$ is the highest frequency in $g$. Therefore, $\deg_{tr}(fg) = \deg_{tr}(f) + \deg_{tr}(g)$.

Since $\deg_{tr}(fg) = \deg_{tr}(f) + \deg_{tr}(g)$, if $f$ and $g$ are non-zero, then $\deg_{tr}(fg) > 0$, so $fg \neq 0$. This shows that $R$ has no zero divisors.

For irreducibility, suppose $\sin x = fg$ where $f, g \in R$ are non-units. Then $\deg_{tr}(f) + \deg_{tr}(g) = \deg_{tr}(\sin x) = 1$. Since $\deg_{tr}(f), \deg_{tr}(g) \geq 0$, one of them must be 0 and the other must be 1. But if $\deg_{tr}(f) = 0$, then $f$ is a constant, and if $\deg_{tr}(g) = 0$, then $g$ is a constant. Since $\sin x$ is not a constant multiple of any other trigonometric polynomial, this is impossible. Therefore, $\sin x$ is irreducible.

Similarly, $\deg_{tr}(1 - \cos x) = 1$, so if $1 - \cos x = fg$, then one of $f$ or $g$ must be a constant. But $1 - \cos x$ is not a constant multiple of any other trigonometric polynomial, so it is irreducible.


\qed
\section{Dedekind Rings}

\noindent\textbf{Definitions and Theorems:}
\begin{itemize}
\item A \textbf{Dedekind ring} is a Noetherian integral domain that is integrally closed and has Krull dimension 1.
\item A \textbf{multiplicative function} $f$ satisfies $f(mn) = f(m)f(n)$ whenever $\gcd(m,n) = 1$.
\item The \textbf{M\"obius function} $\mu(n)$ is defined as $\mu(1) = 1$, $\mu(p_1 \cdots p_r) = (-1)^r$ for distinct primes $p_i$, and $\mu(n) = 0$ if $n$ is divisible by a square.
\item The \textbf{convolution} of two arithmetic functions $f$ and $g$ is $(f * g)(n) = \sum_{d|n} f(d)g(n/d)$.
\end{itemize}

Prove the following statements about a Dedekind ring $o$. To simplify terminology,
by an ideal we shall mean non-zero ideal unless otherwise specified. We let $K$
denote the quotient field of $o$,

\begin{problembox}[2.12: Ring of Arithmetic Functions]
Let $P$ be the set of positive integers and $R$ the set of functions defined on $P$ with values in a commutative ring $K$. Define the sum in $R$ to be the ordinary addition of functions, and define the convolution product by the formula
\[ (f * g)(m) = \sum_{xy=m} f(x)g(y), \]
where the sum is taken over all pairs $(x, y)$ of positive integers such that $xy = m$.
\begin{enumerate}[label=(\alph*)]
    \item Show that $R$ is a commutative ring, whose unit element is the function $\delta$ such that $\delta(1) = 1$ and $\delta(x) = 0$ if $x \neq 1$.
    \item A function $f$ is said to be multiplicative if $f(mn) = f(m)f(n)$ whenever $m, n$ are relatively prime. If $f, g$ are multiplicative, show that $f * g$ is multiplicative.
    \item Let $\mu$ be the M\"obius function such that $\mu(1) = 1$, $\mu(p_1 \cdots p_r) = (-1)^r$ if $p_1, \ldots, p_r$ are distinct primes, and $\mu(m) = 0$ if $m$ is divisible by $p^2$ for some prime $p$. Show that $\mu * \varphi_1 = \delta$, where $\varphi_1$ denotes the constant function having value 1. [Hint: Show first that $\mu$ is multiplicative, and then prove the assertion for prime powers.] The M\"obius inversion formula of elementary number theory is then nothing else but the relation $\mu * \varphi_1 * f = f$.
\end{enumerate}
\end{problembox}

\noindent\textbf{Strategy:} For (a), verify ring axioms directly using the convolution definition. For (b), use the fact that if $m, n$ are coprime, then any divisor of $mn$ can be written uniquely as a product of a divisor of $m$ and a divisor of $n$. For (c), first show $\mu$ is multiplicative, then use multiplicativity to reduce to checking the identity on prime powers.

\noindent\textbf{Solution:}
\begin{enumerate}[label=(\alph*)]
    \item We need to verify the ring axioms. Addition is clearly commutative and associative since it's pointwise addition.
    
    For multiplication, we check associativity:
    \begin{align*}
    ((f * g) * h)(m) &= \sum_{xy=m} (f * g)(x)h(y) = \sum_{xy=m} \sum_{ab=x} f(a)g(b)h(y) \\
    &= \sum_{aby=m} f(a)g(b)h(y) = \sum_{abc=m} f(a)g(b)h(c)
    \end{align*}
    
    Similarly, $(f * (g * h))(m) = \sum_{abc=m} f(a)g(b)h(c)$, so convolution is associative.
    
    The distributive law follows from:
    \begin{align*}
    (f * (g + h))(m) &= \sum_{xy=m} f(x)(g + h)(y) = \sum_{xy=m} f(x)(g(y) + h(y)) \\
    &= \sum_{xy=m} f(x)g(y) + \sum_{xy=m} f(x)h(y) = (f * g)(m) + (f * h)(m)
    \end{align*}
    
    The function $\delta$ is the unit since $(\delta * f)(m) = \sum_{xy=m} \delta(x)f(y) = f(m)$.
    
    \item Let $m, n$ be relatively prime positive integers. Then:
    \begin{align*}
    (f * g)(mn) &= \sum_{xy=mn} f(x)g(y) = \sum_{a_1a_2=m, b_1b_2=n} f(a_1b_1)g(a_2b_2) \\
    &= \sum_{a_1a_2=m, b_1b_2=n} f(a_1)f(b_1)g(a_2)g(b_2) \\
    &= \left(\sum_{a_1a_2=m} f(a_1)g(a_2)\right)\left(\sum_{b_1b_2=n} f(b_1)g(b_2)\right) \\
    &= (f * g)(m)(f * g)(n)
    \end{align*}
    
    \item First, we show that $\mu$ is multiplicative. Let $m, n$ be relatively prime. If either $m$ or $n$ is divisible by a square, then $\mu(mn) = 0 = \mu(m)\mu(n)$. Otherwise, if $m = p_1 \cdots p_r$ and $n = q_1 \cdots q_s$ are products of distinct primes, then $\mu(mn) = (-1)^{r+s} = (-1)^r(-1)^s = \mu(m)\mu(n)$.
    
    Now we show that $\mu * \varphi_1 = \delta$. Since both $\mu$ and $\varphi_1$ are multiplicative, so is $\mu * \varphi_1$. Therefore, it suffices to check the equality for prime powers.
    
    For $p^k$ where $k \geq 1$:
    \begin{align*}
    (\mu * \varphi_1)(p^k) &= \sum_{d|p^k} \mu(d) = \mu(1) + \mu(p) + \mu(p^2) + \cdots + \mu(p^k) \\
    &= 1 + (-1) + 0 + \cdots + 0 = 0
    \end{align*}
    
    For $m = 1$: $(\mu * \varphi_1)(1) = \mu(1) = 1$.
    
    Therefore, $\mu * \varphi_1 = \delta$.
\end{enumerate}


\qed
\begin{problembox}[2.13: Finitely Generated Ideals]
Every ideal is finitely generated. [Hint: Given an ideal $\mathfrak{a}$, let $\mathfrak{b}$ be the fractional ideal such that $\mathfrak{a}\mathfrak{b} = \mathfrak{o}$. Write $1 = \sum a_i b_i$ with $a_i \in \mathfrak{a}$ and $b_i \in \mathfrak{b}$. Show that $\mathfrak{a} = (a_1, \ldots, a_n)$.]
\end{problembox}

\noindent\textbf{Strategy:} Use the fact that in a Dedekind ring, every ideal has an inverse fractional ideal. Write $1$ as a finite sum using elements from $\mathfrak{a}$ and its inverse, then show that any element of $\mathfrak{a}$ can be written as a linear combination of the $a_i$.

\noindent\textbf{Solution:}
Let $\mathfrak{a}$ be an ideal of $o$. Since $o$ is a Dedekind ring, there exists a fractional ideal $\mathfrak{b}$ such that $\mathfrak{a}\mathfrak{b} = o$.

Since $1 \in o = \mathfrak{a}\mathfrak{b}$, we can write $1 = \sum_{i=1}^n a_i b_i$ where $a_i \in \mathfrak{a}$ and $b_i \in \mathfrak{b}$.

We claim that $\mathfrak{a} = (a_1, \ldots, a_n)$.

Let $a \in \mathfrak{a}$. Then $a = a \cdot 1 = a \sum_{i=1}^n a_i b_i = \sum_{i=1}^n (a b_i) a_i$.

Since $a \in \mathfrak{a}$ and $b_i \in \mathfrak{b}$, we have $a b_i \in \mathfrak{a}\mathfrak{b} = o$. Therefore, $a b_i \in o$ for each $i$.

This shows that $a = \sum_{i=1}^n (a b_i) a_i \in (a_1, \ldots, a_n)$.

Therefore, $\mathfrak{a} = (a_1, \ldots, a_n)$ is finitely generated.


\qed
\begin{problembox}[2.14: Unique Factorization of Ideals]
Every ideal has a factorization as a product of prime ideals, uniquely determined up to permutation.
\end{problembox}

\noindent\textbf{Strategy:} This is a fundamental property of Dedekind rings. For existence, use the fact that every ideal is contained in a maximal ideal and the Noetherian property. For uniqueness, use the fact that prime ideals are maximal and the prime ideal property.

\noindent\textbf{Solution:}
This is a fundamental property of Dedekind rings. We prove existence and uniqueness.

For existence: Let $\mathfrak{a}$ be an ideal. If $\mathfrak{a} = o$, then it's the empty product. Otherwise, let $\mathfrak{p}_1$ be a minimal prime ideal containing $\mathfrak{a}$. Then $\mathfrak{a} \subseteq \mathfrak{p}_1$, so there exists an ideal $\mathfrak{a}_1$ such that $\mathfrak{a} = \mathfrak{p}_1 \mathfrak{a}_1$. Since $\mathfrak{p}_1$ is maximal (in Dedekind rings, non-zero prime ideals are maximal), $\mathfrak{a}_1$ properly contains $\mathfrak{a}$. Continue this process, which must terminate since $o$ is Noetherian.

For uniqueness: Suppose $\mathfrak{p}_1 \cdots \mathfrak{p}_r = \mathfrak{q}_1 \cdots \mathfrak{q}_s$ where the $\mathfrak{p}_i$ and $\mathfrak{q}_j$ are prime ideals. Since $\mathfrak{p}_1$ contains the product $\mathfrak{q}_1 \cdots \mathfrak{q}_s$, it must contain one of the $\mathfrak{q}_j$ (by the prime ideal property). Since both are maximal, $\mathfrak{p}_1 = \mathfrak{q}_j$. Cancel and continue by induction.


\qed
\begin{problembox}[2.15: Principal Prime Ideal]
Suppose $\mathfrak{o}$ has only one prime ideal $\mathfrak{p}$. Let $t \in \mathfrak{p}$ and $t \notin \mathfrak{p}^2$. Then $\mathfrak{p} = (t)$ is principal.
\end{problembox}

\noindent\textbf{Strategy:} Use the unique factorization property of ideals in a Dedekind ring. Since there's only one prime ideal, every non-zero element has a unique factorization as a power of $\mathfrak{p}$. The condition $t \in \mathfrak{p}$ but $t \notin \mathfrak{p}^2$ forces $(t) = \mathfrak{p}$.

\noindent\textbf{Solution:}
Since $o$ has only one prime ideal $\mathfrak{p}$, every non-zero element of $o$ has a unique factorization as a power of $\mathfrak{p}$.

Since $t \in \mathfrak{p}$ and $t \notin \mathfrak{p}^2$, the ideal $(t)$ must be exactly $\mathfrak{p}$.

To see this, suppose $(t) = \mathfrak{p}^n$ for some $n \geq 1$. Since $t \notin \mathfrak{p}^2$, we must have $n = 1$. Therefore, $(t) = \mathfrak{p}$.


\qed
\begin{problembox}[2.16: Localization of Dedekind Ring]
Let $\mathfrak{o}$ be any Dedekind ring. Let $\mathfrak{p}$ be a prime ideal. Let $\mathfrak{o}_{\mathfrak{p}}$ be the local ring at $\mathfrak{p}$. Then $\mathfrak{o}_{\mathfrak{p}}$ is Dedekind and has only one prime ideal.
\end{problembox}

\noindent\textbf{Strategy:} Show that localization preserves the three defining properties of a Dedekind ring: Noetherian, integrally closed, and Krull dimension 1. Then use the fact that localization at a prime ideal creates a local ring with unique maximal ideal.

\noindent\textbf{Solution:}
Let $S = o \setminus \mathfrak{p}$. Then $o_{\mathfrak{p}} = S^{-1}o$.

Since $o$ is Noetherian, $o_{\mathfrak{p}}$ is Noetherian. Since $o$ is integrally closed, $o_{\mathfrak{p}}$ is integrally closed. Since $o$ has Krull dimension 1, $o_{\mathfrak{p}}$ has Krull dimension 1.

Therefore, $o_{\mathfrak{p}}$ is a Dedekind ring.

The unique prime ideal of $o_{\mathfrak{p}}$ is $\mathfrak{p} o_{\mathfrak{p}} = \{a/s : a \in \mathfrak{p}, s \notin \mathfrak{p}\}$. This follows from the fact that localization preserves the prime ideal structure, and in a local ring, the unique maximal ideal is the only prime ideal.


\qed
\begin{problembox}[2.17: Divisibility in Dedekind Rings]
As for the integers, we say that $\mathfrak{a} | \mathfrak{b}$ ($\mathfrak{a}$ divides $\mathfrak{b}$) if there exists an ideal $\mathfrak{c}$ such that $\mathfrak{b} = \mathfrak{a}\mathfrak{c}$. Prove:
\begin{enumerate}[label=(\alph*)]
    \item $\mathfrak{a} | \mathfrak{b}$ if and only if $\mathfrak{b} \subset \mathfrak{a}$.
    \item Let $\mathfrak{a}, \mathfrak{b}$ be ideals. Then $\mathfrak{a} + \mathfrak{b}$ is their greatest common divisor. In particular, $\mathfrak{a}, \mathfrak{b}$ are relatively prime if and only if $\mathfrak{a} + \mathfrak{b} = \mathfrak{o}$.
\end{enumerate}
\end{problembox}

\noindent\textbf{Strategy:} For (a), use the fact that every ideal has an inverse fractional ideal in a Dedekind ring. For (b), show that $\mathfrak{a} + \mathfrak{b}$ is the smallest ideal containing both $\mathfrak{a}$ and $\mathfrak{b}$, which makes it the greatest common divisor.

\noindent\textbf{Solution:}
\begin{enumerate}[label=(\alph*)]
    \item If $\mathfrak{a} | \mathfrak{b}$, then $\mathfrak{b} = \mathfrak{a}\mathfrak{c}$ for some ideal $\mathfrak{c}$. Since $\mathfrak{a}\mathfrak{c} \subseteq \mathfrak{a}$, we have $\mathfrak{b} \subseteq \mathfrak{a}$.
    
    Conversely, if $\mathfrak{b} \subseteq \mathfrak{a}$, then there exists a fractional ideal $\mathfrak{c}$ such that $\mathfrak{a}\mathfrak{c} = o$. Then $\mathfrak{b} = \mathfrak{b}o = \mathfrak{b}(\mathfrak{a}\mathfrak{c}) = \mathfrak{a}(\mathfrak{b}\mathfrak{c})$. Since $\mathfrak{b}\mathfrak{c}$ is an ideal, we have $\mathfrak{a} | \mathfrak{b}$.
    
    \item We need to show that $\mathfrak{a} + \mathfrak{b}$ is the smallest ideal containing both $\mathfrak{a}$ and $\mathfrak{b}$.
    
    Clearly, $\mathfrak{a} + \mathfrak{b}$ contains both $\mathfrak{a}$ and $\mathfrak{b}$. If $\mathfrak{c}$ is any ideal containing both $\mathfrak{a}$ and $\mathfrak{b}$, then $\mathfrak{a} + \mathfrak{b} \subseteq \mathfrak{c}$.
    
    Therefore, $\mathfrak{a} + \mathfrak{b}$ is the greatest common divisor of $\mathfrak{a}$ and $\mathfrak{b}$.
    
    Two ideals are relatively prime if their greatest common divisor is $o$. By the above, this means $\mathfrak{a} + \mathfrak{b} = o$.
\end{enumerate}


\qed
\begin{problembox}[2.18: Prime Ideals are Maximal]
Every prime ideal $\mathfrak{p}$ is maximal. (Remember, $\mathfrak{p} \neq 0$ by convention.) In particular, if $\mathfrak{p}_1, \ldots, \mathfrak{p}_n$ are distinct primes, then the Chinese remainder theorem applies to their powers $\mathfrak{p}_1^{r_1}, \ldots, \mathfrak{p}_n^{r_n}$. Use this to prove:
\end{problembox}

\noindent\textbf{Strategy:} Use the definition of Krull dimension 1, which means that every non-zero prime ideal is maximal. If there were a chain of prime ideals $(0) \subset \mathfrak{p} \subset \mathfrak{m}$, it would contradict the Krull dimension being 1.

\noindent\textbf{Solution:}
Since $o$ is a Dedekind ring, it has Krull dimension 1. This means that every non-zero prime ideal is maximal.

To see this, let $\mathfrak{p}$ be a non-zero prime ideal. If $\mathfrak{p}$ is not maximal, then there exists a maximal ideal $\mathfrak{m}$ such that $\mathfrak{p} \subset \mathfrak{m}$. But this would create a chain of prime ideals $(0) \subset \mathfrak{p} \subset \mathfrak{m}$, contradicting the fact that the Krull dimension is 1.

Since prime ideals are maximal, distinct prime ideals $\mathfrak{p}_1, \ldots, \mathfrak{p}_n$ are pairwise coprime. Therefore, the Chinese remainder theorem applies to their powers $\mathfrak{p}_1^{r_1}, \ldots, \mathfrak{p}_n^{r_n}$.

This means that the natural map $o \rightarrow o/\mathfrak{p}_1^{r_1} \times \cdots \times o/\mathfrak{p}_n^{r_n}$ is surjective.


\qed
\begin{problembox}[2.19: Ideal Class Representatives]
Let $\mathfrak{a}, \mathfrak{b}$ be ideals. Show that there exists an element $c \in K$ (the quotient field of $\mathfrak{o}$) such that $c\mathfrak{a}$ is an ideal relatively prime to $\mathfrak{b}$. In particular, every ideal class in $\text{Pic}(\mathfrak{o})$ contains representative ideals prime to a given ideal. For a continuation, see Exercise 7 of Chapter VII; Chapter III, Exercise 11-13.
\end{problembox}

\noindent\textbf{Strategy:} Use the unique factorization of ideals to write $\mathfrak{a}$ and $\mathfrak{b}$ as products of prime ideals. Choose $c$ to be a product of elements from prime ideals that appear in $\mathfrak{a}$ but not in $\mathfrak{b}$, so that $c\mathfrak{a}$ has no prime factors in common with $\mathfrak{b}$.

\noindent\textbf{Solution:}
Let $\mathfrak{a} = \mathfrak{p}_1^{e_1} \cdots \mathfrak{p}_r^{e_r}$ and $\mathfrak{b} = \mathfrak{q}_1^{f_1} \cdots \mathfrak{q}_s^{f_s}$ be the prime factorizations of $\mathfrak{a}$ and $\mathfrak{b}$.

We need to find $c \in K$ such that $c\mathfrak{a}$ is relatively prime to $\mathfrak{b}$. This means that $c\mathfrak{a}$ should not share any prime factors with $\mathfrak{b}$.

Let $c = \prod_{i=1}^r \mathfrak{p}_i^{-e_i}$. Then $c\mathfrak{a} = o$, which is relatively prime to any ideal.

However, this $c$ might not be in $K$ in the sense that $c\mathfrak{a}$ might not be an ideal. Instead, we can choose $c$ to be a product of elements from the prime ideals that appear in $\mathfrak{a}$ but not in $\mathfrak{b}$.

More precisely, let $S = \{\mathfrak{p}_1, \ldots, \mathfrak{p}_r\} \setminus \{\mathfrak{q}_1, \ldots, \mathfrak{q}_s\}$ be the set of prime ideals that appear in $\mathfrak{a}$ but not in $\mathfrak{b}$.

For each $\mathfrak{p} \in S$, choose an element $t_{\mathfrak{p}} \in \mathfrak{p} \setminus \mathfrak{p}^2$. Let $c = \prod_{\mathfrak{p} \in S} t_{\mathfrak{p}}^{e_{\mathfrak{p}}}$ where $e_{\mathfrak{p}}$ is the exponent of $\mathfrak{p}$ in the factorization of $\mathfrak{a}$.

Then $c\mathfrak{a}$ will have the same prime factors as $\mathfrak{a}$ except for those in $S$, which means it will be relatively prime to $\mathfrak{b}$.

This shows that every ideal class in $\text{Pic}(o)$ contains representative ideals prime to a given ideal.

\section{Problem-Solving Techniques}


\subsection*{Proving Prime Ideals}
\begin{itemize}
\item Use proof by contradiction: assume the ideal is not prime and find elements $a, b$ with $ab$ in the ideal but $a, b$ not in it
\item Use maximality properties: if an ideal is maximal with respect to certain conditions, it's often prime
\item Use localization: localizing at a prime ideal creates a local ring with unique maximal ideal
\end{itemize}

\subsection*{Proving Local Rings}
\begin{itemize}
\item Show there is exactly one maximal ideal
\item Use surjective homomorphisms: if $f: A \rightarrow A'$ is surjective and $A$ is local, then $A'$ is local
\item Use localization: $A_{\mathfrak{p}}$ is always a local ring
\end{itemize}

\subsection*{Proving Principal Rings}
\begin{itemize}
\item Use Euclidean algorithm with a norm function (e.g., for $\mathbb{Z}[i]$)
\item Show that every ideal is generated by a single element
\item Use localization: if $A$ is principal, then $S^{-1}A$ is principal
\end{itemize}

\subsection*{Proving Factorial Rings}
\begin{itemize}
\item Show unique factorization of elements into irreducibles
\item Use the fact that principal rings are factorial
\item Use localization: if $A$ is factorial, then $S^{-1}A$ is factorial
\end{itemize}

\subsection*{Proving Irreducibility}
\begin{itemize}
\item Use norm functions in quadratic rings to show elements cannot be factored
\item Use degree functions (e.g., trigonometric degree) to show elements cannot be factored
\item Show that any factorization would require one factor to be a unit
\end{itemize}

\subsection*{Working with Dedekind Rings}
\begin{enumerate}
\item Use unique factorization of ideals into prime ideals
\item Use the fact that every ideal has an inverse fractional ideal
\item Use the fact that prime ideals are maximal (Krull dimension 1)
\item Use localization to reduce to local rings with single prime ideal
\end{enumerate}

\subsection*{Proving Divisibility}
\begin{itemize}
\item In Dedekind rings: $\mathfrak{a} | \mathfrak{b}$ if and only if $\mathfrak{b} \subseteq \mathfrak{a}$
\item Use greatest common divisors: $\mathfrak{a} + \mathfrak{b}$ is the GCD of ideals $\mathfrak{a}$ and $\mathfrak{b}$
\item Use the fact that ideals are relatively prime if and only if their sum is the whole ring
\end{itemize}

\subsection*{Working with Units}
\begin{itemize}
\item Use norm functions to find units in quadratic rings
\item Use the structure of multiplicative groups modulo prime powers
\item For $\mathbb{Z}/p^r\mathbb{Z}$, use induction and exact sequences
\end{itemize}

\subsection*{Proving Ring Properties}
\begin{itemize}
\item Verify ring axioms directly (closure, associativity, distributivity)
\item Use homomorphisms to transfer properties between rings
\item Use localization to transfer properties from a ring to its localizations
\end{itemize}
