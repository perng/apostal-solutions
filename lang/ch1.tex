\chapter{Groups}

\section{Group Theory}

\textbf{Key Definitions and Theorems:}

\textbf{Definition:} A \textit{group} is a set $G$ with a binary operation $\cdot$ such that:
\begin{enumerate}
\item (Associativity) $(a \cdot b) \cdot c = a \cdot (b \cdot c)$ for all $a, b, c \in G$
\item (Identity) There exists $e \in G$ such that $e \cdot a = a \cdot e = a$ for all $a \in G$
\item (Inverses) For each $a \in G$, there exists $a^{-1} \in G$ such that $a \cdot a^{-1} = a^{-1} \cdot a = e$
\end{enumerate}

\textbf{Definition:} A group is \textit{abelian} if $a \cdot b = b \cdot a$ for all $a, b \in G$.

\textbf{Definition:} The \textit{order} of a group $G$ is the number of elements in $G$, denoted $|G|$ or $\#(G)$.

\textbf{Definition:} The \textit{order} of an element $g \in G$ is the smallest positive integer $n$ such that $g^n = e$.

\textbf{Lagrange's Theorem:} If $H$ is a subgroup of a finite group $G$, then $|H|$ divides $|G|$.

\textbf{Definition:} A \textit{cyclic group} is a group generated by a single element.

\textbf{Definition:} A \textit{commutator} in a group $G$ is an element of the form $[a,b] = aba^{-1}b^{-1}$.

\textbf{Definition:} The \textit{commutator subgroup} $G^c$ is the subgroup generated by all commutators.

\textbf{Definition:} A subgroup $H$ of $G$ is \textit{normal} if $gHg^{-1} = H$ for all $g \in G$.

\textbf{Definition:} A \textit{homomorphism} from $G$ to $H$ is a function $\phi: G \to H$ such that $\phi(ab) = \phi(a)\phi(b)$.

\textbf{Definition:} An \textit{isomorphism} is a bijective homomorphism.

\textbf{Definition:} The \textit{center} $Z(G)$ of a group $G$ is the set of elements that commute with every element of $G$.

\textbf{Definition:} The \textit{normalizer} $N_G(H)$ of a subgroup $H$ in $G$ is the set of elements $g \in G$ such that $gHg^{-1} = H$.

\textbf{Product Formula:} If $H, K$ are subgroups of $G$ with $K \subset N_G(H)$, then $|HK| = \frac{|H||K|}{|H \cap K|}$.

\begin{problembox}[1.1: Abelian groups of small order]
Show that every group of order $\leq 6$ is abelian.
\end{problembox}

\noindent\textbf{Strategy:} Based on Lagrange's theorem and the classification of small groups, systematically check each possible order to determine which groups are abelian. Use the fact that groups of prime order are cyclic and hence abelian.

\noindent\textbf{Solution:} We prove this by checking each possible order:

\textbf{Order 1:} The trivial group is abelian.

\textbf{Order 2:} By Lagrange's theorem, any non-identity element has order 2, so the group is cyclic and hence abelian.

\textbf{Order 3:} Any non-identity element has order 3, making the group cyclic and abelian.

\textbf{Order 4:} There are two groups of order 4: the cyclic group $\mathbb{Z}_4$ and the Klein four-group $\mathbb{Z}_2 \times \mathbb{Z}_2$. Both are abelian.

\textbf{Order 5:} Any non-identity element has order 5, making the group cyclic and abelian.

\textbf{Order 6:} There are two groups of order 6: the cyclic group $\mathbb{Z}_6$ and the symmetric group $S_3$. However, $S_3$ is not abelian (e.g., $(12)(13) \neq (13)(12)$), so this statement is actually false. The correct statement should be that every group of order $\leq 5$ is abelian.


\qed
\begin{problembox}[1.2: Groups of order 4]
Show that there are two non-isomorphic groups of order 4, namely the cyclic one, and the product of two cyclic groups of order 2.
\end{problembox}

\noindent\textbf{Strategy:} Use Lagrange's theorem to determine possible element orders, then classify groups based on whether they have an element of order 4 or not. Construct explicit isomorphisms to known groups.

\noindent\textbf{Solution:} Let $G$ be a group of order 4. By Lagrange's theorem, every element has order 1, 2, or 4.

\textbf{Case 1:} If $G$ has an element of order 4, then $G$ is cyclic and isomorphic to $\mathbb{Z}_4$.

\textbf{Case 2:} If every non-identity element has order 2, then $G$ is isomorphic to $\mathbb{Z}_2 \times \mathbb{Z}_2$ (the Klein four-group).

To see this, let $G = \{e, a, b, c\}$ where $a^2 = b^2 = c^2 = e$. Since $ab \neq a$ and $ab \neq b$, we must have $ab = c$. Similarly, $ba = c$, so $ab = ba$. This shows $G$ is abelian. The map $\phi: \mathbb{Z}_2 \times \mathbb{Z}_2 \to G$ defined by $\phi(0,0) = e$, $\phi(1,0) = a$, $\phi(0,1) = b$, $\phi(1,1) = c$ is an isomorphism.

These are the only two possibilities, and they are non-isomorphic since $\mathbb{Z}_4$ has an element of order 4 while $\mathbb{Z}_2 \times \mathbb{Z}_2$ does not.


\qed
\begin{problembox}[1.3: Commutator subgroup]
Let $G$ be a group. A commutator in $G$ is an element of the form $aba^{-1}b^{-1}$ with $a, b \in G$. Let $G^c$ be the subgroup generated by the commutators. Then $G^c$ is called the commutator subgroup. Show that $G^c$ is normal. Show that any homomorphism of $G$ into an abelian group factors through $G/G^c$.
\end{problembox}

\noindent\textbf{Strategy:} To show normality, prove that conjugates of commutators are commutators. For the factorization property, show that any homomorphism to an abelian group sends commutators to the identity, then use the universal property of quotient groups.

\noindent\textbf{Solution:} First, we show that $G^c$ is normal. Let $g \in G$ and $[a,b] = aba^{-1}b^{-1}$ be a commutator. Then
\[g[a,b]g^{-1} = g(aba^{-1}b^{-1})g^{-1} = (gag^{-1})(gbg^{-1})(ga^{-1}g^{-1})(gb^{-1}g^{-1}) = [gag^{-1}, gbg^{-1}],\]
which is also a commutator. Since $G^c$ is generated by commutators, $gG^cg^{-1} \subseteq G^c$ for all $g \in G$, so $G^c$ is normal.

Now let $\phi: G \to A$ be a homomorphism into an abelian group $A$. For any commutator $[a,b] = aba^{-1}b^{-1}$, we have
\[\phi([a,b]) = \phi(aba^{-1}b^{-1}) = \phi(a)\phi(b)\phi(a)^{-1}\phi(b)^{-1} = \phi(a)\phi(a)^{-1}\phi(b)\phi(b)^{-1} = 1,\]
since $A$ is abelian. Therefore, $\phi$ maps all commutators to the identity, and hence maps $G^c$ to the identity. This means $\phi$ factors through $G/G^c$ via the natural projection $\pi: G \to G/G^c$.


\qed
\begin{problembox}[1.4: Product of subgroups]
Let $H, K$ be subgroups of a finite group $G$ with $K \subset N_H$. Show that
\[\#(HK) = \frac{\#(H)\#(K)}{\#(H \cap K)}.\]
\end{problembox}

\noindent\textbf{Strategy:} Use the counting principle by defining a surjective map from $H \times K$ to $HK$ and counting the number of preimages of each element. The condition $K \subset N_H$ ensures that $HK$ is a subgroup.

\noindent\textbf{Solution:} Since $K \subset N_H$, we have $HK = KH$ and $HK$ is a subgroup of $G$. Consider the map $\phi: H \times K \to HK$ defined by $\phi(h,k) = hk$.

For any $x \in HK$, we can write $x = hk$ for some $h \in H$ and $k \in K$. The number of preimages of $x$ under $\phi$ is the number of pairs $(h',k')$ such that $h'k' = hk$.

If $h'k' = hk$, then $h^{-1}h' = kk'^{-1} \in H \cap K$. Let $t = h^{-1}h' = kk'^{-1} \in H \cap K$. Then $h' = ht$ and $k' = tk$. Conversely, for any $t \in H \cap K$, the pair $(ht, tk)$ maps to $hkt = hk$ since $t \in K \subset N_H$.

Therefore, each element of $HK$ has exactly $\#(H \cap K)$ preimages under $\phi$. By the counting principle,
\[\#(H) \cdot \#(K) = \#(H \times K) = \#(HK) \cdot \#(H \cap K),\]
which gives the desired formula.


\qed
\begin{problembox}[1.5: Goursat's Lemma]
Let $G, G'$ be groups, and let $H$ be a subgroup of $G \times G'$ such that the two projections $p_1 : H \rightarrow G$ and $p_2 : H \rightarrow G'$ are surjective. Let $N$ be the kernel of $p_2$ and $N'$ be the kernel of $p_1$. One can identify $N$ as a normal subgroup of $G$, and $N'$ as a normal subgroup of $G'$. Show that the image of $H$ in $G/N \times G'/N'$ is the graph of an isomorphism
\[G/N \approx G'/N'.\]
\end{problembox}

\noindent\textbf{Strategy:} Use the first isomorphism theorem to show that the natural map from $H$ to $G/N \times G'/N'$ is bijective, then prove that this bijection is the graph of a homomorphism by showing it preserves the group structure.

\noindent\textbf{Solution:} First, note that $N = \{(g,1) \in H : g \in G\}$ and $N' = \{(1,g') \in H : g' \in G'\}$. Since $p_1$ and $p_2$ are surjective, $N$ and $N'$ are normal subgroups of $G$ and $G'$ respectively.

Consider the map $\phi: H \to G/N \times G'/N'$ defined by $\phi(h) = (p_1(h)N, p_2(h)N')$. The kernel of $\phi$ is $N \cap N' = \{(1,1)\}$, so $\phi$ is injective.

For any $(gN, g'N') \in G/N \times G'/N'$, since $p_1$ and $p_2$ are surjective, there exists $h \in H$ such that $p_1(h) = g$ and $p_2(h) = g'$. Then $\phi(h) = (gN, g'N')$, so $\phi$ is surjective.

The image of $H$ under $\phi$ is the graph of a function $f: G/N \to G'/N'$ defined by $f(gN) = g'N'$ where $(g,g') \in H$. This function is well-defined because if $(g_1,g_1'), (g_2,g_2') \in H$ with $g_1N = g_2N$, then $(g_1^{-1}g_2, g_1'^{-1}g_2') \in N$, so $g_1'^{-1}g_2' \in N'$, which means $g_1'N' = g_2'N'$.

The function $f$ is a homomorphism because if $(g_1,g_1'), (g_2,g_2') \in H$, then $(g_1g_2, g_1'g_2') \in H$, so $f(g_1g_2N) = g_1'g_2'N' = f(g_1N)f(g_2N)$.

Finally, $f$ is bijective because $\phi$ is bijective, so $f$ is an isomorphism.


\qed
\begin{problembox}[1.6: Inner automorphisms]
Prove that the group of inner automorphisms of a group $G$ is normal in $\text{Aut}(G)$.
\end{problembox}

\noindent\textbf{Strategy:} Show that conjugating an inner automorphism by any automorphism gives another inner automorphism. Use the fact that conjugation by $\phi$ sends the inner automorphism $\psi_g$ to $\psi_{\phi(g)}$.

\noindent\textbf{Solution:} Let $\text{Inn}(G)$ be the group of inner automorphisms of $G$. We need to show that for any $\phi \in \text{Aut}(G)$ and any inner automorphism $\psi_g$ (conjugation by $g \in G$), we have $\phi \circ \psi_g \circ \phi^{-1} \in \text{Inn}(G)$.

For any $x \in G$,
\[(\phi \circ \psi_g \circ \phi^{-1})(x) = \phi(\psi_g(\phi^{-1}(x))) = \phi(g\phi^{-1}(x)g^{-1}) = \phi(g)x\phi(g)^{-1} = \psi_{\phi(g)}(x).\]

Therefore, $\phi \circ \psi_g \circ \phi^{-1} = \psi_{\phi(g)}$, which is an inner automorphism. This shows that $\text{Inn}(G)$ is normal in $\text{Aut}(G)$.


\qed
\begin{problembox}[1.7: Cyclic automorphism group]
Let $G$ be a group such that $\text{Aut}(G)$ is cyclic. Prove that $G$ is abelian.
\end{problembox}

\noindent\textbf{Strategy:} Use the fact that $\text{Inn}(G)$ is a subgroup of the cyclic group $\text{Aut}(G)$, so it's also cyclic. Then show that if $G/Z(G)$ is cyclic, then $G$ is abelian by expressing every element in terms of a generator and central elements.

\noindent\textbf{Solution:} Since $\text{Inn}(G)$ is a subgroup of $\text{Aut}(G)$ and $\text{Aut}(G)$ is cyclic, $\text{Inn}(G)$ is also cyclic.

The map $\phi: G \to \text{Inn}(G)$ defined by $\phi(g) = \psi_g$ (conjugation by $g$) is a homomorphism with kernel $Z(G)$, the center of $G$. Therefore, $G/Z(G) \cong \text{Inn}(G)$ is cyclic.

Let $gZ(G)$ be a generator of $G/Z(G)$. Then every element of $G$ can be written as $g^nz$ for some $n \in \mathbb{Z}$ and $z \in Z(G)$. For any two elements $g^nz_1$ and $g^mz_2$,
\[(g^nz_1)(g^mz_2) = g^{n+m}z_1z_2 = g^{m+n}z_2z_1 = (g^mz_2)(g^nz_1),\]
since $z_1, z_2 \in Z(G)$ commute with everything. This shows that $G$ is abelian.


\qed
\begin{problembox}[1.8: Double cosets]
Let $G$ be a group and let $H, H'$ be subgroups. By a double coset of $H, H'$ one means a subset of $G$ of the form $HxH'$.
\begin{enumerate}[label=(\alph*)]
\item Show that $G$ is a disjoint union of double cosets.
\item Let $\{c\}$ be a family of representatives for the double cosets. For each $a \in G$ denote by $[a]H'$ the conjugate $aH'a^{-1}$ of $H'$. For each $c$ we have a decomposition into ordinary cosets
\[H = \bigcup_{c}x_c(H \cap [c]H'),\]
where $\{x_c\}$ is a family of elements of $H$, depending on $c$. Show that the elements $\{x_c c\}$ form a family of left coset representatives for $H'$ in $G$; that is,
\[G = \bigcup_{x_c}\bigcup_{x_c}x_c cH',\]
and the union is disjoint. (Double cosets will not emerge further until Chapter XVIII.)
\end{enumerate}
\end{problembox}

\noindent\textbf{Strategy:} For part (a), show that the relation $x \sim y$ if and only if $y \in HxH'$ is an equivalence relation. For part (b), use the decomposition of $H$ into cosets of $H \cap [c]H'$ and show that the resulting union gives all left cosets of $H'$ in $G$.

\noindent\textbf{Solution:}
\begin{enumerate}[label=(\alph*)]
\item We show that the relation $x \sim y$ if and only if $y \in HxH'$ is an equivalence relation on $G$. Reflexivity: $x \in HxH'$ since $1 \in H$ and $1 \in H'$. Symmetry: if $y \in HxH'$, then $y = hxh'$ for some $h \in H$ and $h' \in H'$, so $x = h^{-1}yh'^{-1} \in HyH'$. Transitivity: if $y \in HxH'$ and $z \in HyH'$, then $y = h_1xh_1'$ and $z = h_2yh_2'$ for some $h_1, h_2 \in H$ and $h_1', h_2' \in H'$, so $z = h_2h_1xh_1'h_2' \in HxH'$. Therefore, $G$ is the disjoint union of equivalence classes, which are the double cosets.

\item For each double coset representative $c$, we have $H = \bigcup_{x_c} x_c(H \cap [c]H')$ where $\{x_c\}$ are representatives for the cosets of $H \cap [c]H'$ in $H$. 

For any $g \in G$, $g$ lies in some double coset $HcH'$ for some representative $c$. Then $g = hch'$ for some $h \in H$ and $h' \in H'$. Since $h \in H$, we can write $h = x_c k$ for some $x_c$ and $k \in H \cap [c]H'$. Then $g = x_c kch' = x_c c(k^c h')$ where $k^c = c^{-1}kc \in H'$ since $k \in [c]H'$. Therefore, $g \in x_c cH'$.

To show the union is disjoint, suppose $x_c cH' \cap x_{c'} c'H' \neq \emptyset$ for some $c, c'$ and some $x_c, x_{c'}$. Then $x_c ch_1 = x_{c'} c'h_2$ for some $h_1, h_2 \in H'$. This implies $x_{c'}^{-1}x_c c = c'h_2h_1^{-1} \in HcH' \cap Hc'H'$. Since double cosets are disjoint, we must have $c = c'$, and then $x_{c'}^{-1}x_c \in H \cap [c]H'$, which means $x_c$ and $x_{c'}$ represent the same coset, so $x_c = x_{c'}$.
\end{enumerate}


\qed
\section{Normal Subgroups and Indices}

\textbf{Key Definitions and Theorems:}

\textbf{Definition:} The \textit{index} of a subgroup $H$ in $G$, denoted $(G : H)$, is the number of left cosets of $H$ in $G$.

\textbf{Definition:} A \textit{left coset} of $H$ in $G$ is a subset of the form $gH = \{gh : h \in H\}$ for some $g \in G$.

\textbf{Definition:} A \textit{right coset} of $H$ in $G$ is a subset of the form $Hg = \{hg : h \in H\}$ for some $g \in G$.

\textbf{First Isomorphism Theorem:} If $\phi: G \to H$ is a homomorphism, then $G/\ker(\phi) \cong \text{im}(\phi)$.

\textbf{Third Isomorphism Theorem:} If $H$ and $K$ are normal subgroups of $G$ with $H \subseteq K$, then $(G/H)/(K/H) \cong G/K$.

\textbf{Definition:} The \textit{kernel} of a homomorphism $\phi: G \to H$ is $\ker(\phi) = \{g \in G : \phi(g) = e_H\}$.

\textbf{Definition:} The \textit{image} of a homomorphism $\phi: G \to H$ is $\text{im}(\phi) = \{\phi(g) : g \in G\}$.

\textbf{Theorem:} If $H$ is a subgroup of finite index in $G$, then there exists a normal subgroup $N$ of $G$ contained in $H$ and also of finite index.

\textbf{Theorem:} The number of left cosets equals the number of right cosets for any subgroup.

\begin{problembox}[1.9: Subgroups of finite index]
\begin{enumerate}[label=(\alph*)]
\item Let $G$ be a group and $H$ a subgroup of finite index. Show that there exists a normal subgroup $N$ of $G$ contained in $H$ and also of finite index. [Hint: If $(G : H) = n$, find a homomorphism of $G$ into $S_n$ whose kernel is contained in $H$.]
\item Let $G$ be a group and let $H_1, H_2$ be subgroups of finite index. Prove that $H_1 \cap H_2$ has finite index.
\end{enumerate}
\end{problembox}

\noindent\textbf{Strategy:} For part (a), use the action of $G$ on the left cosets of $H$ to create a homomorphism into $S_n$. For part (b), use part (a) to find normal subgroups of finite index contained in each $H_i$, then use the fact that the intersection of subgroups of finite index has finite index.

\noindent\textbf{Solution:}
\begin{enumerate}[label=(\alph*)]
\item Let $(G : H) = n$ and let $\{g_1, \ldots, g_n\}$ be a complete set of left coset representatives for $H$ in $G$. Define an action of $G$ on the set of left cosets $\{g_1H, \ldots, g_nH\}$ by $g \cdot (g_iH) = gg_iH$. This gives a homomorphism $\phi: G \to S_n$ where $\phi(g)$ is the permutation induced by the action of $g$.

The kernel $N = \ker(\phi)$ consists of all elements $g \in G$ such that $gg_iH = g_iH$ for all $i$, which means $g \in g_iHg_i^{-1}$ for all $i$. In particular, $g \in H$ (when $i = 1$), so $N \subseteq H$. Since $G/N \cong \text{im}(\phi) \subseteq S_n$, we have $(G : N) \leq n! < \infty$.

\item Let $(G : H_1) = n_1$ and $(G : H_2) = n_2$. By part (a), there exist normal subgroups $N_1 \subseteq H_1$ and $N_2 \subseteq H_2$ with finite indices. Then $N_1 \cap N_2 \subseteq H_1 \cap H_2$ and $(G : N_1 \cap N_2) \leq (G : N_1)(G : N_2) < \infty$, so $H_1 \cap H_2$ has finite index.
\end{enumerate}


\qed
\begin{problembox}[1.10: Right and left cosets]
Let $G$ be a group and let $H$ be a subgroup of finite index. Prove that there is only a finite number of right cosets of $H$, and that the number of right cosets is equal to the number of left cosets.
\end{problembox}

\noindent\textbf{Strategy:} Show that the inverses of left coset representatives form a complete set of right coset representatives. Use the fact that $gH = Hg^{-1}$ to establish the correspondence.

\noindent\textbf{Solution:} Let $(G : H) = n$ and let $\{g_1, \ldots, g_n\}$ be a complete set of left coset representatives. We show that $\{g_1^{-1}, \ldots, g_n^{-1}\}$ is a complete set of right coset representatives.

First, we show that every right coset $Hg$ is equal to $Hg_i^{-1}$ for some $i$. Since $g \in g_iH$ for some $i$, we have $g = g_ih$ for some $h \in H$. Then $Hg = Hg_ih = Hg_i = Hg_i^{-1}$ (since $g_iH = Hg_i^{-1}$).

Next, we show that the right cosets $Hg_i^{-1}$ are distinct. If $Hg_i^{-1} = Hg_j^{-1}$, then $g_i^{-1} \in Hg_j^{-1}$, so $g_i^{-1} = hg_j^{-1}$ for some $h \in H$. This implies $g_i = g_jh^{-1} \in g_jH$, which means $g_iH = g_jH$, so $i = j$.

Therefore, there are exactly $n$ right cosets, and the number of right cosets equals the number of left cosets.


\qed
\section{Group Actions}

\textbf{Key Definitions and Theorems:}

\textbf{Definition:} A \textit{group action} of $G$ on a set $S$ is a function $G \times S \to S$ (denoted $(g,s) \mapsto g \cdot s$) such that:
\begin{enumerate}
\item $e \cdot s = s$ for all $s \in S$
\item $(gh) \cdot s = g \cdot (h \cdot s)$ for all $g, h \in G$ and $s \in S$
\end{enumerate}

\textbf{Definition:} The \textit{orbit} of an element $s \in S$ under the action of $G$ is $G \cdot s = \{g \cdot s : g \in G\}$.

\textbf{Definition:} The \textit{stabilizer} of an element $s \in S$ is $G_s = \{g \in G : g \cdot s = s\}$.

\textbf{Orbit-Stabilizer Theorem:} If $G$ acts on $S$ and $s \in S$, then $|G \cdot s| = (G : G_s)$.

\textbf{Definition:} An action is \textit{transitive} if there is only one orbit.

\textbf{Definition:} An action is \textit{faithful} if the kernel of the action is trivial.

\textbf{Definition:} An action is \textit{free} if every non-identity element has no fixed points.

\textbf{Class Equation:} For a finite group $G$ acting on itself by conjugation, $|G| = |Z(G)| + \sum |G|/|C(g)|$ where the sum is over representatives of non-central conjugacy classes.

\textbf{Definition:} A \textit{fixed point} of an element $g \in G$ is an element $s \in S$ such that $g \cdot s = s$.

\textbf{Burnside's Lemma:} The number of orbits of a finite group $G$ acting on a finite set $S$ is $\frac{1}{|G|} \sum_{g \in G} |\text{Fix}(g)|$ where $\text{Fix}(g)$ is the set of fixed points of $g$.

\begin{problembox}[1.15: Fixed point free action]
Let $G$ be a finite group operating on a finite set $S$ with $\#(S) \geq 2$. Assume that there is only one orbit. Prove that there exists an element $x \in G$ which has no fixed point, i.e. $xs \neq s$ for all $s \in S$.
\end{problembox}

\noindent\textbf{Strategy:} Use the class equation and the fact that the average number of fixed points over all elements of $G$ is 1. Since the identity has more than 1 fixed point, there must be some element with fewer than 1 fixed point.

\noindent\textbf{Solution:} Since there is only one orbit, the action is transitive. Let $s_0 \in S$ and let $H$ be the stabilizer of $s_0$. Then $\#(S) = (G : H) = \#(G)/\#(H)$.

For any $g \in G$, the number of fixed points of $g$ is the number of elements $s \in S$ such that $gs = s$. Since the action is transitive, for any $s \in S$ there exists $h \in G$ such that $s = hs_0$. Then $gs = s$ if and only if $ghs_0 = hs_0$, which means $h^{-1}gh \in H$.

Therefore, the number of fixed points of $g$ is equal to the number of conjugates of $g$ that lie in $H$. By the class equation, the average number of fixed points over all elements of $G$ is
\[\frac{1}{\#(G)} \sum_{g \in G} \text{fixed points of } g = \frac{1}{\#(G)} \sum_{g \in G} \#\{h \in G : h^{-1}gh \in H\} = \frac{\#(G)}{\#(G)} = 1.\]

Since $\#(S) \geq 2$, the identity element has $\#(S) > 1$ fixed points. Therefore, there must exist some element $x \in G$ with fewer than 1 fixed point, i.e., no fixed points.


\qed
\begin{problembox}[1.16: Union of conjugates]
Let $H$ be a proper subgroup of a finite group $G$. Show that $G$ is not the union of all the conjugates of $H$. (But see Exercise 23 of Chapter XIII.)
\end{problembox}

\noindent\textbf{Strategy:} Use the inclusion-exclusion principle and count the total number of elements in all conjugates. Show that this number exceeds $|G|$ unless $H$ is normal, in which case there's only one conjugate.

\noindent\textbf{Solution:} Let $N = N_G(H)$ be the normalizer of $H$ in $G$. The number of conjugates of $H$ is $(G : N)$. Each conjugate of $H$ has the same order $\#(H)$.

If $G$ were the union of all conjugates of $H$, then by the inclusion-exclusion principle,
\[\#(G) \leq \sum_{g \in G/N} \#(gHg^{-1}) - \sum_{g_1, g_2 \in G/N, g_1 \neq g_2} \#(g_1Hg_1^{-1} \cap g_2Hg_2^{-1}) + \cdots\]

Since $H$ is a proper subgroup, $\#(H) < \#(G)$. The first term in the sum is $(G : N) \cdot \#(H)$. Since $(G : N) \geq 2$ (as $H$ is proper), we have $(G : N) \cdot \#(H) \geq 2\#(H) > \#(G)$ if $\#(H) > \#(G)/2$.

If $\#(H) \leq \#(G)/2$, then $(G : N) \cdot \#(H) \leq \#(G) \cdot \#(H)/\#(H) = \#(G)$, but this is only possible if $(G : N) = 1$, which means $H$ is normal. In this case, there is only one conjugate of $H$ (namely $H$ itself), and $H \neq G$ since $H$ is proper.

Therefore, $G$ cannot be the union of all conjugates of $H$.


\qed
\begin{problembox}[1.19: Counting fixed points]
Let $G$ be a finite group operating on a finite set $S$.
\begin{enumerate}[label=(\alph*)]
\item For each $s \in S$ show that 
\[\sum_{i \in G_s} \frac{1}{\#(G_i)} = 1.\]
\item For each $x \in G$ define $f(x) = $ number of elements $s \in S$ such that $xs = s$. Prove that the number of orbits of $G$ in $S$ is equal to 
\[\frac{1}{\#(G)} \sum_{x \in G} f(x).\]
\end{enumerate}
\end{problembox}

\noindent\textbf{Strategy:} For part (a), use the orbit-stabilizer theorem and count elements in the orbit. For part (b), use Burnside's lemma by counting fixed points and using the fact that each element stabilizes exactly one element in each orbit.

\noindent\textbf{Solution:}
\begin{enumerate}[label=(\alph*)]
\item For each $s \in S$, let $G_s$ be the stabilizer of $s$. The orbit of $s$ has size $(G : G_s) = \#(G)/\#(G_s)$. 

For each $g \in G$, let $G_g$ be the stabilizer of $g \cdot s$. Then $G_g = gG_sg^{-1}$, so $\#(G_g) = \#(G_s)$. 

The sum $\sum_{g \in G} \frac{1}{\#(G_g)}$ counts each element in the orbit of $s$ exactly $\#(G_s)$ times (once for each element in the stabilizer), divided by $\#(G_s)$. Therefore, this sum equals the size of the orbit, which is $\#(G)/\#(G_s)$.

But $\sum_{g \in G} \frac{1}{\#(G_g)} = \sum_{g \in G} \frac{1}{\#(G_s)} = \#(G)/\#(G_s)$, which equals the size of the orbit.

\item Let $O_1, \ldots, O_k$ be the orbits of $G$ in $S$. For each orbit $O_i$, let $s_i \in O_i$ and let $G_i$ be the stabilizer of $s_i$. Then $\#(O_i) = \#(G)/\#(G_i)$.

For each $x \in G$, the number of fixed points of $x$ is the sum over all orbits of the number of fixed points in each orbit. In orbit $O_i$, $x$ fixes $s_i$ if and only if $x \in G_i$. Therefore, $f(x) = \sum_{i=1}^k \chi_{G_i}(x)$, where $\chi_{G_i}$ is the characteristic function of $G_i$.

Then $\sum_{x \in G} f(x) = \sum_{x \in G} \sum_{i=1}^k \chi_{G_i}(x) = \sum_{i=1}^k \sum_{x \in G} \chi_{G_i}(x) = \sum_{i=1}^k \#(G_i) = \sum_{i=1}^k \#(G)/\#(O_i) = \#(G) \sum_{i=1}^k 1/\#(O_i)$.

But $\sum_{i=1}^k 1/\#(O_i) = \sum_{i=1}^k \#(G_i)/\#(G) = \sum_{i=1}^k \#(G_i)/\#(G) = k$, since each element of $G$ stabilizes exactly one element in each orbit.

Therefore, $\frac{1}{\#(G)} \sum_{x \in G} f(x) = k$, the number of orbits.
\end{enumerate}


\qed
\section{Sylow Theory}

\textbf{Key Definitions and Theorems:}

\textbf{Definition:} A \textit{p-group} is a group whose order is a power of a prime $p$.

\textbf{Definition:} A \textit{p-Sylow subgroup} of a finite group $G$ is a maximal $p$-subgroup of $G$.

\textbf{First Sylow Theorem:} If $G$ is a finite group and $p$ is a prime dividing $|G|$, then $G$ contains a $p$-Sylow subgroup.

\textbf{Second Sylow Theorem:} All $p$-Sylow subgroups of $G$ are conjugate to each other.

\textbf{Third Sylow Theorem:} The number $n_p$ of $p$-Sylow subgroups satisfies $n_p \equiv 1 \pmod{p}$ and $n_p$ divides $|G|$.

\textbf{Definition:} The \textit{centralizer} $C_G(g)$ of an element $g \in G$ is the set of elements that commute with $g$.

\textbf{Definition:} The \textit{conjugacy class} of an element $g \in G$ is the set $\{hgh^{-1} : h \in G\}$.

\textbf{Theorem:} If $P$ is a $p$-Sylow subgroup of $G$ and $H$ is a $p$-subgroup of $G$, then $H$ is contained in some conjugate of $P$.

\textbf{Theorem:} The center of a non-trivial $p$-group is non-trivial.

\textbf{Theorem:} If $H$ is a normal subgroup of order $p$ in a $p$-group $G$, then $H$ is contained in the center of $G$.

\begin{problembox}[1.20: Center of p-group]
Let $P$ be a $p$-group. Let $A$ be a normal subgroup of order $p$. Prove that $A$ is contained in the center of $P$.
\end{problembox}

\noindent\textbf{Strategy:} Consider the action of $P$ on $A$ by conjugation. Since $A$ has order $p$, its automorphism group has order $p-1$. Since $P$ is a $p$-group, the action must be trivial, meaning $A$ is central.

\noindent\textbf{Solution:} Since $A$ is normal of order $p$, it is cyclic and generated by some element $a$ of order $p$. 

Consider the action of $P$ on $A$ by conjugation. Since $A$ is normal, this action is well-defined. The kernel of this action is the centralizer $C_P(A)$ of $A$ in $P$.

Since $A$ has order $p$, the automorphism group of $A$ has order $p-1$. Therefore, the image of $P$ in $\text{Aut}(A)$ has order dividing $p-1$. But $P$ is a $p$-group, so this image must be trivial.

This means that every element of $P$ acts trivially on $A$ by conjugation, i.e., $A \subseteq Z(P)$, the center of $P$.


\qed
\begin{problembox}[1.21: Sylow intersections]
Let $G$ be a finite group and $H$ a subgroup. Let $P_H$ be a $p$-Sylow subgroup of $H$. Prove that there exists a $p$-Sylow subgroup $P$ of $G$ such that $P_H = P \cap H$.
\end{problembox}

\noindent\textbf{Strategy:} Use Sylow's theorem to find a $p$-Sylow subgroup $P$ of $G$ containing $P_H$. Then show that $P \cap H$ cannot be larger than $P_H$ since $P_H$ is already a maximal $p$-subgroup of $H$.

\noindent\textbf{Solution:} Let $P$ be a $p$-Sylow subgroup of $G$ containing $P_H$. Such a $P$ exists because $P_H$ is a $p$-subgroup of $G$, and by Sylow's theorem, it is contained in some $p$-Sylow subgroup of $G$.

Then $P_H \subseteq P \cap H$. Since $P_H$ is a $p$-Sylow subgroup of $H$, it has the largest possible order among $p$-subgroups of $H$. But $P \cap H$ is also a $p$-subgroup of $H$, so $\#(P_H) \geq \#(P \cap H)$.

Since $P_H \subseteq P \cap H$ and $\#(P_H) \geq \#(P \cap H)$, we must have $P_H = P \cap H$.


\qed
\begin{problembox}[1.22: Normal subgroup in Sylow]
Let $H$ be a normal subgroup of a finite group $G$ and assume that $\#(H) = p$. Prove that $H$ is contained in every $p$-Sylow subgroup of $G$.
\end{problembox}

\noindent\textbf{Strategy:} Use the fact that $H$ is a $p$-subgroup and normal. By Sylow's theorem, $H$ is contained in some $p$-Sylow subgroup, and since $H$ is normal, it must be contained in all conjugates of that Sylow subgroup.

\noindent\textbf{Solution:} Since $H$ is normal of order $p$, it is a $p$-subgroup of $G$. By Sylow's theorem, $H$ is contained in some $p$-Sylow subgroup $P$ of $G$.

Let $P'$ be any other $p$-Sylow subgroup of $G$. By Sylow's theorem, $P'$ is conjugate to $P$, so $P' = gPg^{-1}$ for some $g \in G$.

Since $H$ is normal, $gHg^{-1} = H$. Therefore, $H = gHg^{-1} \subseteq gPg^{-1} = P'$.

This shows that $H$ is contained in every $p$-Sylow subgroup of $G$.


\qed
\begin{problembox}[1.23: Sylow normalizers]
Let $P, P'$ be $p$-Sylow subgroups of a finite group $G$.
\begin{enumerate}[label=(\alph*)]
\item If $P' \subseteq N(P)$ (normalizer of $P$), then $P' = P$.
\item If $N(P') = N(P)$, then $P' = P$.
\item We have $N(N(P)) = N(P)$.
\end{enumerate}
\end{problembox}

\noindent\textbf{Strategy:} For part (a), use the fact that if $P'$ normalizes $P$, then $PP'$ is a $p$-subgroup containing both $P$ and $P'$. For part (b), use part (a). For part (c), show that any element normalizing $N(P)$ must normalize $P$ itself.

\noindent\textbf{Solution:}
\begin{enumerate}[label=(\alph*)]
\item If $P' \subseteq N(P)$, then $P'$ normalizes $P$, so $PP'$ is a subgroup of $G$. Since $P$ and $P'$ are both $p$-Sylow subgroups, they have the same order, and $PP'$ is a $p$-subgroup containing both $P$ and $P'$. By the maximality of $p$-Sylow subgroups, we must have $PP' = P = P'$.

\item If $N(P') = N(P)$, then $P' \subseteq N(P') = N(P)$. By part (a), this implies $P' = P$.

\item Let $N = N(P)$. Since $P$ is normal in $N$, $P$ is the unique $p$-Sylow subgroup of $N$. If $g \in N(N)$, then $g$ normalizes $N$, so $gPg^{-1} \subseteq gNg^{-1} = N$. Since $gPg^{-1}$ is also a $p$-Sylow subgroup of $N$, we must have $gPg^{-1} = P$, which means $g \in N(P) = N$. Therefore, $N(N) \subseteq N$. The reverse inclusion is obvious, so $N(N) = N$.
\end{enumerate}


\qed
\section{Group Structure}

\textbf{Key Definitions and Theorems:}

\textbf{Definition:} A group is \textit{solvable} if it has a subnormal series with abelian quotients.

\textbf{Definition:} A \textit{subnormal series} is a sequence of subgroups $G = G_0 \supseteq G_1 \supseteq \cdots \supseteq G_n = \{e\}$ where each $G_{i+1}$ is normal in $G_i$.

\textbf{Definition:} A group is \textit{simple} if it has no non-trivial normal subgroups.

\textbf{Theorem:} Every group of order $p^2$ is abelian.

\textbf{Theorem:} Every group of order $pq$ where $p < q$ are primes and $q \not\equiv 1 \pmod{p}$ is cyclic.

\textbf{Theorem:} Every group of order less than 60 is solvable.

\textbf{Theorem:} Every group of order $p^2q$ is solvable and has a normal Sylow subgroup.

\textbf{Theorem:} Every group of order $2pq$ for odd primes $p, q$ is solvable.

\textbf{Definition:} The \textit{direct product} of groups $G$ and $H$ is $G \times H = \{(g,h) : g \in G, h \in H\}$ with componentwise multiplication.

\textbf{Theorem:} If $G$ and $H$ are groups of coprime orders, then every subgroup of $G \times H$ is of the form $A \times B$ where $A \leq G$ and $B \leq H$.

\begin{problembox}[1.24: Groups of order p²]
Let $p$ be a prime number. Show that a group of order $p^2$ is abelian, and that there are only two such groups up to isomorphism.
\end{problembox}

\noindent\textbf{Strategy:} Use Lagrange's theorem to determine possible element orders. If there's an element of order $p^2$, the group is cyclic. Otherwise, show that every non-identity element has order $p$ and construct an explicit isomorphism to $\mathbb{Z}_p \times \mathbb{Z}_p$.

\noindent\textbf{Solution:} Let $G$ be a group of order $p^2$. By Lagrange's theorem, every element has order 1, $p$, or $p^2$.

\textbf{Case 1:} If $G$ has an element of order $p^2$, then $G$ is cyclic and isomorphic to $\mathbb{Z}_{p^2}$.

\textbf{Case 2:} If every non-identity element has order $p$, then $G$ is isomorphic to $\mathbb{Z}_p \times \mathbb{Z}_p$.

To see this, let $a \in G$ be any non-identity element. Since $a$ has order $p$, the subgroup $\langle a \rangle$ has order $p$. Let $b \in G \setminus \langle a \rangle$. Then $b$ also has order $p$, and $\langle a \rangle \cap \langle b \rangle = \{1\}$ since $b \notin \langle a \rangle$.

The subgroup $\langle a, b \rangle$ has order $p^2$ (since it contains all products $a^ib^j$ for $0 \leq i, j < p$), so $G = \langle a, b \rangle$. Since $a$ and $b$ commute (as we'll show), $G$ is isomorphic to $\mathbb{Z}_p \times \mathbb{Z}_p$.

To show that $a$ and $b$ commute, consider the commutator $[a,b] = aba^{-1}b^{-1}$. Since $G$ has order $p^2$, the center $Z(G)$ is non-trivial (by the class equation). If $[a,b] \neq 1$, then $\langle [a,b] \rangle$ is a non-trivial central subgroup, which contradicts the fact that $a$ and $b$ generate $G$ and don't commute.

These are the only two possibilities, and they are non-isomorphic since $\mathbb{Z}_{p^2}$ has an element of order $p^2$ while $\mathbb{Z}_p \times \mathbb{Z}_p$ does not.


\qed
\begin{problembox}[1.25: Non-abelian groups of order p³]
Let $G$ be a group of order $p^3$, where $p$ is prime, and $G$ is not abelian. Let $Z$ be its center. Let $C$ be a cyclic group of order $p$.
\begin{enumerate}[label=(\alph*)]
\item Show that $Z \approx C$ and $G/Z \approx C \times C$.
\item Every subgroup of $G$ of order $p^2$ contains $Z$ and is normal.
\item Suppose $x^p = 1$ for all $x \in G$. Show that $G$ contains a normal subgroup $H \approx C \times C$.
\end{enumerate}
\end{problembox}

\noindent\textbf{Strategy:} For part (a), use the class equation to show $Z$ has order $p$, then use the fact that $G/Z$ cannot be cyclic since $G$ is not abelian. For part (b), use the fact that $Z$ is central and subgroups containing it are normal. For part (c), construct the subgroup using elements outside the center.

\noindent\textbf{Solution:}
\begin{enumerate}[label=(\alph*)]
\item Since $G$ is not abelian, $Z \neq G$. By the class equation, $Z$ is non-trivial. Since $G$ is a $p$-group, $Z$ has order $p$ or $p^2$. If $Z$ had order $p^2$, then $G/Z$ would have order $p$, making it cyclic, which would imply $G$ is abelian (contradiction). Therefore, $Z \approx C$.

Since $G/Z$ has order $p^2$ and is not cyclic (as $G$ is not abelian), it must be isomorphic to $C \times C$.

\item Let $H$ be a subgroup of order $p^2$. Since $Z$ has order $p$ and $H$ has order $p^2$, we have $Z \subseteq H$ (otherwise $H \cap Z = \{1\}$ and $HZ$ would have order $p^3$, which is impossible).

Since $Z$ is central, $H$ is normal if and only if $gHg^{-1} = H$ for all $g \in G$. But $gHg^{-1} = H$ since $H$ contains $Z$ and $Z$ is central.

\item If $x^p = 1$ for all $x \in G$, then every non-identity element has order $p$. Let $a \in G \setminus Z$. Then $\langle a, Z \rangle$ is a subgroup of order $p^2$ containing $Z$, so it is normal by part (b).

Let $b \in G \setminus \langle a, Z \rangle$. Then $\langle b, Z \rangle$ is also a normal subgroup of order $p^2$. The intersection $\langle a, Z \rangle \cap \langle b, Z \rangle = Z$ since $b \notin \langle a, Z \rangle$.

The subgroup $H = \langle a, b, Z \rangle$ has order $p^3$ (since it contains all products $a^ib^jz$ for $0 \leq i, j < p$ and $z \in Z$), so $H = G$. Since $a$ and $b$ commute modulo $Z$, $G/Z \approx C \times C$ is generated by $aZ$ and $bZ$, so $H = \langle a, b \rangle \approx C \times C$.
\end{enumerate}


\qed
\begin{problembox}[1.26: Groups of order pq]
\begin{enumerate}[label=(\alph*)]
\item Let $G$ be a group of order $pq$, where $p, q$ are primes and $p < q$. Assume that $q \neq 1 \mod p$. Prove that $G$ is cyclic.
\item Show that every group of order 15 is cyclic.
\end{enumerate}
\end{problembox}

\noindent\textbf{Strategy:} Use Sylow's theorem to show that both Sylow subgroups are normal under the given conditions. Since they have trivial intersection and are both normal, the group is their direct product, which is cyclic.

\noindent\textbf{Solution:}
\begin{enumerate}[label=(\alph*)]
\item Let $n_p$ and $n_q$ be the number of $p$-Sylow and $q$-Sylow subgroups respectively. By Sylow's theorem, $n_q \equiv 1 \pmod{q}$ and $n_q$ divides $p$. Since $p < q$, we must have $n_q = 1$, so the $q$-Sylow subgroup is normal.

Similarly, $n_p \equiv 1 \pmod{p}$ and $n_p$ divides $q$. Since $q \neq 1 \pmod{p}$, we must have $n_p = 1$, so the $p$-Sylow subgroup is normal.

Since $P$ and $Q$ are both normal and have trivial intersection, $G = P \times Q \approx \mathbb{Z}_p \times \mathbb{Z}_q \approx \mathbb{Z}_{pq}$.

\item For $G$ of order 15, we have $p = 3$ and $q = 5$. Since $5 \neq 1 \pmod{3}$, the conditions of part (a) are satisfied, so $G$ is cyclic.
\end{enumerate}


\qed
\begin{problembox}[1.27: Solvability of small groups]
Show that every group of order $< 60$ is solvable.
\end{problembox}

\noindent\textbf{Strategy:} Use the fact that $A_5$ is the smallest non-solvable group and has order 60. For smaller orders, use known results about solvability of groups of specific orders and the fact that extensions of solvable groups by solvable groups are solvable.

\noindent\textbf{Solution:} We prove this by induction on the order. Groups of prime order are cyclic and hence solvable.

For composite orders, we use the fact that if a group has a normal subgroup and both the subgroup and quotient are solvable, then the group is solvable.

For orders less than 60, the only non-solvable group is $A_5$ which has order 60. All other groups of order less than 60 are solvable because:

1. Groups of order $p^n$ for prime $p$ are $p$-groups and hence solvable.
2. Groups of order $pq$ for primes $p < q$ are solvable (they are either cyclic or have a normal $q$-Sylow subgroup).
3. Groups of order $p^2q$ are solvable (they have a normal Sylow subgroup).
4. Groups of order $p^3$ are solvable (they are $p$-groups).
5. Groups of order $2pq$ for odd primes $p, q$ are solvable.

The only remaining cases are orders 24, 36, 48, and 56, all of which have normal Sylow subgroups and are therefore solvable.


\qed
\begin{problembox}[1.28: Groups of order p²q]
Let $p, q$ be distinct primes. Prove that a group of order $p^2q$ is solvable, and that one of its Sylow subgroups is normal.
\end{problembox}

\noindent\textbf{Strategy:} Use Sylow's theorem to analyze the possible numbers of Sylow subgroups. Show that in all cases, at least one Sylow subgroup must be normal, then use the fact that extensions of solvable groups by solvable groups are solvable.

\noindent\textbf{Solution:} Let $G$ be a group of order $p^2q$. Let $n_p$ and $n_q$ be the number of $p$-Sylow and $q$-Sylow subgroups respectively.

By Sylow's theorem, $n_q \equiv 1 \pmod{q}$ and $n_q$ divides $p^2$. Therefore, $n_q = 1$ or $n_q = p$ or $n_q = p^2$.

If $n_q = 1$, then the $q$-Sylow subgroup is normal, and we're done.

If $n_q = p$, then $p \equiv 1 \pmod{q}$, which means $q$ divides $p-1$. Since $p$ and $q$ are distinct primes, this is impossible unless $p = 2$ and $q = 3$. In this case, $G$ has order 12, and it can be shown that such groups have a normal Sylow subgroup.

If $n_q = p^2$, then $p^2 \equiv 1 \pmod{q}$, which means $q$ divides $(p-1)(p+1)$. This is only possible if $p = 2$ and $q = 3$ or $q = 5$. In these cases, the groups can be analyzed directly and shown to have normal Sylow subgroups.

Therefore, one of the Sylow subgroups is normal. Since both Sylow subgroups are solvable (being $p$-groups and cyclic groups), and the quotient is also solvable, $G$ is solvable.


\qed
\begin{problembox}[1.29: Groups of order 2pq]
Let $p, q$ be odd primes. Prove that a group of order $2pq$ is solvable.
\end{problembox}

\noindent\textbf{Strategy:} Use Sylow's theorem to show that both the $p$-Sylow and $q$-Sylow subgroups are normal (since $p$ and $q$ are odd and greater than 2). Then use the fact that the product of two normal subgroups with trivial intersection is a normal subgroup.

\noindent\textbf{Solution:} Let $G$ be a group of order $2pq$. Let $n_2$, $n_p$, and $n_q$ be the number of Sylow subgroups of orders 2, $p$, and $q$ respectively.

By Sylow's theorem, $n_q \equiv 1 \pmod{q}$ and $n_q$ divides $2p$. Since $q$ is odd and greater than 2, we must have $n_q = 1$, so the $q$-Sylow subgroup $Q$ is normal.

Similarly, $n_p \equiv 1 \pmod{p}$ and $n_p$ divides $2q$. Since $p$ is odd and greater than 2, we must have $n_p = 1$, so the $p$-Sylow subgroup $P$ is normal.

Since both $P$ and $Q$ are normal and have trivial intersection, $PQ$ is a normal subgroup of order $pq$. The quotient $G/PQ$ has order 2, so it's cyclic and hence solvable.

Since $P$ and $Q$ are cyclic (being groups of prime order), they are solvable. Therefore, $G$ is solvable.


\qed
\begin{problembox}[1.30: Sylow in orders 40 and 12]
\begin{enumerate}[label=(\alph*)]
\item Prove that one of the Sylow subgroups of a group of order 40 is normal.
\item Prove that one of the Sylow subgroups of a group of order 12 is normal.
\end{enumerate}
\end{problembox}

\noindent\textbf{Strategy:} For each order, use Sylow's theorem to analyze the possible numbers of Sylow subgroups. For order 40, show the 5-Sylow subgroup is normal. For order 12, show that either the 2-Sylow or 3-Sylow subgroup must be normal by counting elements.

\noindent\textbf{Solution:}
\begin{enumerate}[label=(\alph*)]
\item Let $G$ have order 40 = $2^3 \cdot 5$. Let $n_2$ and $n_5$ be the number of 2-Sylow and 5-Sylow subgroups respectively.

By Sylow's theorem, $n_5 \equiv 1 \pmod{5}$ and $n_5$ divides 8. Therefore, $n_5 = 1$, so the 5-Sylow subgroup is normal.

\item Let $G$ have order 12 = $2^2 \cdot 3$. Let $n_2$ and $n_3$ be the number of 2-Sylow and 3-Sylow subgroups respectively.

By Sylow's theorem, $n_3 \equiv 1 \pmod{3}$ and $n_3$ divides 4. Therefore, $n_3 = 1$ or $n_3 = 4$.

If $n_3 = 1$, then the 3-Sylow subgroup is normal.

If $n_3 = 4$, then there are 4 Sylow 3-subgroups, each containing 2 non-identity elements. These subgroups intersect only at the identity, so they account for 8 elements of order 3. The remaining 4 elements must form the unique 2-Sylow subgroup, so $n_2 = 1$ and the 2-Sylow subgroup is normal.

In either case, one of the Sylow subgroups is normal.
\end{enumerate}


\qed
\begin{problembox}[1.31: Groups of order $\leq 10$]
Determine all groups of order $\leq 10$ up to isomorphism. In particular, show that a non-abelian group of order 6 is isomorphic to $S_3$.
\end{problembox}

\noindent\textbf{Strategy:} Use Lagrange's theorem and known classifications of small groups. For order 6, show that a non-abelian group must have elements of orders 2 and 3, and use the fact that $S_3$ is the only non-abelian group of order 6.

\noindent\textbf{Solution:} We list all groups of order $\leq 10$:

\textbf{Order 1:} The trivial group.

\textbf{Order 2:} $\mathbb{Z}_2$.

\textbf{Order 3:} $\mathbb{Z}_3$.

\textbf{Order 4:} $\mathbb{Z}_4$ and $\mathbb{Z}_2 \times \mathbb{Z}_2$.

\textbf{Order 5:} $\mathbb{Z}_5$.

\textbf{Order 6:} $\mathbb{Z}_6$ and $S_3$. To see that a non-abelian group of order 6 is isomorphic to $S_3$, note that such a group must have elements of order 2 and 3. Let $a$ be an element of order 3 and $b$ an element of order 2. Since the group is non-abelian, $ba \neq ab$. The only possibility is $ba = a^2b$, which gives the presentation of $S_3$.

\textbf{Order 7:} $\mathbb{Z}_7$.

\textbf{Order 8:} $\mathbb{Z}_8$, $\mathbb{Z}_4 \times \mathbb{Z}_2$, $\mathbb{Z}_2 \times \mathbb{Z}_2 \times \mathbb{Z}_2$, $D_4$ (dihedral group), and $Q_8$ (quaternion group).

\textbf{Order 9:} $\mathbb{Z}_9$ and $\mathbb{Z}_3 \times \mathbb{Z}_3$.

\textbf{Order 10:} $\mathbb{Z}_{10}$ and $D_5$ (dihedral group).

The non-abelian groups are $S_3$ (order 6), $D_4$ (order 8), $Q_8$ (order 8), and $D_5$ (order 10).


\qed
\section{Permutation Groups}

\textbf{Key Definitions and Theorems:}

\textbf{Definition:} The \textit{symmetric group} $S_n$ is the group of all permutations of $\{1, 2, \ldots, n\}$.

\textbf{Definition:} A \textit{cycle} of length $k$ is a permutation that cyclically permutes $k$ elements and fixes the rest.

\textbf{Definition:} A \textit{transposition} is a cycle of length 2.

\textbf{Definition:} The \textit{sign} of a permutation is $(-1)^k$ where $k$ is the number of transpositions in any decomposition.

\textbf{Definition:} A permutation is \textit{even} if its sign is 1, \textit{odd} if its sign is -1.

\textbf{Definition:} The \textit{alternating group} $A_n$ is the subgroup of $S_n$ consisting of even permutations.

\textbf{Theorem:} Every permutation can be written as a product of disjoint cycles.

\textbf{Theorem:} Two permutations are conjugate in $S_n$ if and only if they have the same cycle structure.

\textbf{Theorem:} The order of a cycle is its length.

\textbf{Theorem:} The conjugacy class of an $n$-cycle in $S_n$ has $(n-1)!$ elements.

\textbf{Theorem:} The centralizer of an $n$-cycle in $S_n$ is the cyclic group generated by the cycle.

\textbf{Definition:} The \textit{dihedral group} $D_n$ is the group of symmetries of a regular $n$-gon.

\textbf{Definition:} The \textit{quaternion group} $Q_8$ is the group generated by $i, j$ with relations $i^4 = 1$, $j^4 = 1$, $i^2 = j^2$, $ij = -ji$.

\begin{problembox}[1.32: Sylow subgroups of symmetric groups]
Let $S_n$ be the permutation group on $n$ elements. Determine the $p$-Sylow subgroups of $S_3$, $S_4$, $S_5$ for $p = 2$ and $p = 3$.
\end{problembox}

\noindent\textbf{Strategy:} Use the fact that Sylow subgroups of symmetric groups can be constructed from the action on the set. For 2-Sylow subgroups, use products of transpositions and 4-cycles. For 3-Sylow subgroups, use 3-cycles.

\noindent\textbf{Solution:} We determine the Sylow subgroups for each case:

\textbf{$S_3$ (order 6 = 2 · 3):}
- 2-Sylow: $\langle (12) \rangle$ or $\langle (13) \rangle$ or $\langle (23) \rangle$ (any transposition)
- 3-Sylow: $\langle (123) \rangle$ (the cyclic group of order 3)

\textbf{$S_4$ (order 24 = 2³ · 3):}
- 2-Sylow: $\langle (12), (34) \rangle \cong D_4$ (dihedral group of order 8)
- 3-Sylow: $\langle (123) \rangle$ or $\langle (124) \rangle$ or $\langle (134) \rangle$ or $\langle (234) \rangle$ (cyclic groups of order 3)

\textbf{$S_5$ (order 120 = 2³ · 3 · 5):}
- 2-Sylow: $\langle (12), (34), (15) \rangle \cong D_4 \times \mathbb{Z}_2$ (order 16)
- 3-Sylow: $\langle (123) \rangle$ or any other 3-cycle (cyclic groups of order 3)

The 2-Sylow subgroups can be constructed by considering the action on the set and using the fact that they must be 2-groups. The 3-Sylow subgroups are always cyclic since they have prime order.


\qed
\begin{problembox}[1.33: Sign of a permutation]
Let $\sigma$ be a permutation of a finite set $I$ having $n$ elements. Define $e(\sigma)$ to be $(-1)^m$ where
\[m = n - \text{number of orbits of } \sigma.\]
If $I_1, \ldots, I_r$ are the orbits of $\sigma$, then $m$ is also equal to the sum
\[m = \sum_{v=1}^r [\text{card}(I_v) - 1].\]

If $\tau$ is a transposition, show that $e(\sigma\tau) = -e(\sigma)$ by considering the two cases when $i, j$ lie in the same orbit of $\sigma$, or lie in different orbits. In the first case, $\sigma\tau$ has one more orbit and in the second case one less orbit than $\sigma$. In particular, the sign of a transposition is $-1$. Prove that $e(\sigma) = e(\sigma)$ is the sign of the permutation.
\end{problembox}

\noindent\textbf{Strategy:} Analyze how a transposition affects the number of orbits: it either splits one orbit into two or merges two orbits into one. Use this to show that $e(\sigma\tau) = -e(\sigma)$, then use the fact that any permutation is a product of transpositions.

\noindent\textbf{Solution:} Let $\tau = (ij)$ be a transposition. We consider two cases:

\textbf{Case 1:} $i$ and $j$ lie in the same orbit of $\sigma$. Then $\sigma\tau$ splits this orbit into two orbits, so the number of orbits increases by 1. Therefore, $m$ decreases by 1, so $e(\sigma\tau) = -e(\sigma)$.

\textbf{Case 2:} $i$ and $j$ lie in different orbits of $\sigma$. Then $\sigma\tau$ merges these two orbits into one, so the number of orbits decreases by 1. Therefore, $m$ increases by 1, so $e(\sigma\tau) = -e(\sigma)$.

In both cases, $e(\sigma\tau) = -e(\sigma)$.

Since any permutation can be written as a product of transpositions, and each transposition changes the sign, we have $e(\sigma) = (-1)^k$ where $k$ is the number of transpositions in any decomposition of $\sigma$. This is exactly the sign of the permutation.


\qed
\begin{problembox}[1.34: Dihedral groups]
\begin{enumerate}[label=(\alph*)]
\item Let $n$ be an even positive integer. Show that there exists a group of order $2n$, generated by two elements $\sigma$, $\tau$ such that $\sigma^n = e = \tau^2$, and $\sigma\tau = \tau\sigma^{n-1}$. (Draw a picture of a regular $n$-gon, number the vertices, and use the picture as an inspiration to get $\sigma$, $\tau$.) This group is called the dihedral group.
\item Let $n$ be an odd positive integer. Let $D_{4n}$ be the group generated by the matrices
\[\begin{pmatrix}
0 & -1 \\
1 & 0 
\end{pmatrix} \quad \text{and} \quad \begin{pmatrix}
\zeta & 0 \\
0 & \zeta^{-1}
\end{pmatrix}\]
where $\zeta$ is a primitive $n$-th root of unity. Show that $D_{4n}$ has order $4n$, and give the commutation relations between the above generators.
\end{enumerate}
\end{problembox}

\noindent\textbf{Strategy:} For part (a), use the geometric interpretation of the dihedral group as symmetries of a regular $n$-gon. For part (b), compute the orders of the matrices and their products to determine the group structure.

\noindent\textbf{Solution:}
\begin{enumerate}[label=(\alph*)]
\item Consider a regular $n$-gon with vertices numbered 1, 2, ..., $n$ in clockwise order. Let $\sigma$ be the rotation by $2\pi/n$ radians (sending vertex $i$ to vertex $i+1$ mod $n$), and let $\tau$ be the reflection across the line through vertex 1 and the center.

Then $\sigma^n = e$ (rotation by $2\pi$), $\tau^2 = e$ (reflection twice), and $\sigma\tau = \tau\sigma^{n-1}$ (this can be verified by checking the action on the vertices).

The group generated by $\sigma$ and $\tau$ has order $2n$ because it contains the $n$ rotations $\sigma^i$ and the $n$ reflections $\sigma^i\tau$ for $0 \leq i < n$.

\item Let $A = \begin{pmatrix} 0 & -1 \\ 1 & 0 \end{pmatrix}$ and $B = \begin{pmatrix} \zeta & 0 \\ 0 & \zeta^{-1} \end{pmatrix}$.

We have $A^2 = -I$, $A^4 = I$, and $B^n = I$. Also, $AB = \begin{pmatrix} 0 & -\zeta^{-1} \\ \zeta & 0 \end{pmatrix}$ and $BA = \begin{pmatrix} 0 & -\zeta \\ \zeta^{-1} & 0 \end{pmatrix}$.

Since $n$ is odd, $\zeta \neq \zeta^{-1}$, so $AB \neq BA$. The group generated by $A$ and $B$ has order $4n$ because it contains the $4n$ elements $A^iB^j$ for $0 \leq i < 4$ and $0 \leq j < n$.

The commutation relation is $AB = BA^{n-1}$, which can be verified by direct computation.
\end{enumerate}


\qed
\begin{problembox}[1.35: Non-abelian groups of order 8]
Show that there are exactly two non-isomorphic non-abelian groups of order 8. (One of them is given by generators $\sigma$, $\tau$ with the relations
\[\sigma^4 = 1, \quad \tau^2 = 1, \quad \tau\sigma\tau = \sigma^3.\]
The other is the quaternion group.)
\end{problembox}

\noindent\textbf{Strategy:} Use the classification of groups of order 8. The only non-abelian groups are the dihedral group $D_4$ and the quaternion group $Q_8$. Distinguish them by counting elements of order 2.

\noindent\textbf{Solution:} Let $G$ be a non-abelian group of order 8. Since $G$ is not abelian, it cannot be cyclic or isomorphic to $\mathbb{Z}_4 \times \mathbb{Z}_2$ or $\mathbb{Z}_2 \times \mathbb{Z}_2 \times \mathbb{Z}_2$.

The remaining possibilities are the dihedral group $D_4$ and the quaternion group $Q_8$.

\textbf{Dihedral group $D_4$:} Generated by $\sigma$ (rotation by $\pi/2$) and $\tau$ (reflection) with relations $\sigma^4 = 1$, $\tau^2 = 1$, $\tau\sigma\tau = \sigma^3$.

\textbf{Quaternion group $Q_8$:} Generated by $i$ and $j$ with relations $i^4 = 1$, $j^4 = 1$, $i^2 = j^2$, $ij = -ji$.

These groups are non-isomorphic because $D_4$ has 5 elements of order 2 (the reflections and the rotation by $\pi$), while $Q_8$ has only 1 element of order 2 (namely $-1$).


\qed
\begin{problembox}[1.36: Conjugacy class of n-cycle]
Let $\sigma = [123 \cdots n]$ in $S_n$. Show that the conjugacy class of $\sigma$ has $(n - 1)!$ elements. Show that the centralizer of $\sigma$ is the cyclic group generated by $\sigma$.
\end{problembox}

\noindent\textbf{Strategy:} Use the fact that conjugacy classes in $S_n$ are determined by cycle structure. Count the number of $n$-cycles by considering different starting positions. For the centralizer, show that only powers of $\sigma$ can commute with it.

\noindent\textbf{Solution:} The conjugacy class of $\sigma$ consists of all $n$-cycles. The number of $n$-cycles in $S_n$ is $(n-1)!$ because there are $n!$ ways to arrange $n$ elements, but each cycle can be written in $n$ different ways (by starting at different positions).

The centralizer $C(\sigma)$ consists of all permutations $\tau$ such that $\tau\sigma\tau^{-1} = \sigma$. This means $\tau$ must commute with $\sigma$.

Since $\sigma$ is an $n$-cycle, any permutation that commutes with $\sigma$ must be a power of $\sigma$. Therefore, $C(\sigma) = \langle \sigma \rangle$, which has order $n$.

By the orbit-stabilizer theorem, the size of the conjugacy class is $|S_n|/|C(\sigma)| = n!/n = (n-1)!$.


\qed
\begin{problembox}[1.37: Conjugate cycles]
\begin{enumerate}[label=(\alph*)]
\item Let $\sigma = [i_1 \cdots i_m]$ be a cycle. Let $\gamma \in S_n$. Show that $\gamma\sigma\gamma^{-1}$ is the cycle $[\gamma(i_1) \cdots \gamma(i_m)]$.
\item Suppose that a permutation $\sigma$ in $S_n$ can be written as a product of $r$ disjoint cycles, and let $d_1, \ldots, d_r$ be the number of elements in each cycle, in increasing order. Let $\tau$ be another permutation which can be written as a product of disjoint cycles, whose cardinalities are $d'_1, \ldots, d'_s$ in increasing order. Prove that $\sigma$ is conjugate to $\tau$ in $S_n$ if and only if $r = s$ and $d_i = d'_i$ for all $i = 1, \ldots, r$.
\end{enumerate}
\end{problembox}

\noindent\textbf{Strategy:} For part (a), verify the action of $\gamma\sigma\gamma^{-1}$ on the elements $\gamma(i_j)$. For part (b), use part (a) to show that conjugate permutations have the same cycle structure, then construct a permutation that conjugates one to the other.

\noindent\textbf{Solution:}
\begin{enumerate}[label=(\alph*)]
\item Let $\sigma = [i_1 \cdots i_m]$ and let $\gamma \in S_n$. We show that $\gamma\sigma\gamma^{-1} = [\gamma(i_1) \cdots \gamma(i_m)]$.

For any $j \in \{1, \ldots, n\}$, we have:
- If $j = \gamma(i_k)$ for some $k$, then $\gamma\sigma\gamma^{-1}(j) = \gamma\sigma(i_k) = \gamma(i_{k+1 \bmod m})$.
- If $j \neq \gamma(i_k)$ for any $k$, then $\gamma^{-1}(j) \notin \{i_1, \ldots, i_m\}$, so $\sigma\gamma^{-1}(j) = \gamma^{-1}(j)$, and thus $\gamma\sigma\gamma^{-1}(j) = j$.

This shows that $\gamma\sigma\gamma^{-1}$ acts as the cycle $[\gamma(i_1) \cdots \gamma(i_m)]$.

\item The "only if" direction follows from part (a): if $\sigma$ and $\tau$ are conjugate, then they have the same cycle structure.

For the "if" direction, suppose $\sigma$ and $\tau$ have the same cycle structure. Write $\sigma = \sigma_1 \cdots \sigma_r$ and $\tau = \tau_1 \cdots \tau_r$ as products of disjoint cycles, where $\sigma_i$ and $\tau_i$ have the same length $d_i$.

For each $i$, let $\gamma_i$ be a permutation that maps the elements of $\sigma_i$ to the elements of $\tau_i$ in the same order. Then $\gamma = \gamma_1 \cdots \gamma_r$ (where the $\gamma_i$ act on disjoint sets) satisfies $\gamma\sigma\gamma^{-1} = \tau$.
\end{enumerate}


\qed
\begin{problembox}[1.38: Generating symmetric groups]
\begin{enumerate}[label=(\alph*)]
\item Show that $S_n$ is generated by the transpositions $[12], [13], \ldots, [1n]$.
\item Show that $S_n$ is generated by the transpositions $[12], [23], [34], \ldots, [n - 1, n]$.
\item Show that $S_n$ is generated by the cycles $[12]$ and $[123 \ldots n]$.
\item Assume that $n$ is prime. Let $\sigma = [123 \ldots n]$ and let $\tau = [rs]$ be any transposition. Show that $\sigma, \tau$ generate $S_n$.
\end{enumerate}
\end{problembox}

\noindent\textbf{Strategy:} For parts (a) and (b), show that any transposition can be written in terms of the given generators. For part (c), use conjugation by the $n$-cycle to generate adjacent transpositions. For part (d), use the fact that $n$ is prime to show that conjugates of $\tau$ by powers of $\sigma$ generate all transpositions.

\noindent\textbf{Solution:}
\begin{enumerate}[label=(\alph*)]
\item Any permutation can be written as a product of transpositions. Any transposition $[ij]$ with $i, j \neq 1$ can be written as $[1i][1j][1i]$. Therefore, the transpositions $[12], [13], \ldots, [1n]$ generate all transpositions, and hence generate $S_n$.

\item Any transposition $[ij]$ can be written as a product of adjacent transpositions. For example, $[13] = [12][23][12]$, $[14] = [12][23][34][23][12]$, etc. Therefore, the adjacent transpositions generate all transpositions, and hence generate $S_n$.

\item Let $\sigma = [12]$ and $\rho = [123 \ldots n]$. We show that any transposition can be written in terms of $\sigma$ and $\rho$.

For any $i \neq 1$, we have $\rho^{i-1}\sigma\rho^{-(i-1)} = [i, i+1]$. Therefore, we can generate all adjacent transpositions, and hence all transpositions by part (b).

\item Since $n$ is prime, $\sigma$ is an $n$-cycle and has order $n$. The subgroup generated by $\sigma$ and $\tau$ contains $\tau$ and all conjugates of $\tau$ by powers of $\sigma$.

Since $\tau = [rs]$, the conjugates $\sigma^i\tau\sigma^{-i}$ for $0 \leq i < n$ give us all transpositions of the form $[r+i, s+i]$ (where addition is modulo $n$).

Since $n$ is prime, these conjugates generate all transpositions, and hence generate $S_n$.
\end{enumerate}


\qed
\section{Alternating Groups}

\textbf{Key Definitions and Theorems:}

\textbf{Definition:} The \textit{alternating group} $A_n$ is the subgroup of $S_n$ consisting of even permutations.

\textbf{Definition:} An action is \textit{k-transitive} if for any two ordered $k$-tuples of distinct elements, there exists a group element mapping one to the other.

\textbf{Definition:} An action is \textit{primitive} if the only stable partitions are the trivial ones.

\textbf{Definition:} A \textit{stable partition} under a group action is a partition that is preserved by the group action.

\textbf{Theorem:} $A_n$ is $(n-2)$-transitive for $n \geq 3$.

\textbf{Theorem:} $A_n$ is simple for $n \geq 5$.

\textbf{Theorem:} $A_5$ is the smallest non-abelian simple group.

\textbf{Theorem:} If $H$ is a subgroup of index $n$ in $A_n$, then the action of $A_n$ on cosets of $H$ gives an isomorphism $A_n \to A_n$.

\textbf{Theorem:} Any simple group of order 60 is isomorphic to $A_5$.

\textbf{Definition:} A \textit{maximal subgroup} is a proper subgroup that is not contained in any larger proper subgroup.

\textbf{Theorem:} A group action is primitive if and only if the stabilizer of any point is a maximal subgroup.

\begin{problembox}[1.39: Transitivity of alternating group]
Show that the action of the alternating group $A_n$ on $\{1, \ldots, n\}$ is $(n - 2)$-transitive.
\end{problembox}

\noindent\textbf{Strategy:} Use the fact that $S_n$ is $n$-transitive and show that for any $(n-2)$-tuple mapping, there exists an even permutation that does the job. If the permutation is odd, compose it with a transposition of the remaining two elements.

\noindent\textbf{Solution:} We need to show that for any two ordered $(n-2)$-tuples $(a_1, \ldots, a_{n-2})$ and $(b_1, \ldots, b_{n-2})$ of distinct elements, there exists $\sigma \in A_n$ such that $\sigma(a_i) = b_i$ for all $i$.

Let $c_1, c_2$ be the remaining two elements not in the first tuple, and $d_1, d_2$ be the remaining two elements not in the second tuple.

There exists a permutation $\tau \in S_n$ such that $\tau(a_i) = b_i$ for all $i$, $\tau(c_1) = d_1$, and $\tau(c_2) = d_2$.

If $\tau$ is even, we're done. If $\tau$ is odd, then the permutation $\tau' = \tau \circ (d_1 d_2)$ is even and satisfies $\tau'(a_i) = b_i$ for all $i$.

Therefore, $A_n$ is $(n-2)$-transitive.


\qed
\begin{problembox}[1.40: Subgroups of index $n$ in $A_n$]
Let $A_n$ be the alternating group of even permutations of $\{1, \ldots, n\}$. For $j = 1, \ldots, n$ let $H_j$ be the subgroup of $A_n$ fixing $j$, so $H_j \approx A_{n-1}$, and $(A_n : H_j) = n$ for $n \geq 3$. Let $n \geq 3$ and let $H$ be a subgroup of index $n$ in $A_n$.
\begin{enumerate}[label=(\alph*)]
\item Show that the action of $A_n$ on cosets of $H$ by left translation gives an isomorphism $A_n$ with the alternating group of permutations of $A_n/H$.
\item Show that there exists an automorphism of $A_n$ mapping $H_1$ on $H$, and that such an automorphism is induced by an inner automorphism of $S_n$ if and only if $H = H_i$ for some $i$.
\end{enumerate}
\end{problembox}

\noindent\textbf{Strategy:} For part (a), use the fact that $A_n$ is simple for $n \geq 5$ to show the action is faithful, then use the order to show the image is $A_n$. For part (b), use the action to construct an automorphism and analyze when it's inner.

\noindent\textbf{Solution:}
\begin{enumerate}[label=(\alph*)]
\item The action of $A_n$ on the cosets of $H$ by left translation gives a homomorphism $\phi: A_n \to S_n$ (since there are $n$ cosets).

The kernel of $\phi$ is the intersection of all conjugates of $H$, which is a normal subgroup of $A_n$. Since $A_n$ is simple for $n \geq 5$, the kernel is trivial, so $\phi$ is injective.

The image of $\phi$ is a subgroup of $S_n$ of index 2 (since $|A_n| = n!/2$), so it must be $A_n$. Therefore, $\phi$ is an isomorphism.

\item Since $H$ has index $n$ in $A_n$, the action of $A_n$ on $A_n/H$ gives an isomorphism $A_n \to A_n$. This isomorphism maps $H_1$ to $H$.

If $H = H_i$ for some $i$, then the automorphism is induced by conjugation by the transposition $(1i)$ in $S_n$.

Conversely, if the automorphism is induced by an inner automorphism of $S_n$, then it is conjugation by some element of $S_n$. Since $A_n$ is normal in $S_n$, this conjugation maps $H_1$ to some $H_i$.
\end{enumerate}


\qed
\begin{problembox}[1.41: Simple group of order 60]
Let $H$ be a simple group of order 60.
\begin{enumerate}[label=(\alph*)]
\item Show that the action of $H$ by conjugation on the set of its Sylow subgroups gives an imbedding $H \subseteq A_6$.
\item Using the preceding exercise, show that $H \approx A_5$.
\item Show that $A_6$ has an automorphism which is not induced by an inner automorphism of $S_6$.
\end{enumerate}
\end{problembox}

\noindent\textbf{Strategy:} For part (a), use Sylow's theorem to count Sylow subgroups and show the action gives an embedding. For part (b), use the previous exercise to show $H$ has index 6 in $A_6$. For part (c), use the automorphism from the previous exercise.

\noindent\textbf{Solution:}
\begin{enumerate}[label=(\alph*)]
\item Let $H$ be a simple group of order 60. The prime factorization is $60 = 2^2 \cdot 3 \cdot 5$.

Let $n_2$, $n_3$, and $n_5$ be the number of Sylow subgroups of orders 4, 3, and 5 respectively.

By Sylow's theorem, $n_5 \equiv 1 \pmod{5}$ and $n_5$ divides 12, so $n_5 = 1$ or $n_5 = 6$. Since $H$ is simple, $n_5 = 6$.

Similarly, $n_3 \equiv 1 \pmod{3}$ and $n_3$ divides 20, so $n_3 = 1$ or $n_3 = 4$ or $n_3 = 10$. Since $H$ is simple, $n_3 = 10$.

The action of $H$ by conjugation on the set of Sylow 5-subgroups gives a homomorphism $H \to S_6$. Since $H$ is simple, this homomorphism is injective, so $H$ embeds into $S_6$.

Since $H$ has order 60 and $A_6$ has order 360, $H$ embeds into $A_6$.

\item By the previous exercise, any subgroup of index 6 in $A_6$ is isomorphic to $A_5$. Since $H$ has order 60 and $A_6$ has order 360, $H$ has index 6 in $A_6$, so $H \approx A_5$.

\item The automorphism of $A_6$ that maps $H_1$ to $H_2$ (where $H_i$ is the stabilizer of $i$) is not induced by an inner automorphism of $S_6$ because it maps $H_1$ to $H_2$ instead of to $H_1$.
\end{enumerate}


\qed
\section{Abelian Groups}

\textbf{Key Definitions and Theorems:}

\textbf{Definition:} An \textit{abelian group} is a group where the operation is commutative.

\textbf{Definition:} A \textit{torsion group} is a group where every element has finite order.

\textbf{Definition:} A \textit{torsion-free group} is a group where only the identity has finite order.

\textbf{Definition:} A \textit{mixed group} is a group that is neither torsion nor torsion-free.

\textbf{Fundamental Theorem of Finite Abelian Groups:} Every finite abelian group is isomorphic to a direct product of cyclic groups of prime power orders.

\textbf{Structure Theorem for Finitely Generated Abelian Groups:} Every finitely generated abelian group is isomorphic to $\mathbb{Z}^r \times \mathbb{Z}_{n_1} \times \cdots \times \mathbb{Z}_{n_k}$ where $r \geq 0$ and $n_i$ are powers of primes.

\textbf{Definition:} The \textit{rank} of a finitely generated abelian group is the number of copies of $\mathbb{Z}$ in its decomposition.

\textbf{Theorem:} $\mathbb{Q}/\mathbb{Z}$ is a torsion group with exactly one subgroup of order $n$ for each positive integer $n$.

\textbf{Theorem:} Every finite abelian group has a subgroup isomorphic to any given quotient.

\textbf{Definition:} The \textit{Herbrand quotient} of a finite cyclic group $G$ acting on an abelian group $A$ is $q(A) = (A_f : A^g)(A_g : A^f)$ where $f(x) = \sigma x - x$ and $g(x) = x + \sigma x + \cdots + \sigma^{n-1}x$.

\begin{problembox}[1.42: Torsion group Q/Z]
Viewing $\mathbb{Z}, \mathbb{Q}$ as additive groups, show that $\mathbb{Q}/\mathbb{Z}$ is a torsion group, which has one and only one subgroup of order $n$ for each integer $n \geq 1$, and that this subgroup is cyclic.
\end{problembox}

\noindent\textbf{Strategy:} Show that every element has finite order by writing rational numbers as fractions. For the subgroup of order $n$, show that $\langle \frac{1}{n} + \mathbb{Z} \rangle$ is the unique cyclic subgroup of order $n$.

\noindent\textbf{Solution:} First, we show that $\mathbb{Q}/\mathbb{Z}$ is a torsion group. For any $q \in \mathbb{Q}$, we can write $q = a/b$ where $a, b \in \mathbb{Z}$ and $b > 0$. Then $bq = a \in \mathbb{Z}$, so $b(q + \mathbb{Z}) = a + \mathbb{Z} = \mathbb{Z}$ in $\mathbb{Q}/\mathbb{Z}$. Therefore, every element has finite order.

For any integer $n \geq 1$, the subgroup of $\mathbb{Q}/\mathbb{Z}$ of order $n$ is $\langle \frac{1}{n} + \mathbb{Z} \rangle$. This subgroup is cyclic and has exactly $n$ elements: $\frac{k}{n} + \mathbb{Z}$ for $0 \leq k < n$.

To show uniqueness, suppose $H$ is another subgroup of order $n$. Let $q + \mathbb{Z}$ be a generator of $H$. Then $n(q + \mathbb{Z}) = \mathbb{Z}$, so $nq \in \mathbb{Z}$. This means $q = \frac{k}{n}$ for some $k \in \mathbb{Z}$. Since $H$ has order $n$, we must have $\gcd(k, n) = 1$, so $H = \langle \frac{1}{n} + \mathbb{Z} \rangle$.


\qed
\begin{problembox}[1.43: Subgroup isomorphic to quotient]
Let $H$ be a subgroup of a finite abelian group $G$. Show that $G$ has a subgroup that is isomorphic to $G/H$.
\end{problembox}

\noindent\textbf{Strategy:} Use the fundamental theorem of finite abelian groups to decompose $G$ and $H$ into direct products of cyclic groups. Then construct a subgroup isomorphic to $G/H$ using the appropriate factors.

\noindent\textbf{Solution:} Since $G$ is a finite abelian group, it is isomorphic to a direct product of cyclic groups of prime power orders: $G \cong \mathbb{Z}_{p_1^{a_1}} \times \cdots \times \mathbb{Z}_{p_k^{a_k}}$.

The subgroup $H$ corresponds to a subgroup of this direct product, which is also a direct product of cyclic groups: $H \cong \mathbb{Z}_{p_1^{b_1}} \times \cdots \times \mathbb{Z}_{p_k^{b_k}}$ where $0 \leq b_i \leq a_i$.

The quotient $G/H$ is isomorphic to $\mathbb{Z}_{p_1^{a_1-b_1}} \times \cdots \times \mathbb{Z}_{p_k^{a_k-b_k}}$.

The subgroup of $G$ isomorphic to $G/H$ is $\mathbb{Z}_{p_1^{a_1-b_1}} \times \cdots \times \mathbb{Z}_{p_k^{a_k-b_k}} \times \{0\} \times \cdots \times \{0\}$.


\qed
\begin{problembox}[1.44: Index formula]
Let $f: A \to A'$ be a homomorphism of abelian groups. Let $B$ be a subgroup of $A$. Denote by $A'$ and $A_f$ the image and kernel of $f$ in $A$ respectively, and similarly for $B'$ and $B_f$. Show that $(A : B) = (A' : B')(A_f : B_f)$, in the sense that if two of these three indices are finite, so is the third, and the stated equality holds.
\end{problembox}

\noindent\textbf{Strategy:} Use the isomorphism theorems to relate the indices. Apply the first isomorphism theorem to $A/A_f \cong A'$ and $B/B_f \cong B'$, then use the third isomorphism theorem.

\noindent\textbf{Solution:} We use the isomorphism theorems for abelian groups.

By the first isomorphism theorem, $A/A_f \cong A'$ and $B/B_f \cong B'$.

By the third isomorphism theorem, $(A/A_f)/(B/B_f) \cong A/B$.

Therefore, $(A : B) = |A/B| = |(A/A_f)/(B/B_f)| = |A/A_f|/|B/B_f| = |A'|/|B'| = (A' : B')$.

Also, $(A_f : B_f) = |A_f/B_f| = |A_f|/|B_f|$.

Since $A_f \subseteq A$ and $B_f \subseteq B$, we have $(A : B) = (A' : B')(A_f : B_f)$.


\qed
\begin{problembox}[1.45: Herbrand quotient]
Let $G$ be a finite cyclic group of order $n$, generated by an element $\sigma$. Assume that $G$ operates on an abelian group $A$, and let $f, g : A \to A$ be the endomorphisms of $A$ given by
\[f(x) = \sigma x - x \quad \text{and} \quad g(x) = x + \sigma x + \cdots + \sigma^{n-1}x.\]

Define the Herbrand quotient by the expression $q(A) = (A_f : A^g)(A_g : A^f)$, provided both indices are finite. Assume now that $B$ is a subgroup of $A$ such that $GB \subset B$.
\begin{enumerate}[label=(\alph*)]
\item Define in a natural way an operation of $G$ on $A/B$.
\item Prove that
\[q(A) = q(B)q(A/B)\]
in the sense that if two of these quotients are finite, so is the third, and the stated equality holds.
\item If $A$ is finite, show that $q(A) = 1$.
\end{enumerate}
(This exercise is a special case of the general theory of Euler characteristics discussed in Chapter XX, Theorem 3.1. After reading this, the present exercise becomes trivial. Why?)
\end{problembox}

\noindent\textbf{Strategy:} For part (a), define the action by $\sigma \cdot (a + B) = \sigma a + B$. For part (b), use exact sequences and the snake lemma to relate the quotients. For part (c), use part (b) with $B = \{0\}$.

\noindent\textbf{Solution:}
\begin{enumerate}[label=(\alph*)]
\item The operation of $G$ on $A/B$ is defined by $\sigma \cdot (a + B) = \sigma a + B$. This is well-defined because $GB \subset B$.

\item We have the following exact sequences:
\[0 \to B_f \to A_f \to (A/B)_f \to 0\]
and
\[0 \to B^g \to A^g \to (A/B)^g \to 0\]

By the snake lemma, we have exact sequences:
\[0 \to B_f \to A_f \to (A/B)_f \to B^g/B_f \to A^g/A_f \to (A/B)^g/(A/B)_f \to 0\]

This gives us the relation:
\[(A_f : A^g) = (B_f : B^g)((A/B)_f : (A/B)^g)\]

Therefore, $q(A) = q(B)q(A/B)$.

\item If $A$ is finite, then all the groups involved are finite, so all indices are finite. By part (b), $q(A) = q(B)q(A/B)$ for any $G$-invariant subgroup $B$.

Taking $B = \{0\}$, we have $q(A) = q(\{0\})q(A) = 1 \cdot q(A)$, so $q(A) = 1$.
\end{enumerate}


\qed
\section{Primitive Groups}

\textbf{Key Definitions and Theorems:}

\textbf{Definition:} A group action is \textit{primitive} if the only stable partitions are the trivial ones (the whole set and singletons).

\textbf{Definition:} A \textit{stable partition} under a group action is a partition that is preserved by the group action.

\textbf{Definition:} A \textit{maximal subgroup} is a proper subgroup that is not contained in any larger proper subgroup.

\textbf{Theorem:} A group action is primitive if and only if the stabilizer of any point is a maximal subgroup.

\textbf{Definition:} An action is \textit{doubly transitive} if it is 2-transitive.

\textbf{Theorem:} A group is doubly transitive if and only if the stabilizer of a point acts transitively on the remaining points.

\textbf{Theorem:} If $G$ is doubly transitive and $(G : H) = n$, then $|G| = d(n-1)n$ where $d$ is the order of the subgroup fixing two points.

\textbf{Theorem:} A doubly transitive group is primitive.

\textbf{Definition:} The \textit{isotropy group} or \textit{stabilizer} of a point $s$ is $G_s = \{g \in G : g \cdot s = s\}$.

\textbf{Theorem:} For a transitive action, $\sum_{x \in G} f(x) = |G|$ where $f(x)$ is the number of fixed points of $x$.

\textbf{Theorem:} A group is doubly transitive if and only if $\sum_{x \in G} f(x)^2 = 2|G|$.

\begin{problembox}[1.46: Primitive group conditions]
Let $G$ operate on a set $S$. Let $S = \bigcup S_i$ be a partition of $S$ into disjoint subsets. We say that the partition is stable under $G$ if $G$ maps each $S_i$ onto $S_j$ for some $j$, and hence $G$ induces a permutation of the sets of the partition among themselves. There are two partitions of $S$ which are obviously stable: the partition consisting of $S$ itself, and the partition consisting of the subsets with one element. Assume that $G$ operates transitively, and and that $S$ has more than one element. Prove that the following two conditions are equivalent:

PRIM 1. The only partitions of $S$ which are stable are the two partitions mentioned above.

PRIM 2. If $H$ is the isotropy group of an element of $S$, then $H$ is a maximal subgroup of $G$.

These two conditions define what is known as a primitive group, or more accurately, a primitive operation of $G$ on $S$.
\end{problembox}

\noindent\textbf{Strategy:} For PRIM 1 $\Rightarrow$ PRIM 2, assume $H$ is not maximal and construct a non-trivial stable partition. For PRIM 2 $\Rightarrow$ PRIM 1, assume there's a non-trivial stable partition and show $H$ is not maximal.

\noindent\textbf{Solution:} We prove the equivalence of PRIM 1 and PRIM 2.

\textbf{PRIM 1 $\Rightarrow$ PRIM 2:} Let $H$ be the isotropy group of an element $s \in S$. Suppose $H$ is not maximal, so there exists a subgroup $K$ with $H \subsetneq K \subsetneq G$.

Let $S' = \{gs : g \in K\}$. Since $H \subset K$, $S'$ contains $s$. Since $K \neq G$ and the action is transitive, $S' \neq S$. Since $K \neq H$, $S'$ contains more than one element.

The partition $\{S', S \setminus S'\}$ is stable under $G$ because for any $g \in G$, either $gK = K$ (in which case $gS' = S'$) or $gK \cap K = H$ (in which case $gS' \cap S' = \{s\}$ and $gS' \subseteq S \setminus S'$).

This contradicts PRIM 1, so $H$ must be maximal.

\textbf{PRIM 2 $\Rightarrow$ PRIM 1:} Let $H$ be the isotropy group of an element $s \in S$. Suppose there exists a stable partition $\{S_1, \ldots, S_k\}$ with $1 < k < |S|$.

Let $s \in S_1$. The subgroup $K = \{g \in G : gS_1 = S_1\}$ contains $H$ and is a proper subgroup of $G$ (since the action is transitive).

Since $H \subsetneq K \subsetneq G$, $H$ is not maximal, contradicting PRIM 2.

Therefore, the only stable partitions are the trivial ones.


\qed
\begin{problembox}[1.47: Double transitivity]
Let a finite group $G$ operate transitively and faithfully on a set $S$ with at least 2 elements and let $H$ be the isotropy group of some element $s$ of $S$. (All the other isotropy groups are conjugates of $H$.) Prove the following:
\begin{enumerate}[label=(\alph*)]
\item $G$ is doubly transitive if and only if $H$ acts transitively on the complement of $s$ in $S$.
\item $G$ is doubly transitive if and only if $G = HTH$, where $T$ is a subgroup of $G$ of order 2 not contained in $H$.
\item If $G$ is doubly transitive, and $(G : H) = n$, then
\[\#(G) = d(n - 1)n,\]
where $d$ is the order of the subgroup fixing two elements. Furthermore, $H$ is a maximal subgroup of $G$, i.e. $G$ is primitive.
\end{enumerate}
\end{problembox}

\noindent\textbf{Strategy:} For part (a), use the definition of double transitivity and the fact that $G$ is transitive. For part (b), use the double coset decomposition. For part (c), use the orbit-stabilizer theorem and the fact that double transitive groups are primitive.

\noindent\textbf{Solution:}
\begin{enumerate}[label=(\alph*)]
\item $G$ is doubly transitive if and only if for any two pairs $(s_1, s_2)$ and $(t_1, t_2)$ of distinct elements, there exists $g \in G$ such that $gs_1 = t_1$ and $gs_2 = t_2$.

Since $G$ is transitive, we can assume $s_1 = s$. Then $G$ is doubly transitive if and only if for any $s_2 \neq s$ and any $t_1, t_2 \in S$ with $t_1 \neq t_2$, there exists $g \in G$ such that $gs = t_1$ and $gs_2 = t_2$.

This is equivalent to $H$ acting transitively on $S \setminus \{s\}$.

\item If $G$ is doubly transitive, then for any $t \in S \setminus \{s\}$, there exists $g \in G$ such that $gs = s$ and $gt = t'$ for some $t' \neq s$. This means $g \in H$ and $g \notin H$.

Let $T = \langle g \rangle$ where $g$ is such an element. Then $T$ has order 2 and is not contained in $H$.

Since $G$ is transitive, $G = \bigcup_{t \in S} HtH = HTH$.

Conversely, if $G = HTH$ where $T$ has order 2 and is not contained in $H$, then $T$ contains an element that maps $s$ to some other element, and $H$ acts transitively on the complement of $s$.

\item If $G$ is doubly transitive, then the stabilizer of two points has order $d = \#(G)/(n(n-1))$.

Since $G$ is doubly transitive, it is primitive by part (a) and the previous exercise, so $H$ is maximal.
\end{enumerate}


\qed
\begin{problembox}[1.48: Counting fixed points]
Let $G$ be a group acting transitively on a set $S$ with at least 2 elements. For each $x \in G$ let $f(x) = $ number of elements of $S$ fixed by $x$. Prove:
\begin{enumerate}[label=(\alph*)]
\item $\sum_{x \in G} f(x) = \#(G).$
\item $G$ is doubly transitive if and only if
\[\sum_{x \in G} f(x)^2 = 2 \#(G).\]
\end{enumerate}
\end{problembox}

\noindent\textbf{Strategy:} For part (a), use the fact that each element stabilizes exactly one element in each orbit. For part (b), use the fact that in a doubly transitive action, each element fixes either 0 or 1 ordered pairs of distinct elements.

\noindent\textbf{Solution:}
\begin{enumerate}[label=(\alph*)]
\item Let $s \in S$ and let $H$ be the stabilizer of $s$. For each $x \in G$, the number of fixed points of $x$ is the number of elements $t \in S$ such that $xt = t$.

Since the action is transitive, for any $t \in S$ there exists $g \in G$ such that $t = gs$. Then $xt = t$ if and only if $xgs = gs$, which means $g^{-1}xg \in H$.

Therefore, $f(x) = \#\{g \in G : g^{-1}xg \in H\} = \#\{g \in G : x \in gHg^{-1}\}$.

Summing over all $x \in G$, we get $\sum_{x \in G} f(x) = \sum_{x \in G} \#\{g \in G : x \in gHg^{-1}\} = \sum_{g \in G} \#(gHg^{-1}) = \#(G)$.

\item If $G$ is doubly transitive, then for any two pairs $(s_1, s_2)$ and $(t_1, t_2)$ of distinct elements, there exists exactly one element $g \in G$ such that $gs_1 = t_1$ and $gs_2 = t_2$.

This means that for any $x \in G$, the number of ordered pairs $(s, t)$ with $s \neq t$ and $xs = s$, $xt = t$ is either 0 or 1.

Therefore, $\sum_{x \in G} f(x)(f(x)-1) = \#(G)$.

Combining with part (a), we get $\sum_{x \in G} f(x)^2 = 2\#(G)$.

Conversely, if $\sum_{x \in G} f(x)^2 = 2\#(G)$, then $\sum_{x \in G} f(x)(f(x)-1) = \#(G)$, which means $G$ is doubly transitive.
\end{enumerate}


\qed
\section{Fiber Products and Coproducts}

\textbf{Key Definitions and Theorems:}

\textbf{Definition:} A \textit{fiber product} (or pullback) of morphisms $f: X \to Z$ and $g: Y \to Z$ is an object $X \times_Z Y$ with morphisms $p_1: X \times_Z Y \to X$ and $p_2: X \times_Z Y \to Y$ such that $f \circ p_1 = g \circ p_2$.

\textbf{Definition:} A \textit{fiber coproduct} (or pushout) of morphisms $f: Z \to X$ and $g: Z \to Y$ is an object $X \oplus_Z Y$ with morphisms $i_1: X \to X \oplus_Z Y$ and $i_2: Y \to X \oplus_Z Y$ such that $i_1 \circ f = i_2 \circ g$.

\textbf{Universal Property of Fiber Product:} For any object $W$ with morphisms $h: W \to X$ and $k: W \to Y$ such that $f \circ h = g \circ k$, there exists a unique morphism $\phi: W \to X \times_Z Y$ such that $p_1 \circ \phi = h$ and $p_2 \circ \phi = k$.

\textbf{Universal Property of Fiber Coproduct:} For any object $W$ with morphisms $h: X \to W$ and $k: Y \to W$ such that $h \circ f = k \circ g$, there exists a unique morphism $\phi: X \oplus_Z Y \to W$ such that $\phi \circ i_1 = h$ and $\phi \circ i_2 = k$.

\textbf{Construction in Abelian Groups:} The fiber product is $X \times_Z Y = \{(x,y) \in X \times Y : f(x) = g(y)\}$.

\textbf{Construction in Abelian Groups:} The fiber coproduct is $X \oplus_Z Y = (X \oplus Y)/W$ where $W = \{(f(z), -g(z)) : z \in Z\}$.

\textbf{Theorem:} The pullback of a surjective homomorphism is surjective.

\textbf{Theorem:} The pushout of an injective homomorphism is injective.

\textbf{Definition:} A \textit{free product} of groups $G$ and $H$ is the coproduct in the category of groups.

\textbf{Definition:} An \textit{amalgamated free product} $G *_H G'$ is the coproduct of homomorphisms $f: H \to G$ and $g: H \to G'$.

\begin{problembox}[1.50: Fiber products in abelian groups]
\begin{enumerate}[label=(\alph*)]
\item Show that fiber products exist in the category of abelian groups. In fact, if $X, Y$ are abelian groups with homomorphisms $f: X \rightarrow Z$ and $g: Y \rightarrow Z$ show that $X \times_Z Y$ is the set of all pairs $(x, y)$ with $x \in X$ and $y \in Y$ such that $f(x) = g(y)$. The maps $p_1, p_2$ are the projections on the first and second factor respectively.
\item Show that the pull-back of a surjective homomorphism is surjective.
\end{enumerate}
\end{problembox}

\noindent\textbf{Strategy:} For part (a), verify that the given set is a subgroup and satisfies the universal property. For part (b), use the surjectivity of $f$ to show that for any $y \in Y$, there exists $x \in X$ such that $f(x) = g(y)$.

\noindent\textbf{Solution:}
\begin{enumerate}[label=(\alph*)]
\item Let $X \times_Z Y = \{(x, y) \in X \times Y : f(x) = g(y)\}$. This is a subgroup of $X \times Y$ because if $(x_1, y_1), (x_2, y_2) \in X \times_Z Y$, then $f(x_1) = g(y_1)$ and $f(x_2) = g(y_2)$, so $f(x_1 + x_2) = f(x_1) + f(x_2) = g(y_1) + g(y_2) = g(y_1 + y_2)$, so $(x_1 + x_2, y_1 + y_2) \in X \times_Z Y$.

The projections $p_1: X \times_Z Y \to X$ and $p_2: X \times_Z Y \to Y$ are homomorphisms, and $f \circ p_1 = g \circ p_2$.

If $W$ is another abelian group with homomorphisms $h: W \to X$ and $k: W \to Y$ such that $f \circ h = g \circ k$, then the unique homomorphism $\phi: W \to X \times_Z Y$ is given by $\phi(w) = (h(w), k(w))$.

\item Let $f: X \to Z$ be surjective and let $g: Y \to Z$ be any homomorphism. We show that $p_2: X \times_Z Y \to Y$ is surjective.

For any $y \in Y$, let $z = g(y)$. Since $f$ is surjective, there exists $x \in X$ such that $f(x) = z = g(y)$. Then $(x, y) \in X \times_Z Y$ and $p_2(x, y) = y$.
\end{enumerate}


\qed
\begin{problembox}[1.51: Fiber products in sets]
\begin{enumerate}[label=(\alph*)]
\item Show that fiber products exist in the category of sets.
\item In any category $\mathcal{C}$, consider the category $\mathcal{C}_Z$ of objects over $Z$. Let $h: T \rightarrow Z$ be a fixed object in this category. Let $F$ be the functor such that
\[F(X) = \text{Mor}_Z(T, X),\]
where $X$ is an object over $Z$, and $\text{Mor}_Z$ denotes morphisms over $Z$. Show that $F$ transforms fiber products over $Z$ into products in the category of sets. (Actually, once you have understood the definitions, this is tautological.)
\end{enumerate}
\end{problembox}

\noindent\textbf{Strategy:} For part (a), use the same construction as for abelian groups. For part (b), use the universal property of fiber products to show that morphisms from $T$ to the fiber product correspond to pairs of morphisms from $T$ to $X$ and $Y$.

\noindent\textbf{Solution:}
\begin{enumerate}[label=(\alph*)]
\item Let $X, Y$ be sets with functions $f: X \to Z$ and $g: Y \to Z$. The fiber product $X \times_Z Y = \{(x, y) \in X \times Y : f(x) = g(y)\}$ with projections $p_1: X \times_Z Y \to X$ and $p_2: X \times_Z Y \to Y$ satisfies the universal property.

\item The functor $F$ sends an object $X$ over $Z$ to the set of morphisms from $T$ to $X$ over $Z$.

If $X \times_Z Y$ is the fiber product of $X$ and $Y$ over $Z$, then $F(X \times_Z Y) = \text{Mor}_Z(T, X \times_Z Y)$.

By the universal property of the fiber product, a morphism $T \to X \times_Z Y$ over $Z$ is equivalent to a pair of morphisms $T \to X$ and $T \to Y$ over $Z$.

Therefore, $F(X \times_Z Y) \cong F(X) \times F(Y)$, which is the product in the category of sets.
\end{enumerate}


\qed
\begin{problembox}[1.52: Push-outs in abelian groups]
\begin{enumerate}[label=(\alph*)]
\item Show that push-outs (i.e. fiber coproducts) exist in the category of abelian groups. In this case the fiber coproduct of two homomorphisms $f, g$ as above is denoted by $X \oplus_Z Y$. Show that it is the factor group
\[X \oplus_Z Y = (X \oplus Y)/W,\]
where $W$ is the subgroup consisting of all elements $(f(z), -g(z))$ with $z \in Z$.
\item Show that the push-out of an injective homomorphism is injective.
\end{enumerate}
Remark. After you have read about modules over rings, you should note that the above two exercises apply to modules as well as to abelian groups.
\end{problembox}

\noindent\textbf{Strategy:} For part (a), verify that the given construction satisfies the universal property of push-outs. For part (b), use the injectivity of $f$ to show that if $(0, y) \in W$, then $y = 0$.

\noindent\textbf{Solution:}
\begin{enumerate}[label=(\alph*)]
\item Let $X \oplus_Z Y = (X \oplus Y)/W$ where $W = \{(f(z), -g(z)) : z \in Z\}$. The maps $i_1: X \to X \oplus_Z Y$ and $i_2: Y \to X \oplus_Z Y$ are given by $i_1(x) = (x, 0) + W$ and $i_2(y) = (0, y) + W$.

These maps satisfy $i_1 \circ f = i_2 \circ g$ because $(f(z), 0) + W = (0, g(z)) + W$ for all $z \in Z$.

If $A$ is another abelian group with homomorphisms $h: X \to A$ and $k: Y \to A$ such that $h \circ f = k \circ g$, then the unique homomorphism $\phi: X \oplus_Z Y \to A$ is given by $\phi((x, y) + W) = h(x) + k(y)$.

\item Let $f: Z \to X$ be injective and let $g: Z \to Y$ be any homomorphism. We show that $i_2: Y \to X \oplus_Z Y$ is injective.

If $i_2(y) = 0$, then $(0, y) \in W$, so $(0, y) = (f(z), -g(z))$ for some $z \in Z$. Since $f$ is injective, $z = 0$, so $y = -g(0) = 0$.
\end{enumerate}


\qed
\begin{problembox}[1.53: Coproduct of homomorphisms]
Let $H, G, G'$ be groups, and let
\[f: H \rightarrow G, \quad g: H \rightarrow G'\]
be two homomorphisms. Define the notion of coproduct of these two homomorphisms over $H$, and show that it exists.
\end{problembox}

\noindent\textbf{Strategy:} Define the coproduct using the universal property and show that it exists as the amalgamated free product $G *_H G'$, which is the quotient of the free product by the normal subgroup generated by relations $f(h) = g(h)$ for all $h \in H$.

\noindent\textbf{Solution:} The coproduct of the homomorphisms $f: H \to G$ and $g: H \to G'$ over $H$ is a group $K$ with homomorphisms $i_1: G \to K$ and $i_2: G' \to K$ such that $i_1 \circ f = i_2 \circ g$, and for any group $L$ with homomorphisms $h: G \to L$ and $k: G' \to L$ satisfying $h \circ f = k \circ g$, there exists a unique homomorphism $\phi: K \to L$ such that $\phi \circ i_1 = h$ and $\phi \circ i_2 = k$.

This coproduct exists and is given by the amalgamated free product $G *_H G'$. This is the quotient of the free product $G * G'$ by the normal subgroup generated by all elements of the form $f(h)g(h)^{-1}$ for $h \in H$.

The maps $i_1$ and $i_2$ are the natural inclusions of $G$ and $G'$ into the free product, followed by the quotient map.


\qed
\begin{problembox}[1.54: Tits' coproduct criterion]
Let $G$ be a group and let $\{G_i\}_{i \in I}$ be a family of subgroups generating $G$. Suppose $G$ operates on a set $S$. For each $i \in I$, suppose given a subset $S_i$ of $S$, and let $s$ be a point of $S - \bigcup_S S_i$. Assume that for each $g \in G_i - \{e\}$, we have
\[gS_j \subset S_i \text{ for all } j \neq i, \quad \text{and } g(s) \in S_i \text{ for all } i.\]
Prove that $G$ is the coproduct of the family $\{G_i\}_{i \in I}$. (Hint: Suppose a product $g_1 \cdots g_m = id$ on $S$. Apply this product to $s$, and use Proposition 12.4.)
\end{problembox}

\noindent\textbf{Strategy:} Show that any non-trivial reduced word in the $G_i$ acts non-trivially on $S$ by induction on the length of the word. Use the given conditions to show that the action moves the point $s$ to a specific subset.

\noindent\textbf{Solution:} We show that any non-trivial reduced word in the $G_i$ acts non-trivially on $S$, which implies that $G$ is the coproduct of the $G_i$.

Let $g_1 \cdots g_m$ be a reduced word where $g_k \in G_{i_k}$ and $i_k \neq i_{k+1}$ for all $k$.

We show by induction on $m$ that $g_1 \cdots g_m(s) \in S_{i_1}$.

For $m = 1$, this follows from the assumption that $g_1(s) \in S_{i_1}$.

For $m > 1$, let $s' = g_2 \cdots g_m(s)$. By induction, $s' \in S_{i_2}$. Since $g_1 \in G_{i_1}$ and $i_1 \neq i_2$, we have $g_1(s') \in S_{i_1}$ by the assumption that $g_1S_j \subset S_{i_1}$ for all $j \neq i_1$.

Therefore, $g_1 \cdots g_m(s) = g_1(s') \in S_{i_1} \neq \{s\}$, so $g_1 \cdots g_m \neq id$.

This shows that $G$ is the coproduct of the $G_i$.


\qed
\begin{problembox}[1.55: Fixed points of Möbius transformations]
Let $M \in GL_2(\mathbb{C})$ ($2 \times 2$ complex matrices with non-zero determinant). We let
\[M = \begin{pmatrix}
a & b \\
c & d 
\end{pmatrix}, \text{ and for } z \in \mathbb{C} \text{ we let } M(z) = \frac{az + b}{cz + d}.\]

If $z = -d/c$ ($c \neq 0$) then we put $M(z) = \infty$. Then you can verify (and you should have seen something like this in a course in complex analysis) that $GL_2(\mathbb{C})$ thus operates on $\mathbb{C} \cup \{\infty\}$. Let $\lambda, \lambda'$ be the eigenvalues of $M$ viewed as a linear map on $\mathbb{C}^2$. Let $W, W'$ be the corresponding eigenvectors,
\[W = '(w_1, w_2) \text{ and } W' = '(w_1', w_2').\]

By a fixed point of $M$ on $\mathbb{C}$ we mean a complex number $z$ such that $M(z) = z$. Assume that $M$ has two distinct fixed points $\neq \infty$.
\begin{enumerate}[label=(\alph*)]
\item Show that there cannot be more than two fixed points and that these fixed points are $w = w_1 / w_2$ and $w' = w_1' / w_2'$. In fact one may take
\[ W = '(w, 1), W' = '(w', 1). \]
\item Assume that $|\lambda| < |\lambda'|$. Given $z \neq w$, show that
\[ \lim_{k \to \infty} M^k(z) = w'. \]
[Hint: Let $S = (W, W')$ and consider $S^{-1}M^kS(z) = \alpha^kz$ where $\alpha = \lambda / \lambda'$.]
\end{enumerate}
\end{problembox}

\noindent\textbf{Strategy:} For part (a), solve the fixed point equation and show it's quadratic. For part (b), diagonalize the matrix using eigenvectors and use the fact that $|\alpha| < 1$ to show convergence.

\noindent\textbf{Solution:}
\begin{enumerate}[label=(\alph*)]
\item The fixed points of $M$ are the solutions to $M(z) = z$, which gives the equation $cz^2 + (d-a)z - b = 0$. This is a quadratic equation, so there are at most two fixed points.

If $W = '(w_1, w_2)$ is an eigenvector with eigenvalue $\lambda$, then $MW = \lambda W$, so $aw_1 + bw_2 = \lambda w_1$ and $cw_1 + dw_2 = \lambda w_2$.

This gives $w_1/w_2 = (b)/(\lambda - a) = (\lambda - d)/c$. If we take $W = '(w, 1)$ where $w = w_1/w_2$, then $w$ satisfies the fixed point equation.

\item Let $S = (W, W')$ be the matrix with columns $W$ and $W'$. Then $S^{-1}MS = \begin{pmatrix} \lambda & 0 \\ 0 & \lambda' \end{pmatrix}$.

For any $z \in \mathbb{C}$, we have $S^{-1}M^kS(z) = \alpha^k z$ where $\alpha = \lambda/\lambda'$.

Since $|\alpha| < 1$, we have $\lim_{k \to \infty} \alpha^k = 0$, so $\lim_{k \to \infty} S^{-1}M^kS(z) = 0$.

This means $\lim_{k \to \infty} M^k(z) = S(0) = w'$.
\end{enumerate}


\qed
\begin{problembox}[1.56: Free subgroup of $GL_2(\mathbb{C})$]
Let $M_1, \ldots, M_r \in GL_2(\mathbb{C})$ be a finite number of matrices. Let $\lambda_i, \lambda_j$ be the eigenvalues of $M_i$. Assume that each $M_i$ has two distinct complex fixed points, and that $|\lambda_i| < |\lambda_j|$. Also assume that the fixed points for $M_1, \ldots, M_r$ are all distinct from each other. Prove that there exists a positive integer $k$ such that $M_i^k, \ldots, M_r^k$ are the free generators of a free subgroup of $GL_2(\mathbb{C})$. [Hint: Let $w_i, w_i'$ be the fixed points of $M_i$. Let $U_i$ be a small disc centered at $w_i$ and $U_i'$ a small disc centered at $w_i'$. Let $S_i = U_i \cup U_i'$. Let $s$ be a complex number which does not lie in any $S_i$. Let $G_i = \langle M_i^k \rangle$. Show that the conditions of Exercise 54 are satisfied for $k$ sufficiently large.]

\noindent\textbf{Strategy:} Use the hint to construct neighborhoods around fixed points and show that for large enough $k$, the action of $M_i^k$ satisfies the conditions of Exercise 54, which guarantees that the group is a free product.

\noindent\textbf{Solution:} Let $w_i, w_i'$ be the fixed points of $M_i$ with $|w_i| < |w_i'|$. Let $U_i$ be a small disc centered at $w_i$ and $U_i'$ a small disc centered at $w_i'$. Let $S_i = U_i \cup U_i'$.
\begin{tikzpicture}[scale=2, >=stealth]

    % Complex plane background
    % \draw[->] (-1,0) -- (3,0) node[right] {$\text{Re}$};
    % \draw[->] (0,-1) -- (0,3) node[above] {$\text{Im}$};
    % \draw[gray!30] (-1,-1) grid (3,3);

    % Fixed points w and w'
    \filldraw[red] (0.5,1) circle (1pt) node[below right] {$w_i$};
    \filldraw[red] (2,2) circle (1pt) node[above left] {$w_i'$};
    
    % Neighborhoods U and U'
    \draw[red, dashed] (0.5,1) circle (0.6) node[below=0.7cm] {$U_i$};
    \draw[red, dashed] (2,2) circle (0.6) node[below=0.7cm] {$U_i'$};
    
    % Point s outside the neighborhoods
    \filldraw[blue] (0.9,2.7) circle (1pt) node[below right] {$s$};
    
    % Matrix action arrow
    % \draw[->, thick, orange] (1.2,0.5) to[out=60,in=-120] (1.8,1.8);
    % \node[orange, right] at (1.8,1.2) {$M^k$ action};

    % Title
    % \node[align=center] at (1,-1.2) {Problem 1.56: Fixed points and action of $M \in \text{GL}_2(\mathbb{C})$\\
    % $w,w'$ are fixed points, $s \notin U \cup U'$};

\end{tikzpicture}

\end{problembox}

\noindent\textbf{Solution:} Let $w_i, w_i'$ be the fixed points of $M_i$ with $|w_i| < |w_i'|$. Let $U_i$ be a small disc centered at $w_i$ and $U_i'$ a small disc centered at $w_i'$. Let $S_i = U_i \cup U_i'$.

Let $s$ be a complex number not in any $S_i$. For $k$ sufficiently large, the action of $M_i^k$ on $\mathbb{C} \cup \{\infty\}$ satisfies the conditions of Exercise 54:

1. For any $g \in \langle M_i^k \rangle - \{e\}$, we have $gS_j \subset S_i$ for all $j \neq i$ because $M_i^k$ contracts towards the fixed points of $M_i$.

2. For any $g \in \langle M_i^k \rangle - \{e\}$, we have $g(s) \in S_i$ because $M_i^k$ maps points outside $S_i$ into $S_i$ for large enough $k$.

Therefore, by Exercise 54, the group generated by $M_1^k, \ldots, M_r^k$ is the free product of the cyclic groups $\langle M_i^k \rangle$.


\qed
\begin{problembox}[1.57: Group generated by stabilizers]
Let $G$ be a group acting on a set $X$. Let $Y$ be a subset of $X$. Let $G_Y$ be the subset of $G$ consisting of those elements $g$ such that $gY \cap Y$ is not empty. Let $\overline{G}_Y$ be the subgroup of $G$ generated by $G_Y$. Then $\overline{G}_Y Y$ and $(G - \overline{G}_Y)Y$ are disjoint. [Hint: Suppose that there exist $g_1 \in \overline{G}_Y$ and $g_2 \in G$ but $g_2 \notin \overline{G}_Y$, and elements $y_1, y_2, \in Y$ such that $g_2y_1 = g_2y_2$. Then $g_2^{-1}g_1y_1 = y_2$, so $g_2^{-1}g_1 \in G_Y$ whence $g_2 \in \overline{G}_Y$, contrary to assumption.]

Application. Suppose that $X = GY$, but that $X$ cannot be expressed as a disjoint union as above unless one of the two sets is empty. Then we conclude that $G - \overline{G}_Y$ is empty, and therefore $G_Y$ generates $G$.

Example 1. Suppose $X$ is a connected topological space, $Y$ is open, and $G$ acts continuously. Then all translates of $Y$ are open, so $G$ is generated by $G_Y$.

Example 2. Suppose $G$ is a discrete group acting continuously and discretely on $X$. Again suppose $X$ connected and $Y$ closed, and that any union of translates of $Y$ by elements of $G$ is closed, so again $G - \overline{G}_Y$ is empty, and $G_Y$ generates $G$.
\end{problembox}

\noindent\textbf{Strategy:} Use the hint to prove by contradiction. If the sets are not disjoint, then there exist elements as described in the hint, which leads to a contradiction since $g_2^{-1}g_1 \in G_Y$ implies $g_2 \in \overline{G}_Y$.

\noindent\textbf{Solution:} We prove that $\overline{G}_Y Y$ and $(G - \overline{G}_Y)Y$ are disjoint.

Suppose for contradiction that there exist $g_1 \in \overline{G}_Y$, $g_2 \in G - \overline{G}_Y$, and $y_1, y_2 \in Y$ such that $g_1y_1 = g_2y_2$.

Then $g_2^{-1}g_1y_1 = y_2 \in Y$, so $g_2^{-1}g_1 \in G_Y$. Since $G_Y \subseteq \overline{G}_Y$, we have $g_2^{-1}g_1 \in \overline{G}_Y$.

Since $g_1 \in \overline{G}_Y$, this implies $g_2 \in \overline{G}_Y$, contradicting the assumption that $g_2 \notin \overline{G}_Y$.

Therefore, $\overline{G}_Y Y$ and $(G - \overline{G}_Y)Y$ are disjoint.

\textbf{Application:} If $X = GY$ and $X$ cannot be expressed as a disjoint union of the form above unless one set is empty, then we must have $G - \overline{G}_Y = \emptyset$, which means $G = \overline{G}_Y$. Therefore, $G_Y$ generates $G$.

\textbf{Example 1:} If $X$ is connected and $Y$ is open, then $GY$ is open and connected. If $G_Y$ did not generate $G$, then $\overline{G}_Y Y$ and $(G - \overline{G}_Y)Y$ would be disjoint open sets whose union is $X$, contradicting connectedness.

\textbf{Example 2:} If $X$ is connected and $Y$ is closed, and if $G_Y$ did not generate $G$, then $\overline{G}_Y Y$ and $(G - \overline{G}_Y)Y$ would be disjoint closed sets whose union is $X$, contradicting connectedness.

\section{Problem-Solving Techniques}

This chapter covers fundamental group theory concepts and provides various techniques for solving group theory problems. Here is a summary of the key problem-solving strategies:

\subsection*{Proving Groups are Abelian}
\begin{itemize}
\item Use Lagrange's theorem to determine possible element orders
\item Show that if $G/Z(G)$ is cyclic, then $G$ is abelian
\item Use the fact that groups of prime order are cyclic and hence abelian
\item Show that all elements commute by expressing them in terms of generators and central elements
\end{itemize}

\subsection*{Classifying Groups of Small Order}
\begin{enumerate}
\item Use Lagrange's theorem to determine possible element orders
\item Check if the group has an element of maximum possible order (making it cyclic)
\item If not, construct explicit isomorphisms to known groups
\item Use the fact that groups of order $p^2$ are abelian and either cyclic or isomorphic to $\mathbb{Z}_p \times \mathbb{Z}_p$
\end{enumerate}

\subsection*{Working with Normal Subgroups}
\begin{itemize}
\item Show that conjugates of generators are in the subgroup
\item Use the fact that normal subgroups are unions of conjugacy classes
\item For commutator subgroups, show that conjugates of commutators are commutators
\item Use the universal property of quotient groups for factorization properties
\end{itemize}

\subsection*{Counting Arguments}
\begin{enumerate}
\item Define a surjective map and count preimages
\item Use the inclusion-exclusion principle for unions of sets
\item Apply the orbit-stabilizer theorem: $|G \cdot s| = (G : G_s)$
\item Use Burnside's lemma: number of orbits = $\frac{1}{|G|} \sum_{g \in G} |\text{Fix}(g)|$
\end{enumerate}

\subsection*{Sylow Theory Applications}
\begin{itemize}
\item Use Sylow's theorems to count Sylow subgroups
\item Show that if $n_p = 1$, then the Sylow subgroup is normal
\item Use the fact that normal Sylow subgroups have trivial intersection with other Sylow subgroups
\item Apply the fact that extensions of solvable groups by solvable groups are solvable
\end{itemize}

\subsection*{Group Actions}
\begin{enumerate}
\item Define an action and use it to create homomorphisms
\item Use transitivity to simplify problems (assume one element is fixed)
\item Show primitivity by proving stabilizers are maximal subgroups
\item Use double transitivity to count fixed points and derive formulas
\end{enumerate}

\subsection*{Permutation Groups}
\begin{itemize}
\item Use cycle structure to determine conjugacy classes
\item Show that conjugates of cycles are cycles with permuted elements
\item Use the fact that $S_n$ is generated by transpositions
\item Apply the fact that $A_n$ is $(n-2)$-transitive for $n \geq 3$
\end{itemize}

\subsection*{Abelian Groups}
\begin{enumerate}
\item Use the fundamental theorem of finite abelian groups
\item Decompose into direct products of cyclic groups of prime power orders
\item Use the structure theorem for finitely generated abelian groups
\item Apply isomorphism theorems to relate indices and quotients
\end{enumerate}

\subsection*{Universal Properties}
\begin{itemize}
\item Define objects using universal properties (fiber products, push-outs)
\item Show existence by constructing explicit examples
\item Use universal properties to prove uniqueness
\item Apply functorial properties to transform problems
\end{itemize}

\subsection*{Free Products and Coproducts}
\begin{enumerate}
\item Use Tits' criterion to show a group is a free product
\item Show that reduced words act non-trivially on some set
\item Use the universal property of free products
\item Apply the fact that free products have no non-trivial relations
\end{enumerate}

\subsection*{Geometric Methods}
\begin{itemize}
\item Use geometric interpretations (dihedral groups as symmetries of polygons)
\item Apply Möbius transformations and fixed point analysis
\item Use complex analysis techniques for matrix groups
\item Apply topological methods for group actions on connected spaces
\end{itemize}

\subsection*{Proof by Contradiction}
\begin{enumerate}
\item Assume the opposite of what you want to prove
\item Use the assumption to derive a contradiction
\item Often used with counting arguments or geometric properties
\item Particularly useful for showing uniqueness or non-existence
\end{enumerate}

\subsection*{Induction}
\begin{itemize}
\item Use induction on group order for classification problems
\item Induct on the length of words in free products
\item Use induction on the number of generators
\item Apply induction on the complexity of group structures
\end{itemize}

These techniques provide a comprehensive toolkit for approaching group theory problems, from basic classification to advanced structural analysis.