\begin{problembox}[3.1: Open and Closed Intervals ]
Prove that an open interval in $\mathbb{R}^1$ is an open set and that a closed interval is a closed set.
\end{problembox}

\textbf{Solution:} Let $(a,b)$ be an open interval in $\mathbb{R}^1$. To show it's open, we need to prove that every point $x \in (a,b)$ is an interior point. For any $x \in (a,b)$, let $\varepsilon = \min\{x-a, b-x\}$. Then the open ball $B(x,\varepsilon) = (x-\varepsilon, x+\varepsilon)$ is contained entirely within $(a,b)$. This shows that every point in $(a,b)$ is an interior point, so $(a,b)$ is open.

For a closed interval $[a,b]$, we need to show its complement $\mathbb{R} \setminus [a,b] = (-\infty,a) \cup (b,\infty)$ is open. Any point $x$ in this complement is either less than $a$ or greater than $b$. If $x < a$, let $\varepsilon = a-x$, then $B(x,\varepsilon) = (x-\varepsilon, x+\varepsilon) \subset (-\infty,a)$. If $x > b$, let $\varepsilon = x-b$, then $B(x,\varepsilon) \subset (b,\infty)$. This shows the complement is open, so $[a,b]$ is closed.

\begin{problembox}[3.2: Accumulation Points and Set Properties]
Determine all the accumulation points of the following sets in $\mathbb{R}^1$ and decide whether the sets are open or closed (or neither).
\begin{enumerate}[label=\textbf{(\alph*)}]
\item All integers.
\item The interval $(a, b)$.
\item All numbers of the form $1/n$,\quad $(n = 1, 2, 3, \dots)$.
\item All rational numbers.
\item All numbers of the form $2^{-n} + 5^{-m}$,\quad $(m, n = 1, 2, \dots)$.
\item All numbers of the form $(-1)^n + (1/m)$,\quad $(m, n = 1, 2, \dots)$.
\item All numbers of the form $(1/n) + (1/m)$,\quad $(m, n = 1, 2, \dots)$.
\item All numbers of the form $(-1)^n / [1 + (1/n)]$,\quad $(n = 1, 2, \dots)$
.
\end{enumerate}
\end{problembox}

\textbf{Solution:} 
(a) The set of integers has no accumulation points since each integer has a neighborhood containing no other integers. The set is closed (its complement is open) but not open.

(b) The interval $(a,b)$ has accumulation points $[a,b]$. The set is open but not closed.

(c) The set $\{1/n : n \in \mathbb{N}\}$ has 0 as its only accumulation point. The set is neither open nor closed.

(d) The set of rational numbers has all real numbers as accumulation points. The set is neither open nor closed.

(e) The set $\{2^{-n} + 5^{-m} : m,n \in \mathbb{N}\}$ has accumulation points $\{2^{-n} : n \in \mathbb{N}\} \cup \{5^{-m} : m \in \mathbb{N}\} \cup \{0\}$. The set is neither open nor closed.

(f) The set $\{(-1)^n + 1/m : m,n \in \mathbb{N}\}$ has accumulation points $\{-1, 1\}$. The set is neither open nor closed.

(g) The set $\{1/n + 1/m : m,n \in \mathbb{N}\}$ has accumulation points $\{1/n : n \in \mathbb{N}\} \cup \{0\}$. The set is neither open nor closed.

(h) The set $\{(-1)^n/(1+1/n) : n \in \mathbb{N}\}$ has accumulation points $\{-1, 1\}$. The set is neither open nor closed.

\begin{problembox}[3.3: Accumulation Points and Set Properties in $\mathbb{R}^2$]
The same as Exercise 3.2 for the following sets in $\mathbb{R}^2$:
\begin{enumerate}[label=\textbf{(\alph*)}]
\item All complex $z$ such that $|z| > 1$.
\item All complex $z$ such that $|z| \ge 1$.
\item All complex numbers of the form $(1/n) + (i/m)$,\quad $(m, n = 1, 2, \dots)$.
\item All points $(x, y)$ such that $x^2 - y^2 < 1$.
\item All points $(x, y)$ such that $x > 0$.
\item All points $(x, y)$ such that $x \ge 0$.
\end{enumerate}
\end{problembox}

\textbf{Solution:}
(a) The set $\{z \in \mathbb{C} : |z| > 1\}$ has accumulation points $\{z \in \mathbb{C} : |z| \geq 1\}$. The set is open but not closed.

(b) The set $\{z \in \mathbb{C} : |z| \geq 1\}$ has accumulation points $\{z \in \mathbb{C} : |z| \geq 1\}$. The set is closed but not open.

(c) The set $\{1/n + i/m : m,n \in \mathbb{N}\}$ has accumulation points $\{1/n : n \in \mathbb{N}\} \cup \{i/m : m \in \mathbb{N}\} \cup \{0\}$. The set is neither open nor closed.

(d) The set $\{(x,y) : x^2 - y^2 < 1\}$ has accumulation points $\{(x,y) : x^2 - y^2 \leq 1\}$. The set is open but not closed.

(e) The set $\{(x,y) : x > 0\}$ has accumulation points $\{(x,y) : x \geq 0\}$. The set is open but not closed.

(f) The set $\{(x,y) : x \geq 0\}$ has accumulation points $\{(x,y) : x \geq 0\}$. The set is closed but not open.

\begin{problembox}[3.4: Rational and Irrational Elements in Open Sets]
Prove that every nonempty open set $S$ in $\mathbb{R}^1$ contains both rational and irrational numbers.
\end{problembox}

\textbf{Solution:} Let $S$ be a nonempty open set in $\mathbb{R}^1$. Since $S$ is open, for any point $x \in S$, there exists $\varepsilon > 0$ such that the open interval $(x-\varepsilon, x+\varepsilon) \subset S$.

Since the rational numbers are dense in $\mathbb{R}$, there exists a rational number $q$ in $(x-\varepsilon, x+\varepsilon)$, and thus $q \in S$.

Similarly, since the irrational numbers are also dense in $\mathbb{R}$, there exists an irrational number $r$ in $(x-\varepsilon, x+\varepsilon)$, and thus $r \in S$.

Therefore, every nonempty open set contains both rational and irrational numbers.

\begin{problembox}[3.5: Open and Closed Sets in $\mathbb{R}^1$ and $\mathbb{R}^2$]
Prove that the only sets in $\mathbb{R}^1$ which are both open and closed are the empty set and $\mathbb{R}^1$ itself. Is a similar statement true for $\mathbb{R}^2$?
\end{problembox}

\textbf{Solution:} Let $S$ be a subset of $\mathbb{R}^1$ that is both open and closed. If $S$ is empty or $S = \mathbb{R}^1$, we're done. Otherwise, let $a = \inf S$ and $b = \sup S$ (allowing $a = -\infty$ or $b = \infty$).

Since $S$ is closed, $a, b \in S$ if they are finite. Since $S$ is open, there exists $\varepsilon > 0$ such that $(a-\varepsilon, a+\varepsilon) \subset S$ and $(b-\varepsilon, b+\varepsilon) \subset S$. This contradicts the definition of infimum and supremum unless $a = -\infty$ and $b = \infty$, meaning $S = \mathbb{R}^1$.

The same proof works for $\mathbb{R}^2$ using the fact that $\mathbb{R}^2$ is connected. The only subsets of $\mathbb{R}^2$ that are both open and closed are the empty set and $\mathbb{R}^2$ itself.

\begin{problembox}[3.6: Closed Sets as Intersection of Open Sets]
Prove that every closed set in $\mathbb{R}^1$ is the intersection of a countable collection of open sets.
\end{problembox}

\textbf{Solution:} Let $F$ be a closed set in $\mathbb{R}^1$. For each $n \in \mathbb{N}$, define $G_n = \{x \in \mathbb{R} : d(x,F) < 1/n\}$, where $d(x,F) = \inf\{|x-y| : y \in F\}$. Each $G_n$ is open since it's the union of open intervals.

We claim that $F = \bigcap_{n=1}^{\infty} G_n$. Clearly $F \subset \bigcap_{n=1}^{\infty} G_n$ since every point in $F$ has distance 0 to $F$.

For the reverse inclusion, let $x \in \bigcap_{n=1}^{\infty} G_n$. Then $d(x,F) < 1/n$ for all $n$, which means $d(x,F) = 0$. Since $F$ is closed, this implies $x \in F$.

\begin{problembox}[3.7: Structure of Bounded Closed Sets in $\mathbb{R}^1$]
Prove that a nonempty, bounded closed set $S$ in $\mathbb{R}^1$ is either a closed interval, or that $S$ can be obtained from a closed interval by removing a countable disjoint collection of open intervals whose endpoints belong to $S$.
\end{problembox}

\textbf{Solution:} Let $S$ be a nonempty, bounded closed set in $\mathbb{R}^1$. Let $a = \inf S$ and $b = \sup S$. Since $S$ is closed, $a, b \in S$.

If $S = [a,b]$, we're done. Otherwise, the complement $[a,b] \setminus S$ is open and can be written as a countable union of disjoint open intervals $(a_i, b_i)$. Since $S$ is closed, the endpoints $a_i, b_i$ must belong to $S$.

Therefore, $S = [a,b] \setminus \bigcup_{i=1}^{\infty} (a_i, b_i)$, which is the desired representation.

\begin{problembox}[3.8: Open Balls and Intervals in Rn]
Prove that open n-balls and n-dimensional open intervals are open sets in $\mathbb{R}^n$.
\end{problembox}

\textbf{Solution:} Let $B(a;r) = \{x \in \mathbb{R}^n : \|x-a\| < r\}$ be an open ball centered at $a$ with radius $r$. For any $x \in B(a;r)$, let $\varepsilon = r - \|x-a\| > 0$. Then $B(x;\varepsilon) \subset B(a;r)$ by the triangle inequality, showing $B(a;r)$ is open.

For an open interval $I = (a_1,b_1) \times \cdots \times (a_n,b_n)$, let $x = (x_1,\ldots,x_n) \in I$. For each $i$, let $\varepsilon_i = \min\{x_i - a_i, b_i - x_i\}$. Then the ball $B(x;\min\{\varepsilon_1,\ldots,\varepsilon_n\}) \subset I$, showing $I$ is open.

\begin{problembox}[3.9: Interior of a Set is Open]
Prove that the interior of a set in $\mathbb{R}^n$ is open in $\mathbb{R}^n$.
\end{problembox}

\textbf{Solution:} Let $S \subset \mathbb{R}^n$ and let $x \in \text{int } S$. By definition, there exists $\varepsilon > 0$ such that $B(x;\varepsilon) \subset S$.

For any $y \in B(x;\varepsilon)$, let $\delta = \varepsilon - \|y-x\| > 0$. Then $B(y;\delta) \subset B(x;\varepsilon) \subset S$, which shows that $y \in \text{int } S$.

Therefore, $B(x;\varepsilon) \subset \text{int } S$, proving that $\text{int } S$ is open.

\begin{problembox}[3.10: Interior as Union of Open Subsets]
If $S \subseteq \mathbb{R}^n$, prove that int $S$ is the union of all open subsets of $\mathbb{R}^n$ which are contained in $S$. This is described by saying that int $S$ is the largest open subset of $S$.
\end{problembox}

\textbf{Solution:} Let $\mathcal{U}$ be the collection of all open subsets of $\mathbb{R}^n$ contained in $S$. We need to show that $\text{int } S = \bigcup_{U \in \mathcal{U}} U$.

First, if $x \in \text{int } S$, then there exists $\varepsilon > 0$ such that $B(x;\varepsilon) \subset S$. Since $B(x;\varepsilon)$ is open and contained in $S$, we have $x \in B(x;\varepsilon) \in \mathcal{U}$, so $x \in \bigcup_{U \in \mathcal{U}} U$.

Conversely, if $x \in \bigcup_{U \in \mathcal{U}} U$, then $x \in U$ for some open set $U \subset S$. Since $U$ is open, there exists $\varepsilon > 0$ such that $B(x;\varepsilon) \subset U \subset S$, which shows $x \in \text{int } S$.

Therefore, $\text{int } S = \bigcup_{U \in \mathcal{U}} U$, proving that the interior is the largest open subset of $S$.

\begin{problembox}[3.11: Interior of Intersection and Union]
If $S$ and $T$ are subsets of $\mathbb{R}^n$, prove that
int$(S)$ $\cap$ int$(T)$ = int$(S \cap T)$,
and int$(S)$ $\cup$ int$(T)$ $\subseteq$ int$(S \cup T)$.
\end{problembox}    

\textbf{Solution:} For the first equality, let $x \in \text{int}(S) \cap \text{int}(T)$. Then there exist $\varepsilon_1, \varepsilon_2 > 0$ such that $B(x;\varepsilon_1) \subset S$ and $B(x;\varepsilon_2) \subset T$. Let $\varepsilon = \min\{\varepsilon_1, \varepsilon_2\}$. Then $B(x;\varepsilon) \subset S \cap T$, so $x \in \text{int}(S \cap T)$.

Conversely, if $x \in \text{int}(S \cap T)$, there exists $\varepsilon > 0$ such that $B(x;\varepsilon) \subset S \cap T$. This implies $B(x;\varepsilon) \subset S$ and $B(x;\varepsilon) \subset T$, so $x \in \text{int}(S) \cap \text{int}(T)$.

For the second inclusion, if $x \in \text{int}(S) \cup \text{int}(T)$, then $x \in \text{int}(S)$ or $x \in \text{int}(T)$. In either case, there exists $\varepsilon > 0$ such that $B(x;\varepsilon) \subset S$ or $B(x;\varepsilon) \subset T$, which implies $B(x;\varepsilon) \subset S \cup T$. Therefore, $x \in \text{int}(S \cup T)$.

\begin{problembox}[3.12: Properties of Derived Set and Closure]
Let $S'$ denote the derived set and $\overline{S}$ the closure of a set $S$ in $\mathbb{R}^n$. Prove that:
\begin{enumerate}[label=\alph*)]
\item $S'$ is closed in $\mathbb{R}^n$; that is, $\overline{S'} \subseteq S'$.
\item If $S \subseteq T$, then $S' \subseteq T'$.
\item $S' \cup T' = (S \cup T)'$.
\item $\overline{S} = S \cup S'$.
\item $\overline{S}$ is closed in $\mathbb{R}^n$.
\item $\overline{S}$ is the intersection of all closed subsets of $\mathbb{R}^n$ containing $S$. That is, $\overline{S}$ is the smallest closed set containing $S$.
\end{enumerate}
\end{problembox}

\textbf{Solution:} 
(a) Let $x \in \overline{S'}$. Then every neighborhood of $x$ contains a point of $S'$. Let $\varepsilon > 0$ and $y \in B(x;\varepsilon/2) \cap S'$. Since $y$ is an accumulation point of $S$, $B(y;\varepsilon/2)$ contains infinitely many points of $S$. But $B(y;\varepsilon/2) \subset B(x;\varepsilon)$, so $B(x;\varepsilon)$ contains infinitely many points of $S$. This shows $x \in S'$.

(b) If $x \in S'$, then every neighborhood of $x$ contains infinitely many points of $S$. Since $S \subseteq T$, these points are also in $T$, so $x \in T'$.

(c) Since $S \subseteq S \cup T$ and $T \subseteq S \cup T$, we have $S' \subseteq (S \cup T)'$ and $T' \subseteq (S \cup T)'$ by (b). Therefore, $S' \cup T' \subseteq (S \cup T)'$.

For the reverse inclusion, if $x \in (S \cup T)'$, then every neighborhood of $x$ contains infinitely many points of $S \cup T$. If infinitely many of these points are in $S$, then $x \in S'$. Otherwise, infinitely many are in $T$, so $x \in T'$. In either case, $x \in S' \cup T'$.

(d) Clearly $S \cup S' \subseteq \overline{S}$. For the reverse inclusion, if $x \in \overline{S}$, then every neighborhood of $x$ contains a point of $S$. If $x \notin S$, then every neighborhood contains a point of $S$ different from $x$, so $x \in S'$.

(e) By (d), $\overline{S} = S \cup S'$. Since $S'$ is closed by (a), and the union of a set with a closed set is closed, $\overline{S}$ is closed.

(f) Let $\mathcal{F}$ be the collection of all closed sets containing $S$. Since $\overline{S}$ is closed and contains $S$, we have $\bigcap_{F \in \mathcal{F}} F \subseteq \overline{S}$. For the reverse inclusion, since each $F \in \mathcal{F}$ is closed and contains $S$, we have $S' \subseteq F$ for all $F$. Therefore, $\overline{S} = S \cup S' \subseteq F$ for all $F$, so $\overline{S} \subseteq \bigcap_{F \in \mathcal{F}} F$.

\begin{problembox}[3.13: Closure under Intersection of Sets]

Let $S$ and $T$ be subsets of $\mathbb{R}^k$. Prove that $\overline{S \cup T} = \overline{S} \cup \overline{T}$ and that $\overline{S \cap T} \subseteq \overline{S} \cap \overline{T}$ if $S$ is open.

\textit{NOTE.} The statements in Exercises 3.9 through 3.13 are true in any metric space.
\end{problembox}

\textbf{Solution:} For the first equality, by Exercise 3.12(c), $(S \cup T)' = S' \cup T'$. Therefore, $\overline{S \cup T} = (S \cup T) \cup (S \cup T)' = (S \cup T) \cup (S' \cup T') = (S \cup S') \cup (T \cup T') = \overline{S} \cup \overline{T}$.

For the second inclusion, let $x \in \overline{S \cap T}$. Then $x \in S \cap T$ or $x \in (S \cap T)'$. If $x \in S \cap T$, then $x \in \overline{S} \cap \overline{T}$. If $x \in (S \cap T)'$, then every neighborhood of $x$ contains a point of $S \cap T$ different from $x$. Since $S$ is open, there exists $\varepsilon > 0$ such that $B(x;\varepsilon) \subset S$. Any point $y \in B(x;\varepsilon) \cap T$ different from $x$ shows that $x \in T'$, so $x \in \overline{T}$. Similarly, $x \in \overline{S}$. Therefore, $x \in \overline{S} \cap \overline{T}$.

\begin{problembox}[3.14: Properties of Convex Sets]
A set \( S \) in \( \mathbb{R}^n \) is called convex if, for every pair of points \( x \) and \( y \) in \( S \) and every real \( \theta \) satisfying \( 0 < \theta < 1 \), we have \( \theta x + (1 - \theta)y \in S \). Interpret this statement geometrically (in \( \mathbb{R}^2 \) and \( \mathbb{R}^3 \)) and prove that:
\begin{enumerate}[label=\alph*)]
\item Every \( n \)-ball in \( \mathbb{R}^n \) is convex.
\item Every \( n \)-dimensional open interval is convex.
\item The interior of a convex set is convex.
\item The closure of a convex set is convex.
\end{enumerate}
\end{problembox}

\textbf{Solution:} Geometrically, a set is convex if the line segment joining any two points in the set lies entirely within the set.

(a) Let $B(a;r)$ be an $n$-ball and $x, y \in B(a;r)$. For $0 < \theta < 1$, let $z = \theta x + (1-\theta)y$. Then $\|z-a\| = \|\theta(x-a) + (1-\theta)(y-a)\| \leq \theta\|x-a\| + (1-\theta)\|y-a\| < \theta r + (1-\theta)r = r$, so $z \in B(a;r)$.

(b) Let $I = (a_1,b_1) \times \cdots \times (a_n,b_n)$ be an open interval and $x, y \in I$. For $0 < \theta < 1$, let $z = \theta x + (1-\theta)y$. For each $i$, we have $a_i < x_i, y_i < b_i$, so $a_i < \theta x_i + (1-\theta)y_i < b_i$. Therefore, $z \in I$.

(c) Let $S$ be convex and $x, y \in \text{int } S$. There exist $\varepsilon_1, \varepsilon_2 > 0$ such that $B(x;\varepsilon_1) \subset S$ and $B(y;\varepsilon_2) \subset S$. Let $\varepsilon = \min\{\varepsilon_1, \varepsilon_2\}$. For $0 < \theta < 1$, let $z = \theta x + (1-\theta)y$. If $w \in B(z;\varepsilon)$, then $\|w-z\| < \varepsilon$. Let $u = w - z + x$ and $v = w - z + y$. Then $\|u-x\| = \|v-y\| = \|w-z\| < \varepsilon$, so $u, v \in S$. Since $S$ is convex, $w = \theta u + (1-\theta)v \in S$. Therefore, $B(z;\varepsilon) \subset S$, so $z \in \text{int } S$.

(d) Let $S$ be convex and $x, y \in \overline{S}$. There exist sequences $\{x_n\}, \{y_n\} \subset S$ converging to $x, y$ respectively. For $0 < \theta < 1$, let $z = \theta x + (1-\theta)y$ and $z_n = \theta x_n + (1-\theta)y_n$. Since $S$ is convex, $z_n \in S$ for all $n$. Since $z_n \to z$, we have $z \in \overline{S}$.

\begin{problembox}[3.15: Accumulation Points of Intersections and Unions]            
Let $\mathcal{F}$ be a collection of sets in $\mathbb{R}^k$, and let $S = \bigcup_{A \in \mathcal{F}} A$ and $T = \bigcap_{A \in \mathcal{F}} A$. For each of the following statements, either give a proof or exhibit a counterexample:
\begin{enumerate}[label=\alph*)]
\item If $\mathbf{x}$ is an accumulation point of $T$, then $\mathbf{x}$ is an accumulation point of each set $A$ in $\mathcal{F}$.
\item If $\mathbf{x}$ is an accumulation point of $S$, then $\mathbf{x}$ is an accumulation point of at least one set $A$ in $\mathcal{F}$.
\end{enumerate}
\end{problembox}

\textbf{Solution:} 
(a) This statement is false. Let $\mathcal{F} = \{A_1, A_2\}$ where $A_1 = \{1/n : n \in \mathbb{N}\}$ and $A_2 = \{-1/n : n \in \mathbb{N}\}$. Then $T = A_1 \cap A_2 = \emptyset$, so $T$ has no accumulation points. However, if we consider $T = \{0\}$ (a singleton), then $0$ is an accumulation point of $T$ but not of $A_1$ or $A_2$.

(b) This statement is true. Let $x$ be an accumulation point of $S$. Then every neighborhood of $x$ contains infinitely many points of $S$. Since $S = \bigcup_{A \in \mathcal{F}} A$, at least one set $A \in \mathcal{F}$ must contain infinitely many of these points. Therefore, $x$ is an accumulation point of that set $A$.

\begin{problembox}[3.16: Rationals  Not a Countable Intersection of Open Sets]
Prove that the set \( S \) of rational numbers in the interval \( (0, 1) \) cannot be expressed as the intersection of a countable collection of open sets. 

\textit{Hint.} Write \( S = \{x_1, x_2, \ldots\} \), assume \( S = \bigcap_{k=1}^{\infty} S_k \), where each \( S_k \) is open, and construct a sequence \( (Q_n) \) of closed intervals such that \( Q_{n+1} \subseteq Q_n \subseteq S_n \) and such that \( x_n \notin Q_n \). Then use the Cantor intersection theorem to obtain a contradiction.
\end{problembox}

\textbf{Solution:} Suppose for contradiction that $S = \bigcap_{k=1}^{\infty} S_k$ where each $S_k$ is open. Let $S = \{x_1, x_2, \ldots\}$ be an enumeration of the rationals in $(0,1)$.

For each $n$, since $S_n$ is open and contains all rationals in $(0,1)$, we can find a closed interval $Q_n \subset S_n$ such that $x_n \notin Q_n$. We can arrange that $Q_{n+1} \subseteq Q_n$ by taking $Q_{n+1} = Q_n \cap I_{n+1}$ where $I_{n+1}$ is a closed interval in $S_{n+1}$ that doesn't contain $x_{n+1}$.

By the Cantor intersection theorem, $\bigcap_{n=1}^{\infty} Q_n$ is nonempty. Let $x \in \bigcap_{n=1}^{\infty} Q_n$. Then $x \in \bigcap_{k=1}^{\infty} S_k = S$, so $x$ is rational. But $x \neq x_n$ for any $n$ since $x_n \notin Q_n$ for each $n$. This contradicts the fact that $S$ contains all rationals in $(0,1)$.

\begin{problembox}[3.17: Countability of Isolated Points]
If \( S \subseteq \mathbb{R}^n \), prove that the collection of isolated points of \( S \) is countable.
\end{problembox}

\textbf{Solution:} Let $I$ be the set of isolated points of $S$. For each $x \in I$, there exists $\varepsilon_x > 0$ such that $B(x;\varepsilon_x) \cap S = \{x\}$.

For each $x \in I$, let $q_x$ be a rational point in $B(x;\varepsilon_x/2)$ (which exists since rational points are dense). Then $B(q_x;\varepsilon_x/4)$ contains $x$ and no other point of $S$.

If $x, y \in I$ are distinct, then $B(q_x;\varepsilon_x/4) \cap B(q_y;\varepsilon_y/4) = \emptyset$, since otherwise we would have a point in $S$ other than $x$ or $y$ in one of these balls.

Since the collection of balls $\{B(q_x;\varepsilon_x/4) : x \in I\}$ is pairwise disjoint and each contains a rational point, this collection is countable. Therefore, $I$ is countable.

\begin{problembox}[3.18: Countable Covering of the First Quadrant]
Prove that the set of open disks in the \(xy\)-plane with center at \( (x, x) \) and radius \( x > 0 \), where \( x \) is rational, is a countable covering of the set \( \{(x, y) : x > 0, y > 0\} \).
\end{problembox}

\textbf{Solution:} Let $\mathcal{F}$ be the collection of open disks with center at $(x,x)$ and radius $x$ where $x$ is rational and positive. We need to show that every point $(a,b)$ with $a > 0, b > 0$ is contained in some disk in $\mathcal{F}$.

Let $(a,b)$ be a point in the first quadrant. Let $r = \min\{a,b\}$. Since the rationals are dense in $\mathbb{R}$, there exists a rational number $x$ such that $r/2 < x < r$.

Then $\|(a,b) - (x,x)\| = \sqrt{(a-x)^2 + (b-x)^2} < \sqrt{(r-x)^2 + (r-x)^2} = \sqrt{2}(r-x) < \sqrt{2}(r-r/2) = \sqrt{2}(r/2) < r < x$.

Therefore, $(a,b) \in B((x,x); x)$ where $x$ is rational, so $\mathcal{F}$ covers the first quadrant.

Since the rational numbers are countable, $\mathcal{F}$ is countable.

\begin{problembox}[3.19: Non-Finite Subcover of \(0,1\)]
The collection \( \mathcal{F} \) of open intervals of the form \( (1/n, 2/n) \), where \( n = 2, 3, \ldots \), is an open covering of the open interval \( (0, 1) \). Prove (without using Theorem 3.31) that no finite subcollection of \( \mathcal{F} \) covers \( (0, 1) \).
\end{problembox}

\textbf{Solution:} Let $\mathcal{G} = \{(1/n_1, 2/n_1), \ldots, (1/n_k, 2/n_k)\}$ be a finite subcollection of $\mathcal{F}$. Let $N = \max\{n_1, \ldots, n_k\}$.

Then the largest interval in $\mathcal{G}$ is $(1/N, 2/N)$. For any $x \in (0, 1/N)$, we have $x < 1/N < 2/N$, so $x$ is not covered by any interval in $\mathcal{G}$.

Therefore, $\mathcal{G}$ does not cover $(0,1)$, proving that no finite subcollection of $\mathcal{F}$ covers $(0,1)$.

\begin{problembox}[3.20: Closed but Not Bounded Set with Infinite Covering]
Give an example of a set \( S \) which is closed but not bounded and exhibit a countable open covering \( \mathcal{F} \) such that no finite subset of \( \mathcal{F} \) covers \( S \).
\end{problembox}

\textbf{Solution:} Let $S = \mathbb{Z}$ (the set of integers). This set is closed but not bounded.

Let $\mathcal{F} = \{(n-1/2, n+1/2) : n \in \mathbb{Z}\}$. This is a countable open covering of $\mathbb{Z}$ since each integer $n$ is contained in the interval $(n-1/2, n+1/2)$.

However, no finite subcollection of $\mathcal{F}$ covers $\mathbb{Z}$. If $\mathcal{G} = \{(n_1-1/2, n_1+1/2), \ldots, (n_k-1/2, n_k+1/2)\}$ is a finite subcollection, then $\mathcal{G}$ can only cover finitely many integers, but $\mathbb{Z}$ is infinite.

Therefore, $\mathcal{F}$ is a countable open covering of $S$ with no finite subcover.

\begin{problembox}[3.21: Countability via Local Countability]
Given a set \( S \) in \( \mathbb{R}^n \) with the property that for every \( x \) in \( S \) there is an \( n \)-ball \( B(x) \) such that \( B(x) \cap S \) is countable. Prove that \( S \) is countable.
\end{problembox}

\textbf{Solution:} For each $x \in S$, let $B_x$ be an $n$-ball centered at $x$ such that $B_x \cap S$ is countable. Let $\mathcal{B} = \{B_x : x \in S\}$.

Since $\mathbb{R}^n$ is separable, there exists a countable dense subset $D$. For each $B_x \in \mathcal{B}$, there exists a point $d \in D$ such that $d \in B_x$. Let $r_x$ be the radius of $B_x$, and let $q_x$ be a rational number such that $r_x/2 < q_x < r_x$.

Then $B_x$ is uniquely determined by the pair $(d_x, q_x)$ where $d_x$ is the center of $B_x$ and $q_x$ is the rational radius. Since $D$ is countable and the rationals are countable, the set of such pairs is countable.

Therefore, $\mathcal{B}$ is countable, and since each $B_x \cap S$ is countable, we have $S = \bigcup_{B_x \in \mathcal{B}} (B_x \cap S)$ is a countable union of countable sets, hence countable.

\begin{problembox}[3.22: Countability of Disjoint Open Sets]
Prove that a collection of disjoint open sets in \( \mathbb{R}^n \) is necessarily countable. Give an example of a collection of disjoint closed sets which is not countable.
\end{problembox}

\textbf{Solution:} Let $\mathcal{F}$ be a collection of disjoint open sets in $\mathbb{R}^n$. Since $\mathbb{R}^n$ is separable, there exists a countable dense subset $D$.

For each open set $U \in \mathcal{F}$, there exists a point $d \in D$ such that $d \in U$. Since the sets in $\mathcal{F}$ are disjoint, each point $d \in D$ can belong to at most one set in $\mathcal{F}$.

Therefore, the number of sets in $\mathcal{F}$ is at most the number of points in $D$, which is countable.

For an example of uncountably many disjoint closed sets, let $\mathcal{G} = \{\{x\} : x \in \mathbb{R}\}$. Each singleton $\{x\}$ is closed, the sets are disjoint, and there are uncountably many real numbers.

\begin{problembox}[3.23: Existence of Condensation Points]
Assume that \( S \subseteq \mathbb{R}^n \). A point \( x \) in \( \mathbb{R}^n \) is said to be a condensation point of \( S \) if every \( n \)-ball \( B(x) \) has the property that \( B(x) \cap S \) is not countable. Prove that if \( S \) is not countable, then there exists a point \( x \) in \( S \) such that \( x \) is a condensation point of \( S \).
\end{problembox}

\textbf{Solution:} Suppose for contradiction that no point in $S$ is a condensation point of $S$. Then for every $x \in S$, there exists an $n$-ball $B_x$ centered at $x$ such that $B_x \cap S$ is countable.

By Exercise 3.21, this implies that $S$ is countable, which contradicts the hypothesis that $S$ is not countable.

Therefore, there must exist at least one point $x \in S$ that is a condensation point of $S$.

\begin{problembox}[3.24: Properties of Condensation Points]
Assume that \( S \subseteq \mathbb{R}^n \) and that \( S \) is not countable. Let \( T \) denote the set of condensation points of \( S \). Prove that:
\begin{enumerate}[label=\alph*)]
\item \( S - T \) is countable,
\item \( S \cap T \) is not countable,
\item \( T \) is a closed set,
\item \( T \) contains no isolated points.
\end{enumerate}
Note that Exercise 3.23 is a special case of (b).
\end{problembox}

\textbf{Solution:} 
(a) For each $x \in S - T$, there exists an $n$-ball $B_x$ centered at $x$ such that $B_x \cap S$ is countable. By Exercise 3.21, $S - T$ is countable.

(b) Since $S$ is not countable and $S - T$ is countable, $S \cap T$ must be uncountable.

(c) Let $x \in \overline{T}$. Then every neighborhood of $x$ contains a point of $T$. Let $B$ be any $n$-ball centered at $x$. There exists $y \in T \cap B$. Since $y$ is a condensation point, $B(y;r) \cap S$ is uncountable for any $r > 0$. Choose $r$ small enough so that $B(y;r) \subset B$. Then $B \cap S$ contains the uncountable set $B(y;r) \cap S$, so $x$ is a condensation point. Therefore, $T$ is closed.

(d) Let $x \in T$. For any $\varepsilon > 0$, $B(x;\varepsilon) \cap S$ is uncountable. Since $S - T$ is countable, $B(x;\varepsilon) \cap T$ must be uncountable. Therefore, $x$ is not isolated in $T$.

\begin{problembox}[3.25: Cantor-Bendixon Theorem]
A set in \( \mathbb{R}^n \) is called perfect if \( S = S' \), that is, if \( S \) is a closed set which contains no isolated points. Prove that every uncountable closed set \( F \) in \( \mathbb{R}^n \) can be expressed in the form \( F = A \cup B \), where \( A \) is perfect and \( B \) is countable (Cantor-Bendixon theorem).

\textit{Hint.} Use Exercise 3.24.
\end{problembox}

\textbf{Solution:} Let $F$ be an uncountable closed set in $\mathbb{R}^n$. Let $T$ be the set of condensation points of $F$. By Exercise 3.24, $T$ is closed and $F - T$ is countable.

Let $A = T$ and $B = F - T$. Then $F = A \cup B$ where $B$ is countable.

We need to show that $A$ is perfect. Since $T$ is closed by Exercise 3.24(c), $A$ is closed. By Exercise 3.24(d), $T$ contains no isolated points, so $A$ contains no isolated points.

Therefore, $A$ is perfect, and we have the desired decomposition $F = A \cup B$.

\textbf{Metric Spaces}

\begin{problembox}[3.26: Open and Closed Sets in Metric Spaces]
In any metric space \((M, d)\), prove that the empty set \( \emptyset \) and the whole space \( M \) are both open and closed.
\end{problembox}

\textbf{Solution:} The empty set $\emptyset$ is open because the condition "for every point in $\emptyset$, there exists a neighborhood contained in $\emptyset$" is vacuously true (there are no points to check).

The empty set $\emptyset$ is closed because its complement $M$ is open.

The whole space $M$ is open because for any point $x \in M$ and any $\varepsilon > 0$, the ball $B(x;\varepsilon) \subset M$.

The whole space $M$ is closed because its complement $\emptyset$ is open.

\begin{problembox}[3.27: Metric Balls in Different Metrics]
Consider the following two metrics in \( \mathbb{R}^n \):
\[d_1(x, y) = \max_{1 \leq i \leq n} |x_i - y_i|, \quad d_2(x, y) = \sum_{i=1}^n |x_i - y_i|.\]

In each of the following metric spaces prove that the ball \( B(a; r) \) has the geometric appearance indicated:
\begin{enumerate}[label=\alph*)]
\item In \( (\mathbb{R}^2, d_1) \), a square with sides parallel to the coordinate axes.
\item In \( (\mathbb{R}^2, d_2) \), a square with diagonals parallel to the axes.
\item A cube in \( (\mathbb{R}^3, d_1) \).
\item An octahedron in \( (\mathbb{R}^3, d_2) \).
\end{enumerate}
\end{problembox}

\textbf{Solution:} 
(a) In $(\mathbb{R}^2, d_1)$, the ball $B(a;r) = \{(x,y) : \max\{|x-a_1|, |y-a_2|\} < r\}$. This means $|x-a_1| < r$ and $|y-a_2| < r$, which defines a square with center $(a_1,a_2)$ and sides of length $2r$ parallel to the coordinate axes.

(b) In $(\mathbb{R}^2, d_2)$, the ball $B(a;r) = \{(x,y) : |x-a_1| + |y-a_2| < r\}$. This defines a diamond-shaped region (square rotated 45 degrees) with diagonals parallel to the axes.

(c) In $(\mathbb{R}^3, d_1)$, the ball $B(a;r) = \{(x,y,z) : \max\{|x-a_1|, |y-a_2|, |z-a_3|\} < r\}$. This defines a cube with center $(a_1,a_2,a_3)$ and sides of length $2r$ parallel to the coordinate axes.

(d) In $(\mathbb{R}^3, d_2)$, the ball $B(a;r) = \{(x,y,z) : |x-a_1| + |y-a_2| + |z-a_3| < r\}$. This defines an octahedron with center $(a_1,a_2,a_3)$.

\begin{problembox}[3.28: Metric Inequalities]
Let \( d_1 \) and \( d_2 \) be the metrics of Exercise 3.27 and let \( \|x - y\| \) denote the usual Euclidean metric. Prove the following inequalities for all \( x \) and \( y \) in \( \mathbb{R}^n \):
\[d_1(x, y) \leq \|x - y\| \leq d_2(x, y) \quad \text{and} \quad d_2(x, y) \leq \sqrt{n} \|x - y\| \leq n\,d_1(x, y).\]
\end{problembox}

\textbf{Solution:} Let $x, y \in \mathbb{R}^n$. For the first set of inequalities:

Since $d_1(x,y) = \max_{1 \leq i \leq n} |x_i - y_i|$, we have $d_1(x,y)^2 = \max_{1 \leq i \leq n} |x_i - y_i|^2 \leq \sum_{i=1}^n |x_i - y_i|^2 = \|x-y\|^2$. Taking square roots gives $d_1(x,y) \leq \|x-y\|$.

For the upper bound, by the triangle inequality for the absolute value, $\|x-y\| = \sqrt{\sum_{i=1}^n |x_i - y_i|^2} \leq \sqrt{\sum_{i=1}^n |x_i - y_i|^2 + 2\sum_{i < j} |x_i - y_i||x_j - y_j|} = \sum_{i=1}^n |x_i - y_i| = d_2(x,y)$.

For the second set of inequalities:

By the Cauchy-Schwarz inequality, $d_2(x,y)^2 = (\sum_{i=1}^n |x_i - y_i|)^2 \leq n\sum_{i=1}^n |x_i - y_i|^2 = n\|x-y\|^2$. Taking square roots gives $d_2(x,y) \leq \sqrt{n}\|x-y\|$.

For the last inequality, $\sqrt{n}\|x-y\| = \sqrt{n}\sqrt{\sum_{i=1}^n |x_i - y_i|^2} \leq \sqrt{n}\sqrt{n}\max_{1 \leq i \leq n} |x_i - y_i| = n\,d_1(x,y)$.

\begin{problembox}[3.29: Bounded Metric]
If \( (M, d) \) is a metric space, define
\[d'(x, y) = \frac{d(x, y)}{1 + d(x, y)}.\]
Prove that \( d' \) is also a metric for \( M \). Note that \( 0 \leq d'(x, y) < 1 \) for all \( x, y \) in \( M \).
\end{problembox}

\textbf{Solution:} We need to verify the three properties of a metric:

(1) $d'(x,y) \geq 0$ since $d(x,y) \geq 0$ and $1 + d(x,y) > 0$.

(2) $d'(x,y) = 0$ if and only if $d(x,y) = 0$, which occurs if and only if $x = y$.

(3) $d'(x,y) = d'(y,x)$ since $d(x,y) = d(y,x)$.

(4) For the triangle inequality, let $f(t) = \frac{t}{1+t}$. Then $f'(t) = \frac{1}{(1+t)^2} > 0$, so $f$ is increasing. Therefore, $d'(x,z) = f(d(x,z)) \leq f(d(x,y) + d(y,z)) = \frac{d(x,y) + d(y,z)}{1 + d(x,y) + d(y,z)} \leq \frac{d(x,y)}{1 + d(x,y)} + \frac{d(y,z)}{1 + d(y,z)} = d'(x,y) + d'(y,z)$.

The last inequality follows from the fact that $\frac{a+b}{1+a+b} \leq \frac{a}{1+a} + \frac{b}{1+b}$ for $a,b \geq 0$.

\begin{problembox}[3.30: Finite Sets in Metric Spaces]
Prove that every finite subset of a metric space is closed.
\end{problembox}

\textbf{Solution:} Let $S = \{x_1, x_2, \ldots, x_n\}$ be a finite subset of a metric space $(M,d)$. We need to show that the complement $M \setminus S$ is open.

Let $x \in M \setminus S$. Let $\varepsilon = \min\{d(x,x_i) : i = 1,2,\ldots,n\}$. Since $x \notin S$, we have $\varepsilon > 0$.

Then $B(x;\varepsilon) \cap S = \emptyset$, so $B(x;\varepsilon) \subset M \setminus S$. This shows that every point in $M \setminus S$ is an interior point, so $M \setminus S$ is open.

Therefore, $S$ is closed.

\begin{problembox}[3.31: Closed Balls in Metric Spaces]
In a metric space \((M, d)\) the closed ball of radius \( r > 0 \) about a point \( a \) in \( M \) is the set \( \overline{B}(a; r) = \{x : d(x, a) \leq r\} \).
\begin{enumerate}[label=\alph*)]
\item Prove that \( \overline{B}(a; r) \) is a closed set.
\item Give an example of a metric space in which \( \overline{B}(a; r) \) is not the closure of the open ball \( B(a; r) \).
\end{enumerate}
\end{problembox}

\textbf{Solution:} 
(a) Let $x \in M \setminus \overline{B}(a;r)$. Then $d(x,a) > r$. Let $\varepsilon = d(x,a) - r > 0$. For any $y \in B(x;\varepsilon)$, we have $d(y,a) \geq d(x,a) - d(x,y) > d(x,a) - \varepsilon = r$. Therefore, $B(x;\varepsilon) \subset M \setminus \overline{B}(a;r)$, showing that $M \setminus \overline{B}(a;r)$ is open. Hence, $\overline{B}(a;r)$ is closed.

(b) Consider the discrete metric space $(M,d)$ where $d(x,y) = 1$ if $x \neq y$ and $d(x,y) = 0$ if $x = y$. Let $a \in M$ and $r = 1$. Then $B(a;1) = \{a\}$ and $\overline{B}(a;1) = M$. The closure of $B(a;1)$ is $\{a\}$, which is not equal to $\overline{B}(a;1) = M$.

\begin{problembox}[3.32: Transitivity of Density]
In a metric space \( M \), if subsets satisfy \( A \subseteq S \subseteq \overline{A} \), where \(\overline{A}\) is the closure of \( A \), then \( A \) is said to be dense in \( S \). For example, the set \( \mathbb{Q} \) of rational numbers is dense in \( \mathbb{R} \). If \( A \) is dense in \( S \) and if \( S \) is dense in \( T \), prove that \( A \) is dense in \( T \).
\end{problembox}

\textbf{Solution:} We need to show that $A \subseteq T \subseteq \overline{A}$.

Since $A \subseteq S \subseteq T$, we have $A \subseteq T$.

Since $S$ is dense in $T$, we have $T \subseteq \overline{S}$. Since $A$ is dense in $S$, we have $S \subseteq \overline{A}$. Therefore, $\overline{S} \subseteq \overline{\overline{A}} = \overline{A}$.

Combining these, we get $T \subseteq \overline{S} \subseteq \overline{A}$, so $T \subseteq \overline{A}$.

Therefore, $A \subseteq T \subseteq \overline{A}$, showing that $A$ is dense in $T$.

\begin{problembox}[3.33: Separability of Euclidean Spaces]
A metric space \( M \) is said to be separable if there is a countable subset \( A \) which is dense in \( M \). For example, \( \mathbb{R} \) is separable because the set \( \mathbb{Q} \) of rational numbers is a countable dense subset. Prove that every Euclidean space \( \mathbb{R}^k \) is separable.
\end{problembox}

\textbf{Solution:} Let $A$ be the set of all points in $\mathbb{R}^k$ with rational coordinates. That is, $A = \{(q_1, q_2, \ldots, q_k) : q_i \in \mathbb{Q} \text{ for } i = 1,2,\ldots,k\}$.

Since $\mathbb{Q}$ is countable, the Cartesian product $A = \mathbb{Q}^k$ is countable.

To show that $A$ is dense in $\mathbb{R}^k$, let $x = (x_1, x_2, \ldots, x_k) \in \mathbb{R}^k$ and $\varepsilon > 0$. Since $\mathbb{Q}$ is dense in $\mathbb{R}$, for each $i$ there exists $q_i \in \mathbb{Q}$ such that $|x_i - q_i| < \varepsilon/\sqrt{k}$.

Then $q = (q_1, q_2, \ldots, q_k) \in A$ and $\|x - q\| = \sqrt{\sum_{i=1}^k (x_i - q_i)^2} < \sqrt{k(\varepsilon/\sqrt{k})^2} = \varepsilon$.

Therefore, $A$ is a countable dense subset of $\mathbb{R}^k$, so $\mathbb{R}^k$ is separable.

\begin{problembox}[3.34: Lindelöf Theorem in Separable Spaces]
Prove that the Lindelöf covering theorem (Theorem 3.28) is valid in any separable metric space.
\end{problembox}

\textbf{Solution:} Let $M$ be a separable metric space with countable dense subset $D = \{d_1, d_2, \ldots\}$. Let $\mathcal{F}$ be an open covering of $M$.

For each $d_i \in D$ and each positive rational $r$, if there exists a set $F \in \mathcal{F}$ such that $B(d_i;r) \subset F$, let $F_{i,r}$ be one such set.

The collection $\{F_{i,r} : i \in \mathbb{N}, r \in \mathbb{Q}^+, B(d_i;r) \subset F_{i,r} \text{ for some } F \in \mathcal{F}\}$ is countable.

We claim this collection covers $M$. Let $x \in M$. Since $\mathcal{F}$ covers $M$, there exists $F \in \mathcal{F}$ such that $x \in F$. Since $F$ is open, there exists $\varepsilon > 0$ such that $B(x;\varepsilon) \subset F$.

Since $D$ is dense, there exists $d_i \in D$ such that $d_i \in B(x;\varepsilon/2)$. Let $r$ be a rational number such that $d(x,d_i) < r < \varepsilon/2$. Then $B(d_i;r) \subset B(x;\varepsilon) \subset F$.

Therefore, $F_{i,r}$ exists and contains $x$, showing that the countable subcollection covers $M$.

\begin{problembox}[3.35: Density and Open Sets]
If \( A \) is dense in \( S \) and if \( B \) is open in \( S \), prove that \( B \subseteq \overline{A \cap B} \).

\textit{Hint.} Exercise 3.13.
\end{problembox}

\textbf{Solution:} Let $x \in B$. Since $B$ is open in $S$, there exists $\varepsilon > 0$ such that $B(x;\varepsilon) \cap S \subset B$.

Since $A$ is dense in $S$, every neighborhood of $x$ contains a point of $A$. In particular, $B(x;\varepsilon) \cap A \neq \emptyset$.

Let $y \in B(x;\varepsilon) \cap A$. Since $y \in S$ and $B(x;\varepsilon) \cap S \subset B$, we have $y \in B$.

Therefore, $y \in A \cap B$, so $B(x;\varepsilon) \cap (A \cap B) \neq \emptyset$.

This shows that every neighborhood of $x$ contains a point of $A \cap B$, so $x \in \overline{A \cap B}$.

Therefore, $B \subseteq \overline{A \cap B}$.

\begin{problembox}[3.36: Intersection of Dense and Open Sets]
If each of \( A \) and \( B \) is dense in \( S \) and if \( B \) is open in \( S \), prove that \( A \cap B \) is dense in \( S \).
\end{problembox}

\textbf{Solution:} We need to show that $S \subseteq \overline{A \cap B}$.

Let $x \in S$. Since $B$ is open in $S$, there exists $\varepsilon > 0$ such that $B(x;\varepsilon) \cap S \subset B$.

Since $A$ is dense in $S$, $B(x;\varepsilon) \cap A \neq \emptyset$. Let $y \in B(x;\varepsilon) \cap A$. Since $y \in S$ and $B(x;\varepsilon) \cap S \subset B$, we have $y \in B$.

Therefore, $y \in A \cap B$, so $B(x;\varepsilon) \cap (A \cap B) \neq \emptyset$.

This shows that every neighborhood of $x$ contains a point of $A \cap B$, so $x \in \overline{A \cap B}$.

Therefore, $S \subseteq \overline{A \cap B}$, showing that $A \cap B$ is dense in $S$.

\begin{problembox}[3.37: Product Metrics]
Given two metric spaces \((S_1, d_1)\) and \((S_2, d_2)\), a metric \( \rho \) for the Cartesian product \( S_1 \times S_2 \) can be constructed from \( d_1 \) and \( d_2 \) in many ways. For example, if \( x = (x_1, x_2) \) and \( y = (y_1, y_2) \) are in \( S_1 \times S_2 \), let \( \rho(x, y) = d_1(x_1, y_1) + d_2(x_2, y_2) \). Prove that \( \rho \) is a metric for \( S_1 \times S_2 \) and construct further examples.
\end{problembox}

\textbf{Solution:} We need to verify the three properties of a metric for $\rho(x,y) = d_1(x_1,y_1) + d_2(x_2,y_2)$:

(1) $\rho(x,y) \geq 0$ since $d_1(x_1,y_1) \geq 0$ and $d_2(x_2,y_2) \geq 0$.

(2) $\rho(x,y) = 0$ if and only if $d_1(x_1,y_1) = 0$ and $d_2(x_2,y_2) = 0$, which occurs if and only if $x_1 = y_1$ and $x_2 = y_2$, i.e., $x = y$.

(3) $\rho(x,y) = \rho(y,x)$ since $d_1(x_1,y_1) = d_1(y_1,x_1)$ and $d_2(x_2,y_2) = d_2(y_2,x_2)$.

(4) For the triangle inequality, let $z = (z_1,z_2)$. Then $\rho(x,z) = d_1(x_1,z_1) + d_2(x_2,z_2) \leq d_1(x_1,y_1) + d_1(y_1,z_1) + d_2(x_2,y_2) + d_2(y_2,z_2) = \rho(x,y) + \rho(y,z)$.

Other examples of product metrics include:
- $\rho(x,y) = \max\{d_1(x_1,y_1), d_2(x_2,y_2)\}$
- $\rho(x,y) = \sqrt{d_1(x_1,y_1)^2 + d_2(x_2,y_2)^2}$
- $\rho(x,y) = (d_1(x_1,y_1)^p + d_2(x_2,y_2)^p)^{1/p}$ for $p \geq 1$

\begin{problembox}[3.38: Relative Compactness]
Assume \( S \subseteq T \subseteq M \). Then \( S \) is compact in \((M, d)\) if, and only if, \( S \) is compact in the metric subspace \((T, d)\).
\end{problembox}

\textbf{Solution:} Suppose $S$ is compact in $(M,d)$. Let $\mathcal{F}$ be an open covering of $S$ in the subspace $(T,d)$. Then each $F \in \mathcal{F}$ is of the form $F = U \cap T$ where $U$ is open in $(M,d)$.

The collection $\{U : U \text{ is open in } (M,d) \text{ and } U \cap T \in \mathcal{F}\}$ is an open covering of $S$ in $(M,d)$. Since $S$ is compact in $(M,d)$, there exists a finite subcollection $\{U_1, \ldots, U_n\}$ that covers $S$.

Then $\{U_1 \cap T, \ldots, U_n \cap T\}$ is a finite subcollection of $\mathcal{F}$ that covers $S$, showing that $S$ is compact in $(T,d)$.

Conversely, suppose $S$ is compact in $(T,d)$. Let $\mathcal{G}$ be an open covering of $S$ in $(M,d)$. Then $\{G \cap T : G \in \mathcal{G}\}$ is an open covering of $S$ in $(T,d)$. Since $S$ is compact in $(T,d)$, there exists a finite subcollection $\{G_1 \cap T, \ldots, G_n \cap T\}$ that covers $S$.

Then $\{G_1, \ldots, G_n\}$ is a finite subcollection of $\mathcal{G}$ that covers $S$, showing that $S$ is compact in $(M,d)$.

\begin{problembox}[3.39: Intersection with Compact Sets]
If \( S \) is closed and \( T \) is compact, then \( S \cap T \) is compact.
\end{problembox}

\textbf{Solution:} Since $T$ is compact, it is closed. Therefore, $S \cap T$ is the intersection of two closed sets, so it is closed.

Since $S \cap T \subseteq T$ and $T$ is compact, by Exercise 3.38, $S \cap T$ is compact in $(T,d)$. Since compactness is independent of the ambient space, $S \cap T$ is compact in $(M,d)$.

\begin{problembox}[3.40: Intersection of Compact Sets]
The intersection of an arbitrary collection of compact subsets of \( M \) is compact.
\end{problembox}

\textbf{Solution:} Let $\{K_\alpha\}$ be a collection of compact subsets of $M$. Since each $K_\alpha$ is closed, the intersection $\bigcap K_\alpha$ is closed.

Let $K_1$ be any member of the collection. Then $\bigcap K_\alpha \subseteq K_1$ and $K_1$ is compact. Since $\bigcap K_\alpha$ is closed and contained in a compact set, by Exercise 3.39, $\bigcap K_\alpha$ is compact.

\begin{problembox}[3.41: Finite Union of Compact Sets]
The union of a finite number of compact subsets of \( M \) is compact.
\end{problembox}

\textbf{Solution:} Let $K_1, K_2, \ldots, K_n$ be compact subsets of $M$. Since each $K_i$ is closed, their union $\bigcup_{i=1}^n K_i$ is closed.

Let $\mathcal{F}$ be an open covering of $\bigcup_{i=1}^n K_i$. Then $\mathcal{F}$ is also an open covering of each $K_i$. Since each $K_i$ is compact, there exists a finite subcollection $\mathcal{F}_i$ of $\mathcal{F}$ that covers $K_i$.

Then $\bigcup_{i=1}^n \mathcal{F}_i$ is a finite subcollection of $\mathcal{F}$ that covers $\bigcup_{i=1}^n K_i$.

Since $\bigcup_{i=1}^n K_i$ is closed and every open covering has a finite subcover, it is compact.

\begin{problembox}[3.42: Non-Compact Closed and Bounded Set]
Consider the metric space \( \mathbb{Q} \) of rational numbers with the Euclidean metric of \( \mathbb{R} \). Let \( S \) consist of all rational numbers in the open interval \((a, b)\), where \( a \) and \( b \) are irrational. Then \( S \) is a closed and bounded subset of \( \mathbb{Q} \) which is not compact.
\end{problembox}

\textbf{Solution:} Let $S = \mathbb{Q} \cap (a,b)$ where $a, b$ are irrational numbers.

$S$ is bounded since it is contained in the bounded interval $(a,b)$.

$S$ is closed in $\mathbb{Q}$ because its complement $\mathbb{Q} \setminus S = \mathbb{Q} \cap ((-\infty,a] \cup [b,\infty))$ is open in $\mathbb{Q}$.

However, $S$ is not compact. Let $\{q_n\}$ be a sequence of rational numbers in $(a,b)$ that converges to $a$ (which exists since $\mathbb{Q}$ is dense in $\mathbb{R}$). Then $\{q_n\}$ is a sequence in $S$ that has no convergent subsequence in $S$ (since $a \notin S$).

Therefore, $S$ is closed and bounded but not compact.

\begin{problembox}[Miscellaneous Properties of Interior and Boundary]
The following problems involve arbitrary subsets \( A \) and \( B \) of a metric space \( M \).
\end{problembox}

\begin{problembox}[3.43: Interior via Closure]
Prove that \(\text{int } A = M - \overline{M - A}\).
\end{problembox}

\textbf{Solution:} Let $x \in \text{int } A$. Then there exists $\varepsilon > 0$ such that $B(x;\varepsilon) \subset A$. This means $B(x;\varepsilon) \cap (M - A) = \emptyset$, so $x \notin \overline{M - A}$. Therefore, $x \in M - \overline{M - A}$.

Conversely, let $x \in M - \overline{M - A}$. Then $x \notin \overline{M - A}$, so there exists $\varepsilon > 0$ such that $B(x;\varepsilon) \cap (M - A) = \emptyset$. This means $B(x;\varepsilon) \subset A$, so $x \in \text{int } A$.

\begin{problembox}[3.44: Interior of Complement]
Prove that \(\text{int }(M - A) = M - \overline{A}\).
\end{problembox}

\textbf{Solution:} Let $x \in \text{int }(M - A)$. Then there exists $\varepsilon > 0$ such that $B(x;\varepsilon) \subset M - A$. This means $B(x;\varepsilon) \cap A = \emptyset$, so $x \notin \overline{A}$. Therefore, $x \in M - \overline{A}$.

Conversely, let $x \in M - \overline{A}$. Then $x \notin \overline{A}$, so there exists $\varepsilon > 0$ such that $B(x;\varepsilon) \cap A = \emptyset$. This means $B(x;\varepsilon) \subset M - A$, so $x \in \text{int }(M - A)$.

\begin{problembox}[3.45: Idempotence of Interior]
Prove that \(\text{int }(\text{int } A) = \text{int } A\).
\end{problembox}

\textbf{Solution:} Since $\text{int } A \subseteq A$, we have $\text{int }(\text{int } A) \subseteq \text{int } A$.

For the reverse inclusion, let $x \in \text{int } A$. Then there exists $\varepsilon > 0$ such that $B(x;\varepsilon) \subset A$. Since $B(x;\varepsilon)$ is open and contained in $A$, we have $B(x;\varepsilon) \subset \text{int } A$. Therefore, $x \in \text{int }(\text{int } A)$.

\begin{problembox}[3.46: Interior of Intersections]
\begin{enumerate}[label=\alph*)]
\item Prove that \(\text{int } \left(\bigcap_{i=1}^n A_i\right) = \bigcap_{i=1}^n (\text{int } A_i)\), where each \( A_i \subseteq M \).
\item Show that \(\text{int } \left(\bigcap_{A \in F} A\right) \subseteq \bigcap_{A \in F} (\text{int } A)\) if \( F \) is an infinite collection of subsets of \( M \).
\item Give an example where equality does not hold in (b).
\end{enumerate}
\end{problembox}

\textbf{Solution:} 
(a) Let $x \in \text{int }(\bigcap_{i=1}^n A_i)$. Then there exists $\varepsilon > 0$ such that $B(x;\varepsilon) \subset \bigcap_{i=1}^n A_i$. This means $B(x;\varepsilon) \subset A_i$ for each $i$, so $x \in \text{int } A_i$ for each $i$. Therefore, $x \in \bigcap_{i=1}^n (\text{int } A_i)$.

Conversely, let $x \in \bigcap_{i=1}^n (\text{int } A_i)$. Then for each $i$, there exists $\varepsilon_i > 0$ such that $B(x;\varepsilon_i) \subset A_i$. Let $\varepsilon = \min\{\varepsilon_1, \ldots, \varepsilon_n\}$. Then $B(x;\varepsilon) \subset \bigcap_{i=1}^n A_i$, so $x \in \text{int }(\bigcap_{i=1}^n A_i)$.

(b) Let $x \in \text{int }(\bigcap_{A \in F} A)$. Then there exists $\varepsilon > 0$ such that $B(x;\varepsilon) \subset \bigcap_{A \in F} A$. This means $B(x;\varepsilon) \subset A$ for each $A \in F$, so $x \in \text{int } A$ for each $A \in F$. Therefore, $x \in \bigcap_{A \in F} (\text{int } A)$.

(c) Let $F = \{A_n : n \in \mathbb{N}\}$ where $A_n = (-1/n, 1/n)$. Then $\bigcap_{A \in F} A = \{0\}$, so $\text{int }(\bigcap_{A \in F} A) = \emptyset$. However, $\text{int } A_n = A_n$ for each $n$, so $\bigcap_{A \in F} (\text{int } A) = \bigcap_{n=1}^{\infty} A_n = \{0\}$.

\begin{problembox}[3.47: Interior of Unions]
\begin{enumerate}[label=\alph*)]
\item Prove that \(\bigcup_{A \in F} (\text{int } A) \subseteq \text{int } \left(\bigcup_{A \in F} A\right)\).
\item Give an example of a finite collection \( F \) in which equality does not hold in (a).
\end{enumerate}
\end{problembox}

\textbf{Solution:} 
(a) Let $x \in \bigcup_{A \in F} (\text{int } A)$. Then $x \in \text{int } A$ for some $A \in F$. There exists $\varepsilon > 0$ such that $B(x;\varepsilon) \subset A \subset \bigcup_{A \in F} A$. Therefore, $x \in \text{int }(\bigcup_{A \in F} A)$.

(b) Let $F = \{A, B\}$ where $A = [0,1]$ and $B = [1,2]$. Then $\text{int } A = (0,1)$ and $\text{int } B = (1,2)$, so $\bigcup_{A \in F} (\text{int } A) = (0,1) \cup (1,2)$. However, $\bigcup_{A \in F} A = [0,2]$, so $\text{int }(\bigcup_{A \in F} A) = (0,2)$, which properly contains $(0,1) \cup (1,2)$.

\begin{problembox}[3.48: Interior of Boundary]
\begin{enumerate}[label=\alph*)]
\item Prove that \(\text{int } (\partial A) = \emptyset\) if \( A \) is open or if \( A \) is closed in \( M \).
\item Give an example in which \(\text{int } (\partial A) = M\).
\end{enumerate}
\end{problembox}

\textbf{Solution:} 
(a) If $A$ is open, then $\partial A = \overline{A} \setminus \text{int } A = \overline{A} \setminus A$. If $A$ is closed, then $\partial A = A \setminus \text{int } A$.

In both cases, $\partial A$ contains no open balls, so $\text{int } (\partial A) = \emptyset$.

(b) Let $A = \mathbb{Q}$ in the metric space $\mathbb{R}$. Then $\partial A = \mathbb{R}$, so $\text{int } (\partial A) = \mathbb{R} = M$.

\begin{problembox}[3.49: Interior of Union of Sets with Empty Interior]
If \(\text{int } A = \text{int } B = \emptyset\) and if \(A\) is closed in \(M\), then \(\text{int } (A \cup B) = \emptyset\).
\end{problembox}

\textbf{Solution:} Since $A$ is closed, $\text{int } A = \emptyset$ implies that $A$ has no isolated points. Therefore, every point in $A$ is a limit point of $A$.

Let $x \in A \cup B$. If $x \in A$, then every neighborhood of $x$ contains points of $A$ different from $x$. Since $A \subset A \cup B$, every neighborhood of $x$ contains points of $A \cup B$ different from $x$, so $x$ is not an interior point of $A \cup B$.

If $x \in B \setminus A$, then since $\text{int } B = \emptyset$, every neighborhood of $x$ contains points not in $B$. Since $A$ is closed and $x \notin A$, there exists a neighborhood of $x$ that doesn't intersect $A$. This neighborhood contains points not in $A \cup B$, so $x$ is not an interior point of $A \cup B$.

Therefore, $\text{int } (A \cup B) = \emptyset$.

\begin{problembox}[3.50: Counterexample for Union of Sets with Empty Interior]
Give an example in which \(\text{int } A = \text{int } B = \emptyset\) but \(\text{int } (A \cup B) = M\).
\end{problembox}

\textbf{Solution:} Let $A = \mathbb{Q}$ and $B = \mathbb{R} \setminus \mathbb{Q}$ in the metric space $\mathbb{R}$. Then $\text{int } A = \emptyset$ and $\text{int } B = \emptyset$, but $A \cup B = \mathbb{R}$, so $\text{int } (A \cup B) = \mathbb{R} = M$.

\begin{problembox}[3.51: Properties of Boundary]
Prove that:
\[
\partial A = \overline{A} \cap \overline{M - A} \quad \text{and} \quad \partial A = \partial(M - A).
\]
\end{problembox}

\textbf{Solution:} For the first equality, $x \in \partial A$ if and only if every neighborhood of $x$ contains both points of $A$ and points of $M - A$. This means $x \in \overline{A}$ and $x \in \overline{M - A}$, so $x \in \overline{A} \cap \overline{M - A}$.

For the second equality, $\partial A = \overline{A} \cap \overline{M - A} = \overline{M - A} \cap \overline{A} = \partial(M - A)$.

\begin{problembox}[3.52: Boundary of Union under Disjoint Closures]
If \(\overline{A} \cap \overline{B} = \emptyset\), then \(\partial(A \cup B) = \partial A \cup \partial B\).
\end{problembox}

\textbf{Solution:} Since $\overline{A} \cap \overline{B} = \emptyset$, we have $\overline{A \cup B} = \overline{A} \cup \overline{B}$.

Let $x \in \partial(A \cup B)$. Then $x \in \overline{A \cup B} = \overline{A} \cup \overline{B}$ and $x \in \overline{M - (A \cup B)} = \overline{(M - A) \cap (M - B)} \subseteq \overline{M - A} \cap \overline{M - B}$.

If $x \in \overline{A}$, then $x \in \overline{A} \cap \overline{M - A} = \partial A$. If $x \in \overline{B}$, then $x \in \overline{B} \cap \overline{M - B} = \partial B$. Therefore, $x \in \partial A \cup \partial B$.

Conversely, let $x \in \partial A \cup \partial B$. Without loss of generality, assume $x \in \partial A$. Then $x \in \overline{A} \subseteq \overline{A \cup B}$ and $x \in \overline{M - A} \subseteq \overline{M - (A \cup B)}$. Therefore, $x \in \partial(A \cup B)$.