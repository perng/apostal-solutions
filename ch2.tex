\chapter{Some Basic Notions of Set Theory}

\begin{problembox}[2.1: Equality of Ordered Pairs]
Prove Theorem 2.2. Hint: $(a, b) = (c, d)$ means $\{\{a\}, \{a, b\}\} = \{\{c\}, \{c, d\}\}$. Now appeal to the definition of set equality.
\end{problembox}

\textbf{Solution:}  
Using the Kuratowski definition of ordered pairs:  
\[
(a, b) = \{\{a\}, \{a, b\}\}, \quad (c, d) = \{\{c\}, \{c, d\}\}
\]  
If these sets are equal, then their elements must match.  
From the structure, it must be that $a = c$ and $b = d$. Hence the ordered pairs are equal if and only if their components are equal.


\begin{problembox}[2.2: Properties of Relations]
Determine which of the following relations $S$ on $\mathbb{R}^2$ are reflexive, symmetric, and transitive:
\begin{enumerate}[label=(\alph*)]
\item $S = \{(x,y) \in \mathbb{R}^2 : x \leq y\}$
\item $S = \{(x,y) \in \mathbb{R}^2 : x < y\}$
\item $S = \{(x,y) \in \mathbb{R}^2 : x > y\}$
\item $S = \{(x,y) \in \mathbb{R}^2 : x^2 + y^2 \geq 1\}$
\item $S = \{(x,y) \in \mathbb{R}^2 : x^2 + y^2 < 0\}$
\item $S = \{(x,y) \in \mathbb{R}^2 : x^2 + x \leq y^2 + y\}$
\end{enumerate}
\end{problembox}

\textbf{Solution:}  
\begin{enumerate}[label=(\alph*)]
\item \textbf{Reflexive:} Yes - for all $x \in \mathbb{R}$, we have $x \leq x$ \\
\textbf{Symmetric:} No - if $x \leq y$ and $x \neq y$, then $y \not\leq x$ \\
\textbf{Transitive:} Yes - if $x \leq y$ and $y \leq z$, then $x \leq z$

\item \textbf{Reflexive:} No - $x < x$ is never true \\
\textbf{Symmetric:} No - if $x < y$, then $y \not< x$ \\
\textbf{Transitive:} Yes - if $x < y$ and $y < z$, then $x < z$

\item \textbf{Reflexive:} No - $x > x$ is never true \\
\textbf{Symmetric:} No - if $x > y$, then $y \not> x$ \\
\textbf{Transitive:} Yes - if $x > y$ and $y > z$, then $x > z$

\item \textbf{Reflexive:} Depends on values - $(x,x) \in S$ if and only if $2x^2 \geq 1$ \\
\textbf{Symmetric:} Yes - if $x^2 + y^2 \geq 1$, then $y^2 + x^2 \geq 1$ \\
\textbf{Transitive:} No - counterexample: $(1,0) \in S$, $(0,1) \in S$, but $(1,1) \notin S$

\item \textbf{Reflexive:} No - $x^2 + x^2 = 2x^2 \geq 0$ for all $x \in \mathbb{R}$ \\
\textbf{Symmetric:} Yes - if $x^2 + y^2 < 0$, then $y^2 + x^2 < 0$ \\
\textbf{Transitive:} Vacuously true - the relation is empty

\item \textbf{Reflexive:} Yes - for all $x \in \mathbb{R}$, $x^2 + x \leq x^2 + x$ \\
\textbf{Symmetric:} No - if $x^2 + x \leq y^2 + y$ and $x \neq y$, then $y^2 + y \not\leq x^2 + x$ \\
\textbf{Transitive:} Yes - if $x^2 + x \leq y^2 + y$ and $y^2 + y \leq z^2 + z$, then $x^2 + x \leq z^2 + z$
\end{enumerate}

\begin{problembox}[2.3: Composition and Inversion of Functions]
The following functions \( F \) and \( G \) are defined for all real \( x \) by the equations given below.  
\textbf{Part 1.} In each case where the composite function \( G \circ F \) can be formed, give the domain of \( G \circ F \) and a formula (or formulas) for \( (G \circ F)(x) \):  
\begin{itemize}
\item[(a)] \( F(x) = 1 - x \), \quad \( G(x) = x^2 + 2x \)
\item[(b)] \( F(x) = x + 5 \), \quad \( G(x) = \frac{|x|}{x} \), \( G(0) = 1 \)
\item[(c)] \( F(x) = \begin{cases} 2x, & 0 \le x \le 1 \\ 11, & \text{otherwise} \end{cases} \),  
\( G(x) = \begin{cases} x^2, & 0 \le x \le 1 \\ 10, & \text{otherwise} \end{cases} \)
\end{itemize}
\textbf{Part 2.} In the following, find \( F(x) \) if \( G(x) \) and \( G[F(x)] \) are given:  
\begin{itemize}
\item[(d)] \( G(x) = x^3 \), \quad \( G[F(x)] = x^3 - 3x^2 + 3x - 1 \)
\item[(e)] \( G(x) = 3 + x + x^2 \), \quad \( G[F(x)] = x^2 - 3x + 5 \)
\end{itemize}
\end{problembox}

\textbf{Solution:}

\textbf{(a)}  
\( G \circ F(x) = G(1 - x) = (1 - x)^2 + 2(1 - x) = x^2 - 4x + 3 \)  
Domain: \( \mathbb{R} \)

\textbf{(b)}  
\( G \circ F(x) = \frac{|x + 5|}{x + 5} \), undefined when \( x = -5 \)  
Domain: \( \mathbb{R} \setminus \{-5\} \)

\textbf{(c)}  
For \( x \in [0, 1] \), \( F(x) = 2x \in [0,2] \).  
But \( G(x) = x^2 \) is defined only on \( [0, 1] \), otherwise \( G(x) = 10 \).  
So:
- If \( x \in [0, 0.5] \), then \( F(x) \in [0,1] \Rightarrow G(F(x)) = (2x)^2 = 4x^2 \)
- If \( x \in (0.5, 1] \), then \( F(x) > 1 \Rightarrow G(F(x)) = 10 \)
- If \( x \notin [0,1] \), then \( F(x) = 11 \Rightarrow G(F(x)) = 10 \)

Thus:
\[
(G \circ F)(x) = 
\begin{cases}
4x^2, & 0 \le x \le 0.5 \\
10, & x > 0.5 \text{ or } x < 0
\end{cases}
\]
Domain: \( \mathbb{R} \)

\textbf{(d)}  
We are given: \( G(x) = x^3 \), \( G[F(x)] = x^3 - 3x^2 + 3x - 1 \)  
Note: \( x^3 - 3x^2 + 3x - 1 = (x - 1)^3 \)  
So: \( F(x) = x - 1 \)

\textbf{(e)}  
We are given: \( G(x) = 3 + x + x^2 \), \( G[F(x)] = x^2 - 3x + 5 \)  
We solve:
\[
3 + F(x) + F(x)^2 = x^2 - 3x + 5 \\
\Rightarrow F(x)^2 + F(x) = x^2 - 3x + 2 \\
\Rightarrow F(x)^2 + F(x) - (x^2 - 3x + 2) = 0
\]
This is a quadratic in \( F(x) \):  
\[
F(x)^2 + F(x) - x^2 + 3x - 2 = 0
\Rightarrow F(x)^2 + F(x) + (3x - x^2 - 2) = 0
\]
Solve this quadratic for \( F(x) \); it's context-dependent. No unique closed form without more constraints, so \( F(x) \) is defined implicitly.

\begin{problembox}[2.4: Associativity of Function Composition]
Given three functions \( F, G, H \), what restrictions must be placed on their domains so that the following four composite functions can be defined?
\[
G \circ F, \quad H \circ G, \quad H \circ (G \circ F), \quad (H \circ G) \circ F
\]
Assuming that \( H \circ (G \circ F) \) and \( (H \circ G) \circ F \) can be defined, prove the associative law:
\[
H \circ (G \circ F) = (H \circ G) \circ F
\]
\end{problembox}

\textbf{Solution:}  
To define \( G \circ F \), the range of \( F \) must be contained in the domain of \( G \).  
To define \( H \circ G \), the range of \( G \) must be contained in the domain of \( H \).  
Under these conditions,  
\[
(H \circ (G \circ F))(x) = H(G(F(x))) = ((H \circ G) \circ F)(x)
\]  
So function composition is associative wherever defined.

\begin{problembox}[2.5: Set-Theoretic Identities]
Prove the following set-theoretic identities:
\begin{itemize}
\item[(a)] \( A \cup (B \cup C) = (A \cup B) \cup C \), \quad \( A \cap (B \cap C) = (A \cap B) \cap C \)
\item[(b)] \( A \cap (B \cup C) = (A \cap B) \cup (A \cap C) \)
\item[(c)] \( (A \cap B) \cup (A \cap C) = A \cap (B \cup C) \)
\item[(d)] \( (A \cup B)(B \cup C)(C \cup A) = (A \cap B) \cup (A \cap C) \cup (B \cap C) \)
\item[(e)] \( A \cap (B - C) = (A \cap B) - (A \cap C) \)
\item[(f)] \( (A - C) \cap (B - C) = (A \cap B) - C \)
\item[(g)] \( (A - B) \cup B = A \) if and only if \( B \subseteq A \)
\end{itemize}
\end{problembox}

\textbf{Solution:}  
Each identity can be proven by element-chasing: assuming \( x \in \) one side and showing \( x \in \) the other side.  
For example, for (b), if \( x \in A \cap (B \cup C) \), then \( x \in A \) and \( x \in B \cup C \Rightarrow x \in (A \cap B) \cup (A \cap C) \). Similar for the reverse.

\begin{problembox}[2.6: Image of Unions and Intersections]
Let \( f: S \to T \) be a function. If \( A \) and \( B \subseteq S \), prove:
\[
f(A \cup B) = f(A) \cup f(B), \quad f(A \cap B) \subseteq f(A) \cap f(B)
\]
Generalize to arbitrary unions and intersections.
\end{problembox}

\textbf{Solution:}  
For any \( x \in A \cup B \), \( f(x) \in f(A) \cup f(B) \).  
For intersections, \( x \in A \cap B \Rightarrow f(x) \in f(A) \cap f(B) \), but the converse need not hold.  
Generalization:  
\[
f\left( \bigcup_i A_i \right) = \bigcup_i f(A_i), \quad f\left( \bigcap_i A_i \right) \subseteq \bigcap_i f(A_i)
\]

\begin{problembox}[2.7: Inverse Image Laws]
Let \( f: S \to T \), and for any \( Y \subseteq T \), define the inverse image:
\[
f^{-1}(Y) = \{x \in S \mid f(x) \in Y \}
\]
Prove:
\begin{itemize}
\item[(a)] \( f^{-1}(Y_1 \cup Y_2) = f^{-1}(Y_1) \cup f^{-1}(Y_2) \)
\item[(b)] \( f^{-1}(T - Y) = S - f^{-1}(Y) \)
\end{itemize}
Generalize to arbitrary unions and intersections.
\end{problembox}

\textbf{Solution:}  
(a) If \( x \in f^{-1}(Y_1 \cup Y_2) \), then \( f(x) \in Y_1 \cup Y_2 \Rightarrow x \in f^{-1}(Y_1) \cup f^{-1}(Y_2) \), and vice versa.  
(b) \( f(x) \notin Y \iff x \notin f^{-1}(Y) \Rightarrow x \in S - f^{-1}(Y) \)

\begin{problembox}[2.8: Image of Preimage and Surjectivity]
Prove that \( f[f^{-1}(Y)] = Y \) for every \( Y \subseteq T \) if and only if \( f \) is surjective.
\end{problembox}

\textbf{Solution:}  
If \( f \) is surjective, every \( y \in Y \) has a preimage in \( S \), so is included in \( f[f^{-1}(Y)] \).  
If \( f \) is not surjective, then some \( y \notin f(S) \), and so not in the image of any preimage — thus excluded from \( f[f^{-1}(Y)] \).

\begin{problembox}[2.9: Equivalent Conditions for Injectivity]
Let \( f: S \to T \) be a function. Show the following are equivalent:
\begin{itemize}
\item[(a)] \( f \) is injective
\item[(b)] \( f(A \cap B) = f(A) \cap f(B) \) for all \( A, B \subseteq S \)
\item[(c)] \( f^{-1}[f(A)] = A \) for all \( A \subseteq S \)
\item[(d)] For disjoint sets \( A, B \subseteq S \), \( f(A) \cap f(B) = \emptyset \)
\item[(e)] If \( B \subseteq A \), then \( f(A - B) = f(A) - f(B) \)
\end{itemize}
\end{problembox}

\textbf{Solution:}  
Each condition implies the others under the assumption that \( f(x_1) = f(x_2) \Rightarrow x_1 = x_2 \).  
E.g., (c) implies \( f^{-1}[f(\{x\})] = \{x\} \Rightarrow \) only one \( x \) maps to any \( f(x) \).

\begin{problembox}[2.10: Subset Transitivity]
Prove that if \( A \subseteq B \) and \( B \subseteq C \), then \( A \subseteq C \).
\end{problembox}

\textbf{Solution:}  
If \( x \in A \), then since \( A \subseteq B \), we have \( x \in B \), and since \( B \subseteq C \), we get \( x \in C \).  
Thus, every element of \( A \) is in \( C \), so \( A \subseteq C \).

\begin{problembox}[2.11: Finite Set Bijection Implies Equal Size]
If \( \{1, 2, \ldots, n\} \sim \{1, 2, \ldots, m\} \), prove that \( m = n \).
\end{problembox}

\textbf{Solution:}  
A bijection between two finite sets implies they have the same number of elements.  
So if such a bijection exists, then \( \#\{1, \ldots, n\} = n = m = \#\{1, \ldots, m\} \), hence \( n = m \).

\begin{problembox}[2.12: Infinite Sets Contain Countable Subsets]
If \( S \) is an infinite set, prove that \( S \) contains a countably infinite subset.
\end{problembox}

\textbf{Solution:}  
We can construct an injection from \( \mathbb{N} \) into \( S \):  
Select \( a_1 \in S \), then pick \( a_2 \in S \setminus \{a_1\} \), then \( a_3 \in S \setminus \{a_1, a_2\} \), and so on.  
Since \( S \) is infinite, this process never terminates. Thus, \( \{a_1, a_2, \ldots\} \subseteq S \) is countably infinite.

\begin{problembox}[2.13: Infinite Set Similar to a Proper Subset]
Prove that every infinite set \( S \) contains a proper subset similar (i.e., bijective) to \( S \) itself.
\end{problembox}

\textbf{Solution:}  
Let \( T \subset S \) be a countably infinite subset (from Exercise 2.12). Since \( S \) is infinite, there exists a bijection \( f: S \to T \cup \{s_0\} \subset S \).  
Then \( f \) is a bijection from \( S \) onto a proper subset \( T' \subset S \), so \( S \sim T' \).

\begin{problembox}[2.14: Removing Countable from Uncountable]
If \( A \) is a countable set and \( B \) an uncountable set, prove that \( B - A \sim B \).
\end{problembox}

\textbf{Solution:}  
Since \( A \) is countable and \( B \) is uncountable, \( B - A \) is uncountable.  
Also, \( A \cup (B - A) = B \). Define a bijection \( f \) from \( B \) to \( B - A \cup \{a_0\} \subset B \) by remapping countably many points.  
Thus, \( B \sim B - A \).

\begin{problembox}[2.15: Algebraic Numbers are Countable]
A real number is called \emph{algebraic} if it is a root of a polynomial with integer coefficients.  
Prove that the set of all polynomials with integer coefficients is countable, and deduce that the set of algebraic numbers is also countable.
\end{problembox}

\textbf{Solution:}  
Each polynomial can be represented by a finite tuple of integers (its coefficients). The set of finite sequences of integers is countable (a countable union of countable sets).  
Each polynomial has finitely many roots, so the set of all algebraic numbers is a countable union of finite sets → countable.


\begin{problembox}[2.16: Power Set of Finite Set]
Let \( S \) be a finite set with \( n \) elements, and let \( T \) be the collection of all subsets of \( S \).  
Show that \( T \) is finite, and determine how many elements it contains.
\end{problembox}

\textbf{Solution:}  
Each element of \( S \) may either be in or not in a subset.  
So the number of subsets is \( 2^n \). Hence \( \#T = 2^n \), and \( T \) is finite.

\begin{problembox}[2.17: Real Functions vs Real Numbers]
Let \( R \) be the set of real numbers and \( S \) the set of all real-valued functions with domain \( R \).  
Show that \( S \) and \( R \) are not equinumerous.
\end{problembox}

\textbf{Solution:}  
Assume toward contradiction that there is a bijection \( f: R \to S \).  
Define a function \( h(x) = f(x)(x) + 1 \). Then \( h \in S \), but there is no \( x \in R \) such that \( f(x) = h \), since \( f(x)(x) \ne h(x) \).  
Contradiction → no such bijection. Thus, \( S \) has strictly greater cardinality than \( R \).

\begin{problembox}[2.18: Binary Sequences are Uncountable]
Let \( S \) be the set of all infinite sequences of 0s and 1s. Show that \( S \) is uncountable.
\end{problembox}

\textbf{Solution:}  
Use Cantor's diagonal argument: assume \( S \) is countable and list all sequences.  
Construct a new sequence differing from the \( n \)-th sequence at the \( n \)-th place.  
This sequence is not in the list — contradiction. So \( S \) is uncountable.

\begin{problembox}[2.19: Countability of Specific Sets]
Show that the following sets are countable:
\begin{itemize}
\item[(a)] Circles in the complex plane with rational radii and centers with rational coordinates.
\item[(b)] Any collection of disjoint intervals of positive length.
\end{itemize}
\end{problembox}

\textbf{Solution:}  
(a) Each circle is determined by a rational radius and two rational coordinates → set is countable.  
(b) Each disjoint interval must contain a distinct rational number → inject into \( \mathbb{Q} \), which is countable.

\begin{problembox}[2.20: Countable Support for Real Function]
Let \( f \) be a real-valued function on \( [0,1] \). Suppose there exists \( M > 0 \) such that for any finite set of points \( \{x_1, \dots, x_n\} \subset [0,1] \),  
\[
|f(x_1)| + \dots + |f(x_n)| \le M
\]  
Let \( S = \{x \in [0,1] \mid f(x) \ne 0\} \). Prove that \( S \) is countable.
\end{problembox}

\textbf{Solution:}  
Suppose \( S \) were uncountable. Then there would be infinitely many disjoint points where \( |f(x)| > \varepsilon \), summing to a total > \( M \), violating the hypothesis.  
So \( S \) must be countable.

\begin{problembox}[2.21: Fallacy in Countability of Intervals]
Find the fallacy in the following "proof" that the set of all intervals of positive length is countable:  
Let \( \{x_1, x_2, \ldots\} \) be the rationals. Every interval contains a rational \( x_n \) with minimal index \( n \).  
Assign to the interval the smallest such \( n \). This gives a function from intervals to \( \mathbb{N} \), so the set of intervals is countable.
\end{problembox}

\textbf{Solution:}  
The function \( F \) is not injective — many intervals may have the same smallest-index rational.  
So this does not establish a one-to-one correspondence between intervals and \( \mathbb{N} \).  
Hence, the proof is invalid.

\begin{problembox}[2.22: Additive Set Functions]
Let \( S \) be the collection of all subsets of a given set \( T \).  
A function \( f: S \to \mathbb{R} \) is additive if:
\[
f(A \cup B) = f(A) + f(B)
\quad \text{whenever } A \cap B = \emptyset
\]  
Prove:  
\[
f(A \cup B) = f(A) + f(B - A), \quad 
f(A \cup B) = f(A) + f(B) - f(A \cap B)
\]
\end{problembox}

\textbf{Solution:}  
From additivity and set identity \( B = (B - A) \cup (A \cap B) \):  
\[
f(A \cup B) = f(A \cup (B - A)) = f(A) + f(B - A)
\]  
Also:  
\[
f(A \cup B) = f((A - B) \cup (A \cap B) \cup (B - A)) = f(A) + f(B) - f(A \cap B)
\]
\begin{problembox}[2.23: Solving for Total Measure from Functional Equations]
Refer to Exercise 2.22. Assume \( f \) is additive and also that the following relations hold for two particular subsets \( A \) and \( B \) of a set \( T \):
\[
f(A \cup B) = f(A) + f(B) - f(A)f(B), \quad f(A^c \cap B^c) = f(A)f(B),
\]
and assume that \( f(A) + f(B) \ne f(T) \).  
Prove that these relations determine \( f(T) \), and compute the value of \( f(T) \).
\end{problembox}

\textbf{Solution:}  

We are given:
- \( f(A \cup B) = f(A) + f(B) - f(A)f(B) \)
- \( f(A^c \cap B^c) = f(A)f(B) \)
- \( f \) is additive: \( f(X \cup Y) = f(X) + f(Y) \) when \( X \cap Y = \emptyset \)

Note that:
\[
A^c \cap B^c = (A \cup B)^c \Rightarrow T = (A \cup B) \cup (A^c \cap B^c), \quad \text{and these sets are disjoint}.
\]

Using the additivity of \( f \), we get:
\[
f(T) = f(A \cup B) + f(A^c \cap B^c)
\]

Substitute the given values:
\[
f(T) = \big(f(A) + f(B) - f(A)f(B)\big) + f(A)f(B)
\]

Simplify:
\[
f(T) = f(A) + f(B)
\]

Thus, despite the more complex expressions involving products, the total measure \( f(T) \) ends up being the simple sum:
\[
\boxed{f(T) = f(A) + f(B)}
\]
