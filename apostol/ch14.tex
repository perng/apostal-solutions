\chapter{Multiple Riemann Integrals}
\section{Multiple Integrals}
\noindent\textbf{Definitions and Theorems needed.}
\begin{enumerate}[label=(\roman*)]
    \item Riemann integral on a rectangle: upper and lower sums, oscillation on subrectangles, and the integrability criterion via vanishing total oscillation.
    \item Product formula for separable integrands: if $f \in R[a,b]$ and $g \in R[c,d]$, then $\int_c^d \! \int_a^b f(x)g(y)\,dx\,dy = \left(\int_a^b f\right)\!\left(\int_c^d g\right)$.
    \item If a bounded function on a rectangle is monotone in each variable separately, then it is Riemann integrable on that rectangle.
    \item For continuous (or piecewise continuous) integrands, iterated integrals agree with the double integral (Fubini for Riemann), and symmetry/geometric decompositions may be used to evaluate integrals.
\end{enumerate}



\begin{problembox}[14.1: Product of Riemann Integrable Functions]
If \( f_1 \in R \) on \([a_1, b_1], \ldots, f_n \in R \) on \([a_n, b_n]\), prove that
\[ \int_{S} f_1(x_1) \cdots f_n(x_n) \, d(x_1, \ldots, x_n) = \left( \int_{a_1}^{b_1} f_1(x_1) \, dx_1 \right) \cdots \left( \int_{a_n}^{b_n} f_n(x_n) \, dx_n \right), \]
where \( S = [a_1, b_1] \times \cdots \times [a_n, b_n] \).
\end{problembox}

\noindent\textbf{Solution:}
For $n=2$, write $g(x,y)=f_1(x)f_2(y)$. Since $f_1\in R[a_1,b_1]$ and $f_2\in R[a_2,b_2]$, for any $\varepsilon>0$ there are partitions $\mathcal P_1,\mathcal P_2$ such that $U(f_i,\mathcal P_i)-L(f_i,\mathcal P_i)<\varepsilon$ $(i=1,2)$. For the product partition $\mathcal P=\mathcal P_1\times\mathcal P_2$ on $S$, the oscillation of $g$ on each rectangle factors, and one obtains
\[ U(g,\mathcal P)-L(g,\mathcal P) \le (U(f_1,\mathcal P_1)-L(f_1,\mathcal P_1))\,\int f_2 + (U(f_2,\mathcal P_2)-L(f_2,\mathcal P_2))\,\int f_1 + O(\varepsilon^2), \]
which can be made arbitrarily small. Hence $g\in R$ and
\[ \iint_S f_1(x)f_2(y)\,d(x,y) = \int_{a_1}^{b_1} f_1(x)\,dx\,\int_{a_2}^{b_2} f_2(y)\,dy. \]
The case $n>2$ follows by induction by grouping variables two at a time and applying the $n=2$ case repeatedly.\qed


\begin{problembox}[14.2: Riemann Integrability of Monotone Functions]
Let \( f \) be defined and bounded on a compact rectangle \( Q = [a, b] \times [c, d] \) in \( \mathbb{R}^2 \). Assume that for each fixed \( y \) in \([c, d]\), \( f(x, y) \) is an increasing function of \( x \), and that for each fixed \( x \) in \([a, b]\), \( f(x, y) \) is an increasing function of \( y \). Prove that \( f \in R \) on \( Q \).
\end{problembox}

\noindent\textbf{Solution:}
Partition $[a,b]$ and $[c,d]$ into subintervals of mesh smaller than $\delta>0$. On each rectangle $R=I\times J$, monotonicity in each variable gives
\[ \operatorname{osc}(f;R) \le \max_{x\in I}\!f(x,\sup J)-\min_{x\in I}\!f(x,\inf J) \le f(\sup I,\sup J)-f(\inf I,\inf J). \]
Summing over the grid yields
\[ U(f)-L(f) \le \sum_{i,j} \big(f(x_{i+1},y_{j+1})-f(x_i,y_j)\big) \le (\text{Var}_x f)\,\delta + (\text{Var}_y f)\,\delta, \]
which can be made $<\varepsilon$ by choosing $\delta$ small. Thus $U(f)=L(f)$ and $f\in R$ on $Q$.\qed


\begin{problembox}[14.3: Evaluation of Double Integrals]
Evaluate each of the following double integrals.
\begin{enumerate}[label=(\alph*)]
    \item \[ \iint_{Q} \sin^2 x \, \sin^2 y \, dx \, dy, \quad \text{where } Q = [0, \pi] \times [0, \pi]. \]
    \item \[ \iint_{Q} |\cos (x + y)| \, dx \, dy, \quad \text{where } Q = [0, \pi] \times [0, \pi]. \]
    \item \[ \iint_{Q} [x + y] \, dx \, dy, \quad \text{where } Q = [0, 2] \times [0, 2], \text{ and } [t] \text{ is the greatest integer } \leq t. \]
\end{enumerate}
\end{problembox}

\noindent\textbf{Solution:}
\begin{enumerate}[label=(\alph*)]
    \item The integrand factors, so
    \[ \iint_Q \sin^2 x\,\sin^2 y\,dx\,dy = \left(\int_0^{\pi} \sin^2 x\,dx\right)^2 = \left(\tfrac{\pi}{2}\right)^2 = \tfrac{\pi^2}{4}. \]
    \item Let $t=x+y$. For $t\in[0,\pi]$ the fiber length is $t$; for $t\in[\pi,2\pi]$ it is $2\pi-t$. Hence
    \[ \iint_Q |\cos(x+y)|\,dx\,dy = \int_0^{2\pi} |\cos t|\,m(t)\,dt = 2\int_0^{\pi} t\,|\cos t|\,dt = 2\pi. \]
    \item With $m(t)$ the fiber length in $[0,2]^2$, $m(t)=t$ on $[0,2]$ and $m(t)=4-t$ on $[2,4]$. Thus
    \[ \iint_Q [x+y]\,dx\,dy = \sum_{k=0}^3 k\int_k^{k+1} m(t)\,dt = 0\cdot\tfrac{1}{2} + 1\cdot\tfrac{3}{2} + 2\cdot\tfrac{3}{2} + 3\cdot\tfrac{1}{2} = 6. \]
\end{enumerate}\qed


\begin{problembox}[14.4: Integrals over Unit Square]
Let \( Q = [0, 1] \times [0, 1] \) and calculate \( \int_{Q} f(x, y) \, dx \, dy \) in each case.
\begin{enumerate}[label=(\alph*)]
    \item \( f(x, y) = 1 - x - y \) if \( x + y \leq 1, \quad f(x, y) = 0 \) otherwise.
    \item \( f(x, y) = x^2 + y^2 \) if \( x^2 + y^2 \leq 1, \quad f(x, y) = 0 \) otherwise.
    \item \( f(x, y) = x + y \) if \( x^2 \leq y \leq 2x^2, \quad f(x, y) = 0 \) otherwise.
\end{enumerate}
\end{problembox}

\noindent\textbf{Solution:}
\begin{enumerate}[label=(\alph*)]
    \item Over $\{x+y\le 1\}$,
    \[ \int_0^1 \!\int_0^{1-x} (1-x-y)\,dy\,dx = \int_0^1 \tfrac{(1-x)^2}{2}\,dx = \tfrac{1}{6}. \]
    \item Quarter-disk of radius 1: in polar coordinates,
    \[ \int_0^{\pi/2}\!\int_0^1 (x^2+y^2)\,r\,dr\,d\theta = \int_0^{\pi/2}\!\int_0^1 r^3\,dr\,d\theta = \tfrac{\pi}{8}. \]
    \item Decompose by $x$:
    \[ \int_0^{1/\sqrt{2}} \!\int_{x^2}^{2x^2} (x+y)\,dy\,dx\; +\; \int_{1/\sqrt{2}}^{1} \!\int_{x^2}^{1} (x+y)\,dy\,dx. \]
    The first term equals $\int_0^{1/\sqrt{2}} \big(x^3+\tfrac{3}{2}x^4\big)dx= \tfrac{1}{16}+\tfrac{3}{40\sqrt{2}}$. The second term equals
    \[ \int_{1/\sqrt{2}}^{1} \Big(x(1-x^2)+\tfrac{1-x^4}{2}\Big)dx = \left[\tfrac{x^2}{2}-\tfrac{x^4}{4}+\tfrac{x}{2}-\tfrac{x^5}{10}\right]_{1/\sqrt{2}}^{1} = \tfrac{37}{80}-\tfrac{19}{40\sqrt{2}}. \]
    Summing gives $\tfrac{21}{40}-\tfrac{2}{5\sqrt{2}}$.
\end{enumerate}\qed


\begin{problembox}[14.5: Mixed Partial Integrals]
Define \( f \) on the square \( Q = [0, 1] \times [0, 1] \) as follows:
\[ f(x, y) = 
\begin{cases} 
1 & \text{if } x \text{ is rational}, \\
2y & \text{if } x \text{ is irrational}. 
\end{cases} \]
\begin{enumerate}[label=(\alph*)]
    \item Prove that \( \int_{0}^{t} f(x, y) \, dy \) exists for \( 0 \leq t \leq 1 \) and that
    \[ \underline\int_{0}^{1} \left[ \int_{0}^{t} f(x, y) \, dy \right] \, dx = t^2, \]
    and \[ \overline\int_{0}^{1} \left[ \int_{0}^{t} f(x, y) \, dy \right] \, dx = t. \]
    This shows that \( \int_{0}^{1} \left[ \int_{0}^{1} f(x, y) \, dy \right] \, dx \) exists and equals 1.
    
    \item Prove that \( \int_{0}^{1} \left[ \overline\int_{0}^{1} f(x, y) \, dx \right] \, dy \) exists and find its value.
    \item Prove that the double integral \( \int_{Q} f(x, y) \, d(x, y) \) does not exist.
\end{enumerate}
\end{problembox}



\noindent\textbf{Solution:}
\begin{enumerate}[label=(\alph*)]
\item For fixed $x$ and $t\in[0,1]$, the function $y\mapsto f(x,y)$ is Riemann integrable on $[0,t]$ and
\[\int_0^t f(x,y)\,dy = \begin{cases} t, & x\in\mathbb Q, \\[4pt] t^2, & x\notin\mathbb Q.\end{cases}\]
As a function of $x$, this takes the two values $t$ and $t^2$ on dense sets. Hence on every subinterval of $[0,1]$ the supremum is $\max\{t,t^2\}=t$ and the infimum is $\min\{t,t^2\}=t^2$. Therefore
\[ \underline\int_0^1\!\Big[\int_0^t f(x,y)\,dy\Big]dx = t^2,\qquad \overline\int_0^1\!\Big[\int_0^t f(x,y)\,dy\Big]dx = t. \]
In particular, for $t=1$ we have $\int_0^1\!\big[\int_0^1 f(x,y)\,dy\big]dx=1$ since both cases give the value $1$.

\item For fixed $y\in[0,1]$, as a function of $x$ the values $1$ and $2y$ occur on dense sets, so on every subinterval the supremum is $\max\{1,2y\}$ and the infimum is $\min\{1,2y\}$. Hence the upper integral in $x$ exists and equals
\[ \overline\int_0^1 f(x,y)\,dx = \max\{1,2y\} = \begin{cases} 1, & 0\le y\le \tfrac12, \\[4pt] 2y, & \tfrac12< y\le 1. \end{cases} \]
Thus
\[ \int_0^1\!\Big[\overline\int_0^1 f(x,y)\,dx\Big]dy = \int_0^{1/2}\!1\,dy + \int_{1/2}^1\!2y\,dy = \tfrac12 + (1-\tfrac14) = \tfrac{5}{4}. \]

\item Let $R=I\times J$ with $J=[\alpha,\beta]$. Because rationals and irrationals in $x$ are dense, we have
\[ \sup_R f = \max\{1,2\beta\},\qquad \inf_R f=\min\{1,2\alpha\}. \]
Consequently, for any partition $\mathcal P$ of $Q$,
\[ L(f,\mathcal P) \le \int_0^1 \min\{1,2y\}\,dy = \tfrac{3}{4},\qquad U(f,\mathcal P) \ge \int_0^1 \max\{1,2y\}\,dy = \tfrac{5}{4}. \]
Hence $\underline{\iint_Q} f \le \tfrac34 < \tfrac54 \le \overline{\iint_Q} f$, so the double Riemann integral $\int_Q f$ does not exist.
\end{enumerate}\qed


\begin{problembox}[14.6: Discontinuous Integrand]
Define \( f \) on the square \( Q = [0, 1] \times [0, 1] \) as follows:
\[f(x, y) = 
\begin{cases} 
0 & \text{if at least one of } x, y \text{ is irrational}, \\ 
1/n & \text{if } y \text{ is rational and } x = m/n,
\end{cases}\]
where \( m \) and \( n \) are relatively prime integers, \( n > 0 \). Prove that
\[\int_{0}^{1} f(x, y) \, dx = \int_{0}^{1} \left[ \int_{0}^{1} f(x, y) \, dx \right] \, dy = \int_{Q} f(x, y) \, d(x, y) = 0\]
but that \( \int_{0}^{1} f(x, y) \, dy \) does not exist for rational \( x \).
\end{problembox}

\noindent\textbf{Solution:}
For fixed $y$, $f(\cdot,y)$ is zero except possibly on the countable set $\{m/n\}$ when $y$ is rational. Given $\varepsilon>0$, choose a partition of $[0,1]$ so that the total length of intervals covering those rationals is $<\varepsilon$; the contribution to upper sums is then $<\varepsilon\cdot\sup f\le \varepsilon$, hence $\int_0^1 f(x,y)\,dx=0$. Therefore $\int_0^1\!\Big[\int_0^1 f(x,y)\,dx\Big]dy=0$.
Similarly, on any rectangle in $Q$ the set where $f\ne 0$ is a countable subset with total contribution to upper sums made arbitrarily small, so $\int_Q f=0$. However, for rational $x=m/n$, the function $y\mapsto f(m/n,y)$ equals $\tfrac{1}{n}$ on rationals and $0$ on irrationals, which is not Riemann integrable, so $\int_0^1 f(x,y)\,dy$ does not exist for rational $x$.\qed


\begin{problembox}[14.7: Dense Set with Finite Cross-Sections]
If \( p_k \) denotes the \( k \)th prime number, let
\[S(p_k) = \left\{ \begin{pmatrix}
n & m \\
p_k & p_k
\end{pmatrix} : n = 1, 2, \ldots, p_k - 1, \quad m = 1, 2, \ldots, p_k - 1 \right\},\]
let \( S = \bigcup_{k=1}^{\infty} S(p_k) \), and let \( Q = [0, 1] \times [0, 1] \).

\begin{enumerate}[label=(\alph*)]
    \item Prove that \( S \) is dense in \( Q \) (that is, the closure of \( S \) contains \( Q \)) but that any line parallel to the coordinate axes contains at most a finite subset of \( S \).
    
    \item Define \( f \) on \( Q \) as follows:
    \[f(x, y) = 0 \quad \text{if } (x, y) \in S, \quad f(x, y) = 1 \quad \text{if } (x, y) \in Q - S.\]
    Prove that \( \int_{0}^{1} \left[ \int_{0}^{1} f(x, y) \, dy \right] \, dx = \int_{0}^{1} \left[ \int_{0}^{1} f(x, y) \, dx \right] \, dy = 1 \), but that the double integral \( \int_{Q} f(x, y) \, d(x, y) \) does not exist.
\end{enumerate}
\end{problembox}

\noindent\textbf{Solution:}
\begin{enumerate}[label=(\alph*)]
    \item For any rectangle in $Q$ and any large prime $p$, the grid $\{(n/p,m/p):1\le n,m\le p-1\}$ lies $1/p$-dense, hence $S$ is dense. A vertical line $x=x_0$ meets $S$ only when $x_0=n/p$ with $p$ prime; for a reduced rational $x_0=a/b$, this forces $b$ prime and yields at most $p-1$ points, otherwise none. Similarly for horizontal lines; hence each axis-parallel line meets $S$ in a finite set.
    \item For fixed $x$, the vertical section $\{y:(x,y)\in S\}$ is finite, so it has Jordan content zero and $\int_0^1 f(x,y)\,dy=1$. Thus $\int_0^1\![\int_0^1 f(x,y)dy]dx=1$. The same holds with $x$ and $y$ interchanged. However, every rectangle contains points of $S$ and of $Q\setminus S$, so the oscillation of $f$ on every subrectangle is $1$; lower sums are $0$ and upper sums are $1$, hence $\int_Q f$ does not exist.
\end{enumerate}\qed
\section{Jordan Content}
\noindent\textbf{Definitions and Theorems needed.}
\begin{enumerate}[label=(\roman*)]
    \item Jordan outer/inner content; a bounded set is Jordan-measurable iff its boundary has content zero.
    \item Sets of content zero: countable sets and images of small coverings have arbitrarily small total content.
    \item Tube/strip coverings: graphs of continuous functions and rectifiable curves can be covered by thin rectangles with arbitrarily small total area.
    \item Cavalieri’s principle for Jordan content: content of an ordinate set of a continuous function over a Jordan region equals the integral of the function over the base.
\end{enumerate}



\begin{problembox}[14.8: Jordan Content of Finite Accumulation Points]
Let \( S \) be a bounded set in \( \mathbb{R}^n \) having at most a finite number of accumulation points. Prove that \( c(S) = 0 \).
\end{problembox}

\noindent\textbf{Solution:}
Let $\{a_1,\dots,a_m\}$ be the accumulation points (possibly empty). Cover each $a_j$ by a cube of content $<\varepsilon/(2m)$. The remaining points of $S$ are isolated; choose disjoint cubes around each, with total content $<\varepsilon/2$ (possible since only finitely many lie outside any given compact exhaustion). Thus $S$ is covered by cubes of total content $<\varepsilon$. As $\varepsilon>0$ is arbitrary, $c(S)=0$.\qed


\begin{problembox}[14.9: Graph of Continuous Function has Zero Content]
Let \( f \) be a continuous real-valued function defined on \([a, b]\). Let \( S \) denote the graph of \( f \), that is, \( S = \{(x, y) : y = f(x), a \leq x \leq b\} \). Prove that \( S \) has two-dimensional Jordan content zero.
\end{problembox}

\noindent\textbf{Solution:}
By uniform continuity of $f$ on $[a,b]$, for $\varepsilon>0$ choose $\delta$ so that $|x-x'|<\delta$ implies $|f(x)-f(x')|<\varepsilon/(b-a)$. Partition $[a,b]$ into subintervals of length $<\delta$ and cover the graph over each subinterval by a rectangle of width $\Delta x$ and height $<\varepsilon/(b-a)$. The total area is $<\varepsilon$. Hence the graph has content zero.\qed


\begin{problembox}[14.10: Rectifiable Curve has Zero Content]
Let \( \Gamma \) be a rectifiable curve in \( \mathbb{R}^n \). Prove that \( \Gamma \) has \( n \)-dimensional Jordan content zero.
\end{problembox}

\noindent\textbf{Solution:}
Approximate $\Gamma$ by a polygonal path of length within $\varepsilon$ of its total length $L$. Cover each segment by a tube of thickness $\eta>0$; the $n$-dimensional content of the tube is bounded by $C_n\,L\,\eta$, which can be made arbitrarily small by choosing $\eta$. Therefore $c(\Gamma)=0$.\qed


\begin{problembox}[14.11: Ordinate Set Content]
Let \( f \) be a nonnegative function defined on a set \( S \) in \( \mathbb{R}^n \). The ordinate set of \( f \) over \( S \) is defined to be the following subset of \( \mathbb{R}^{n+1} \):
\[\{(x_1, \ldots, x_n, x_{n+1}) : (x_1, \ldots, x_n) \in S, \quad 0 \leq x_{n+1} \leq f(x_1, \ldots, x_n)\}.\]
If \( S \) is a Jordan-measurable region in \( \mathbb{R}^n \) and if \( f \) is continuous on \( S \), prove that the ordinate set of \( f \) over \( S \) has \( (n + 1) \)-dimensional Jordan content whose value is
\[\int_{S} f(x_1, \ldots, x_n) \, d(x_1, \ldots, x_n).\]
Interpret this problem geometrically when \( n = 1 \) and \( n = 2 \).
\end{problembox}

\noindent\textbf{Solution:}
Vertical sections at $x\in S$ are intervals of length $f(x)$; outside $S$ they are empty. By continuity of $f$ and Jordan-measurability of $S$, the ordinate set is Jordan-measurable and its content equals the integral of section lengths:
\[ c_{n+1}(\text{ord}(f,S)) = \int_S f(x)\,dx. \]
Geometrically: for $n=1$, the area under the curve $y=f(x)$ over $S\subset\mathbb R$; for $n=2$, the volume under the surface $z=f(x,y)$ over a planar region $S$.\qed
\section{Advanced Topics}
\noindent\textbf{Definitions and Theorems needed.}
\begin{enumerate}[label=(\roman*)]
    \item Content-zero sets and ``almost everywhere'' statements for Riemann integrable functions.
    \item Mean Value Theorem for integrals on Jordan regions with continuous integrands.
    \item Equality of mixed partials: if $f$ is continuous, then $\partial_2\partial_1$ and $\partial_1\partial_2$ of the iterated integral coincide and equal $f$.
    \item One-dimensional Mean Value Theorem applied along line segments; Fundamental Theorem of Calculus in each variable.
\end{enumerate}



\begin{problembox}[14.12: Zero Integral Implies Zero Function]
Assume that \( f \in R \) on \( S \) and suppose \( \int_S f(x) \, dx = 0 \). (\( S \) is a subset of \( \mathbb{R}^n \)). Let \( A = \{ x : x \in S, f(x) < 0 \} \) and assume that \( c(A) = 0 \). Prove that there exists a set \( B \) of measure zero such that \( f(x) = 0 \) for each \( x \) in \( S - B \).
\end{problembox}

\noindent\textbf{Solution:}
Let $E_\varepsilon=\{x\in S:f(x)>\varepsilon\}$. If $c(E_\varepsilon)>0$ for some $\varepsilon>0$, then $\int_S f\,dx\ge \varepsilon\,c(E_\varepsilon)>0$, a contradiction. Hence $c(E_\varepsilon)=0$ for all $\varepsilon>0$. Let $B=A\cup\bigcup_{m=1}^\infty E_{1/m}$. Then $c(B)=0$ and for $x\in S\setminus B$ we have $f(x)\ge 0$ and $f(x)\le 1/m$ for all $m$, hence $f(x)=0$.\qed


\begin{problembox}[14.13: Mean Value Theorem for Integrals]
Assume that \( f \in R \) on \( S \), where \( S \) is a region in \( \mathbb{R}^n \) and \( f \) is continuous on \( S \). Prove that there exists an interior point \( x_0 \) of \( S \) such that
\[\int_S f(x) \, dx = f(x_0)c(S).\]
\end{problembox}

\noindent\textbf{Solution:}
Let $m=\min_{\overline S} f$ and $M=\max_{\overline S} f$ (attained by continuity). Then $m\,c(S)\le \int_S f\,dx\le M\,c(S)$. Choose $x_-,x_+\in S$ with $f(x_-)\le \frac{1}{c(S)}\int_S f\,dx\le f(x_+)$ (possible since $f(S)$ is an interval on each path-connected component and $S$ has nonempty interior). By continuity, along a path in $S$ joining $x_-$ to $x_+$, the intermediate value $\frac{1}{c(S)}\int_S f$ is assumed at some interior point $x_0$.\qed


\begin{problembox}[14.14: Mixed Partial Derivatives]
Let \( f \) be continuous on a rectangle \( Q = [a, b] \times [c, d] \). For each interior point \( (x_1, x_2) \) in \( Q \), define
\[F(x_1, x_2) = \int_a^{x_1} \left( \int_c^{x_2} f(x, y) \, dy \right) \, dx.\]
Prove that \( D_{1,2} F(x_1, x_2) = D_{2,1} F(x_1, x_2) = f(x_1, x_2) \).
\end{problembox}

\noindent\textbf{Solution:}
By the one-variable Fundamental Theorem of Calculus and continuity of $f$,
\[ D_2 F(x_1,x_2) = \int_a^{x_1} f(x,x_2)\,dx, \qquad D_1 D_2 F(x_1,x_2) = f(x_1,x_2). \]
Symmetrically, $D_1 F(x_1,x_2)=\int_c^{x_2} f(x_1,y)\,dy$ and then $D_2 D_1 F(x_1,x_2)=f(x_1,x_2)$.\qed


\begin{problembox}[14.15: Integral of Mixed Partial Derivative]
Let \( T \) denote the following triangular region in the plane:
\[T = \left\{ (x, y) : 0 \leq \frac{x}{a} + \frac{y}{b} \leq 1 \right\}, \quad \text{where } a > 0, \, b > 0.\]
Assume that \( f \) has a continuous second-order partial derivative \( D_{1,2} f \) on \( T \). Prove that there is a point \( (x_0, y_0) \) on the segment joining \( (a, 0) \) and \( (0, b) \) such that
\[\int_T D_{1,2} f(x, y) \, d(x, y) = f(0, 0) - f(a, 0) + aD_1 f(x_0, y_0).\]
\end{problembox}

\noindent\textbf{Solution:}
Integrate $D_{1,2}f$ first in $y$ over $[0,\,b\,(1-\tfrac{x}{a})]$:
\[ \int_0^{b(1-x/a)} D_{1,2}f(x,y)\,dy = D_1 f\big(x, b(1-\tfrac{x}{a})\big) - D_1 f(x,0). \]
Integrating in $x\in[0,a]$ gives
\[ \int_T D_{1,2}f = \int_0^a D_1 f\big(x, b(1-\tfrac{x}{a})\big)\,dx - \int_0^a D_1 f(x,0)\,dx = f(0,0)-f(a,0) + \int_0^a D_1 f\big(x, b(1-\tfrac{x}{a})\big)\,dx. \]
By the one-dimensional Mean Value Theorem, there exists $\xi\in(0,a)$ such that the last integral equals $a\,D_1 f\big(\xi, b(1-\tfrac{\xi}{a})\big)$. This point lies on the segment joining $(a,0)$ and $(0,b)$, completing the proof.