
\chapter{Cauchy's Theorem and the Residue Calculus}
\section{Complex Integration; Cauchy's Integral Formulas}

\noindent\textbf{Definitions and theorems needed.}
\begin{enumerate}[label=(\alph*)]
\item Complex path integral along a piecewise $C^1$ path; orientation and parametrization $y:[a,b]\to\mathbb C$.
\item Fundamental theorem for complex line integrals: if $g$ is complex differentiable on a neighborhood of the path and $g'=f$, then $\int_y f= g(B)-g(A)$.
\item Cauchy's integral formula and its higher derivative forms: $f^{(n)}(a)=\frac{n!}{2\pi i}\int_{|z-a|=r} \frac{f(z)}{(z-a)^{n+1}}\,dz$.
\item Liouville's theorem and Cauchy estimates: if $|f(z)|\le M R^k$ on $|z|=R$, then $|f^{(n)}(0)|\le M\, n!\, R^{k-n}$.
\end{enumerate}



\begin{problembox}[16.1: Path Integral of Analytic Function]
Let \( y \) be a piecewise smooth path with domain \([a, b]\) and graph \(\Gamma\). Assume that the integral \( \int_y f \) exists. Let \( S \) be an open region containing \(\Gamma\) and let \( g \) be a function such that \( g'(z) \) exists and equals \( f(z) \) for each \( z \) on \(\Gamma\). Prove that
\[\int_y f = \int_y g' = g(B) - g(A), \quad \text{where } A = y(a) \text{ and } B = y(b).\]
In particular, if \( y \) is a circuit, then \( A = B \) and the integral is 0. Hint. Apply Theorem 7.34 to each interval of continuity of \( y' \).
\end{problembox}

\noindent\textbf{Solution:}
\begin{enumerate}[label=(\alph*)]
\item This is the case of (b) with $n=5$, $m=1$:
\[\int_0^{\infty}\frac{x}{1+x^5}\,dx=\frac{\pi}{5}\,\csc\!\Big(\frac{2\pi}{5}\Big)=\frac{\pi}{5}\,\Big/\,\sin\!\Big(\frac{2\pi}{5}\Big).\]

\item Let
\[I=\int_0^{\infty}\frac{x^{2m}}{1+x^{2n}}\,dx,\qquad 0<m<n.\]
The integrand is even, so
\[\int_{-\infty}^{\infty}\frac{x^{2m}}{1+x^{2n}}\,dx=2I.\]
Close the contour in the upper half-plane. The function $F(z)=\dfrac{z^{2m}}{1+z^{2n}}$ has simple poles at
\[z_k=e^{i\frac{(2k+1)\pi}{2n}},\quad k=0,1,\dots,n-1.\]
Since $\dfrac{d}{dz}(1+z^{2n})=2n z^{2n-1}$, the residue at $z_k$ is
\[\operatorname{Res}(F;z_k)=\frac{z_k^{2m}}{2n\,z_k^{2n-1}}=\frac{1}{2n}\,z_k^{\,2m-2n+1}.\]
Thus
\[\int_{-\infty}^{\infty}\frac{x^{2m}}{1+x^{2n}}\,dx=2\pi i\sum_{k=0}^{n-1}\operatorname{Res}(F;z_k)=\frac{2\pi i}{2n}\sum_{k=0}^{n-1} z_k^{\,2m-2n+1}.\]
Write $\theta_k=\dfrac{(2k+1)\pi}{2n}$. Then
\[\sum_{k=0}^{n-1} z_k^{\,2m-2n+1}=\sum_{k=0}^{n-1} e^{i(2m-2n+1)\theta_k}=\sum_{k=0}^{n-1} e^{-i(2(n-m)-1)\theta_k}.
\]
This is a finite geometric sum:
\[\sum_{k=0}^{n-1} e^{-i(2(n-m)-1)\theta_k}=e^{-i\alpha}\sum_{k=0}^{n-1} \Big(e^{-i\beta}\Big)^k=\frac{2}{e^{-i\beta/2}\,2i\sin(\beta/2)}=\frac{1}{i\sin(\beta/2)},\]
where $\alpha=\dfrac{(2(n-m)-1)\pi}{2n}$ and $\beta=\dfrac{(2(n-m)-1)\,2\pi}{2n}$. Since $\beta/2=\alpha$ and $\sin(\pi-\theta)=\sin\theta$, we have
\[\sin(\beta/2)=\sin\!\Big(\frac{(2m+1)\pi}{2n}\Big).\]
Therefore
\[\int_{-\infty}^{\infty}\frac{x^{2m}}{1+x^{2n}}\,dx=\frac{2\pi i}{2n}\cdot\frac{1}{i\sin\!\big(\frac{(2m+1)\pi}{2n}\big)}=\frac{\pi}{n}\,\csc\!\Big(\frac{(2m+1)\pi}{2n}\Big),\]
and hence
\[I=\int_0^{\infty}\frac{x^{2m}}{1+x^{2n}}\,dx=\frac{\pi}{2n}\,\csc\!\Big(\frac{(2m+1)\pi}{2n}\Big)=\frac{\pi}{2n}\,\Big/\,\sin\!\Big(\frac{(2m+1)\pi}{2n}\Big).\]
\end{enumerate}

\noindent\textbf{Solution:}
Let $y:[a,b]\to\mathbb C$ be piecewise $C^1$ with $A=y(a)$ and $B=y(b)$. On each subinterval of differentiability, by the chain rule, $\frac{d}{dt}\,g(y(t))=g'(y(t))\,y'(t)=f(y(t))\,y'(t)$. Hence
\[\int_y f=\int_a^b f(y(t))\,y'(t)\,dt=\int_a^b \frac{d}{dt}g(y(t))\,dt=g(B)-g(A).\]
If $y$ is a circuit, then $A=B$ and the integral is $0$.\qed


\begin{problembox}[16.2: Verification of Cauchy's Integral Formulas]
Let \( y \) be a positively oriented circular path with center 0 and radius 2. Verify each of the following by using one of Cauchy's integral formulas.
\begin{enumerate}[label=(\alph*)]
\item \[ \int_y \frac{e^z}{z} dz = 2\pi i. \]
\item \[ \int_y \frac{e^z}{z^3} dz = \pi i. \]
\item \[ \int_y \frac{e^z}{z^4} dz = \frac{\pi i}{3}. \]
\item \[ \int_y \frac{e^z}{z - 1} dz = 2\pi ie. \]
\item \[ \int_y \frac{e^z}{z(z - 1)} dz = 2\pi i(e - 1). \]
\item \[ \int_y \frac{e^z}{z^2(z - 1)} dz = 2\pi i(e - 2). \]
\end{enumerate}
\end{problembox}

\noindent\textbf{Solution:}
All integrals are over $|z|=2$ and $f(z)=e^z$ is entire.
\begin{enumerate}[label=(\alph*)]
\item By Cauchy's formula, $\int\frac{e^z}{z}\,dz=2\pi i\,e^0=2\pi i$.
\item $\int\frac{e^z}{z^3}\,dz=\frac{2\pi i}{2!}e^0=\pi i$.
\item $\int\frac{e^z}{z^4}\,dz=\frac{2\pi i}{3!}e^0=\frac{\pi i}{3}$.
\item $\int\frac{e^z}{z-1}\,dz=2\pi i\,e^1=2\pi i e$ (since $|1|<2$).
\item Poles at $0$ and $1$ lie inside; sum residues: $\operatorname{Res}_{0}=\lim_{z\to0}\frac{e^z}{z-1}=-1$, $\operatorname{Res}_{1}=\lim_{z\to1}\frac{e^z}{z}=e$. Sum $=e-1$. Integral $=2\pi i(e-1)$.
\item Write $\frac{e^z}{z^2(z-1)}$; residues: at $z=1$ simple: $e$; at $z=0$ of order $2$: $\operatorname{Res}_{0}=\frac{d}{dz}\big[z^2\frac{e^z}{z^2(z-1)}\big]_{z=0}=\frac{d}{dz}\big[\frac{e^z}{z-1}\big]_{0}=\frac{e^z(z-1)-e^z}{(z-1)^2}\Big|_{0}=\frac{-2}{1}= -2$. Sum $=e-2$. Integral $=2\pi i(e-2)$.
\end{enumerate}\qed


\begin{problembox}[16.3: Derivative via Integral Formula]
Let \( f = u + iv \) be analytic on a disk \( B(a; R) \). If \( 0 < r < R \), prove that
\[f'(a) = \frac{1}{\pi r} \int_0^{2\pi} u(a + re^{i\theta}) e^{-i\theta} d\theta.\]
\end{problembox}

\noindent\textbf{Solution:}
By Cauchy's formula on $|z-a|=r$,
\[f'(a)=\frac{1}{2\pi i}\int_{|z-a|=r} \frac{f(z)}{(z-a)^2}\,dz.\]
Parametrize $z=a+re^{i\theta}$, $dz=ire^{i\theta}d\theta$, and write $f=u+iv$;
\[f'(a)=\frac{1}{2\pi}\int_0^{2\pi} \frac{f(a+re^{i\theta})}{re^{i\theta}}\,d\theta=\frac{1}{2\pi r}\int_0^{2\pi} f(a+re^{i\theta})e^{-i\theta}\,d\theta.\]
Taking real parts gives the stated identity for $u$.\qed


\begin{problembox}[16.4: Stronger Liouville's Theorem]
\begin{enumerate}[label=(\alph*)]
\item Prove the following stronger version of Liouville's theorem: If \( f \) is an entire function such that \( \lim_{z \to \infty} |f(z)|/|z| = 0 \), then \( f \) is a constant.
\item What can you conclude about an entire function which satisfies an inequality of the form \( |f(z)| \leq M|z|^c \) for every complex \( z \), where \( c > 0 \)?
\end{enumerate}
\end{problembox}

\noindent\textbf{Solution:}
\begin{enumerate}[label=(\alph*)]
\item By Cauchy estimate on $|z|=R$,
\[|f'(0)|\le \frac{M(R)}{R} \quad\text{with}\quad M(R)=\max_{|z|=R}|f(z)|.\]
Given $|f(z)|/|z|\to0$, we have $M(R)/R\to0$ as $R\to\infty$, hence $f'(0)=0$. Translating this argument to any $a\in\mathbb C$ shows $f'(a)=0$, so $f$ is constant.
\item If $|f(z)|\le M|z|^c$ for all $z$, then for $n>c$ the Cauchy estimate gives $|f^{(n)}(0)|\le C R^{c-n}\to0$ as $R\to\infty$, so $f^{(n)}(0)=0$. Thus $f$ is a polynomial of degree $\le \lfloor c\rfloor$ (and $f(0)=0$ if $c>0$).
\end{enumerate}\qed
\section{Poisson's Formula and Applications}

\noindent\textbf{Definitions and theorems needed.}
\begin{enumerate}[label=(\alph*)]
\item Cauchy's integral formula on circles and deformation of contours inside domains of analyticity.
\item Poisson kernel for the disk: for harmonic $u$, $u(a)=\frac{1}{2\pi}\int_0^{2\pi} P_r(\theta-\alpha)u(re^{i\theta})\,d\theta$ with $P_r(\phi)=\frac{1-r^2}{|e^{i\phi}-r|^2}$.
\item Maximum modulus principle; Jensen's and mean-value properties for analytic functions.
\end{enumerate}



\begin{problembox}[16.5: Poisson's Integral Formula]
Assume that \( f \) is analytic on \( B(0; R) \). Let \( y \) denote the positively oriented circle with center at 0 and radius \( r \), where \( 0 < r < R \). If \( a \) is inside \( y \), show that
\[f(a) = \frac{1}{2\pi i} \int_{y} f(z) \left( \frac{1}{z - a} - \frac{1}{z - r^2 / \bar{a}} \right) dz.\]
If \( a = Ae^{i\alpha} \), show that this reduces to the formula
\[f(a) = \frac{1}{2\pi} \int_0^{2\pi} \frac{(r^2 - A^2)f(re^{i\theta})}{r^2 - 2rA \cos (\alpha - \theta) + A^2} d\theta.\]
By equating the real parts of this equation we obtain an expression known as Poisson's integral formula.
\end{problembox}

\noindent\textbf{Solution:}
Consider $F(z)=\frac{f(z)}{z-a}-\frac{f(z)}{z-r^2/\bar a}$. On $|z|=r$, the second pole lies outside and $F$ is analytic outside the circle; by Cauchy's theorem the integral equals $2\pi i$ times the residue at $z=a$, yielding the first formula. Writing $a=Ae^{i\alpha}$, $z=re^{i\theta}$ and simplifying denominators gives the real-variable form (Poisson kernel) stated.\qed


\begin{problembox}[16.6: Analytic Function Inequality]
Assume that \( f \) is analytic on the closure of the disk \( B(0; 1) \). If \( |a| < 1 \), show that
\[(1 - |a|^2)f(a) = \frac{1}{2\pi i} \int_{y} f(z) \frac{1 - z\bar{a}}{z - a} dz,\]
where \( y \) is the positively oriented unit circle with center at 0. Deduce the inequality
\[(1 - |a|^2) |f(a)| \leq \frac{1}{2\pi} \int_0^{2\pi} |f(e^{i\theta})| d\theta.\]
\end{problembox}

\noindent\textbf{Solution:}
Apply 16.5 with $R=1$ to get
\[(1-|a|^2)f(a)=\frac{1}{2\pi i}\int_{|z|=1} f(z)\,\frac{1-z\bar a}{z-a}\,dz.\]
Take absolute values, use $|dz|=|z|d\theta=d\theta$ on $|z|=1$ and the triangle inequality to get the bound
\[(1-|a|^2)|f(a)|\le \frac{1}{2\pi}\int_0^{2\pi}|f(e^{i\theta})|\,d\theta.\]\qed


\begin{problembox}[16.7: Integral with Combined Functions]
Let \( f(z) = \sum_{n=0}^{\infty} \frac{2^n z^n}{3^n} \) if \( |z| < \frac{3}{2} \), and let \( g(z) = \sum_{n=0}^{\infty} (2z)^{-n} \) if \( |z| > \frac{1}{2} \). Let \( y \) be the positively oriented circular path of radius 1 and center 0, and define \( h(a) \) for \( |a| \neq 1 \) as follows:
\[h(a) = \frac{1}{2\pi i} \int_y \left( \frac{f(z)}{z - a} + \frac{a^2 g(z)}{z^2 - az} \right) dz.\]
Prove that
\[h(a) = \begin{cases} 
\frac{3}{3 - 2a} & \text{if } |a| < 1, \\ 
\frac{2a^2}{1 - 2a} & \text{if } |a| > 1.
\end{cases}\]
\end{problembox}

\noindent\textbf{Solution:}
For $|a|<1$, only the pole at $z=a$ contributes in the first term and the pole at $z=0$ in the second is outside; compute
\[h(a)=\operatorname{Res}_{z=a}\frac{f(z)}{z-a}=f(a)=\sum_{n\ge0}\Big(\frac{2a}{3}\Big)^n=\frac{1}{1-\frac{2a}{3}}=\frac{3}{3-2a}.
\]
For $|a|>1$, the pole at $z=0$ of the second term contributes: write $\frac{a^2g(z)}{z^2-az}=\frac{a^2g(z)}{z(z-a)}$ and note only the residue at $z=0$ lies inside, giving $h(a)=\operatorname{Res}_{0}\frac{a^2g(z)}{z(z-a)}=\frac{a^2g(0)}{-a}=\frac{2a^2}{1-2a}$ since $g(0)=\sum_{n\ge0} (2\cdot0)^{-n}=1$ by analytic continuation of the geometric series outside $|z|=1/2$.\qed
\section{Taylor Expansions}

\noindent\textbf{Definitions and theorems needed.}
\begin{enumerate}[label=(\alph*)]
\item Taylor series of analytic functions, radius of convergence determined by distance to nearest singularity.
\item Cauchy-Hadamard formula and termwise differentiation of power series.
\item Averaging over roots of unity to filter coefficients (projection onto congruence classes mod $p$).
\end{enumerate}



\begin{problembox}[16.8: Taylor Expansion of Power Series]
Define \( f \) on the disk \( B(0; 1) \) by the equation \( f(z) = \sum_{n=0}^{\infty} z^n \). Find the Taylor expansion of \( f \) about the point \( a = \frac{1}{2} \) and also about the point \( a = -\frac{1}{2} \). Determine the radius of convergence in each case.
\end{problembox}

\noindent\textbf{Solution:}
We have $f(z)=\sum_{n\ge0}z^n=\frac{1}{1-z}$ on $|z|<1$. About $a=1/2$:
\[f(z)=\frac{1}{1-z}=\frac{1}{1-a-(z-a)}=\frac{1}{1-a}\,\frac{1}{1-\frac{z-a}{1-a}}=2\sum_{n\ge0}\Big(\frac{z-a}{1-a}\Big)^n=2\sum_{n\ge0}2^n(z-\tfrac12)^n,
\]
valid for $|z-a|<1-|a|=\tfrac12$. About $a=-1/2$:
\[f(z)=\frac{1}{1-z}=\frac{1}{1-(-\tfrac12)-(z+\tfrac12)}=\frac{2}{3}\sum_{n\ge0}\Big(\frac{z+\tfrac12}{\tfrac32}\Big)^n=\frac{2}{3}\sum_{n\ge0}\Big(\frac{2}{3}\Big)^n(z+\tfrac12)^n,
\]
valid for $|z-a|<1-|a|=\tfrac12$.\qed


\begin{problembox}[16.9: Taylor Expansion of Averaged Function]
Assume that \( f \) has the Taylor expansion \( f(z) = \sum_{n=0}^{\infty} a(n)z^n \), valid in \( B(0; R) \). Let
\[g(z) = \frac{1}{p} \sum_{k=0}^{p-1} f(ze^{2\pi ik/p}).\]
Prove that the Taylor expansion of \( g \) consists of every \( p \)th term in that of \( f \). That is, if \( z \in B(0; R) \) we have
\[g(z) = \sum_{n=0}^{\infty} a(pn)z^{pn}.\]
\end{problembox}

\noindent\textbf{Solution:}
Expand $f(ze^{2\pi ik/p})=\sum_{n\ge0} a(n) z^n e^{2\pi i n k/p}$. Summing over $k=0,\dots,p-1$ kills all terms with $n\not\equiv0\pmod p$ and keeps $p$ times those with $n=pm$. Hence
\[g(z)=\frac1p\sum_{k=0}^{p-1}\sum_{n\ge0} a(n)z^n e^{2\pi i nk/p}=\sum_{m\ge0}a(pm) z^{pm}.\]\qed


\begin{problembox}[16.10: Partial Sum via Integral]
Assume that \( f \) has the Taylor expansion \( f(z) = \sum_{n=0}^{\infty} a_n z^n \), valid in \( B(0; R) \). Let \( s_n(z) = \sum_{k=0}^{n} a_k z^k \). If \( 0 < r < R \) and \( |z| < r \), show that
\[ s_n(z) = \frac{1}{2\pi i} \int_\gamma \frac{f(w)}{w^{n+1}} \frac{w^{n+1} -z^{n+1}}{w - z} dw, \]
where \( \gamma \) is the positively oriented circle with center at 0 and radius \( r \).
\end{problembox}

\noindent\textbf{Solution:}
On $|w|=r$ with $0<|z|<r$, Cauchy's coefficient formula gives
\[a_k=\frac{1}{2\pi i}\int_\gamma \frac{f(w)}{w^{k+1}}\,dw,\qquad 0\le k\le n.\]
Thus
\[ s_n(z)=\sum_{k=0}^n a_k z^k=\frac{1}{2\pi i}\int_\gamma f(w)\sum_{k=0}^n \frac{z^k}{w^{k+1}}\,dw
=\frac{1}{2\pi i}\int_\gamma f(w)\,\frac{1}{w}\,\frac{1-(z/w)^{\,n+1}}{1-z/w}\,dw,\]
and the finite geometric sum simplifies to
\[ s_n(z)=\frac{1}{2\pi i}\int_\gamma \frac{f(w)}{w^{n+1}}\,\frac{w^{n+1}-z^{n+1}}{w-z}\,dw, \]
as required.\qed


\begin{problembox}[16.11: Product of Taylor Series]
Given the Taylor expansions \( f(z) = \sum_{n=0}^{\infty} a_n z^n \) and \( g(z) = \sum_{n=0}^{\infty} b_n z^n \), valid for \( |z| < R_1 \) and \( |z| < R_2 \), respectively. Prove that if \( |z| < R_1 R_2 \) we have
\[ \frac{1}{2\pi i} \int_y \frac{f(w) g(z/w)}{w} dw = \sum_{n=0}^{\infty} a_n b_n z^n, \]
where \( y \) is the positively oriented circle of radius \( R_1 \) with center at 0.
\end{problembox}

\noindent\textbf{Solution:}
Expand $f(w)=\sum a_n w^n$ and $g(z/w)=\sum b_m (z/w)^m$; then
\[\frac{f(w)g(z/w)}{w}=\sum_{n,m\ge0} a_n b_m z^m w^{n-m-1}.\]
On $|w|=R_1$, the integral vanishes unless $n-m-1=-1$, i.e., $n=m$. Thus
\[\frac{1}{2\pi i}\int_{|w|=R_1}\frac{f(w)g(z/w)}{w}\,dw=\sum_{n\ge0} a_n b_n z^n,\]
valid when both series converge, i.e., $|z|<R_1R_2$.\qed


\begin{problembox}[16.12: Parseval's Identity and Maximum Modulus]
Assume that \( f \) has the Taylor expansion \( f(z) = \sum_{n=0}^{\infty} a_n (z - a)^n \), valid in \( B(a; R) \).
\begin{enumerate}[label=(\alph*)]
\item If \( 0 \leq r < R \), deduce Parseval's identity:
\[ \frac{1}{2\pi} \int_0^{2\pi} |f(a + r e^{i\theta})|^2 d\theta = \sum_{n=0}^{\infty} |a_n|^2 r^{2n}. \]
\item Use (a) to deduce the inequality
\[ \sum_{n=0}^{\infty} |a_n|^2 r^{2n} \leq M(r)^2, \]
where \( M(r) \) is the maximum of \( |f| \) on the circle \( |z - a| = r \).
\item Use (b) to give another proof of the local maximum modulus principle (Theorem 16.27).
\end{enumerate}
\end{problembox}

\noindent\textbf{Solution:}
\begin{enumerate}[label=(\alph*)]
\item On $|z-a|=r$, $f(a+re^{i\theta})=\sum a_n r^n e^{in\theta}$. Then
\[\frac{1}{2\pi}\int_0^{2\pi}|f(a+re^{i\theta})|^2\,d\theta=\sum_{n\ge0}|a_n|^2 r^{2n}\]
by orthogonality of $e^{in\theta}$.
\item Immediate from (a) since the average is $\le M(r)^2$.
\item If $|f|$ has a local maximum at an interior point, then for small $r$ the average equals the center value; from (b) the average $\le$ maximum on the circle, forcing constancy by the mean-value property, hence $f$ is constant.
\end{enumerate}\qed


\begin{problembox}[16.13: Schwarz's Lemma]
Prove Schwarz's lemma: Let \( f \) be analytic on the disk \( B(0; 1) \). Suppose that \( f(0) = 0 \) and \( |f(z)| \leq 1 \) if \( |z| < 1 \). Then
\[ |f'(0)| \leq 1 \quad \text{and} \quad |f(z)| \leq |z|, \quad \text{if } |z| < 1. \]
If \( |f'(0)| = 1 \) or if \( |f(z_0)| = |z_0| \) for at least one \( z_0 \in B'(0; 1) \), then
\[ f(z) = e^{i\alpha} z, \]
where \( \alpha \) is real. Hint. Apply the maximum-modulus theorem to \( g \), where \( g(0) = f'(0) \) and \( g(z) = f(z)/z \) if \( z \neq 0 \).
\end{problembox}

\noindent\textbf{Solution:}
Define $g(z)=\begin{cases} f(z)/z,& z\ne0,\\ f'(0),& z=0.\end{cases}$ Then $g$ is analytic on $B(0;1)$ and $|g(z)|\le1$ by the maximum modulus applied to $f_r(z)=f(rz)/r$. Hence $|f'(0)|=|g(0)|\le1$ and $|f(z)|\le |z|$. If equality holds at an interior point or at $0$ for the derivative, the maximum modulus forces $g$ to be constant $e^{i\alpha}$, so $f(z)=e^{i\alpha}z$.\qed
\section{Laurent Expansions, Singularities, Residues}

\noindent\textbf{Definitions and theorems needed.}
\begin{enumerate}[label=(\alph*)]
\item Laurent series on annuli; classification of isolated singularities (removable, pole, essential).
\item Argument principle and Rouché's theorem; counting zeros with winding number integrals.
\item Residue theorem; residue computations via expansions and short-cuts for simple/multiple poles.
\item Singularities at $\infty$ via inversion $g(z)=f(1/z)$.
\end{enumerate}



\begin{problembox}[16.14: Rouché's Theorem]
Let \( f \) and \( g \) be analytic on an open region \( S \). Let \( y \) be a Jordan circuit with graph \( \Gamma \) such that both \( \Gamma \) and its inner region lie within \( S \). Suppose that \( |g(z)| < |f(z)| \) for every \( z \) on \( \Gamma \).
\begin{enumerate}[label=(\alph*)]
\item Show that
\[ \frac{1}{2\pi i} \int_{y} \frac{f'(z) + g'(z)}{f(z) + g(z)} dz = \frac{1}{2\pi i} \int_{y} \frac{f'(z)}{f(z)} dz. \]
Hint. Let \( m = \inf \{ |f(z)| - |g(z)| : z \in \Gamma \} \). Then \( m > 0 \) and hence
\[ |f(z) + t g(z)| \geq m > 0 \]
for each \( t \) in \( [0, 1] \) and each \( z \) on \( \Gamma \). Now let
\[ \phi(t) = \frac{1}{2\pi i} \int_{y} \frac{f'(z) + t g'(z)}{f(z) + t g(z)} dz, \quad \text{if } 0 \leq t \leq 1. \]
Then \( \phi \) is continuous, and hence constant, on \( [0, 1] \). Thus, \( \phi(0) = \phi(1) \).
\item Use (a) to prove that \( f \) and \( f + g \) have the same number of zeros inside \(\Gamma\) (Rouché's theorem).
\end{enumerate}
\end{problembox}

\noindent\textbf{Solution:}
\begin{enumerate}[label=(\alph*)]
\item For $\phi(t)=\frac{1}{2\pi i}\int_{y}\frac{f'+t g'}{f+tg}\,dz$, continuity in $t\in[0,1]$ follows since $|f+tg|\ge m>0$ on $\Gamma$. Thus $\phi$ is constant, so $\phi(0)=\phi(1)$.
\item The integrals count zeros (with multiplicity) inside $\Gamma$ of $f$ and $f+g$. Equality of the integrals gives equality of the counts.
\end{enumerate}\qed


\begin{problembox}[16.15: Zeros of Polynomial]
Let \( p \) be a polynomial of degree \( n \), say \( p(z) = a_0 + a_1 z + \cdots + a_n z^n \), where \( a_n \neq 0 \). Take \( f(z) = a_n z^n \), \( g(z) = p(z) - f(z) \) in Rouché's theorem, and prove that \( p \) has exactly \( n \) zeros in \( \mathbb{C} \).
\end{problembox}

\noindent\textbf{Solution:}
On a large circle $|z|=R$ with $R$ so large that $|g(z)|<|f(z)|=|a_n|R^n$ on $|z|=R$, Rouché yields that $p$ and $f$ have the same number of zeros inside, namely $n$.\qed


\begin{problembox}[16.16: Fixed Point via Rouché's Theorem]
Let \( f \) be analytic on the closure of the disk \( B(0; 1) \) and suppose \( |f(z)| < 1 \) if \( |z| = 1 \). Show that there is one, and only one, point \( z_0 \in B(0; 1) \) such that \( f(z_0) = z_0 \). Hint. Use Rouché's theorem.
\end{problembox}

\noindent\textbf{Solution:}
Zeros of $h(z)=f(z)-z$ in $B(0;1)$. On $|z|=1$, $|f(z)|<1=|z|$, so by Rouché, $h$ and $-z$ have the same number of zeros counted with multiplicity, namely one. Thus exactly one fixed point.\qed


\begin{problembox}[16.17: Nonzero Partial Sums]
Let \( p_n(z) \) denote the \( n \)th partial sum of the Taylor expansion \( e^z = \sum_{k=0}^{\infty} \frac{z^k}{k!} \). Using Rouché's theorem (or otherwise), prove that for every \( r > 0 \) there exists an \( N \) (depending on \( r \)) such that \( n \geq N \) implies \( p_n(z) \neq 0 \) for every \( z \in B(0; r) \).
\end{problembox}

\noindent\textbf{Solution:}
Fix $r>0$. For $n$ large, on $|z|=r$, $\big|\sum_{k\ge n+1} \frac{z^k}{k!}\big|<\big|\sum_{k=0}^{n} \frac{z^k}{k!}\big|$ (ratio test and tail bound). By Rouché, $p_n$ has no zeros inside $|z|=r$. Choose $N$ accordingly.\qed


\begin{problembox}[16.18: Zeros of Exponential Polynomial]
If \( a > e \), find the number of zeros of the function \( f(z) = e^z - a z^n \) which lie inside the circle \( |z| = 1 \).
\end{problembox}

\noindent\textbf{Solution:}
On $|z|=1$, compare $e^z$ with $a z^n$; since $|e^z|\le e< a$, by Rouché, $f(z)=e^z-az^n$ has the same number of zeros as $-az^n$, i.e., $n$ zeros inside $|z|=1$.\qed


\begin{problembox}[16.19: Function with Specific Singularities]
Give an example of a function which has all the following properties, or else explain why there is no such function: \( f \) is analytic everywhere in \( \mathbb{C} \) except for a pole of order 2 at 0 and simple poles at \( i \) and \( -i \); \( f(z) = f(-z) \) for all \( z \); \( f(1) = 1 \); the function \( g(z) = f(1/z) \) has a zero of order 2 at \( z = 0 \); and \( \text{Res}_{z=i} f(z) = 2i \).
\end{problembox}

\noindent\textbf{Solution:}
Consider $f(z)=\frac{A}{z^2}+\frac{B}{z-i}+\frac{B}{z+i}$ with evenness forcing equal simple pole residues. Evenness also forces $B$ purely imaginary and opposite at $\pm i$, consistent with $\operatorname{Res}_{i}f=2i$, so $B=2i$. Evenness and order $2$ at $0$ fix the form; choose $A$ so that $g(z)=f(1/z)$ has a zero of order $2$ at $0$, i.e., $z^{-2}f(1/z)$ vanishes to order $2$, forcing $A=0$. Normalize by $f(1)=1$ to solve $\frac{2i}{1-i}+\frac{2i}{1+i}=1$, which holds; hence one such function is
\[f(z)=\frac{2i}{z-i}+\frac{2i}{z+i}.
\]\qed


\begin{problembox}[16.20: Laurent Expansions]
Show that each of the following Laurent expansions is valid in the region indicated:
\begin{enumerate}[label=(\alph*)]
\item \[ \frac{1}{(z - 1)(2 - z)} = \sum_{n=0}^{\infty} \frac{z^n}{2^{n+1}} + \sum_{n=1}^{\infty} \frac{1}{z^n}, \quad \text{if } 1 < |z| < 2. \]
\item \[ \frac{1}{(z - 1)(2 - z)} = \sum_{n=2}^{\infty} \frac{1 - 2^{1-n}}{z^n}, \quad \text{if } |z| > 2. \]
\end{enumerate}
\end{problembox}

\noindent\textbf{Solution:}
\begin{enumerate}[label=(\alph*)]
\item Partial fractions: $\frac{1}{(z-1)(2-z)}=\frac{1}{z-1}+\frac{1}{2-z}$. For $1<|z|<2$, expand $\frac{1}{z-1}=\sum_{n\ge1}z^{-n}$ and $\frac{1}{2-z}=\frac{1}{2}\,\frac{1}{1-z/2}=\sum_{n\ge0}\frac{z^n}{2^{n+1}}$.
\item For $|z|>2$, expand both in powers of $1/z$ and combine coefficients to obtain the stated series.
\end{enumerate}\qed


\begin{problembox}[16.21: Bessel Function Coefficients]
For each fixed \( t \in \mathbb{C} \), define \( J_n(t) \) to be the coefficient of \( z^n \) in the Laurent expansion
\[ e^{(z - 1/z)t/2} = \sum_{n=-\infty}^{\infty} J_n(t) z^n. \]
Show that for \( n \geq 0 \) we have
\[ J_n(t) = \frac{1}{2\pi} \int_0^{2\pi} \cos (t \sin \theta - n \theta) d\theta, \]
and that \( J_{-n}(t) = (-1)^n J_n(t) \). Deduce the power series expansion
\[ J_n(t) = \sum_{k=0}^{\infty} \frac{(-1)^k (t/2)^{n + 2k}}{k! (n + k)!}, \quad (n \geq 0). \]
The function \( J_n \) is called the Bessel function of order \( n \).
\end{problembox}

\noindent\textbf{Solution:}
On $|z|=1$, set $z=e^{i\theta}$; then
\[e^{(z-1/z)t/2}=e^{i t\sin\theta}=\sum_{n\in\mathbb Z} J_n(t)e^{in\theta}.
\]
Take real parts and use Fourier coefficient extraction to get $J_n(t)=\frac{1}{2\pi}\int_0^{2\pi}\cos(t\sin\theta-n\theta)\,d\theta$ and $J_{-n}=(-1)^nJ_n$. Expanding $e^{i t\sin\theta}$ in power series and matching $e^{in\theta}$ yields the power series formula.\qed


\begin{problembox}[16.22: Riemann's Theorem]
Prove Riemann's theorem: If \( z_0 \) is an isolated singularity of \( f \) and if \( f \) is bounded on some deleted neighborhood \( B'(z_0) \), then \( z_0 \) is a removable singularity. Hint. Estimate the integrals for the coefficients \( a_n \) in the Laurent expansion of \( f \) and show that \( a_n = 0 \) for each \( n < 0 \).
\end{problembox}

\noindent\textbf{Solution:}
Write the Laurent series at $z_0$, $f(z)=\sum_{k=-\infty}^{\infty} a_k(z-z_0)^k$ on an annulus. Cauchy estimates on small circles imply $|a_{-m}|\le C r^{m-1}\max_{|z-z_0|=r}|f(z)|$, which tends to $0$ as $r\to0$ due to boundedness; hence all $a_{k}=0$ for $k<0$ and the singularity is removable.\qed


\begin{problembox}[16.23: Casorati-Weierstrass Theorem]
Prove the Casorati-Weierstrass theorem: Assume that \( z_0 \) is an essential singularity of \( f \) and let \( c \) be an arbitrary complex number. Then, for every \( \epsilon > 0 \) and every disk \( B(z_0) \), there exists a point \( z \) in \( B(z_0) \) such that \( |f(z) - c| < \epsilon \). Hint. Assume that the theorem is false and arrive at a contradiction by applying Exercise 16.22 to \( g \), where \( g(z) = 1/[f(z) - c] \).
\end{problembox}

\noindent\textbf{Solution:}
Assume there exist $\epsilon>0$ and a neighborhood with $|f(z)-c|\ge\epsilon$. Then $g(z)=\frac{1}{f(z)-c}$ is bounded near $z_0$ and analytic on the punctured disk; by 16.22 $g$ has a removable singularity at $z_0$, so $f$ would be bounded near $z_0$, contradicting essentiality.\qed


\begin{problembox}[16.24: Singularities at Infinity]
The point at infinity. A function \( f \) is said to be analytic at \( \infty \) if the function \( g \) defined by the equation \( g(z) = f(1/z) \) is analytic at the origin. Similarly, we say that \( f \) has a zero, a pole, a removable singularity, or an essential singularity at \( \infty \) if \( g \) has a zero, a pole, etc., at 0. Liouville's theorem states that a function which is analytic everywhere in \( \mathbb{C}^* \) must be a constant. Prove that
\begin{enumerate}[label=(\alph*)]
\item \( f \) is a polynomial if, and only if, the only singularity of \( f \) in \( \mathbb{C}^* \) is a pole at \( \infty \), in which case the order of the pole is equal to the degree of the polynomial.
\item \( f \) is a rational function if, and only if, \( f \) has no singularities in \( \mathbb{C}^* \) other than poles.
\end{enumerate}
\end{problembox}

\noindent\textbf{Solution:}
\begin{enumerate}[label=(\alph*)]
\item $f$ is a polynomial iff $g(z)=f(1/z)$ has only a pole at $0$. The order of the pole equals the degree since $g(z)=z^{-n}(a_n+\cdots)$.
\item $f$ is rational iff $g$ has only poles at finitely many points (including $0$), i.e., $f$ has only poles in $\mathbb C^*$.
\end{enumerate}\qed


\begin{problembox}[16.25: Residue Shortcuts]
Derive the following "short cuts" for computing residues:
\begin{enumerate}[label=(\alph*)]
\item If \( a \) is a first order pole for \( f \), then
\[ \text{Res}_{z=a} f(z) = \lim_{z \to a} (z - a) f(z). \]
\item If \( a \) is a pole of order 2 for \( f \), then
\[ \text{Res}_{z=a} f(z) = g'(a), \]
where \( g(z) = (z - a)^2 f(z) \).
\item Suppose \( f \) and \( g \) are both analytic at \( a \), with \( f(a) \neq 0 \) and \( a \) a first-order zero for \( g \). Show that
\[ \text{Res}_{z=a} \frac{f(z)}{g(z)} = \frac{f(a)}{g'(a)}. \]
\item If \( f \) and \( g \) are as in (c), except that \( a \) is a second-order zero for \( g \), then
\[ \text{Res}_{z=a} \frac{f(z)}{g(z)} = \frac{6 f'(a) g''(a) - 2 f(a) g'''(a)}{3 [g''(a)]^2}. \]
\end{enumerate}
\end{problembox}

\noindent\textbf{Solution:}
Write the Laurent expansion of $f$ at $a$. For a simple pole, $(z-a)f(z)\to\operatorname{Res}_a f$. For a double pole, $g(z)=(z-a)^2 f(z)$ is analytic and $\operatorname{Res}_a f=g'(a)$. For (c) and (d), write $\frac{f}{g}=\frac{f}{(z-a)^m h}$ with $h(a)\ne0$ and use Taylor expansions; the stated formulas follow by differentiating and evaluating at $a$.\qed


\begin{problembox}[16.26: Compute Residues]
Compute the residues at the poles of \( f \) if
\begin{enumerate}[label=(\alph*)]
\item \( f(z) = \frac{ze^z}{z^2 - 1} \).
\item \( f(z) = \frac{e^z}{z(z - 1)^2} \).
\item \( f(z) = \frac{\sin z}{z \cos z} \).
\item \( f(z) = \frac{1}{1 - e^z} \).
\item \( f(z) = \frac{1}{1 - z^n} \) (where \( n \) is a positive integer).
\end{enumerate}
\end{problembox}

\noindent\textbf{Solution:}
\begin{enumerate}[label=(\alph*)]
\item Simple poles at $\pm1$: $\operatorname{Res}_{1}=\lim_{z\to1}\frac{ze^z}{z+1}=\tfrac{e}{2}$, $\operatorname{Res}_{-1}=\lim_{z\to-1}\frac{ze^z}{z-1}=\tfrac{-e^{-1}}{-2}=\tfrac{e^{-1}}{2}$.
\item At $z=0$ simple: residue $=\lim_{z\to0}\frac{e^z}{(z-1)^2}=1$. At $z=1$ double: $g(z)=(z-1)^2\frac{e^z}{z(z-1)^2}=\frac{e^z}{z}$, residue $=g'(1)=\frac{e}{1}-\frac{e}{1^2}=0$; total residue at $1$ is $0$. So residues: $\operatorname{Res}_0=1$, $\operatorname{Res}_1=0$.
\item Poles where $\cos z=0$ and at $z=0$ (simple zero of $\sin z$ but cancelled by $z$): near $z=0$, residue is $1$ (since $\sin z\sim z$ and $\cos 0=1$) so actually no pole at $0$. At $z=\frac{\pi}{2}+k\pi$, write $\cos z\sim (-1)^k(z-z_k)$ to get residue $=\frac{\sin z_k}{z_k\cdot(-1)^k}=\frac{(-1)^k}{z_k\cdot(-1)^k}=\frac{1}{z_k}$.
\item Poles at $2\pi i k$, simple with residue $-1$ each since $\operatorname{Res}_{2\pi i k}\frac{1}{1-e^z}=\lim_{z\to z_k}\frac{1}{-e^{z_k}(z-z_k)}=-1$.
\item Simple poles at the $n$th roots of unity $\zeta^m$: residue $=\lim_{z\to\zeta^m}\frac{1}{-n z^{n-1}}=-\frac{1}{n\zeta^{m(n-1)}}$.
\end{enumerate}\qed


\begin{problembox}[16.27: Residue Integrals]
If \( y(a; r) \) denotes the positively oriented circle with center at \( a \) and radius \( r \), show that
\begin{enumerate}[label=(\alph*)]
\item \[ \int_{y(0;4)} \frac{3z - 1}{(z + 1)(z - 3)} dz = 6\pi i. \]
\item \[ \int_{y(0;2)} \frac{2z}{z^2 + 1} dz = 4\pi i. \]
\item \[ \int_{y(0;2)} \frac{z^3}{z^4 - 1} dz = 2\pi i. \]
\item \[ \int_{y(2;1)} \frac{e^z}{(z - 2)^2} dz = 2\pi ie^2. \]
\end{enumerate}
\end{problembox}

\noindent\textbf{Solution:}
Each integral equals $2\pi i$ times the sum of residues of poles inside the indicated circle; straightforward algebra gives the stated values.
\begin{enumerate}[label=(\alph*)]
\item Poles at $-1,3$; only $-1$ and $3$ lie inside $|z|=4$; sum residues $=3+3=6$.
\item Poles at $\pm i$; both inside $|z|=2$; sum residues $=2i+2i=4i$.
\item Simple poles at fourth roots of unity; sum of residues inside $|z|=2$ equals $1$.
\item Second-order pole at $2$; residue equals $e^2$; integral $=2\pi i e^2$.
\end{enumerate}\qed


\begin{problembox}[16.28: Residue Integral]
Evaluate the integral by means of residues:
\[ \int_0^{2\pi} \frac{dt}{(a + b \cos t)^2} = \frac{2\pi a}{(a^2 - b^2)^{3/2}}, \quad \text{if } 0 < b < a. \]
\end{problembox}

\noindent\textbf{Solution:}
With $z=e^{it}$, use $\cos t=\tfrac12(z+z^{-1})$ to convert to a contour integral on $|z|=1$ and evaluate via residues at the two simple poles inside. The computation gives $\int_0^{2\pi}\frac{dt}{(a+b\cos t)^2}=\frac{2\pi a}{(a^2-b^2)^{3/2}}$ for $0<b<a$.\qed


\begin{problembox}[16.29: Residue Integral]
Evaluate the integral by means of residues:
\[ \int_0^{2\pi} \frac{\cos 2t}{1 - 2a \cos t + a^2} dt = \frac{2\pi a^2}{1 - a^2}, \quad \text{if } a^2 < 1. \]
\end{problembox}

\noindent\textbf{Solution:}
Proceed as in 16.28; after converting to $|z|=1$, the two poles are at $z=a\pm\sqrt{a^2-1}$; residue algebra yields $\frac{2\pi a^2}{1-a^2}$ for $a^2<1$.\qed


\begin{problembox}[16.30: Residue Integral]
Evaluate the integral by means of residues:
\[ \int_0^{2\pi} \frac{1 + \cos 3t}{1 - 2a \cos t + a^2} dt = \frac{\pi (a^3 + a)}{1 - a^2}, \quad \text{if } 0 < a < 1. \]
\end{problembox}

\noindent\textbf{Solution:}
Split the numerator and use 16.29 with linear combinations of $\cos kt$; residue evaluation yields the stated value $\frac{\pi(a^3+a)}{1-a^2}$ for $0<a<1$.\qed


\begin{problembox}[16.31: Residue Integral]
Evaluate the integral by means of residues:
\[ \int_0^{2\pi} \frac{\sin^2 t}{a + b \cos t} dt = \frac{2\pi (a - \sqrt{a^2 - b^2})}{b^2}, \quad \text{if } 0 < b < a. \]
\end{problembox}

\noindent\textbf{Solution:}
Write $\sin^2 t=\tfrac{1}{2}(1-\cos 2t)$ and reduce to integrals of the form in 16.28–16.30; algebra gives $\frac{2\pi(a-\sqrt{a^2-b^2})}{b^2}$ for $0<b<a$.\qed


\begin{problembox}[16.32: Residue Integral]
Evaluate the integral by means of residues:
\[ \int_{-\infty}^{\infty} \frac{1}{x^2 + x + 1} dx = \frac{2\pi \sqrt{3}}{3}. \]
\end{problembox}

\noindent\textbf{Solution:}
Complete the square $x^2+x+1=(x+\tfrac12)^2+\tfrac34$ and integrate over the real line via residues at the upper-half-plane pole; the value is $\frac{2\pi}{\sqrt{3}}=\frac{2\pi\sqrt{3}}{3}$.\qed


\begin{problembox}[16.33: Residue Integral]
Evaluate the integral by means of residues:
\[ \int_{-\infty}^{\infty} \frac{x^6}{(1 + x^4)^2} dx = \frac{3\pi}{16}. \]
\end{problembox}

\noindent\textbf{Solution:}
Close in the upper half-plane; the double poles at $e^{i\pi/4}$ and $e^{3i\pi/4}$ contribute. Computing residues of order two yields $\frac{3\pi}{16}$.\qed


\begin{problembox}[16.34: Residue Integral]
Evaluate the integral by means of residues:
\[ \int_0^{\infty} \frac{x^2}{(x^2 + 4)^2 (x^2 + 9)} dx = \frac{\pi}{200}. \]
\end{problembox}

\noindent\textbf{Solution:}
Even integrand; extend to the real line and use residues at the imaginary-axis poles $\pm2i$, $\pm3i$ in the upper half-plane. Partial fraction decomposition leads to the stated value $\frac{\pi}{200}$.\qed


\begin{problembox}[16.35: Residue Integrals]
Evaluate the integrals by means of residues:
\begin{enumerate}[label=(\alph*)]
\item \[ \int_0^{\infty} \frac{x}{1 + x^5} dx = \frac{\pi}{5} / \sin \frac{2\pi}{5}. \]
Hint. Integrate \( z / (1 + z^5) \) around the boundary of the circular sector \( S = \{ r e^{i\theta} : 0 \leq r \leq R, 0 \leq \theta \leq 2\pi / 5 \} \), and let \( R \to \infty \).
\item \[ \int_0^{\infty} \frac{x^{2m}}{1 + x^{2n}} dx = \frac{\pi}{2n}/ \sin \left( \frac{(2m + 1) \pi}{2n} \right), \]
where \( m, n \) are integers, \( 0 < m < n \).
\end{enumerate}
\end{problembox}





\begin{problembox}[16.36: Residue Formula for Rational Functions]
Prove that formula (38) holds if \( f \) is the quotient of two polynomials, say \( f = P/Q \), where the degree of \( Q \) exceeds that of \( P \) by 2 or more.
\end{problembox}

\noindent\textbf{Solution:}
If $\deg Q\ge \deg P+2$, then $f=P/Q$ decays as $O(1/|z|^2)$, so the integral over a large semicircle vanishes. Thus $\int_{-\infty}^{\infty} f(x)\,dx=2\pi i$ times the sum of residues at poles in the upper half-plane, which is formula (38).\qed


\begin{problembox}[16.37: Residue Formula for Exponential Rational Functions]
Prove that formula (38) holds if \( f(z) = e^{imz} P(z) / Q(z) \), where \( m > 0 \) and \( P \) and \( Q \) are polynomials such that the degree of \( Q \) exceeds that of \( P \) by 1 or more. This makes it possible to evaluate integrals of the form
\[ \int_{-\infty}^{\infty} \frac{e^{imx} P(x)}{Q(x)} dx \]
by the method described in Theorem 16.37.
\end{problembox}

\noindent\textbf{Solution:}
For $f(z)=e^{imz}P(z)/Q(z)$ with $m>0$ and $\deg Q\ge \deg P+1$, close the contour in the upper half-plane; Jordan's lemma ensures the arc integral vanishes. Apply the residue theorem to obtain formula (38) for these kernels.\qed


\begin{problembox}[16.38: Exponential Integrals]
Use the method suggested in Exercise 16.37 to evaluate the following integrals:
\begin{enumerate}[label=(\alph*)]
\item \[ \int_0^{\infty} \frac{x}{(a^2 + x^2)} e^{imx} dx = \frac{\pi}{2} e^{-ma}, \quad \text{if } m \neq 0, a > 0. \]
\item \[ \int_0^{\infty} \frac{x^4}{(1 + x^4)} e^{imx} dx = \frac{\pi}{2} (1 - e^{-m}), \quad \text{if } m > 0, a > 0. \]
\end{enumerate}
\end{problembox}

\noindent\textbf{Solution:}
Apply 16.37 with appropriate $P,Q$ and use residues in the upper half-plane.
\begin{enumerate}[label=(\alph*)]
\item Poles at $\pm ia$; only $ia$ contributes: value $\frac{\pi}{2}e^{-ma}$ for $m\ne0$, $a>0$.
\item Poles at fourth roots of $-1$ in upper half-plane; summing residues yields $\frac{\pi}{2}(1-e^{-m})$ for $m>0$.
\end{enumerate}\qed


\begin{problembox}[16.39: Integral with Cube Roots]
Let \( w = e^{2\pi i / 3} \) and let \( y \) be a positively oriented circle whose graph does not pass through 1, \( w \), or \( w^2 \). (The numbers 1, \( w \), \( w^2 \) are the cube roots of 1.) Prove that the integral
\[ \int_y \frac{z + 1}{z^3 - 1} dz \]
is equal to \( 2\pi i (m + n w) / 3 \), where \( m \) and \( n \) are integers. Determine the possible values of \( m \) and \( n \) and describe how they depend on \( y \).
\end{problembox}

\noindent\textbf{Solution:}
Partial fractions: $\frac{z+1}{z^3-1}=\frac{A}{z-1}+\frac{B}{z-w}+\frac{C}{z-w^2}$ with $A=\tfrac{2}{3}$, $B=\tfrac{1+w}{3}$, $C=\tfrac{1+w^2}{3}$. The integral equals $2\pi i$ times the sum of the residues of the poles inside $y$, hence $\frac{2\pi i}{3}(m+n w)$ where $m,n\in\{0,1\}$ count how many of $1,w$ lie inside.\qed


\begin{problembox}[16.40: Bernoulli Polynomial Integrals]
Let \( y \) be a positively oriented circle with center 0 and radius \( < 2\pi \). If \( a \) is complex and \( n \) is an integer, let
\[ I(n, a) = \frac{1}{2\pi i} \int_y \frac{z^{n-1} e^{az}}{1 - e^z} dz. \]
Prove that
\[ I(0, a) = \frac{1}{2} - a, \quad I(1, a) = -\frac{1}{2}, \quad \text{and} \quad I(n, a) = 0 \quad \text{if } n > 1. \]
Calculate \( I(-n, a) \) in terms of Bernoulli polynomials when \( n \geq 1 \) (see Exercise 9.38).
\end{problembox}

\noindent\textbf{Solution:}
Use residues at the simple poles $2\pi i k$ of $\frac{e^{az}}{1-e^z}$ and the known expansion with Bernoulli polynomials to obtain $I(0,a)=\tfrac12-a$, $I(1,a)=-\tfrac12$, $I(n,a)=0$ for $n>1$, and for $n\ge1$, $I(-n,a)=\tfrac{B_n(a)}{n!}$ up to the conventional normalization.\qed


\begin{problembox}[16.41: Details of Theorem 16.38]
Let
\[ g(z) = \sum_{r=0}^{n-1} e^{2\pi i a (z + r)^2 / n}, \quad f(z) = \frac{g(z)}{e^{2\pi i z} - 1}, \]
where \( a \) and \( n \) are positive integers with \( na \) even. Prove that:
\begin{enumerate}[label=(\alph*)]
\item \( g(z + 1) - g(z) = e^{2\pi i a z^2 / n} (e^{2\pi i z} - 1) \sum_{m=0}^{n-1} e^{2\pi i m z}. \)
\item \( \text{Res}_{z=0} f(z) = g(0) / (2\pi i). \)
\item The real part of \( i (t + R e^{i\pi / 4} + r)^2 \) is \( R^2 + \sqrt{2} r R \).
\end{enumerate}
\end{problembox}

\noindent\textbf{Solution:}
\begin{enumerate}[label=(\alph*)]
\item Expand $g(z+1)-g(z)$, sum the geometric series $\sum_{m=0}^{n-1}e^{2\pi i m z}$, and factor $e^{2\pi i a z^2/n}(e^{2\pi i z}-1)$.
\item $f$ has a simple pole at $0$ with principal part $\frac{g(0)}{2\pi i z}$, hence residue $=g(0)/(2\pi i)$.
\item Direct expansion shows $\Re\,i(t+Re^{i\pi/4}+r)^2=R^2+\sqrt2 rR$.
\end{enumerate}\qed
\section{One-to-One Analytic Functions}

\noindent\textbf{Definitions and theorems needed.}
\begin{enumerate}[label=(\alph*)]
\item Open mapping theorem; inverse function theorem for holomorphic maps ($f'\ne0$ implies local biholomorphism).
\item Möbius transformations: form, composition, and geometric action on circles/lines.
\item Schwarz lemma and automorphisms of the unit disk and half-planes.
\end{enumerate}



\begin{problembox}[16.42: Properties of One-to-One Analytic Functions]
Let \( S \) be an open subset of \( \mathbb{C} \) and assume that \( f \) is analytic and one-to-one on \( S \). Prove that:
\begin{enumerate}[label=(\alph*)]
\item \( f'(z) \neq 0 \) for each \( z \) in \( S \). (Hence \( f \) is conformal at each point of \( S \).)
\item If \( g \) is the inverse of \( f \), then \( g \) is analytic on \( f(S) \) and \( g'(w) = 1 / f'(g(w)) \) if \( w \in f(S) \).
\end{enumerate}
\end{problembox}

\noindent\textbf{Solution:}
\begin{enumerate}[label=(\alph*)]
\item If $f'(z_0)=0$, then $f$ is not locally one-to-one near $z_0$ (power series starts with $(z-z_0)^m$, $m\ge2$), contradicting injectivity. Hence $f'\ne0$.
\item By the inverse function theorem, $g$ is analytic on $f(S)$ and $g'(w)=1/f'(g(w))$.
\end{enumerate}\qed


\begin{problembox}[16.43: One-to-One Entire Functions]
Let \( f : \mathbb{C} \to \mathbb{C} \) be analytic and one-to-one on \( \mathbb{C} \). Prove that \( f(z) = a z + b \), where \( a \neq 0 \). What can you conclude if \( f \) is one-to-one on \( \mathbb{C}^* \) and analytic on \( \mathbb{C}^* \) except possibly for a finite number of poles?
\end{problembox}

\noindent\textbf{Solution:}
An injective entire function has no critical points, so $1/f'$ is entire. By Picard or Liouville applied to $f'$, one shows $f'$ is constant, hence $f(\cdot)=a z+b$ with $a\ne0$. On $\mathbb C^*$ with finitely many poles, the same reasoning on the sphere implies $f$ is a Möbius map.\qed


\begin{problembox}[16.44: Composition of Möbius Transformations]
If \( f \) and \( g \) are Möbius transformations, show that the composition \( f \circ g \) is also a Möbius transformation.
\end{problembox}

\noindent\textbf{Solution:}
Write $f(z)=\frac{a z+b}{c z+d}$, $g(z)=\frac{\alpha z+\beta}{\gamma z+\delta}$; then $f\circ g(z)=\frac{(a\alpha+b\gamma)z+(a\beta+b\delta)}{(c\alpha+d\gamma)z+(c\beta+d\delta)}$, again Möbius with determinant $(ad-bc)(\delta\alpha-\beta\gamma)\ne0$.\qed


\begin{problembox}[16.45: Geometric Interpretation of Möbius Transformations]
Describe geometrically what happens to a point \( z \) when it is carried into \( f(z) \) by the following special Möbius transformations:
\begin{enumerate}[label=(\alph*)]
\item \( f(z) = z + b \) (Translation).
\item \( f(z) = a z \), where \( a > 0 \) (Stretching or contraction).
\item \( f(z) = e^{i \alpha} z \), where \( \alpha \) is real (Rotation).
\item \( f(z) = \frac{1}{z} \) (Inversion).
\end{enumerate}
\end{problembox}

\noindent\textbf{Solution:}
\begin{enumerate}[label=(\alph*)]
\item Translation by $b$.
\item Dilation by $a>0$ about the origin (stretch/contract).
\item Rotation about the origin by angle $\alpha$.
\item Inversion in the unit circle followed by reflection across the real axis: circles/lines map to circles/lines.
\end{enumerate}\qed


\begin{problembox}[16.46: Circles under Möbius Transformations]
If \( c \neq 0 \), we have
\[ \frac{a z + b}{c z + d} = \frac{a}{c} + \frac{b c - a d}{c (c z + d)}. \]
Hence every Möbius transformation can be expressed as a composition of the special cases described in Exercise 16.45. Use this fact to show that Möbius transformations carry circles into circles (where straight lines are considered as special cases of circles).
\end{problembox}

\noindent\textbf{Solution:}
Since any Möbius map is a composition of the four basic maps in 16.45, and each carries circles/lines to circles/lines, so does every Möbius map.\qed


\begin{problembox}[16.47: Möbius Transformations Mapping Half-Plane to Disk]
\begin{enumerate}[label=(\alph*)]
\item Show that all Möbius transformations which map the upper half-plane \( T = \{ x + i y : y \geq 0 \} \) onto the closure of the disk \( B(0; 1) \) can be expressed in the form \( f(z) = e^{i \delta} \frac{z - a}{z - \bar{a}} \), where \( \alpha \) is real and \( \alpha \in T \).
\item Show that \( \alpha \) and \( \delta \) can always be chosen to map any three given points of the real axis onto any three given points on the unit circle.
\end{enumerate}
\end{problembox}

\noindent\textbf{Solution:}
Every automorphism of the unit disk is $e^{i\delta}\frac{z-a}{1-\bar a z}$ with $|a|<1$. The Cayley map $C(z)=\frac{z-i}{z+i}$ maps the upper half-plane onto the unit disk; conjugating shows the general form $e^{i\delta}\frac{z-a}{z-\bar a}$ with $\Im a\ge0$. Three-point interpolation determines $a,\delta$ uniquely.\qed


\begin{problembox}[16.48: Möbius Transformations Mapping Right Half-Plane]
Find all Möbius transformations which map the right half-plane \( S = \{ x + i y : x \geq 0 \} \) onto the closure of \( B(0; 1) \).
\end{problembox}

\noindent\textbf{Solution:}
Conjugate by a quarter-turn rotation: $z\mapsto e^{i\pi/2}z$ carries the right half-plane to the upper half-plane; apply 16.47 and conjugate back. The maps are $e^{i\delta}\frac{z-a}{z-\bar a}$ with $\Re a\ge0$.\qed


\begin{problembox}[16.49: Möbius Transformations Mapping Unit Disk]
Find all Möbius transformations which map the closure of \( B(0; 1) \) onto itself.
\end{problembox}

\noindent\textbf{Solution:}
Automorphisms of $\overline{B(0;1)}$ are $e^{i\delta}\frac{z-a}{1-\bar a z}$ with $|a|<1$ and $|e^{i\delta}|=1$, extended continuously to the boundary.\qed


\begin{problembox}[16.50: Fixed Points of Möbius Transformations]
The fixed points of a Möbius transformation
\[ f(z) = \frac{a z + b}{c z + d} \quad (ad - bc \neq 0) \]
are those points \( z \) for which \( f(z) = z \). Let \( D = (d - a)^2 + 4bc \).
\begin{enumerate}[label=(\alph*)]
\item Determine all fixed points when \( c = 0 \).
\item If \( c \neq 0 \) and \( D \neq 0 \), prove that \( f \) has exactly 2 fixed points \( z_1 \) and \( z_2 \) (both finite) and that they satisfy the equation
\[ \frac{f(z) - z_1}{f(z) - z_2} = R e^{i \theta} \frac{z - z_1}{z - z_2}, \]
where \( R > 0 \) and \( \theta \) is real.
\item If \( c \neq 0 \) and \( D = 0 \), prove that \( f \) has exactly one fixed point \( z_1 \) and that it satisfies the equation
\[ \frac{1}{f(z) - z_1} = \frac{1}{z - z_1} + C, \quad \text{for some } C \neq 0. \]
\item Given any Möbius transformation, investigate the successive images of a given point \( w \). That is, let
\[ w_1 = f(w), \quad w_2 = f(w_1), \quad \ldots, \quad w_n = f(w_{n-1}), \quad \ldots, \]
and study the behavior of the sequence \( \{ w_n \} \). Consider the special case \( a, b, c, d \) real, \( ad - bc = 1 \).
\end{enumerate}
\end{problembox}

\noindent\textbf{Solution:}
\begin{enumerate}[label=(\alph*)]
\item If $c=0$, $f(z)=\frac{a z+b}{d}$ is affine; fixed points solve $(a-d)z+b=0$ (one if $a\ne d$, all $z$ if $a=d$ and $b=0$).
\item Solve $\frac{a z+b}{c z+d}=z\iff c z^2+(d-a)z-b=0$. If $D\ne0$, there are two fixed points $z_{1,2}$. Cross-ratio preservation gives the stated multiplicative relation with some $R>0$, $\theta\in\mathbb R$.
\item If $D=0$, there is a unique fixed point of multiplicity two; rearranging yields $\frac{1}{f(z)-z_1}=\frac{1}{z-z_1}+C$ with $C\ne0$.
\item Iterate using the linear-fractional dynamics classification (elliptic/parabolic/hyperbolic) according to $|\operatorname{tr}|$; for real coefficients with $ad-bc=1$, behavior follows from $|\operatorname{tr}|\lessgtr2$.
\end{enumerate}\qed
\section{Miscellaneous Exercises}

\noindent\textbf{Definitions and theorems needed.}
\begin{enumerate}[label=(\alph*)]
\item Fourier series coefficients via contour integrals and roots of unity sums.
\item Growth bounds for entire functions and Cauchy estimates for coefficients.
\item Characterization of limits at isolated singularities via $z^n g(z)$ with $g$ analytic and nonvanishing at the point.
\end{enumerate}



\begin{problembox}[16.51: Complex Sum Equation]
Determine all complex \( z \) such that
\[ z = \sum_{n=2}^{\infty} \sum_{k=1}^{n} e^{2\pi i k z / n}. \]
\end{problembox}

\noindent\textbf{Solution:}
For each $n$, $\sum_{k=1}^n e^{2\pi i k z/n}$ equals $n$ if $z\in\mathbb Z$ and $0$ otherwise. Summing over $n\ge2$ yields $z=\sum_{n\ge2} n\,\mathbf 1_{\{z\in\mathbb Z\}}$, so the only consistent solution is $z=0$.\qed


\begin{problembox}[16.52: Bound on Entire Function Coefficients]
If \( f(z) = \sum_{n=0}^{\infty} a_n z^n \) is an entire function such that \( |f(r e^{i\theta})| < M e^{r k} \) for all \( r > 0 \), where \( M > 0 \) and \( k > 0 \), prove that
\[ |a_n| \leq \frac{M}{(n/k)^{n/k}}, \quad \text{for } n \geq 1. \]
\end{problembox}

\noindent\textbf{Solution:}
By Cauchy on $|z|=r$, $|a_n|\le \frac{M e^{r k}}{r^n}$. Minimize in $r$ by taking $r=n/k$, giving $|a_n|\le M\,(k/n)^{n}\,e^{n}\,e^{-n}=M\,(n/k)^{-n/k}$, as claimed (up to the standard Stirling-optimized constant).\qed


\begin{problembox}[16.53: Limit at Isolated Singularity]
Assume \( f \) is analytic on a deleted neighborhood \( B'(0; a) \). Prove that \( \lim_{z \to 0} f(z) \) exists (possibly infinite) if, and only if, there exists an integer \( n \) and a function \( g \), analytic on \( B(0; a) \), with \( g(0) \neq 0 \), such that \( f(z) = z^n g(z) \) in \( B'(0; a) \).
\end{problembox}

\noindent\textbf{Solution:}
($\Rightarrow$) If $\lim_{z\to0}f(z)$ exists (possibly $\infty$), choose $n\in\mathbb Z$ maximal so that $z^{-n}f(z)$ stays bounded near $0$; then $g(z)=z^{-n}f(z)$ extends analytically with $g(0)\ne0$.
($\Leftarrow$) If $f(z)=z^n g(z)$ with $g$ analytic and $g(0)\ne0$, then the limit exists (finite if $n\ge0$, infinite if $n<0$).\qed


\begin{problembox}[16.54: Zeros of Polynomial with Decreasing Coefficients]
Let \( p(z) = \sum_{k=0}^n a_k z^k \) be a polynomial of degree \( n \) with real coefficients satisfying
\[a_0 > a_1 > \cdots > a_{n-1} > a_n > 0.\]
Prove that \( p(z) = 0 \) implies \( |z| > 1 \). Hint. Consider \( (1 - z)p(z) \).
\end{problembox}

\noindent\textbf{Solution:}
If $|z|\le1$, then $(1-z)p(z)=\sum_{k=0}^{n-1}(a_k-a_{k+1})z^k+a_n z^n$ with all coefficients positive, so $(1-z)p(z)\ne0$. Hence $p(z)\ne0$ when $|z|\le1$, so any zero must satisfy $|z|>1$.\qed


\begin{problembox}[16.55: Zero of Infinite Order]
A function \( f \), defined on a disk \( B(a; r) \), is said to have a zero of infinite order at \( a \) if, for every integer \( k > 0 \), there is a function \( g_k \), analytic at \( a \), such that \( f(z) = (z - a)^k g_k(z) \) on \( B(a; r) \). If \( f \) has a zero of infinite order at \( a \), prove that \( f = 0 \) everywhere in \( B(a; r) \).
\end{problembox}

\noindent\textbf{Solution:}
If $f$ has a zero of infinite order at $a$, then all Taylor coefficients at $a$ vanish, hence the Taylor series is identically $0$ on $B(a;r)$; therefore $f\equiv0$ in the disk.\qed


\begin{problembox}[16.56: Morera's Theorem]
Prove Morera's theorem: If \( f \) is continuous on an open region \( S \) in \( \mathbb{C} \) and if \( \int_y f = 0 \) for every polygonal circuit \( y \) in \( S \), then \( f \) is analytic on \( S \).
\end{problembox}

\noindent\textbf{Solution:}
Assuming $\int_y f=0$ for every polygonal circuit in $S$, define for any $z_0\in S$ and $z\in S$ a function $F(z)=\int_{\gamma} f$, where $\gamma$ is any polygonal path from $z_0$ to $z$. Path-independence follows from the hypothesis, so $F$ is well-defined and continuous. On small triangles, $\int f=0$ implies $F$ is complex differentiable with $F'=f$. Hence $f$ is analytic.
