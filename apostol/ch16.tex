\chapter{Cauchy's Theorem and the Residue Calculus}
\section{Complex Integration; Cauchy's Integral Formulas}

\subsection*{Essential Definitions and Theorems}

\begin{definition}[Complex Path Integral]
For a piecewise $C^1$ path $\gamma: [a,b] \to \mathbb{C}$ and a continuous function $f$ defined on the image of $\gamma$, the complex path integral is:
\[\int_\gamma f(z) \, dz = \int_a^b f(\gamma(t)) \gamma'(t) \, dt\]
\end{definition}

\noindent\textbf{Importance:} Complex path integrals are the fundamental tool for integrating functions of a complex variable. They extend the concept of line integrals to the complex plane and are essential for understanding complex analysis. The parametrization of the path allows us to convert complex integrals into real integrals.



\begin{definition}[Analytic Function]
A function $f$ is analytic (or holomorphic) at a point $z_0$ if it has a complex derivative at every point in some neighborhood of $z_0$. A function is analytic on a region if it is analytic at every point in that region.
\end{definition}

\noindent\textbf{Importance:} Analytic functions are the central objects of study in complex analysis. They have remarkable properties such as infinite differentiability, power series representations, and the maximum principle. Analytic functions appear naturally in many areas of mathematics and physics.



\begin{theorem}[Fundamental Theorem for Complex Line Integrals]
If $f$ is analytic on a simply connected region $D$ and $\gamma$ is a piecewise $C^1$ path from $A$ to $B$ in $D$, then:
\[\int_\gamma f(z) \, dz = F(B) - F(A)\]
where $F$ is any antiderivative of $f$ in $D$.
\end{theorem}

\noindent\textbf{Importance:} This theorem shows that for analytic functions, the integral depends only on the endpoints of the path, not on the path itself. This is the complex analogue of the fundamental theorem of calculus and is crucial for many applications in complex analysis.



\begin{theorem}[Cauchy's Integral Formula]
If $f$ is analytic inside and on a simple closed contour $C$ and $a$ is a point inside $C$, then:
\[f(a) = \frac{1}{2\pi i} \int_C \frac{f(z)}{z-a} \, dz\]
\end{theorem}

\noindent\textbf{Importance:} Cauchy's integral formula is one of the most fundamental results in complex analysis. It shows that the value of an analytic function at any point inside a contour can be computed from its values on the contour. This formula is the foundation for many other results in complex analysis.



\begin{theorem}[Cauchy's Integral Formula for Derivatives]
If $f$ is analytic inside and on a simple closed contour $C$ and $a$ is a point inside $C$, then:
\[f^{(n)}(a) = \frac{n!}{2\pi i} \int_C \frac{f(z)}{(z-a)^{n+1}} \, dz\]
\end{theorem}

\noindent\textbf{Importance:} This theorem shows that analytic functions are infinitely differentiable and provides a formula for computing derivatives using contour integrals. This is a remarkable property that has no analogue in real analysis and is essential for many applications.



\begin{theorem}[Liouville's Theorem]
If $f$ is entire (analytic on the entire complex plane) and bounded, then $f$ is constant.
\end{theorem}

\noindent\textbf{Importance:} Liouville's theorem is a powerful result that shows the rigidity of analytic functions. It has important applications in algebra (fundamental theorem of algebra) and physics (bounded solutions to differential equations). The theorem demonstrates the strong constraints that analyticity imposes.





\begin{problembox}[16.1: Path Integral of Analytic Function]
\begin{problemstatement}
Let \( y \) be a piecewise smooth path with domain \([a, b]\) and graph \(\Gamma\). Assume that the integral \( \int_y f \) exists. Let \( S \) be an open region containing \(\Gamma\) and let \( g \) be a function such that \( g'(z) \) exists and equals \( f(z) \) for each \( z \) on \(\Gamma\). Prove that
\[\int_y f = \int_y g' = g(B) - g(A), \quad \text{where } A = y(a) \text{ and } B = y(b).\]
In particular, if \( y \) is a circuit, then \( A = B \) and the integral is 0. Hint. Apply Theorem 7.34 to each interval of continuity of \( y' \).
\end{problemstatement}
\end{problembox}

\noindent\textbf{Strategy:} Apply the fundamental theorem for complex line integrals by using the chain rule on each subinterval where \( y' \) is continuous, then integrate to obtain the difference of values at endpoints.

\bigskip\noindent\textbf{Solution:}
\begin{enumerate}[label=(\alph*)]
\item This is the case of (b) with $n=5$, $m=1$:
\[\int_0^{\infty}\frac{x}{1+x^5}\,dx=\frac{\pi}{5}\,\csc\!\Big(\frac{2\pi}{5}\Big)=\frac{\pi}{5}\,\Big/\,\sin\!\Big(\frac{2\pi}{5}\Big).\]

\item Let
\[I=\int_0^{\infty}\frac{x^{2m}}{1+x^{2n}}\,dx,\qquad 0<m<n.\]
The integrand is even, so
\[\int_{-\infty}^{\infty}\frac{x^{2m}}{1+x^{2n}}\,dx=2I.\]
Close the contour in the upper half-plane. The function $F(z)=\dfrac{z^{2m}}{1+z^{2n}}$ has simple poles at
\[z_k=e^{i\frac{(2k+1)\pi}{2n}},\quad k=0,1,\dots,n-1.\]
Since $\dfrac{d}{dz}(1+z^{2n})=2n z^{2n-1}$, the residue at $z_k$ is
\[\operatorname{Res}(F;z_k)=\frac{z_k^{2m}}{2n\,z_k^{2n-1}}=\frac{1}{2n}\,z_k^{\,2m-2n+1}.\]
Thus
\[\int_{-\infty}^{\infty}\frac{x^{2m}}{1+x^{2n}}\,dx=2\pi i\sum_{k=0}^{n-1}\operatorname{Res}(F;z_k)=\frac{2\pi i}{2n}\sum_{k=0}^{n-1} z_k^{\,2m-2n+1}.\]
Write $\theta_k=\dfrac{(2k+1)\pi}{2n}$. Then
\[\sum_{k=0}^{n-1} z_k^{\,2m-2n+1}=\sum_{k=0}^{n-1} e^{i(2m-2n+1)\theta_k}=\sum_{k=0}^{n-1} e^{-i(2(n-m)-1)\theta_k}.
\]
This is a finite geometric sum:
\begin{align*}
\sum_{k=0}^{n-1} e^{-i(2(n-m)-1)\theta_k}=& e^{-i\alpha}\sum_{k=0}^{n-1} \Big(e^{-i\beta}\Big)^k \\
=&\frac{2}{e^{-i\beta/2}\,2i\sin(\beta/2)}=\frac{1}{i\sin(\beta/2)},
\end{align*}

where $\alpha=\dfrac{(2(n-m)-1)\pi}{2n}$ and $\beta=\dfrac{(2(n-m)-1)\,2\pi}{2n}$. Since $\beta/2=\alpha$ and $\sin(\pi-\theta)=\sin\theta$, we have
\[\sin(\beta/2)=\sin\!\Big(\frac{(2m+1)\pi}{2n}\Big).\]
Therefore
\[\int_{-\infty}^{\infty}\frac{x^{2m}}{1+x^{2n}}\,dx=\frac{2\pi i}{2n}\cdot\frac{1}{i\sin\!\big(\frac{(2m+1)\pi}{2n}\big)}=\frac{\pi}{n}\,\csc\!\Big(\frac{(2m+1)\pi}{2n}\Big),\]
and hence
\[I=\int_0^{\infty}\frac{x^{2m}}{1+x^{2n}}\,dx=\frac{\pi}{2n}\,\csc\!\Big(\frac{(2m+1)\pi}{2n}\Big)=\frac{\pi}{2n}\,\Big/\,\sin\!\Big(\frac{(2m+1)\pi}{2n}\Big).\]
\end{enumerate}


\bigskip\noindent\textbf{Solution:}
Let $y:[a,b]\to\mathbb C$ be piecewise $C^1$ with $A=y(a)$ and $B=y(b)$. On each subinterval of differentiability, by the chain rule, $\frac{d}{dt}\,g(y(t))=g'(y(t))\,y'(t)=f(y(t))\,y'(t)$. Hence
\[\int_y f=\int_a^b f(y(t))\,y'(t)\,dt=\int_a^b \frac{d}{dt}g(y(t))\,dt=g(B)-g(A).\]
If $y$ is a circuit, then $A=B$ and the integral is $0$.\qed


\begin{problembox}[16.2: Verification of Cauchy's Integral Formulas]
\begin{problemstatement}
Let \( y \) be a positively oriented circular path with center 0 and radius 2. Verify each of the following by using one of Cauchy's integral formulas.
\begin{enumerate}[label=(\alph*)]
\item \[ \int_y \frac{e^z}{z} dz = 2\pi i. \]
\item \[ \int_y \frac{e^z}{z^3} dz = \pi i. \]
\item \[ \int_y \frac{e^z}{z^4} dz = \frac{\pi i}{3}. \]
\item \[ \int_y \frac{e^z}{z - 1} dz = 2\pi ie. \]
\item \[ \int_y \frac{e^z}{z(z - 1)} dz = 2\pi i(e - 1). \]
\item \[ \int_y \frac{e^z}{z^2(z - 1)} dz = 2\pi i(e - 2). \]
\end{enumerate}
\end{problemstatement}
\end{problembox}

\noindent\textbf{Strategy:} Use Cauchy's integral formula and its higher derivative forms for the first three cases, then apply the residue theorem for the remaining cases by identifying poles inside the circle and computing their residues.

\bigskip\noindent\textbf{Solution:}
All integrals are over $|z|=2$ and $f(z)=e^z$ is entire.
\begin{enumerate}[label=(\alph*)]
\item By Cauchy's formula, $\int\frac{e^z}{z}\,dz=2\pi i\,e^0=2\pi i$.
\item $\int\frac{e^z}{z^3}\,dz=\frac{2\pi i}{2!}e^0=\pi i$.
\item $\int\frac{e^z}{z^4}\,dz=\frac{2\pi i}{3!}e^0=\frac{\pi i}{3}$.
\item $\int\frac{e^z}{z-1}\,dz=2\pi i\,e^1=2\pi i e$ (since $|1|<2$).
\item Poles at $0$ and $1$ lie inside; sum residues: $\operatorname{Res}_{0}=\lim_{z\to0}\frac{e^z}{z-1}=-1$, $\operatorname{Res}_{1}=\lim_{z\to1}\frac{e^z}{z}=e$. Sum $=e-1$. Integral $=2\pi i(e-1)$.
\item Write $\frac{e^z}{z^2(z-1)}$; residues: at $z=1$ simple: $e$; at $z=0$ of order $2$: $\operatorname{Res}_{0}=\frac{d}{dz}\big[z^2\frac{e^z}{z^2(z-1)}\big]_{z=0}=\frac{d}{dz}\big[\frac{e^z}{z-1}\big]_{0}=\frac{e^z(z-1)-e^z}{(z-1)^2}\Big|_{0}=\frac{-2}{1}= -2$. Sum $=e-2$. Integral $=2\pi i(e-2)$.
\end{enumerate}\qed


\begin{problembox}[16.3: Derivative via Integral Formula]
\begin{problemstatement}
Let \( f = u + iv \) be analytic on a disk \( B(a; R) \). If \( 0 < r < R \), prove that
\[f'(a) = \frac{1}{\pi r} \int_0^{2\pi} u(a + re^{i\theta}) e^{-i\theta} d\theta.\]
\end{problemstatement}
\end{problembox}

\noindent\textbf{Strategy:} Apply Cauchy's integral formula for the derivative, parametrize the circle, and take the real part of the resulting expression to isolate the real component \( u \).

\bigskip\noindent\textbf{Solution:}
By Cauchy's formula on $|z-a|=r$,
\[f'(a)=\frac{1}{2\pi i}\int_{|z-a|=r} \frac{f(z)}{(z-a)^2}\,dz.\]
Parametrize $z=a+re^{i\theta}$, $dz=ire^{i\theta}d\theta$, and write $f=u+iv$;
\[f'(a)=\frac{1}{2\pi}\int_0^{2\pi} \frac{f(a+re^{i\theta})}{re^{i\theta}}\,d\theta=\frac{1}{2\pi r}\int_0^{2\pi} f(a+re^{i\theta})e^{-i\theta}\,d\theta.\]
Taking real parts gives the stated identity for $u$.\qed


\begin{problembox}[16.4: Stronger Liouville's Theorem]
\begin{problemstatement}
\begin{enumerate}[label=(\alph*)]
\item Prove the following stronger version of Liouville's theorem: If \( f \) is an entire function such that \( \lim_{z \to \infty} |f(z)|/|z| = 0 \), then \( f \) is a constant.
\item What can you conclude about an entire function which satisfies an inequality of the form \( |f(z)| \leq M|z|^c \) for every complex \( z \), where \( c > 0 \)?
\end{enumerate}
\end{problemstatement}
\end{problembox}

\noindent\textbf{Strategy:} Use Cauchy estimates to bound derivatives in terms of the growth condition, then show that higher derivatives vanish as the radius increases, forcing the function to be a polynomial of bounded degree.

\bigskip\noindent\textbf{Solution:}
\begin{enumerate}[label=(\alph*)]
\item By Cauchy estimate on $|z|=R$,
\[|f'(0)|\le \frac{M(R)}{R} \quad\text{with}\quad M(R)=\max_{|z|=R}|f(z)|.\]
Given $|f(z)|/|z|\to0$, we have $M(R)/R\to0$ as $R\to\infty$, hence $f'(0)=0$. Translating this argument to any $a\in\mathbb C$ shows $f'(a)=0$, so $f$ is constant.
\item If $|f(z)|\le M|z|^c$ for all $z$, then for $n>c$ the Cauchy estimate gives $|f^{(n)}(0)|\le C R^{c-n}\to0$ as $R\to\infty$, so $f^{(n)}(0)=0$. Thus $f$ is a polynomial of degree $\le \lfloor c\rfloor$ (and $f(0)=0$ if $c>0$).
\end{enumerate}\qed
\section{Poisson's Formula and Applications}

\subsection*{Essential Definitions and Theorems}

\begin{definition}[Poisson Kernel]
The Poisson kernel for the unit disk is:
\[P_r(\theta) = \frac{1-r^2}{1-2r\cos\theta + r^2} = \frac{1-r^2}{|e^{i\theta}-r|^2}\]
where $0 \leq r < 1$ and $\theta \in [0, 2\pi]$.
\end{definition}

\noindent\textbf{Importance:} The Poisson kernel is fundamental for solving boundary value problems in the unit disk. It provides a way to extend harmonic functions from the boundary to the interior and is essential for understanding the relationship between boundary values and interior values of analytic functions.



\begin{definition}[Harmonic Function]
A function $u$ is harmonic in a region $D$ if it satisfies Laplace's equation:
\[\frac{\partial^2 u}{\partial x^2} + \frac{\partial^2 u}{\partial y^2} = 0\]
in $D$. The real and imaginary parts of analytic functions are harmonic.
\end{definition}

\noindent\textbf{Importance:} Harmonic functions are fundamental in many areas of mathematics and physics. They represent equilibrium states in physical systems and have important properties such as the maximum principle and mean value property. Harmonic functions appear naturally as solutions to many partial differential equations.



\begin{theorem}[Poisson's Integral Formula]
If $f$ is analytic on the closed unit disk and $a = re^{i\alpha}$ with $0 \leq r < 1$, then:
\[f(a) = \frac{1}{2\pi} \int_0^{2\pi} \frac{(1-r^2)f(e^{i\theta})}{1-2r\cos(\alpha-\theta) + r^2} \, d\theta\]
\end{theorem}

\noindent\textbf{Importance:} Poisson's integral formula provides a way to compute the value of an analytic function at any point inside the unit disk from its values on the boundary. This is a powerful tool for understanding the behavior of analytic functions and for solving boundary value problems.



\begin{theorem}[Maximum Modulus Principle]
If $f$ is analytic in a bounded region $D$ and continuous on its closure, then $|f|$ attains its maximum on the boundary of $D$.
\end{theorem}

\noindent\textbf{Importance:} The maximum modulus principle is a fundamental property of analytic functions that shows they cannot have local maxima in their domain of analyticity. This principle has important applications in complex analysis and provides powerful tools for estimating function values.



\begin{theorem}[Mean Value Property]
If $f$ is analytic in a region containing the closed disk $|z-a| \leq r$, then:
\[f(a) = \frac{1}{2\pi} \int_0^{2\pi} f(a + re^{i\theta}) \, d\theta\]
\end{theorem}

\noindent\textbf{Importance:} The mean value property shows that the value of an analytic function at a point equals the average of its values on any circle centered at that point. This property is fundamental for understanding the behavior of analytic functions and has important applications in complex analysis.





\begin{problembox}[16.5: Poisson's Integral Formula]
\begin{problemstatement}
Assume that \( f \) is analytic on \( B(0; R) \). Let \( y \) denote the positively oriented circle with center at 0 and radius \( r \), where \( 0 < r < R \). If \( a \) is inside \( y \), show that
\[f(a) = \frac{1}{2\pi i} \int_{y} f(z) \left( \frac{1}{z - a} - \frac{1}{z - r^2 / \bar{a}} \right) dz.\]
If \( a = Ae^{i\alpha} \), show that this reduces to the formula
\[f(a) = \frac{1}{2\pi} \int_0^{2\pi} \frac{(r^2 - A^2)f(re^{i\theta})}{r^2 - 2rA \cos (\alpha - \theta) + A^2} d\theta.\]
By equating the real parts of this equation we obtain an expression known as Poisson's integral formula.
\end{problemstatement}
\end{problembox}

\noindent\textbf{Strategy:} Use Cauchy's integral formula and the fact that the second pole \( r^2/\bar{a} \) lies outside the circle, then parametrize the circle and simplify the denominator to obtain the Poisson kernel form.

\bigskip\noindent\textbf{Solution:}
Consider $F(z)=\frac{f(z)}{z-a}-\frac{f(z)}{z-r^2/\bar a}$. On $|z|=r$, the second pole lies outside and $F$ is analytic outside the circle; by Cauchy's theorem the integral equals $2\pi i$ times the residue at $z=a$, yielding the first formula. Writing $a=Ae^{i\alpha}$, $z=re^{i\theta}$ and simplifying denominators gives the real-variable form (Poisson kernel) stated.\qed


\begin{problembox}[16.6: Analytic Function Inequality]
\begin{problemstatement}
Assume that \( f \) is analytic on the closure of the disk \( B(0; 1) \). If \( |a| < 1 \), show that
\[(1 - |a|^2)f(a) = \frac{1}{2\pi i} \int_{y} f(z) \frac{1 - z\bar{a}}{z - a} dz,\]
where \( y \) is the positively oriented unit circle with center at 0. Deduce the inequality
\[(1 - |a|^2) |f(a)| \leq \frac{1}{2\pi} \int_0^{2\pi} |f(e^{i\theta})| d\theta.\]
\end{problemstatement}
\end{problembox}

\noindent\textbf{Strategy:} Apply the result from Problem 16.5 with \( R = 1 \), then take absolute values and use the triangle inequality to bound the integral.

\bigskip\noindent\textbf{Solution:}
Apply 16.5 with $R=1$ to get
\[(1-|a|^2)f(a)=\frac{1}{2\pi i}\int_{|z|=1} f(z)\,\frac{1-z\bar a}{z-a}\,dz.\]
Take absolute values, use $|dz|=|z|d\theta=d\theta$ on $|z|=1$ and the triangle inequality to get the bound
\[(1-|a|^2)|f(a)|\le \frac{1}{2\pi}\int_0^{2\pi}|f(e^{i\theta})|\,d\theta.\]\qed


\begin{problembox}[16.7: Integral with Combined Functions]
\begin{problemstatement}
Let \( f(z) = \sum_{n=0}^{\infty} \frac{2^n z^n}{3^n} \) if \( |z| < \frac{3}{2} \), and let \( g(z) = \sum_{n=0}^{\infty} (2z)^{-n} \) if \( |z| > \frac{1}{2} \). Let \( y \) be the positively oriented circular path of radius 1 and center 0, and define \( h(a) \) for \( |a| \neq 1 \) as follows:
\[h(a) = \frac{1}{2\pi i} \int_y \left( \frac{f(z)}{z - a} + \frac{a^2 g(z)}{z^2 - az} \right) dz.\]
Prove that
\[h(a) = \begin{cases} 
\frac{3}{3 - 2a} & \text{if } |a| < 1, \\ 
\frac{2a^2}{1 - 2a} & \text{if } |a| > 1.
\end{cases}\]
\end{problemstatement}
\end{problembox}

\noindent\textbf{Strategy:} Use the residue theorem by identifying which poles lie inside the unit circle for each case \( |a| < 1 \) and \( |a| > 1 \), then compute the residues using the power series representations of \( f \) and \( g \).

\bigskip\noindent\textbf{Solution:}
For $|a|<1$, only the pole at $z=a$ contributes in the first term and the pole at $z=0$ in the second is outside; compute
\[h(a)=\operatorname{Res}_{z=a}\frac{f(z)}{z-a}=f(a)=\sum_{n\ge0}\Big(\frac{2a}{3}\Big)^n=\frac{1}{1-\frac{2a}{3}}=\frac{3}{3-2a}.
\]
For $|a|>1$, the pole at $z=0$ of the second term contributes: write $\frac{a^2g(z)}{z^2-az}=\frac{a^2g(z)}{z(z-a)}$ and note only the residue at $z=0$ lies inside, giving $h(a)=\operatorname{Res}_{0}\frac{a^2g(z)}{z(z-a)}=\frac{a^2g(0)}{-a}=\frac{2a^2}{1-2a}$ since $g(0)=\sum_{n\ge0} (2\cdot0)^{-n}=1$ by analytic continuation of the geometric series outside $|z|=1/2$.\qed
\section{Taylor Expansions}

\subsection*{Essential Definitions and Theorems}

\begin{definition}[Taylor Series]
For an analytic function $f$ in a neighborhood of $a$, the Taylor series of $f$ about $a$ is:
\[f(z) = \sum_{n=0}^{\infty} \frac{f^{(n)}(a)}{n!}(z-a)^n\]
\end{definition}

\noindent\textbf{Importance:} Taylor series are fundamental for understanding analytic functions. They provide a way to represent analytic functions as infinite polynomials and are essential for computing function values, derivatives, and understanding local behavior. Taylor series are the foundation for many approximation methods.



\begin{definition}[Radius of Convergence]
The radius of convergence $R$ of a power series $\sum_{n=0}^{\infty} a_n(z-a)^n$ is the largest number such that the series converges for all $z$ with $|z-a| < R$.
\end{definition}

\noindent\textbf{Importance:} The radius of convergence determines the domain where a power series represents an analytic function. It's typically determined by the distance to the nearest singularity of the function. Understanding convergence is crucial for using power series effectively.



\begin{theorem}[Cauchy-Hadamard Formula]
The radius of convergence $R$ of the power series $\sum_{n=0}^{\infty} a_n(z-a)^n$ is given by:
\[\frac{1}{R} = \limsup_{n \to \infty} |a_n|^{1/n}\]
\end{theorem}

\noindent\textbf{Importance:} The Cauchy-Hadamard formula provides a practical way to compute the radius of convergence of a power series. It connects the growth rate of the coefficients with the convergence properties of the series, which is essential for understanding when power series representations are valid.



\begin{theorem}[Termwise Differentiation]
If $f(z) = \sum_{n=0}^{\infty} a_n(z-a)^n$ has radius of convergence $R$, then:
\[f'(z) = \sum_{n=1}^{\infty} n a_n(z-a)^{n-1}\]
and this series also has radius of convergence $R$.
\end{theorem}

\noindent\textbf{Importance:} This theorem shows that power series can be differentiated term by term within their radius of convergence. This is a remarkable property that has no analogue for general infinite series and is fundamental for many applications in analysis and differential equations.



\begin{theorem}[Uniqueness of Power Series]
If two power series $\sum_{n=0}^{\infty} a_n(z-a)^n$ and $\sum_{n=0}^{\infty} b_n(z-a)^n$ converge to the same function in a neighborhood of $a$, then $a_n = b_n$ for all $n$.
\end{theorem}

\noindent\textbf{Importance:} This theorem shows that power series representations are unique, which is crucial for many applications. It allows us to identify functions by their power series coefficients and is fundamental for techniques like coefficient comparison and series manipulation.





\begin{problembox}[16.8: Taylor Expansion of Power Series]
\begin{problemstatement}
Define \( f \) on the disk \( B(0; 1) \) by the equation \( f(z) = \sum_{n=0}^{\infty} z^n \). Find the Taylor expansion of \( f \) about the point \( a = \frac{1}{2} \) and also about the point \( a = -\frac{1}{2} \). Determine the radius of convergence in each case.
\end{problemstatement}
\end{problembox}

\noindent\textbf{Strategy:} Recognize that \( f(z) = \frac{1}{1-z} \) and use the geometric series expansion about each center by rewriting the function in terms of \( z-a \), then determine the radius of convergence as the distance to the nearest singularity.

\bigskip\noindent\textbf{Solution:}
We have $f(z)=\sum_{n\ge0}z^n=\frac{1}{1-z}$ on $|z|<1$. About $a=1/2$:
\begin{align*}
f(z)=&\frac{1}{1-z}=\frac{1}{1-a-(z-a)}=\frac{1}{1-a}\,\frac{1}{1-\frac{z-a}{1-a}}=2\sum_{n\ge0}\Big(\frac{z-a}{1-a}\Big)^n \\
=&2\sum_{n\ge0}2^n(z-\tfrac12)^n,
\end{align*}
valid for $|z-a|<1-|a|=\tfrac12$. About $a=-1/2$:
\begin{align*}
f(z)=&\frac{1}{1-z}=\frac{1}{1-(-\tfrac12)-(z+\tfrac12)}\\
=&\frac{2}{3}\sum_{n\ge0}\Big(\frac{z+\tfrac12}{\tfrac32}\Big)^n=\frac{2}{3}\sum_{n\ge0}\Big(\frac{2}{3}\Big)^n(z+\tfrac12)^n,
\end{align*}
valid for $|z-a|<1-|a|=\tfrac12$.\qed


\begin{problembox}[16.9: Taylor Expansion of Averaged Function]
\begin{problemstatement}
Assume that \( f \) has the Taylor expansion \( f(z) = \sum_{n=0}^{\infty} a(n)z^n \), valid in \( B(0; R) \). Let
\[g(z) = \frac{1}{p} \sum_{k=0}^{p-1} f(ze^{2\pi ik/p}).\]
Prove that the Taylor expansion of \( g \) consists of every \( p \)th term in that of \( f \). That is, if \( z \in B(0; R) \) we have
\[g(z) = \sum_{n=0}^{\infty} a(pn)z^{pn}.\]
\end{problemstatement}
\end{problembox}

\noindent\textbf{Strategy:} Expand each term \( f(ze^{2\pi ik/p}) \) using the Taylor series, then use the orthogonality property of roots of unity to show that only terms with exponents divisible by \( p \) survive the averaging.

\bigskip\noindent\textbf{Solution:}
Expand $f(ze^{2\pi ik/p})=\sum_{n\ge0} a(n) z^n e^{2\pi i n k/p}$. Summing over $k=0,\dots,p-1$ kills all terms with $n\not\equiv0\pmod p$ and keeps $p$ times those with $n=pm$. Hence
\[g(z)=\frac1p\sum_{k=0}^{p-1}\sum_{n\ge0} a(n)z^n e^{2\pi i nk/p}=\sum_{m\ge0}a(pm) z^{pm}.\]\qed


\begin{problembox}[16.10: Partial Sum via Integral]
\begin{problemstatement}
Assume that \( f \) has the Taylor expansion \( f(z) = \sum_{n=0}^{\infty} a_n z^n \), valid in \( B(0; R) \). Let \( s_n(z) = \sum_{k=0}^{n} a_k z^k \). If \( 0 < r < R \) and \( |z| < r \), show that
\[ s_n(z) = \frac{1}{2\pi i} \int_\gamma \frac{f(w)}{w^{n+1}} \frac{w^{n+1} -z^{n+1}}{w - z} dw, \]
where \( \gamma \) is the positively oriented circle with center at 0 and radius \( r \).
\end{problemstatement}
\end{problembox}

\noindent\textbf{Strategy:} Use Cauchy's coefficient formula to express each \( a_k \) as an integral, then sum the geometric series \( \sum_{k=0}^n \frac{z^k}{w^{k+1}} \) to obtain the desired formula.

\bigskip\noindent\textbf{Solution:}
On $|w|=r$ with $0<|z|<r$, Cauchy's coefficient formula gives
\[a_k=\frac{1}{2\pi i}\int_\gamma \frac{f(w)}{w^{k+1}}\,dw,\qquad 0\le k\le n.\]
Thus
\begin{align*}
s_n(z)=&\sum_{k=0}^n a_k z^k=\frac{1}{2\pi i}\int_\gamma f(w)\sum_{k=0}^n \frac{z^k}{w^{k+1}}\,dw \\
=&\frac{1}{2\pi i}\int_\gamma f(w)\,\frac{1}{w}\,\frac{1-(z/w)^{\,n+1}}{1-z/w}\,dw,
\end{align*}
and the finite geometric sum simplifies to
\begin{align*}
s_n(z)=&\frac{1}{2\pi i}\int_\gamma \frac{f(w)}{w^{n+1}}\,\frac{w^{n+1}-z^{n+1}}{w-z}\,dw,
\end{align*}
as required.\qed


\begin{problembox}[16.11: Product of Taylor Series]
\begin{problemstatement}
Given the Taylor expansions \( f(z) = \sum_{n=0}^{\infty} a_n z^n \) and \( g(z) = \sum_{n=0}^{\infty} b_n z^n \), valid for \( |z| < R_1 \) and \( |z| < R_2 \), respectively. Prove that if \( |z| < R_1 R_2 \) we have
\[ \frac{1}{2\pi i} \int_y \frac{f(w) g(z/w)}{w} dw = \sum_{n=0}^{\infty} a_n b_n z^n, \]
where \( y \) is the positively oriented circle of radius \( R_1 \) with center at 0.
\end{problemstatement}
\end{problembox}

\noindent\textbf{Strategy:} Expand both \( f(w) \) and \( g(z/w) \) in their respective power series, multiply them, and use the fact that the integral vanishes unless the exponent of \( w \) is \(-1\), which occurs when \( n = m \) in the double sum.

\bigskip\noindent\textbf{Solution:}
Expand $f(w)=\sum a_n w^n$ and $g(z/w)=\sum b_m (z/w)^m$; then
\[\frac{f(w)g(z/w)}{w}=\sum_{n,m\ge0} a_n b_m z^m w^{n-m-1}.\]
On $|w|=R_1$, the integral vanishes unless $n-m-1=-1$, i.e., $n=m$. Thus
\[\frac{1}{2\pi i}\int_{|w|=R_1}\frac{f(w)g(z/w)}{w}\,dw=\sum_{n\ge0} a_n b_n z^n,\]
valid when both series converge, i.e., $|z|<R_1R_2$.\qed


\begin{problembox}[16.12: Parseval's Identity and Maximum Modulus]
\begin{problemstatement}
Assume that \( f \) has the Taylor expansion \( f(z) = \sum_{n=0}^{\infty} a_n (z - a)^n \), valid in \( B(a; R) \).
\begin{enumerate}[label=(\alph*)]
\item If \( 0 \leq r < R \), deduce Parseval's identity:
\[ \frac{1}{2\pi} \int_0^{2\pi} |f(a + r e^{i\theta})|^2 d\theta = \sum_{n=0}^{\infty} |a_n|^2 r^{2n}. \]
\item Use (a) to deduce the inequality
\[ \sum_{n=0}^{\infty} |a_n|^2 r^{2n} \leq M(r)^2, \]
where \( M(r) \) is the maximum of \( |f| \) on the circle \( |z - a| = r \).
\item Use (b) to give another proof of the local maximum modulus principle (Theorem 16.27).
\end{enumerate}
\end{problemstatement}
\end{problembox}

\noindent\textbf{Strategy:} Use the orthogonality of \( e^{in\theta} \) to compute the integral of \( |f|^2 \), then apply the mean-value property and maximum principle to show that a local maximum forces the function to be constant.

\bigskip\noindent\textbf{Solution:}
\begin{enumerate}[label=(\alph*)]
\item On $|z-a|=r$, $f(a+re^{i\theta})=\sum a_n r^n e^{in\theta}$. Then
\[\frac{1}{2\pi}\int_0^{2\pi}|f(a+re^{i\theta})|^2\,d\theta=\sum_{n\ge0}|a_n|^2 r^{2n}\]
by orthogonality of $e^{in\theta}$.
\item Immediate from (a) since the average is $\le M(r)^2$.
\item If $|f|$ has a local maximum at an interior point, then for small $r$ the average equals the center value; from (b) the average $\le$ maximum on the circle, forcing constancy by the mean-value property, hence $f$ is constant.
\end{enumerate}\qed


\begin{problembox}[16.13: Schwarz's Lemma]
\begin{problemstatement}
Prove Schwarz's lemma: Let \( f \) be analytic on the disk \( B(0; 1) \). Suppose that \( f(0) = 0 \) and \( |f(z)| \leq 1 \) if \( |z| < 1 \). Then
\[ |f'(0)| \leq 1 \quad \text{and} \quad |f(z)| \leq |z|, \quad \text{if } |z| < 1. \]
If \( |f'(0)| = 1 \) or if \( |f(z_0)| = |z_0| \) for at least one \( z_0 \in B'(0; 1) \), then
\[ f(z) = e^{i\alpha} z, \]
where \( \alpha \) is real. Hint. Apply the maximum-modulus theorem to \( g \), where \( g(0) = f'(0) \) and \( g(z) = f(z)/z \) if \( z \neq 0 \).
\end{problemstatement}
\end{problembox}

\noindent\textbf{Strategy:} Define \( g(z) = f(z)/z \) for \( z \neq 0 \) and \( g(0) = f'(0) \), then apply the maximum modulus principle to show \( |g| \leq 1 \), which gives the desired bounds and forces \( g \) to be constant when equality holds.

\bigskip\noindent\textbf{Solution:}
Define $g(z)=\begin{cases} f(z)/z,& z\ne0,\\ f'(0),& z=0.\end{cases}$ Then $g$ is analytic on $B(0;1)$ and $|g(z)|\le1$ by the maximum modulus applied to $f_r(z)=f(rz)/r$. Hence $|f'(0)|=|g(0)|\le1$ and $|f(z)|\le |z|$. If equality holds at an interior point or at $0$ for the derivative, the maximum modulus forces $g$ to be constant $e^{i\alpha}$, so $f(z)=e^{i\alpha}z$.\qed
\section{Laurent Expansions, Singularities, Residues}

\subsection*{Essential Definitions and Theorems}

\begin{definition}[Laurent Series]
A Laurent series is a series of the form:
\[f(z) = \sum_{n=-\infty}^{\infty} a_n(z-a)^n\]
which converges in an annulus $r < |z-a| < R$.
\end{definition}

\noindent\textbf{Importance:} Laurent series are essential for understanding functions with singularities. They provide a way to represent functions in annular regions and are crucial for analyzing the behavior of functions near isolated singularities. Laurent series are fundamental for residue theory and many applications in complex analysis.



\begin{definition}[Isolated Singularities]
A point $a$ is an isolated singularity of $f$ if $f$ is analytic in a punctured neighborhood of $a$ but not at $a$ itself. Isolated singularities are classified as:
\begin{enumerate}[label=(\alph*)]
\item Removable: $\lim_{z \to a} f(z)$ exists and is finite
\item Pole: $\lim_{z \to a} f(z) = \infty$
\item Essential: $\lim_{z \to a} f(z)$ does not exist
\end{enumerate}
\end{definition}

\noindent\textbf{Importance:} Understanding isolated singularities is crucial for complex analysis. The classification helps determine the behavior of functions near singular points and is essential for computing residues and understanding function behavior. Different types of singularities require different analytical techniques.



\begin{theorem}[Residue Theorem]
If $f$ is analytic inside and on a simple closed contour $C$ except for isolated singularities $a_1, \ldots, a_n$ inside $C$, then:
\[\int_C f(z) \, dz = 2\pi i \sum_{k=1}^n \text{Res}(f; a_k)\]
\end{theorem}

\noindent\textbf{Importance:} The residue theorem is one of the most powerful tools in complex analysis. It allows us to compute contour integrals by summing residues, which is often much simpler than direct integration. This theorem is fundamental for many applications in physics, engineering, and applied mathematics.



\begin{theorem}[Rouché's Theorem]
If $f$ and $g$ are analytic inside and on a simple closed contour $C$ and $|g(z)| < |f(z)|$ on $C$, then $f$ and $f + g$ have the same number of zeros inside $C$.
\end{theorem}

\noindent\textbf{Importance:} Rouché's theorem provides a powerful method for counting zeros of functions. It's particularly useful for proving the existence of solutions to equations and for understanding how zeros change under perturbations. This theorem has important applications in many areas of mathematics and physics.



\begin{theorem}[Argument Principle]
If $f$ is analytic inside and on a simple closed contour $C$ except for poles, and has no zeros or poles on $C$, then:
\[\frac{1}{2\pi i} \int_C \frac{f'(z)}{f(z)} \, dz = N - P\]
where $N$ is the number of zeros and $P$ is the number of poles inside $C$, counting multiplicities.
\end{theorem}

\noindent\textbf{Importance:} The argument principle provides a way to count zeros and poles of functions using contour integrals. It's fundamental for understanding the behavior of analytic functions and has important applications in control theory, signal processing, and many other areas.





\begin{problembox}[16.14: Rouché's Theorem]
\begin{problemstatement}
Let \( f \) and \( g \) be analytic on an open region \( S \). Let \( y \) be a Jordan circuit with graph \( \Gamma \) such that both \( \Gamma \) and its inner region lie within \( S \). Suppose that \( |g(z)| < |f(z)| \) for every \( z \) on \( \Gamma \).
\begin{enumerate}[label=(\alph*)]
\item Show that
\[ \frac{1}{2\pi i} \int_{y} \frac{f'(z) + g'(z)}{f(z) + g(z)} dz = \frac{1}{2\pi i} \int_{y} \frac{f'(z)}{f(z)} dz. \]
Hint. Let \( m = \inf \{ |f(z)| - |g(z)| : z \in \Gamma \} \). Then \( m > 0 \) and hence
\[ |f(z) + t g(z)| \geq m > 0 \]
for each \( t \) in \( [0, 1] \) and each \( z \) on \( \Gamma \). Now let
\[ \phi(t) = \frac{1}{2\pi i} \int_{y} \frac{f'(z) + t g'(z)}{f(z) + t g(z)} dz, \quad \text{if } 0 \leq t \leq 1. \]
Then \( \phi \) is continuous, and hence constant, on \( [0, 1] \). Thus, \( \phi(0) = \phi(1) \).
\item Use (a) to prove that \( f \) and \( f + g \) have the same number of zeros inside\(\Gamma\) (Rouché's theorem).
\end{enumerate}
\end{problemstatement}
\end{problembox}

\noindent\textbf{Strategy:} Use a homotopy argument by considering the family of functions \( f + t g \) for \( t \in [0,1] \), show that the integral is continuous and hence constant in \( t \), then apply the argument principle to count zeros.

\bigskip\noindent\textbf{Solution:}
\begin{enumerate}[label=(\alph*)]
\item For $\phi(t)=\frac{1}{2\pi i}\int_{y}\frac{f'+t g'}{f+tg}\,dz$, continuity in $t\in[0,1]$ follows since $|f+tg|\ge m>0$ on $\Gamma$. Thus $\phi$ is constant, so $\phi(0)=\phi(1)$.
\item The integrals count zeros (with multiplicity) inside $\Gamma$ of $f$ and $f+g$. Equality of the integrals gives equality of the counts.
\end{enumerate}\qed


\begin{problembox}[16.15: Zeros of Polynomial]
\begin{problemstatement}
Let \( p \) be a polynomial of degree \( n \), say \( p(z) = a_0 + a_1 z + \cdots + a_n z^n \), where \( a_n \neq 0 \). Take \( f(z) = a_n z^n \), \( g(z) = p(z) - f(z) \) in Rouché's theorem, and prove that \( p \) has exactly \( n \) zeros in \( \mathbb{C} \).
\end{problemstatement}
\end{problembox}

\noindent\textbf{Strategy:} Apply Rouché's theorem on a large circle where the leading term \( a_n z^n \) dominates the lower degree terms, showing that \( p \) and \( a_n z^n \) have the same number of zeros inside the circle.

\bigskip\noindent\textbf{Solution:}
On a large circle $|z|=R$ with $R$ so large that $|g(z)|<|f(z)|=|a_n|R^n$ on $|z|=R$, Rouché yields that $p$ and $f$ have the same number of zeros inside, namely $n$.\qed


\begin{problembox}[16.16: Fixed Point via Rouché's Theorem]
\begin{problemstatement}
Let \( f \) be analytic on the closure of the disk \( B(0; 1) \) and suppose \( |f(z)| < 1 \) if \( |z| = 1 \). Show that there is one, and only one, point \( z_0 \in B(0; 1) \) such that \( f(z_0) = z_0 \). Hint. Use Rouché's theorem.
\end{problemstatement}
\end{problembox}

\noindent\textbf{Strategy:} Apply Rouché's theorem to the function \( h(z) = f(z) - z \) and compare it with \( -z \) on the unit circle, showing they have the same number of zeros inside the disk.

\bigskip\noindent\textbf{Solution:}
Zeros of $h(z)=f(z)-z$ in $B(0;1)$. On $|z|=1$, $|f(z)|<1=|z|$, so by Rouché, $h$ and $-z$ have the same number of zeros counted with multiplicity, namely one. Thus exactly one fixed point.\qed


\begin{problembox}[16.17: Exponential Series Zeros]
\begin{problemstatement}
Let \( p_n(z) \) denote the \( n \)th partial sum of the Taylor expansion \( e^z = \sum_{k=0}^{\infty} \frac{z^k}{k!} \). Using Rouché's theorem (or otherwise), prove that for every \( r > 0 \) there exists an \( N \) (depending on \( r \)) such that \( n \geq N \) implies \( p_n(z) \neq 0 \) for every \( z \in B(0; r) \).
\end{problemstatement}
\end{problembox}

\noindent\textbf{Strategy:} Apply Rouché's theorem by comparing \( p_n(z) \) with the tail of the series on a circle of radius \( r \), showing that for large enough \( n \), the partial sum dominates the remainder term.

\bigskip\noindent\textbf{Solution:}
Fix $r>0$. For $n$ large, on $|z|=r$, $\big|\sum_{k\ge n+1} \frac{z^k}{k!}\big|<\big|\sum_{k=0}^{n} \frac{z^k}{k!}\big|$ (ratio test and tail bound). By Rouché, $p_n$ has no zeros inside $|z|=r$. Choose $N$ accordingly.\qed


\begin{problembox}[16.18: Exponential vs Power Battle]
\begin{problemstatement}
If \( a > e \), find the number of zeros of the function \( f(z) = e^z - a z^n \) which lie inside the circle \( |z| = 1 \).
\end{problemstatement}
\end{problembox}

\noindent\textbf{Strategy:} Apply Rouché's theorem by comparing \( e^z \) with \( a z^n \) on the unit circle, using the fact that \( |e^z| \leq e < a \) when \( |z| = 1 \).

\bigskip\noindent\textbf{Solution:}
On $|z|=1$, compare $e^z$ with $a z^n$; since $|e^z|\le e< a$, by Rouché, $f(z)=e^z-az^n$ has the same number of zeros as $-az^n$, i.e., $n$ zeros inside $|z|=1$.\qed


\begin{problembox}[16.19: The Perfect Function Puzzle]
\begin{problemstatement}
Give an example of a function which has all the following properties, or else explain why there is no such function: \( f \) is analytic everywhere in \( \mathbb{C} \) except for a pole of order 2 at 0 and simple poles at \( i \) and \( -i \); \( f(z) = f(-z) \) for all \( z \); \( f(1) = 1 \); the function \( g(z) = f(1/z) \) has a zero of order 2 at \( z = 0 \); and \( \text{Res}_{z=i} f(z) = 2i \).
\end{problemstatement}
\end{problembox}

\noindent\textbf{Strategy:} Use the evenness condition to determine the form of the function, then use the residue condition and the behavior at infinity to find the specific coefficients that satisfy all requirements.

\bigskip\noindent\textbf{Solution:}
Consider $f(z)=\frac{A}{z^2}+\frac{B}{z-i}+\frac{B}{z+i}$ with evenness forcing equal simple pole residues. Evenness also forces $B$ purely imaginary and opposite at $\pm i$, consistent with $\operatorname{Res}_{i}f=2i$, so $B=2i$. Evenness and order $2$ at $0$ fix the form; choose $A$ so that $g(z)=f(1/z)$ has a zero of order $2$ at $0$, i.e., $z^{-2}f(1/z)$ vanishes to order $2$, forcing $A=0$. Normalize by $f(1)=1$ to solve $\frac{2i}{1-i}+\frac{2i}{1+i}=1$, which holds; hence one such function is
\[f(z)=\frac{2i}{z-i}+\frac{2i}{z+i}.
\]\qed


\begin{problembox}[16.20: Laurent Series Adventures]
\begin{problemstatement}
Show that each of the following Laurent expansions is valid in the region indicated:
\begin{enumerate}[label=(\alph*)]
\item \[ \frac{1}{(z - 1)(2 - z)} = \sum_{n=0}^{\infty} \frac{z^n}{2^{n+1}} + \sum_{n=1}^{\infty} \frac{1}{z^n}, \quad \text{if } 1 < |z| < 2. \]
\item \[ \frac{1}{(z - 1)(2 - z)} = \sum_{n=2}^{\infty} \frac{1 - 2^{1-n}}{z^n}, \quad \text{if } |z| > 2. \]
\end{enumerate}
\end{problemstatement}
\end{problembox}

\noindent\textbf{Strategy:} Use partial fractions to decompose the function, then expand each term in the appropriate geometric series based on the region of convergence.

\bigskip\noindent\textbf{Solution:}
\begin{enumerate}[label=(\alph*)]
\item Partial fractions: $\frac{1}{(z-1)(2-z)}=\frac{1}{z-1}+\frac{1}{2-z}$. For $1<|z|<2$, expand $\frac{1}{z-1}=\sum_{n\ge1}z^{-n}$ and $\frac{1}{2-z}=\frac{1}{2}\,\frac{1}{1-z/2}=\sum_{n\ge0}\frac{z^n}{2^{n+1}}$.
\item For $|z|>2$, expand both in powers of $1/z$ and combine coefficients to obtain the stated series.
\end{enumerate}\qed


\begin{problembox}[16.21: Bessel Functions Unveiled]
\begin{problemstatement}
For each fixed \( t \in \mathbb{C} \), define \( J_n(t) \) to be the coefficient of \( z^n \) in the Laurent expansion
\[ e^{(z - 1/z)t/2} = \sum_{n=-\infty}^{\infty} J_n(t) z^n. \]
Show that for \( n \geq 0 \) we have
\[ J_n(t) = \frac{1}{2\pi} \int_0^{2\pi} \cos (t \sin \theta - n \theta) d\theta, \]
and that \( J_{-n}(t) = (-1)^n J_n(t) \). Deduce the power series expansion
\[ J_n(t) = \sum_{k=0}^{\infty} \frac{(-1)^k (t/2)^{n + 2k}}{k! (n + k)!}, \quad (n \geq 0). \]
The function \( J_n \) is called the Bessel function of order \( n \).
\end{problemstatement}
\end{problembox}

\noindent\textbf{Strategy:} Parametrize the unit circle \( z = e^{i\theta} \) to convert the Laurent coefficient formula into a Fourier integral, then use the power series expansion of the exponential to derive the power series formula.

\bigskip\noindent\textbf{Solution:}
On $|z|=1$, set $z=e^{i\theta}$; then
\[e^{(z-1/z)t/2}=e^{i t\sin\theta}=\sum_{n\in\mathbb Z} J_n(t)e^{in\theta}.
\]
Take real parts and use Fourier coefficient extraction to get $J_n(t)=\frac{1}{2\pi}\int_0^{2\pi}\cos(t\sin\theta-n\theta)\,d\theta$ and $J_{-n}=(-1)^nJ_n$. Expanding $e^{i t\sin\theta}$ in power series and matching $e^{in\theta}$ yields the power series formula.\qed


\begin{problembox}[16.22: Riemann's Removable Singularity Magic]
\begin{problemstatement}
Prove Riemann's theorem: If \( z_0 \) is an isolated singularity of \( f \) and if \( f \) is bounded on some deleted neighborhood \( B'(z_0) \), then \( z_0 \) is a removable singularity. Hint. Estimate the integrals for the coefficients \( a_n \) in the Laurent expansion of \( f \) and show that \( a_n = 0 \) for each \( n < 0 \).
\end{problemstatement}
\end{problembox}

\noindent\textbf{Strategy:} Use Cauchy estimates on the Laurent coefficients to show that the negative coefficients vanish as the radius approaches zero, since the function is bounded near the singularity.

\bigskip\noindent\textbf{Solution:}
Write the Laurent series at $z_0$, $f(z)=\sum_{k=-\infty}^{\infty} a_k(z-z_0)^k$ on an annulus. Cauchy estimates on small circles imply $|a_{-m}|\le C r^{m-1}\max_{|z-z_0|=r}|f(z)|$, which tends to $0$ as $r\to0$ due to boundedness; hence all $a_{k}=0$ for $k<0$ and the singularity is removable.\qed


\begin{problembox}[16.23: Casorati-Weierstrass: The Wild Behavior]
\begin{problemstatement}
Prove the Casorati-Weierstrass theorem: Assume that \( z_0 \) is an essential singularity of \( f \) and let \( c \) be an arbitrary complex number. Then, for every \( \epsilon > 0 \) and every disk \( B(z_0) \), there exists a point \( z \) in \( B(z_0) \) such that \( |f(z) - c| < \epsilon \). Hint. Assume that the theorem is false and arrive at a contradiction by applying Exercise 16.22 to \( g \), where \( g(z) = 1/[f(z) - c] \).
\end{problemstatement}
\end{problembox}

\noindent\textbf{Strategy:} Use proof by contradiction: assume there exists a neighborhood where \( |f(z) - c| \geq \epsilon \), then show that \( g(z) = 1/(f(z) - c) \) would be bounded near \( z_0 \), contradicting the essential singularity by Riemann's theorem.

\bigskip\noindent\textbf{Solution:}
Assume there exist $\epsilon>0$ and a neighborhood with $|f(z)-c|\ge\epsilon$. Then $g(z)=\frac{1}{f(z)-c}$ is bounded near $z_0$ and analytic on the punctured disk; by 16.22 $g$ has a removable singularity at $z_0$, so $f$ would be bounded near $z_0$, contradicting essentiality.\qed


\begin{problembox}[16.24: Infinity: The Final Frontier]
\begin{problemstatement}
The point at infinity. A function \( f \) is said to be analytic at \( \infty \) if the function \( g \) defined by the equation \( g(z) = f(1/z) \) is analytic at the origin. Similarly, we say that \( f \) has a zero, a pole, a removable singularity, or an essential singularity at \( \infty \) if \( g \) has a zero, a pole, etc., at 0. Liouville's theorem states that a function which is analytic everywhere in \( \mathbb{C}^* \) must be a constant. Prove that
\begin{enumerate}[label=(\alph*)]
\item \( f \) is a polynomial if, and only if, the only singularity of \( f \) in \( \mathbb{C}^* \) is a pole at \( \infty \), in which case the order of the pole is equal to the degree of the polynomial.
\item \( f \) is a rational function if, and only if, \( f \) has no singularities in \( \mathbb{C}^* \) other than poles.
\end{enumerate}
\end{problemstatement}
\end{problembox}

\noindent\textbf{Strategy:} Use the transformation \( g(z) = f(1/z) \) to analyze the behavior at infinity, then apply the classification of singularities to characterize polynomials and rational functions based on their pole structure.

\bigskip\noindent\textbf{Solution:}
\begin{enumerate}[label=(\alph*)]
\item $f$ is a polynomial iff $g(z)=f(1/z)$ has only a pole at $0$. The order of the pole equals the degree since $g(z)=z^{-n}(a_n+\cdots)$.
\item $f$ is rational iff $g$ has only poles at finitely many points (including $0$), i.e., $f$ has only poles in $\mathbb C^*$.
\end{enumerate}\qed


\begin{problembox}[16.25: Residue Calculation Tricks]
\begin{problemstatement}
Derive the following "short cuts" for computing residues:
\begin{enumerate}[label=(\alph*)]
\item If \( a \) is a first order pole for \( f \), then
\[ \text{Res}_{z=a} f(z) = \lim_{z \to a} (z - a) f(z). \]
\item If \( a \) is a pole of order 2 for \( f \), then
\[ \text{Res}_{z=a} f(z) = g'(a), \]
where \( g(z) = (z - a)^2 f(z) \).
\item Suppose \( f \) and \( g \) are both analytic at \( a \), with \( f(a) \neq 0 \) and \( a \) a first-order zero for \( g \). Show that
\[ \text{Res}_{z=a} \frac{f(z)}{g(z)} = \frac{f(a)}{g'(a)}. \]
\item If \( f \) and \( g \) are as in (c), except that \( a \) is a second-order zero for \( g \), then
\[ \text{Res}_{z=a} \frac{f(z)}{g(z)} = \frac{6 f'(a) g''(a) - 2 f(a) g'''(a)}{3 [g''(a)]^2}. \]
\end{enumerate}
\end{problemstatement}
\end{problembox}

\noindent\textbf{Strategy:} Use the Laurent expansion of \( f \) at \( a \) and extract the coefficient of \( (z-a)^{-1} \) by multiplying by appropriate powers of \( (z-a) \) and taking limits or derivatives.

\bigskip\noindent\textbf{Solution:}
Write the Laurent expansion of $f$ at $a$. For a simple pole, $(z-a)f(z)\to\operatorname{Res}_a f$. For a double pole, $g(z)=(z-a)^2 f(z)$ is analytic and $\operatorname{Res}_a f=g'(a)$. For (c) and (d), write $\frac{f}{g}=\frac{f}{(z-a)^m h}$ with $h(a)\ne0$ and use Taylor expansions; the stated formulas follow by differentiating and evaluating at $a$.\qed


\begin{problembox}[16.26: Residue Detective Work]
\begin{problemstatement}
Compute the residues at the poles of \( f \) if
\begin{enumerate}[label=(\alph*)]
\item \( f(z) = \frac{ze^z}{z^2 - 1} \).
\item \( f(z) = \frac{e^z}{z(z - 1)^2} \).
\item \( f(z) = \frac{\sin z}{z \cos z} \).
\item \( f(z) = \frac{1}{1 - e^z} \).
\item \( f(z) = \frac{1}{1 - z^n} \) (where \( n \) is a positive integer).
\end{enumerate}
\end{problemstatement}
\end{problembox}

\noindent\textbf{Strategy:} Identify the poles and their orders, then apply the residue shortcuts from Problem 16.25, using limits for simple poles and derivatives for higher-order poles.

\bigskip\noindent\textbf{Solution:}
\begin{enumerate}[label=(\alph*)]
\item Simple poles at $\pm1$: $\operatorname{Res}_{1}=\lim_{z\to1}\frac{ze^z}{z+1}=\tfrac{e}{2}$, $\operatorname{Res}_{-1}=\lim_{z\to-1}\frac{ze^z}{z-1}=\tfrac{-e^{-1}}{-2}=\tfrac{e^{-1}}{2}$.
\item At $z=0$ simple: residue $=\lim_{z\to0}\frac{e^z}{(z-1)^2}=1$. At $z=1$ double: $g(z)=(z-1)^2\frac{e^z}{z(z-1)^2}=\frac{e^z}{z}$, residue $=g'(1)=\frac{e}{1}-\frac{e}{1^2}=0$; total residue at $1$ is $0$. So residues: $\operatorname{Res}_0=1$, $\operatorname{Res}_1=0$.
\item Poles where $\cos z=0$ and at $z=0$ (simple zero of $\sin z$ but cancelled by $z$): near $z=0$, residue is $1$ (since $\sin z\sim z$ and $\cos 0=1$) so actually no pole at $0$. At $z=\frac{\pi}{2}+k\pi$, write $\cos z\sim (-1)^k(z-z_k)$ to get residue $=\frac{\sin z_k}{z_k\cdot(-1)^k}=\frac{(-1)^k}{z_k\cdot(-1)^k}=\frac{1}{z_k}$.
\item Poles at $2\pi i k$, simple with residue $-1$ each since $\operatorname{Res}_{2\pi i k}\frac{1}{1-e^z}=\lim_{z\to z_k}\frac{1}{-e^{z_k}(z-z_k)}=-1$.
\item Simple poles at the $n$th roots of unity $\zeta^m$: residue $=\lim_{z\to\zeta^m}\frac{1}{-n z^{n-1}}=-\frac{1}{n\zeta^{m(n-1)}}$.
\end{enumerate}\qed


\begin{problembox}[16.27: Circle Integration Challenge]
\begin{problemstatement}
If \( y(a; r) \) denotes the positively oriented circle with center at \( a \) and radius \( r \), show that
\begin{enumerate}[label=(\alph*)]
\item \[ \int_{y(0;4)} \frac{3z - 1}{(z + 1)(z - 3)} dz = 6\pi i. \]
\item \[ \int_{y(0;2)} \frac{2z}{z^2 + 1} dz = 4\pi i. \]
\item \[ \int_{y(0;2)} \frac{z^3}{z^4 - 1} dz = 2\pi i. \]
\item \[ \int_{y(2;1)} \frac{e^z}{(z - 2)^2} dz = 2\pi ie^2. \]
\end{enumerate}
\end{problemstatement}
\end{problembox}

\noindent\textbf{Strategy:} Apply the residue theorem by identifying which poles lie inside each circle, then compute the residues at those poles and sum them.

\bigskip\noindent\textbf{Solution:}
Each integral equals $2\pi i$ times the sum of residues of poles inside the indicated circle; straightforward algebra gives the stated values.
\begin{enumerate}[label=(\alph*)]
\item Poles at $-1,3$; only $-1$ and $3$ lie inside $|z|=4$; sum residues $=3+3=6$.
\item Poles at $\pm i$; both inside $|z|=2$; sum residues $=2i+2i=4i$.
\item Simple poles at fourth roots of unity; sum of residues inside $|z|=2$ equals $1$.
\item Second-order pole at $2$; residue equals $e^2$; integral $=2\pi i e^2$.
\end{enumerate}\qed


\begin{problembox}[16.28: Trigonometric Integral Magic]
\begin{problemstatement}
Evaluate the integral by means of residues:
\[ \int_0^{2\pi} \frac{dt}{(a + b \cos t)^2} = \frac{2\pi a}{(a^2 - b^2)^{3/2}}, \quad \text{if } 0 < b < a. \]
\end{problemstatement}
\end{problembox}

\noindent\textbf{Strategy:} Convert to a contour integral on the unit circle using \( z = e^{it} \) and \( \cos t = \frac{1}{2}(z + z^{-1}) \), then apply the residue theorem to the resulting rational function.

\bigskip\noindent\textbf{Solution:}
With $z=e^{it}$, use $\cos t=\tfrac12(z+z^{-1})$ to convert to a contour integral on $|z|=1$ and evaluate via residues at the two simple poles inside. The computation gives $\int_0^{2\pi}\frac{dt}{(a+b\cos t)^2}=\frac{2\pi a}{(a^2-b^2)^{3/2}}$ for $0<b<a$.\qed


\begin{problembox}[16.29: Cosine Double Angle Adventure]
\begin{problemstatement}
Evaluate the integral by means of residues:
\[ \int_0^{2\pi} \frac{\cos 2t}{1 - 2a \cos t + a^2} dt = \frac{2\pi a^2}{1 - a^2}, \quad \text{if } a^2 < 1. \]
\end{problemstatement}
\end{problembox}

\noindent\textbf{Strategy:} Use the same trigonometric substitution as in Problem 16.28, converting \( \cos 2t \) to \( \frac{1}{2}(z^2 + z^{-2}) \) and applying the residue theorem.

\bigskip\noindent\textbf{Solution:}
Proceed as in 16.28; after converting to $|z|=1$, the two poles are at $z=a\pm\sqrt{a^2-1}$; residue algebra yields $\frac{2\pi a^2}{1-a^2}$ for $a^2<1$.\qed


\begin{problembox}[16.30: Triple Cosine Challenge]
\begin{problemstatement}
Evaluate the integral by means of residues:
\[ \int_0^{2\pi} \frac{1 + \cos 3t}{1 - 2a \cos t + a^2} dt = \frac{\pi (a^3 + a)}{1 - a^2}, \quad \text{if } 0 < a < 1. \]
\end{problemstatement}
\end{problembox}

\noindent\textbf{Strategy:} Split the numerator into two terms and use the results from Problems 16.28-16.29, converting \( \cos 3t \) to complex exponentials and applying the residue theorem.

\bigskip\noindent\textbf{Solution:}
Split the numerator and use 16.29 with linear combinations of $\cos kt$; residue evaluation yields the stated value $\frac{\pi(a^3+a)}{1-a^2}$ for $0<a<1$.\qed


\begin{problembox}[16.31: Sine Squared Surprise]
\begin{problemstatement}
Evaluate the integral by means of residues:
\[ \int_0^{2\pi} \frac{\sin^2 t}{a + b \cos t} dt = \frac{2\pi (a - \sqrt{a^2 - b^2})}{b^2}, \quad \text{if } 0 < b < a. \]
\end{problemstatement}
\end{problembox}

\noindent\textbf{Strategy:} Use the identity \( \sin^2 t = \frac{1}{2}(1 - \cos 2t) \) to reduce to integrals of the form in Problems 16.28-16.30, then apply the residue theorem.

\bigskip\noindent\textbf{Solution:}
Write $\sin^2 t=\tfrac{1}{2}(1-\cos 2t)$ and reduce to integrals of the form in 16.28–16.30; algebra gives $\frac{2\pi(a-\sqrt{a^2-b^2})}{b^2}$ for $0<b<a$.\qed


\begin{problembox}[16.32: Real Line Integration Quest]
\begin{problemstatement}
Evaluate the integral by means of residues:
\[ \int_{-\infty}^{\infty} \frac{1}{x^2 + x + 1} dx = \frac{2\pi \sqrt{3}}{3}. \]
\end{problemstatement}
\end{problembox}

\noindent\textbf{Strategy:} Complete the square in the denominator, then close the contour in the upper half-plane and apply the residue theorem at the pole in the upper half-plane.

\bigskip\noindent\textbf{Solution:}
Complete the square $x^2+x+1=(x+\tfrac12)^2+\tfrac34$ and integrate over the real line via residues at the upper-half-plane pole; the value is $\frac{2\pi}{\sqrt{3}}=\frac{2\pi\sqrt{3}}{3}$.\qed


\begin{problembox}[16.33: Power of Six Exploration]
\begin{problemstatement}
Evaluate the integral by means of residues:
\[ \int_{-\infty}^{\infty} \frac{x^6}{(1 + x^4)^2} dx = \frac{3\pi}{16}. \]
\end{problemstatement}
\end{problembox}

\noindent\textbf{Strategy:} Close the contour in the upper half-plane and compute residues at the double poles at the fourth roots of unity in the upper half-plane.

\bigskip\noindent\textbf{Solution:}
Close in the upper half-plane; the double poles at $e^{i\pi/4}$ and $e^{3i\pi/4}$ contribute. Computing residues of order two yields $\frac{3\pi}{16}$.\qed


\begin{problembox}[16.34: Mixed Powers Mystery]
\begin{problemstatement}
Evaluate the integral by means of residues:
\[ \int_0^{\infty} \frac{x^2}{(x^2 + 4)^2 (x^2 + 9)} dx = \frac{\pi}{200}. \]
\end{problemstatement}
\end{problembox}

\noindent\textbf{Strategy:} Use the evenness of the integrand to extend to the real line, then close in the upper half-plane and compute residues at the poles \( \pm 2i \) (double) and \( \pm 3i \) (simple).

\bigskip\noindent\textbf{Solution:}
Even integrand; extend to the real line and use residues at the imaginary-axis poles $\pm2i$, $\pm3i$ in the upper half-plane. Partial fraction decomposition leads to the stated value $\frac{\pi}{200}$.\qed


\begin{problembox}[16.35: Sector Contour Adventures]
\begin{problemstatement}
Evaluate the integrals by means of residues:
\begin{enumerate}[label=(\alph*)]
\item \[ \int_0^{\infty} \frac{x}{1 + x^5} dx = \frac{\pi}{5} / \sin \frac{2\pi}{5}. \]
Hint. Integrate \( z / (1 + z^5) \) around the boundary of the circular sector \( S = \{ r e^{i\theta} : 0 \leq r \leq R, 0 \leq \theta \leq 2\pi / 5 \} \), and let \( R \to \infty \).
\item \[ \int_0^{\infty} \frac{x^{2m}}{1 + x^{2n}} dx = \frac{\pi}{2n}/ \sin \left( \frac{(2m + 1) \pi}{2n} \right), \]
where \( m, n \) are integers, \( 0 < m < n \).
\end{enumerate}
\end{problemstatement}
\end{problembox}

\noindent\textbf{Strategy:} Use a sector contour with angle \( 2\pi/n \) to exploit the symmetry of the roots of unity, then apply the residue theorem and use the geometric sum formula for roots of unity.


\begin{problembox}[16.36: Residue Formula for Rational Functions]
\begin{problemstatement}
Prove that formula (38) holds if \( f \) is the quotient of two polynomials, say \( f = P/Q \), where the degree of \( Q \) exceeds that of \( P \) by 2 or more.
\end{problemstatement}
\end{problembox}

\noindent\textbf{Strategy:} Show that the integral over a large semicircle vanishes due to the degree condition, then apply the residue theorem to the poles in the upper half-plane.

\bigskip\noindent\textbf{Solution:}
If $\deg Q\ge \deg P+2$, then $f=P/Q$ decays as $O(1/|z|^2)$, so the integral over a large semicircle vanishes. Thus $\int_{-\infty}^{\infty} f(x)\,dx=2\pi i$ times the sum of residues at poles in the upper half-plane, which is formula (38).\qed


\begin{problembox}[16.37: Residue Formula for Exponential Rational Functions]
\begin{problemstatement}
Prove that formula (38) holds if \( f(z) = e^{imz} P(z) / Q(z) \), where \( m > 0 \) and \( P \) and \( Q \) are polynomials such that the degree of \( Q \) exceeds that of \( P \) by 1 or more. This makes it possible to evaluate integrals of the form
\[ \int_{-\infty}^{\infty} \frac{e^{imx} P(x)}{Q(x)} dx \]
by the method described in Theorem 16.37.
\end{problemstatement}
\end{problembox}

\noindent\textbf{Strategy:} Use Jordan's lemma to show that the integral over a large semicircle in the upper half-plane vanishes due to the exponential decay, then apply the residue theorem.

\bigskip\noindent\textbf{Solution:}
For $f(z)=e^{imz}P(z)/Q(z)$ with $m>0$ and $\deg Q\ge \deg P+1$, close the contour in the upper half-plane; Jordan's lemma ensures the arc integral vanishes. Apply the residue theorem to obtain formula (38) for these kernels.\qed


\begin{problembox}[16.38: Exponential Integrals]
\begin{problemstatement}
Use the method suggested in Exercise 16.37 to evaluate the following integrals:
\begin{enumerate}[label=(\alph*)]
\item \[ \int_0^{\infty} \frac{x}{(a^2 + x^2)} e^{imx} dx = \frac{\pi}{2} e^{-ma}, \quad \text{if } m \neq 0, a > 0. \]
\item \[ \int_0^{\infty} \frac{x^4}{(1 + x^4)} e^{imx} dx = \frac{\pi}{2} (1 - e^{-m}), \quad \text{if } m > 0, a > 0. \]
\end{enumerate}
\end{problemstatement}
\end{problembox}

\noindent\textbf{Strategy:} Apply the method from Problem 16.37 by closing the contour in the upper half-plane and computing residues at the poles in the upper half-plane.

\bigskip\noindent\textbf{Solution:}
Apply 16.37 with appropriate $P,Q$ and use residues in the upper half-plane.
\begin{enumerate}[label=(\alph*)]
\item Poles at $\pm ia$; only $ia$ contributes: value $\frac{\pi}{2}e^{-ma}$ for $m\ne0$, $a>0$.
\item Poles at fourth roots of $-1$ in upper half-plane; summing residues yields $\frac{\pi}{2}(1-e^{-m})$ for $m>0$.
\end{enumerate}\qed


\begin{problembox}[16.39: Integral with Cube Roots]
\begin{problemstatement}
Let \( w = e^{2\pi i / 3} \) and let \( y \) be a positively oriented circle whose graph does not pass through 1, \( w \), or \( w^2 \). (The numbers 1, \( w \), \( w^2 \) are the cube roots of 1.) Prove that the integral
\[ \int_y \frac{z + 1}{z^3 - 1} dz \]
is equal to \( 2\pi i (m + n w) / 3 \), where \( m \) and \( n \) are integers. Determine the possible values of \( m \) and \( n \) and describe how they depend on \( y \).
\end{problemstatement}
\end{problembox}

\noindent\textbf{Strategy:} Use partial fractions to decompose the integrand, then apply the residue theorem to count which poles lie inside the contour \( y \).

\bigskip\noindent\textbf{Solution:}
Partial fractions: $\frac{z+1}{z^3-1}=\frac{A}{z-1}+\frac{B}{z-w}+\frac{C}{z-w^2}$ with $A=\tfrac{2}{3}$, $B=\tfrac{1+w}{3}$, $C=\tfrac{1+w^2}{3}$. The integral equals $2\pi i$ times the sum of the residues of the poles inside $y$, hence $\frac{2\pi i}{3}(m+n w)$ where $m,n\in\{0,1\}$ count how many of $1,w$ lie inside.\qed


\begin{problembox}[16.40: Bernoulli Polynomial Integrals]
\begin{problemstatement}
Let \( y \) be a positively oriented circle with center 0 and radius \( < 2\pi \). If \( a \) is complex and \( n \) is an integer, let
\[ I(n, a) = \frac{1}{2\pi i} \int_y \frac{z^{n-1} e^{az}}{1 - e^z} dz. \]
Prove that
\[ I(0, a) = \frac{1}{2} - a, \quad I(1, a) = -\frac{1}{2}, \quad \text{and} \quad I(n, a) = 0 \quad \text{if } n > 1. \]
Calculate \( I(-n, a) \) in terms of Bernoulli polynomials when \( n \geq 1 \) (see Exercise 9.38).
\end{problemstatement}
\end{problembox}

\noindent\textbf{Strategy:} Use the residue theorem at the simple poles \( 2\pi i k \) of \( \frac{e^{az}}{1-e^z} \), then use the known expansion involving Bernoulli polynomials to evaluate the series.

\bigskip\noindent\textbf{Solution:}
Use residues at the simple poles $2\pi i k$ of $\frac{e^{az}}{1-e^z}$ and the known expansion with Bernoulli polynomials to obtain $I(0,a)=\tfrac12-a$, $I(1,a)=-\tfrac12$, $I(n,a)=0$ for $n>1$, and for $n\ge1$, $I(-n,a)=\tfrac{B_n(a)}{n!}$ up to the conventional normalization.\qed


\begin{problembox}[16.41: Details of Theorem 16.38]
\begin{problemstatement}
Let
\[ g(z) = \sum_{r=0}^{n-1} e^{2\pi i a (z + r)^2 / n}, \quad f(z) = \frac{g(z)}{e^{2\pi i z} - 1}, \]
where \( a \) and \( n \) are positive integers with \( na \) even. Prove that:
\begin{enumerate}[label=(\alph*)]
\item \( g(z + 1) - g(z) = e^{2\pi i a z^2 / n} (e^{2\pi i z} - 1) \sum_{m=0}^{n-1} e^{2\pi i m z}. \)
\item \( \text{Res}_{z=0} f(z) = g(0) / (2\pi i). \)
\item The real part of \( i (t + R e^{i\pi / 4} + r)^2 \) is \( R^2 + \sqrt{2} r R \).
\end{enumerate}
\end{problemstatement}
\end{problembox}

\noindent\textbf{Strategy:} Use direct computation for (a) by expanding the difference, apply the residue formula for simple poles for (b), and compute the real part directly for (c).

\bigskip\noindent\textbf{Solution:}
\begin{enumerate}[label=(\alph*)]
\item Expand $g(z+1)-g(z)$, sum the geometric series $\sum_{m=0}^{n-1}e^{2\pi i m z}$, and factor $e^{2\pi i a z^2/n}(e^{2\pi i z}-1)$.
\item $f$ has a simple pole at $0$ with principal part $\frac{g(0)}{2\pi i z}$, hence residue $=g(0)/(2\pi i)$.
\item Direct expansion shows $\Re\,i(t+Re^{i\pi/4}+r)^2=R^2+\sqrt2 rR$.
\end{enumerate}\qed
\section{One-to-One Analytic Functions}

\subsection*{Essential Definitions and Theorems}

\begin{definition}[Univalent Function]
A function $f$ is univalent (or schlicht) in a region $D$ if it is analytic and one-to-one in $D$. That is, $f(z_1) = f(z_2)$ implies $z_1 = z_2$ for all $z_1, z_2 \in D$.
\end{definition}

\noindent\textbf{Importance:} Univalent functions are important in conformal mapping theory and preserve angles between curves.

\begin{definition}[Möbius Transformation]
A Möbius transformation (or linear fractional transformation) is a function of the form:
\[f(z) = \frac{az + b}{cz + d}\]
where $a, b, c, d \in \mathbb{C}$ and $ad - bc \neq 0$.
\end{definition}

\noindent\textbf{Importance:} Möbius transformations are important conformal mappings that preserve circles and lines.

\begin{theorem}[Open Mapping Theorem]
If $f$ is analytic and non-constant in a region $D$, then $f(D)$ is an open set.
\end{theorem}

\noindent\textbf{Importance:} The open mapping theorem shows that non-constant analytic functions preserve openness. This is fundamental for understanding the behavior of analytic functions and is crucial for many results in complex analysis, including the inverse function theorem.



\begin{theorem}[Inverse Function Theorem for Analytic Functions]
If $f$ is analytic at $z_0$ and $f'(z_0) \neq 0$, then $f$ has a local analytic inverse in a neighborhood of $f(z_0)$.
\end{theorem}

\noindent\textbf{Importance:} This theorem provides conditions under which analytic functions have local inverses. It's essential for understanding when functions can be "undone" and is fundamental for many applications in complex analysis and differential equations.



\begin{theorem}[Schwarz Lemma]
If $f$ is analytic in the unit disk with $f(0) = 0$ and $|f(z)| \leq 1$ for $|z| < 1$, then $|f(z)| \leq |z|$ for all $z$ in the unit disk, and $|f'(0)| \leq 1$.
\end{theorem}

\noindent\textbf{Importance:} The Schwarz lemma provides sharp bounds on the behavior of analytic functions in the unit disk. It's fundamental for understanding the geometry of analytic functions and has important applications in geometric function theory and extremal problems.





\begin{problembox}[16.42: Properties of One-to-One Analytic Functions]
\begin{problemstatement}
Let \( S \) be an open subset of \( \mathbb{C} \) and assume that \( f \) is analytic and one-to-one on \( S \). Prove that:
\begin{enumerate}[label=(\alph*)]
\item \( f'(z) \neq 0 \) for each \( z \) in \( S \). (Hence \( f \) is conformal at each point of \( S \).)
\item If \( g \) is the inverse of \( f \), then \( g \) is analytic on \( f(S) \) and \( g'(w) = 1 / f'(g(w)) \) if \( w \in f(S) \).
\end{enumerate}
\end{problemstatement}
\end{problembox}

\noindent\textbf{Strategy:} Use the open mapping theorem and the fact that a zero derivative would make the function not locally injective, then apply the inverse function theorem for holomorphic functions.

\bigskip\noindent\textbf{Solution:}
\begin{enumerate}[label=(\alph*)]
\item If $f'(z_0)=0$, then $f$ is not locally one-to-one near $z_0$ (power series starts with $(z-z_0)^m$, $m\ge2$), contradicting injectivity. Hence $f'\ne0$.
\item By the inverse function theorem, $g$ is analytic on $f(S)$ and $g'(w)=1/f'(g(w))$.
\end{enumerate}\qed


\begin{problembox}[16.43: One-to-One Entire Functions]
\begin{problemstatement}
Let \( f : \mathbb{C} \to \mathbb{C} \) be analytic and one-to-one on \( \mathbb{C} \). Prove that \( f(z) = a z + b \), where \( a \neq 0 \). What can you conclude if \( f \) is one-to-one on \( \mathbb{C}^* \) and analytic on \( \mathbb{C}^* \) except possibly for a finite number of poles?
\end{problemstatement}
\end{problembox}

\noindent\textbf{Strategy:} Use the fact that an injective entire function has no critical points, so \( 1/f' \) is entire, then apply Picard's theorem or Liouville's theorem to show \( f' \) is constant.

\bigskip\noindent\textbf{Solution:}
An injective entire function has no critical points, so $1/f'$ is entire. By Picard or Liouville applied to $f'$, one shows $f'$ is constant, hence $f(\cdot)=a z+b$ with $a\ne0$. On $\mathbb C^*$ with finitely many poles, the same reasoning on the sphere implies $f$ is a Möbius map.\qed


\begin{problembox}[16.44: Composition of Möbius Transformations]
\begin{problemstatement}
If \( f \) and \( g \) are Möbius transformations, show that the composition \( f \circ g \) is also a Möbius transformation.
\end{problemstatement}
\end{problembox}

\noindent\textbf{Strategy:} Write both transformations in the form \( \frac{az+b}{cz+d} \) and compute the composition directly, showing it has the same form with a non-zero determinant.

\bigskip\noindent\textbf{Solution:}
Write $f(z)=\frac{a z+b}{c z+d}$, $g(z)=\frac{\alpha z+\beta}{\gamma z+\delta}$; then \\ $f\circ g(z)=\frac{(a\alpha+b\gamma)z+(a\beta+b\delta)}{(c\alpha+d\gamma)z+(c\beta+d\delta)}$, again Möbius with determinant $(ad-bc)(\delta\alpha-\beta\gamma)\ne0$.\qed


\begin{problembox}[16.45: Geometric Interpretation of Möbius Transformations]
\begin{problemstatement}
Describe geometrically what happens to a point \( z \) when it is carried into \( f(z) \) by the following special Möbius transformations:
\begin{enumerate}[label=(\alph*)]
\item \( f(z) = z + b \) (Translation).
\item \( f(z) = a z \), where \( a > 0 \) (Stretching or contraction).
\item \( f(z) = e^{i \alpha} z \), where \( \alpha \) is real (Rotation).
\item \( f(z) = \frac{1}{z} \) (Inversion).
\end{enumerate}
\end{problemstatement}
\end{problembox}

\noindent\textbf{Strategy:} Analyze each transformation by considering its effect on the complex plane: translation moves points, dilation scales distances, rotation turns points around the origin, and inversion reflects across the unit circle.

\bigskip\noindent\textbf{Solution:}
\begin{enumerate}[label=(\alph*)]
\item Translation by $b$.
\item Dilation by $a>0$ about the origin (stretch/contract).
\item Rotation about the origin by angle $\alpha$.
\item Inversion in the unit circle followed by reflection across the real axis: circles/lines map to circles/lines.
\end{enumerate}\qed


\begin{problembox}[16.46: Circles under Möbius Transformations]
\begin{problemstatement}
If \( c \neq 0 \), we have
\[ \frac{a z + b}{c z + d} = \frac{a}{c} + \frac{b c - a d}{c (c z + d)}. \]
Hence every Möbius transformation can be expressed as a composition of the special cases described in Exercise 16.45. Use this fact to show that Möbius transformations carry circles into circles (where straight lines are considered as special cases of circles).
\end{problemstatement}
\end{problembox}

\noindent\textbf{Strategy:} Use the decomposition of Möbius transformations into basic transformations and the fact that each basic transformation preserves circles and lines.

\bigskip\noindent\textbf{Solution:}
Since any Möbius map is a composition of the four basic maps in 16.45, and each carries circles/lines to circles/lines, so does every Möbius map.\qed


\begin{problembox}[16.47: Möbius Transformations Mapping Half-Plane to Disk]
\begin{problemstatement}
\begin{enumerate}[label=(\alph*)]
\item Show that all Möbius transformations which map the upper half-plane \( T = \{ x + i y : y \geq 0 \} \) onto the closure of the disk \( B(0; 1) \) can be expressed in the form \( f(z) = e^{i \delta} \frac{z - a}{z - \bar{a}} \), where \( \alpha \) is real and \( \alpha \in T \).
\item Show that \( \alpha \) and \( \delta \) can always be chosen to map any three given points of the real axis onto any three given points on the unit circle.
\end{enumerate}
\end{problemstatement}
\end{problembox}

\noindent\textbf{Strategy:} Use the Cayley transform to map the upper half-plane to the unit disk, then conjugate by automorphisms of the unit disk to get the general form, and use three-point interpolation to determine the parameters.

\bigskip\noindent\textbf{Solution:}
Every automorphism of the unit disk is $e^{i\delta}\frac{z-a}{1-\bar a z}$ with $|a|<1$. The Cayley map $C(z)=\frac{z-i}{z+i}$ maps the upper half-plane onto the unit disk; conjugating shows the general form $e^{i\delta}\frac{z-a}{z-\bar a}$ with $\Im a\ge0$. Three-point interpolation determines $a,\delta$ uniquely.\qed


\begin{problembox}[16.48: Möbius Transformations Mapping Right Half-Plane]
\begin{problemstatement}
Find all Möbius transformations which map the right half-plane \( S = \{ x + i y : x \geq 0 \} \) onto the closure of \( B(0; 1) \).
\end{problemstatement}
\end{problembox}

\noindent\textbf{Strategy:} Conjugate the result from Problem 16.47 by a quarter-turn rotation to map the right half-plane to the upper half-plane, then apply the known transformation.

\bigskip\noindent\textbf{Solution:}
Conjugate by a quarter-turn rotation: $z\mapsto e^{i\pi/2}z$ carries the right half-plane to the upper half-plane; apply 16.47 and conjugate back. The maps are $e^{i\delta}\frac{z-a}{z-\bar a}$ with $\Re a\ge0$.\qed


\begin{problembox}[16.49: Möbius Transformations Mapping Unit Disk]
\begin{problemstatement}
Find all Möbius transformations which map the closure of \( B(0; 1) \) onto itself.
\end{problemstatement}
\end{problembox}

\noindent\textbf{Strategy:} Use the known form of automorphisms of the unit disk, which are the Möbius transformations that preserve the unit disk.

\bigskip\noindent\textbf{Solution:}
Automorphisms of $\overline{B(0;1)}$ are $e^{i\delta}\frac{z-a}{1-\bar a z}$ with $|a|<1$ and $|e^{i\delta}|=1$, extended continuously to the boundary.\qed


\begin{problembox}[16.50: Fixed Points of Möbius Transformations]
\begin{problemstatement}
The fixed points of a Möbius transformation
\[ f(z) = \frac{a z + b}{c z + d} \quad (ad - bc \neq 0) \]
are those points \( z \) for which \( f(z) = z \). Let \( D = (d - a)^2 + 4bc \).
\begin{enumerate}[label=(\alph*)]
\item Determine all fixed points when \( c = 0 \).
\item If \( c \neq 0 \) and \( D \neq 0 \), prove that \( f \) has exactly 2 fixed points \( z_1 \) and \( z_2 \) (both finite) and that they satisfy the equation
\[ \frac{f(z) - z_1}{f(z) - z_2} = R e^{i \theta} \frac{z - z_1}{z - z_2}, \]
where \( R > 0 \) and \( \theta \) is real.
\item If \( c \neq 0 \) and \( D = 0 \), prove that \( f \) has exactly one fixed point \( z_1 \) and that it satisfies the equation
\[ \frac{1}{f(z) - z_1} = \frac{1}{z - z_1} + C, \quad \text{for some } C \neq 0. \]
\item Given any Möbius transformation, investigate the successive images of a given point \( w \). That is, let
\[ w_1 = f(w), \quad w_2 = f(w_1), \quad \ldots, \quad w_n = f(w_{n-1}), \quad \ldots, \]
and study the behavior of the sequence \( \{ w_n \} \). Consider the special case \( a, b, c, d \) real, \( ad - bc = 1 \).
\end{enumerate}
\end{problemstatement}
\end{problembox}

\noindent\textbf{Strategy:} Solve the fixed point equation \( f(z) = z \) to get a quadratic equation, then use cross-ratio preservation and the classification of Möbius transformations based on their fixed points to analyze the dynamics.

\bigskip\noindent\textbf{Solution:}
\begin{enumerate}[label=(\alph*)]
\item If $c=0$, $f(z)=\frac{a z+b}{d}$ is affine; fixed points solve $(a-d)z+b=0$ (one if $a\ne d$, all $z$ if $a=d$ and $b=0$).
\item Solve $\frac{a z+b}{c z+d}=z\iff c z^2+(d-a)z-b=0$. If $D\ne0$, there are two fixed points $z_{1,2}$. Cross-ratio preservation gives the stated multiplicative relation with some $R>0$, $\theta\in\mathbb R$.
\item If $D=0$, there is a unique fixed point of multiplicity two; rearranging yields $\frac{1}{f(z)-z_1}=\frac{1}{z-z_1}+C$ with $C\ne0$.
\item Iterate using the linear-fractional dynamics classification (elliptic/parabolic/hyperbolic) according to $|\operatorname{tr}|$; for real coefficients with $ad-bc=1$, behavior follows from $|\operatorname{tr}|\lessgtr2$.
\end{enumerate}\qed
\section{Miscellaneous Exercises}

\subsection*{Essential Definitions and Theorems}

\begin{definition}[Entire Function]
A function $f$ is entire if it is analytic on the entire complex plane $\mathbb{C}$. Examples include polynomials, exponential functions, and trigonometric functions.
\end{definition}

\noindent\textbf{Importance:} Entire functions are fundamental objects in complex analysis. They have remarkable properties such as infinite differentiability and power series representations that converge everywhere. Understanding entire functions is crucial for many applications in mathematics and physics.



\begin{definition}[Growth Order]
An entire function $f$ has growth order $\rho$ if:
\[\limsup_{r \to \infty} \frac{\log \log M(r)}{\log r} = \rho\]
where $M(r) = \max_{|z| = r} |f(z)|$.
\end{definition}

\noindent\textbf{Importance:} Growth order provides a way to classify entire functions by how fast they grow as $|z| \to \infty$.

\begin{theorem}[Cauchy Estimates]
If $f(z) = \sum_{n=0}^{\infty} a_n z^n$ is analytic in $|z| < R$ and $|f(z)| \leq M$ for $|z| = r < R$, then:
\[|a_n| \leq \frac{M}{r^n}\]
\end{theorem}

\noindent\textbf{Importance:} Cauchy estimates provide bounds on the coefficients of power series in terms of the maximum modulus of the function. These estimates are fundamental for understanding the relationship between function values and power series coefficients and have many applications in complex analysis.



\begin{theorem}[Morera's Theorem]
If $f$ is continuous on a region $D$ and $\int_\gamma f(z) \, dz = 0$ for every closed contour $\gamma$ in $D$, then $f$ is analytic in $D$.
\end{theorem}

\noindent\textbf{Importance:} Morera's theorem provides a converse to Cauchy's theorem, showing that if all contour integrals vanish, then the function must be analytic. This is a powerful tool for proving analyticity and has important applications in complex analysis and function theory.



\begin{theorem}[Characterization of Isolated Singularities]
If $f$ has an isolated singularity at $a$, then $\lim_{z \to a} f(z)$ exists (possibly infinite) if and only if there exists an integer $n$ and a function $g$ analytic at $a$ with $g(a) \neq 0$ such that $f(z) = (z-a)^n g(z)$ near $a$.
\end{theorem}

\noindent\textbf{Importance:} This theorem provides a complete characterization of when limits exist at isolated singularities. It connects the behavior of functions near singularities with their Laurent series representations and is essential for understanding function behavior near singular points.





\begin{problembox}[16.51: Complex Sum Equation]
\begin{problemstatement}
Determine all complex \( z \) such that
\[ z = \sum_{n=2}^{\infty} \sum_{k=1}^{n} e^{2\pi i k z / n}. \]
\end{problemstatement}
\end{problembox}

\noindent\textbf{Strategy:} Use the fact that \( \sum_{k=1}^n e^{2\pi i k z/n} \) equals \( n \) if \( z \) is an integer and \( 0 \) otherwise, then analyze the resulting equation.

\bigskip\noindent\textbf{Solution:}
For each $n$, $\sum_{k=1}^n e^{2\pi i k z/n}$ equals $n$ if $z\in\mathbb Z$ and $0$ otherwise. Summing over $n\ge2$ yields $z=\sum_{n\ge2} n\,\mathbf 1_{\{z\in\mathbb Z\}}$, so the only consistent solution is $z=0$.\qed


\begin{problembox}[16.52: Bound on Entire Function Coefficients]
\begin{problemstatement}
If \( f(z) = \sum_{n=0}^{\infty} a_n z^n \) is an entire function such that \( |f(r e^{i\theta})| < M e^{r k} \) for all \( r > 0 \), where \( M > 0 \) and \( k > 0 \), prove that
\[ |a_n| \leq \frac{M}{(n/k)^{n/k}}, \quad \text{for } n \geq 1. \]
\end{problemstatement}
\end{problembox}

\noindent\textbf{Strategy:} Use Cauchy estimates on circles of radius \( r \) and optimize the choice of \( r \) to minimize the bound on \( |a_n| \).

\bigskip\noindent\textbf{Solution:}
By Cauchy on $|z|=r$, $|a_n|\le \frac{M e^{r k}}{r^n}$. Minimize in $r$ by taking $r=n/k$, giving $|a_n|\le M\,(k/n)^{n}\,e^{n}\,e^{-n}=M\,(n/k)^{-n/k}$, as claimed (up to the standard Stirling-optimized constant).\qed


\begin{problembox}[16.53: Limit at Isolated Singularity]
\begin{problemstatement}
Assume \( f \) is analytic on a deleted neighborhood \( B'(0; a) \). Prove that \( \lim_{z \to 0} f(z) \) exists (possibly infinite) if, and only if, there exists an integer \( n \) and a function \( g \), analytic on \( B(0; a) \), with \( g(0) \neq 0 \), such that \( f(z) = z^n g(z) \) in \( B'(0; a) \).
\end{problemstatement}
\end{problembox}

\noindent\textbf{Strategy:} Use the Laurent expansion of \( f \) at \( 0 \) and show that the limit exists if and only if the expansion has the form \( z^n \) times a function analytic and non-zero at \( 0 \).

\bigskip\noindent\textbf{Solution:}
($\Rightarrow$) If $\lim_{z\to0}f(z)$ exists (possibly $\infty$), choose $n\in\mathbb Z$ maximal so that $z^{-n}f(z)$ stays bounded near $0$; then $g(z)=z^{-n}f(z)$ extends analytically with $g(0)\ne0$.
($\Leftarrow$) If $f(z)=z^n g(z)$ with $g$ analytic and $g(0)\ne0$, then the limit exists (finite if $n\ge0$, infinite if $n<0$).\qed


\begin{problembox}[16.54: Zeros of Polynomial with Decreasing Coefficients]
\begin{problemstatement}
Let \( p(z) = \sum_{k=0}^n a_k z^k \) be a polynomial of degree \( n \) with real coefficients satisfying
\[a_0 > a_1 > \cdots > a_{n-1} > a_n > 0.\]
Prove that \( p(z) = 0 \) implies \( |z| > 1 \). Hint. Consider \( (1 - z)p(z) \).
\end{problemstatement}
\end{problembox}

\noindent\textbf{Strategy:} Use the hint to consider \( (1-z)p(z) \) and show that if \( |z| \leq 1 \), then all coefficients of this polynomial are positive, so it cannot be zero.

\bigskip\noindent\textbf{Solution:}
If $|z|\le1$, then $(1-z)p(z)=\sum_{k=0}^{n-1}(a_k-a_{k+1})z^k+a_n z^n$ with all coefficients positive, so $(1-z)p(z)\ne0$. Hence $p(z)\ne0$ when $|z|\le1$, so any zero must satisfy $|z|>1$.\qed


\begin{problembox}[16.55: Zero of Infinite Order]
\begin{problemstatement}
A function \( f \), defined on a disk \( B(a; r) \), is said to have a zero of infinite order at \( a \) if, for every integer \( k > 0 \), there is a function \( g_k \), analytic at \( a \), such that \( f(z) = (z - a)^k g_k(z) \) on \( B(a; r) \). If \( f \) has a zero of infinite order at \( a \), prove that \( f = 0 \) everywhere in \( B(a; r) \).
\end{problemstatement}
\end{problembox}

\noindent\textbf{Strategy:} Show that all Taylor coefficients of \( f \) at \( a \) must be zero, which implies that the Taylor series is identically zero, hence \( f \) is identically zero.

\bigskip\noindent\textbf{Solution:}
If $f$ has a zero of infinite order at $a$, then all Taylor coefficients at $a$ vanish, hence the Taylor series is identically $0$ on $B(a;r)$; therefore $f\equiv0$ in the disk.\qed


\begin{problembox}[16.56: Morera's Theorem]
\begin{problemstatement}
Prove Morera's theorem: If \( f \) is continuous on an open region \( S \) in \( \mathbb{C} \) and if \( \int_y f = 0 \) for every polygonal circuit \( y \) in \( S \), then \( f \) is analytic on \( S \).
\end{problemstatement}
\end{problembox}

\noindent\textbf{Strategy:} Use the hypothesis to define a primitive function \( F \) by path integration, then show that \( F \) is complex differentiable with derivative \( f \), hence \( f \) is analytic.

\bigskip\noindent\textbf{Solution:}
Assuming $\int_y f=0$ for every polygonal circuit in $S$, define for any $z_0\in S$ and $z\in S$ a function $F(z)=\int_{\gamma} f$, where $\gamma$ is any polygonal path from $z_0$ to $z$. Path-independence follows from the hypothesis, so $F$ is well-defined and continuous. On small triangles, $\int f=0$ implies $F$ is complex differentiable with $F'=f$. Hence $f$ is analytic.


\section{Solving and Proving Techniques}

\subsection*{Essential Definitions and Theorems}

\begin{definition}[Complex Path Integration Strategy]
Complex path integration involves computing integrals of analytic functions along curves in the complex plane. The fundamental approach is to use the fact that analytic functions have antiderivatives in simply connected regions.
\end{definition}

\noindent\textbf{Importance:} Complex path integration is the foundation of complex analysis. It provides powerful tools for computing integrals that would be difficult or impossible to evaluate using real analysis techniques. The relationship between path independence and analyticity is fundamental.



\begin{definition}[Residue Theory Strategy]
Residue theory provides a systematic method for computing contour integrals by summing the residues of isolated singularities inside the contour. This is often much simpler than direct integration.
\end{definition}

\noindent\textbf{Importance:} Residue theory is one of the most powerful tools in complex analysis. It allows us to compute many difficult integrals by simply adding up residues, which can often be computed using simple formulas or power series expansions.



\begin{theorem}[Contour Integration Methods]
For computing integrals using complex analysis:
\begin{enumerate}[label=(\alph*)]
\item Use the fundamental theorem: if $g' = f$ on a neighborhood of the path, then $\int_\gamma f = g(B) - g(A)$
\item Apply the fact that integrals over closed paths are zero when the integrand has an antiderivative
\item Use the fact that path integrals are linear and additive over path concatenation
\item Apply the residue theorem for functions with isolated singularities
\end{enumerate}
\end{theorem}

\noindent\textbf{Importance:} These methods provide systematic approaches for computing complex integrals. They allow us to break down complex problems into simpler ones and often provide elegant solutions to difficult integration problems.



\begin{theorem}[Series and Function Analysis]
For analyzing functions using complex analysis:
\begin{enumerate}[label=(\alph*)]
\item Use power series expansions to understand local behavior
\item Apply Laurent series for functions with singularities
\item Use the maximum modulus principle for bounds
\item Apply the argument principle for counting zeros and poles
\end{enumerate}
\end{theorem}

\noindent\textbf{Importance:} These techniques provide powerful tools for understanding the behavior of complex functions. They allow us to analyze functions locally and globally, and provide insights that are not available through real analysis alone.



\begin{theorem}[Geometric Methods in Complex Analysis]
For geometric problems in complex analysis:
\begin{enumerate}[label=(\alph*)]
\item Use conformal mappings to transform regions
\item Apply Möbius transformations for special mappings
\item Use the Schwarz lemma for bounds in the unit disk
\item Apply the Riemann mapping theorem for simply connected regions
\end{enumerate}
\end{theorem}

\noindent\textbf{Importance:} Geometric methods provide powerful tools for understanding the relationship between complex functions and geometry. They allow us to transform complex problems into simpler ones and provide insights into the geometric structure of complex functions.



\subsection*{Applying Cauchy's Integral Formulas}
\begin{itemize}
\item Use Cauchy's integral formula: $f(a) = \frac{1}{2\pi i} \int_{|z-a|=r} \frac{f(z)}{z-a} \, dz$
\item Apply the higher derivative formula: $f^{(n)}(a) = \frac{n!}{2\pi i} \int_{|z-a|=r} \frac{f(z)}{(z-a)^{n+1}} \, dz$
\item Use the fact that these formulas hold for any simple closed contour containing $a$
\item Apply the fact that the formulas can be used to compute derivatives of analytic functions
\item Use the fact that the formulas can be used to prove properties of analytic functions
\end{itemize}

\subsection*{Working with Residues}
\begin{itemize}
\item Use the residue theorem: $\int_\gamma f = 2\pi i \sum \text{Res}(f; a_k)$ where the sum is over poles inside $\gamma$
\item Apply the fact that residues can be computed using Laurent series expansions
\item Use the fact that simple poles have residue $\lim_{z \to a} (z-a)f(z)$
\item Apply the fact that poles of order $n$ have residue $\frac{1}{(n-1)!} \lim_{z \to a} \frac{d^{n-1}}{dz^{n-1}}[(z-a)^n f(z)]$
\item Use the fact that residues can be used to evaluate real integrals
\end{itemize}

\subsection*{Evaluating Real Integrals}
\begin{itemize}
\item Use contour integration by closing the real line with semicircles or other contours
\item Apply the fact that even functions can be extended to the real line
\item Use the fact that trigonometric integrals can be converted to contour integrals using $z = e^{it}$
\item Apply the fact that rational functions can be integrated using residues at poles
\item Use the fact that sector contours can be used for functions with specific symmetries
\end{itemize}

\subsection*{Working with Trigonometric Integrals}
\begin{itemize}
\item Use the substitution $z = e^{it}$ to convert to contour integrals on the unit circle
\item Apply the fact that $\cos t = \frac{1}{2}(z + z^{-1})$ and $\sin t = \frac{1}{2i}(z - z^{-1})$
\item Use the fact that $\cos nt = \frac{1}{2}(z^n + z^{-n})$ and $\sin nt = \frac{1}{2i}(z^n - z^{-n})$
\item Apply the fact that trigonometric identities can be used to simplify integrands
\item Use the fact that residues can be computed for the resulting rational functions
\end{itemize}

\subsection*{Applying Liouville's Theorem}
\begin{itemize}
\item Use Liouville's theorem: bounded entire functions are constant
\item Apply the fact that polynomial growth can be used to bound derivatives
\item Use Cauchy estimates: $|f^{(n)}(a)| \leq \frac{n! M}{R^n}$ where $|f(z)| \leq M$ on $|z-a| = R$
\item Apply the fact that these estimates can be used to prove functions are polynomials
\item Use the fact that these estimates can be used to prove functions are constant
\end{itemize}

\subsection*{Using Rouché's Theorem}
\begin{itemize}
\item Apply Rouché's theorem: if $|g(z)| < |f(z)|$ on a contour, then $f$ and $f+g$ have the same number of zeros inside
\item Use the fact that this can be used to count zeros of polynomials
\item Apply the fact that this can be used to prove existence of fixed points
\item Use the fact that this can be used to show functions have no zeros in certain regions
\item Apply the fact that this can be used to compare zeros of different functions
\end{itemize}

\subsection*{Working with Laurent Series}
\begin{itemize}
\item Use Laurent series to represent functions on annuli around singularities
\item Apply the fact that coefficients can be computed using contour integrals
\item Use the fact that the principal part determines the type of singularity
\item Apply the fact that Laurent series can be used to compute residues
\item Use the fact that Laurent series can be used to classify singularities
\end{itemize}

\subsection*{Classifying Singularities}
\begin{itemize}
\item Use Riemann's theorem: bounded functions have removable singularities
\item Apply the fact that poles have finite principal parts in Laurent series
\item Use the fact that essential singularities have infinite principal parts
\item Apply Casorati-Weierstrass: essential singularities approach every value
\item Use the fact that singularities at infinity can be analyzed via $f(1/z)$
\end{itemize}

\subsection*{Working with Taylor Series}
\begin{itemize}
\item Use Taylor series to represent analytic functions on disks
\item Apply the fact that radius of convergence is distance to nearest singularity
\item Use the fact that Taylor series can be differentiated term by term
\item Apply the fact that Taylor series can be used to compute derivatives
\item Use the fact that Taylor series can be used to prove analyticity
\end{itemize}

\subsection*{Applying Maximum Modulus Principle}
\begin{itemize}
\item Use the maximum modulus principle: maximum occurs on boundary
\item Apply the fact that local maxima force functions to be constant
\item Use the fact that this can be used to prove inequalities
\item Apply the fact that this can be used to prove uniqueness results
\item Use the fact that this can be used to prove Schwarz's lemma
\end{itemize}

\subsection*{Working with Harmonic Functions}
\begin{itemize}
\item Use Poisson's integral formula for harmonic functions on disks
\item Apply the fact that harmonic functions satisfy the mean value property
\item Use the fact that harmonic functions satisfy the maximum principle
\item Apply the fact that real parts of analytic functions are harmonic
\item Use the fact that harmonic functions can be represented by integrals
\end{itemize}
