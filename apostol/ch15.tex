\chapter{Multiple Lebesgue Integrals}


\section{Fubini--Tonelli and Slicing}
\noindent\textbf{Tools used in this section.} Fubini's Theorem and Tonelli's Theorem for nonnegative functions; slicing of sets as in Definition 15.4; the calculus identity \(\int_0^{\infty} e^{-ay} \sin(by) \, dy = \dfrac{b}{a^2 + b^2}\) for \(a>0\), \(b\in\mathbb{R}\).
 


\begin{problembox}[15.1: Integral over Triangular Region]
\begin{problemstatement}
If \( f \in L(T) \), where \( T \) is the triangular region in \( \mathbb{R}^2 \) with vertices at \((0, 0)\), \((1, 0)\), and \((0, 1)\), prove that
\[
\int_T f(x, y) \, d(x, y) = \int_0^1 \left[ \int_0^x f(x, y) \, dy \right] \, dx = \int_0^1 \left[ \int_y^1 f(x, y) \, dx \right] \, dy.
\]
\end{problemstatement}
\end{problembox}

\noindent\textbf{Strategy:} Apply Fubini's Theorem to the indicator function of the triangular region. Express the double integral as an iterated integral using two different slicing methods: first by vertical lines (fixing x and integrating over y), then by horizontal lines (fixing y and integrating over x).

\bigskip\noindent\textbf{Solution:}
Write the indicator of the triangle \(T=\{(x,y): 0\le y \le x \le 1\}\) and apply Fubini to
\[
\int_{\mathbb{R}^2} f(x,y) \mathbf{1}_T(x,y)\, d(x,y).
\]
Slicing by vertical lines gives \(\int_0^1\!\big[\int_0^x f(x,y)\,dy\big]dx\). Slicing by horizontal lines gives \(\int_0^1\!\big[\int_y^1 f(x,y)\,dx\big]dy\). Hence the displayed equalities.\qed


\begin{problembox}[15.2: Double Integral Calculation]
\begin{problemstatement}
For fixed \( c, 0 < c < 1 \), define \( f \) on \( \mathbb{R}^2 \) as follows:
\[
f(x, y) = 
\begin{cases} 
(1 - y)^c / (x - y)^c & \text{if } 0 \leq y < x, 0 < x < 1, \\
0 & \text{otherwise}.
\end{cases}
\]
Prove that \( f \in L(\mathbb{R}^2) \) and calculate the double integral 
\[
\int_{\mathbb{R}^2} f(x, y) \, d(x, y).
\]
\end{problemstatement}
\end{problembox}

\noindent\textbf{Strategy:} First identify the support of the function as a triangular region. Then use a change of variables \(u = x-y\), \(v = y\) to simplify the integrand and transform the region to a more manageable shape. The Jacobian is 1, making the transformation straightforward.

\bigskip\noindent\textbf{Solution:}
The support is the triangle \(0\le y < x < 1\), so
\[
\int_{\mathbb{R}^2} f = \int_0^1\!\int_0^x \frac{(1-y)^c}{(x-y)^c}\,dy\,dx.
\]
Let \(u=x-y\), \(v=y\). The Jacobian is 1 and the region is \(0<v<1\), \(0<u<1-v\). Then
\[
\int_0^1\!\int_0^{1-v} u^{-c}(1-v)^c\,du\,dv
= \frac{1}{1-c}\int_0^1 (1-v)\,dv = \frac{1}{2(1-c)}.
\]
Thus \(f\in L(\mathbb{R}^2)\) for \(0<c<1\) and the value is \(1/[2(1-c)]\).\qed


\begin{problembox}[15.3: Measure of a Subset]
\begin{problemstatement}
Let \( S \) be a measurable subset of \( \mathbb{R}^2 \) with finite measure \( \mu(S) \). Using the notation of Definition 15.4, prove that
\[
\mu(S) = \int_{-\infty}^{\infty} \mu(S^x) \, dx = \int_{-\infty}^{\infty} \mu(S_y) \, dy.
\]
\end{problemstatement}
\end{problembox}

\noindent\textbf{Strategy:} Apply Fubini's Theorem to the indicator function \(\mathbf{1}_S\). Use the definition of the sliced sets \(S^x\) and \(S_y\) from Definition 15.4, which represent the cross-sections of \(S\) at fixed \(x\) and \(y\) values respectively.

\bigskip\noindent\textbf{Solution:}
Apply Fubini to the indicator function \(\mathbf{1}_S\). By Definition 15.4, \(\mu(S^x)=\int \mathbf{1}_S(x,y)\,dy\) and \(\mu(S_y)=\int \mathbf{1}_S(x,y)\,dx\). Hence
\[
\mu(S)=\iint \mathbf{1}_S\,d(x,y)=\int_{-\infty}^{\infty}\!\mu(S^x)\,dx=\int_{-\infty}^{\infty}\!\mu(S_y)\,dy.
\]\qed


\begin{problembox}[15.4: Iterated Integrals vs Double Integral]
\begin{problemstatement}
Let \( f(x, y) = e^{-xy} \sin x \sin y \) if \( x \geq 0, y \geq 0 \), and let \( f(x, y) = 0 \) otherwise. Prove that both iterated integrals
\[
\int_{\mathbb{R}^1} \left[ \int_{\mathbb{R}^1} f(x, y) \, dx \right] \, dy \quad \text{and} \quad \int_{\mathbb{R}^1} \left[ \int_{\mathbb{R}^1} f(x, y) \, dy \right] \, dx
\]
exist and are equal, but that the double integral of \( f \) over \( \mathbb{R}^2 \) does not exist. Also, explain why this does not contradict the Tonelli-Hobson test (Theorem 15.8).
\end{problemstatement}
\end{problembox}

\noindent\textbf{Strategy:} For the iterated integrals, use the Laplace transform formula for \(\int_0^\infty e^{-at} \sin t \, dt\). Show that both iterated integrals converge absolutely. For the double integral, demonstrate that \(|f|\) is not integrable by showing divergence near the axes. Explain why Tonelli's theorem doesn't apply since \(f\) changes sign and is not absolutely integrable.

\bigskip\noindent\textbf{Solution:}
Since \(f\) is supported in the first quadrant, we may write the iterated integrals as
\[
\int_0^\infty \left[ \int_0^\infty e^{-xy} \sin x \sin y \, dx \right] \, dy,
\quad
\int_0^\infty \left[ \int_0^\infty e^{-xy} \sin x \sin y \, dy \right] \, dx.
\]

\emph{First iterated integral.}
For fixed \(y \ge 0\),
\[
\int_0^\infty e^{-xy} \sin x \, dx
= \frac{1}{y^2+1},
\]
by the standard Laplace transform formula 
\(\int_0^\infty e^{-at} \sin t\, dt = \frac{1}{a^2+1}\) for \(a > -1\).
Thus
\[
\int_0^\infty e^{-xy} \sin x \sin y \, dx
= \sin y \cdot \frac{1}{y^2+1}.
\]
The outer integral becomes
\[
\int_0^\infty \frac{\sin y}{y^2+1} \, dy,
\]
which converges absolutely since \(\frac{|\sin y|}{y^2+1} \le \frac{1}{y^2+1}\) and \(\int_0^\infty \frac{dy}{y^2+1} < \infty\).

\emph{Second iterated integral.}
For fixed \(x \ge 0\),
\[
\int_0^\infty e^{-xy} \sin y \, dy
= \frac{1}{x^2+1},
\]
again by the Laplace transform formula (now with roles of \(x\) and \(y\) interchanged).
Thus
\[
\int_0^\infty e^{-xy} \sin x \sin y \, dy
= \sin x \cdot \frac{1}{x^2+1}.
\]
The outer integral becomes
\[
\int_0^\infty \frac{\sin x}{x^2+1} \, dx,
\]
which converges absolutely by the same comparison as above.  

Therefore both iterated integrals exist and
\[
\int_{\mathbb{R}^1} \left[ \int_{\mathbb{R}^1} f(x, y) \, dx \right] \, dy
=
\int_{\mathbb{R}^1} \left[ \int_{\mathbb{R}^1} f(x, y) \, dy \right] \, dx
= \int_0^\infty \frac{\sin t}{t^2+1} \, dt.
\]

\emph{Non-existence of the double integral.}
The double integral
\[
\iint_{\mathbb{R}^2} f(x,y) \, dx \, dy
\]
in the Lebesgue sense exists only if \(f\) is absolutely integrable:
\[
\iint_{\mathbb{R}^2} |f(x,y)| \, dx \, dy < \infty.
\]
But for \(x,y \ge 0\),
\[
|f(x,y)| = e^{-xy} |\sin x| \, |\sin y| \ge 0.
\]
Separate variables:
\[
\iint_{[0,\infty)^2} e^{-xy} |\sin x|\, |\sin y| \, dx\, dy
\]
fails to converge. Indeed, near the \(x\)-axis (\(y \to 0^+\)), \(e^{-xy} \approx 1\) and the inner \(x\)-integral over \([0,\infty)\) of \(|\sin x|\) diverges, giving divergence of the absolute integral.  
Hence \(f \notin L^1(\mathbb{R}^2)\) and the double integral does not exist in the Lebesgue sense.

\emph{Why no contradiction to Tonelli--Hobson.}
Tonelli's theorem applies to nonnegative functions and ensures that if \(\iint |f| < \infty\) then Fubini's theorem allows exchanging order of integration.  
Here \(f\) changes sign and is not absolutely integrable, so Tonelli's theorem does not apply.  
Hobson's test also requires certain boundedness conditions on one of the iterated integrals of \(|f|\), which fail here because \(\int |f(x,y)|\, dx = \infty\) for every small \(y > 0\).  
Therefore there is no contradiction: both iterated integrals exist and are equal, yet the double integral does not exist because \(f \notin L^1(\mathbb{R}^2)\).\qed
\section{Non-Integrable Examples and Iterated Integrals}

\noindent\textbf{Tools used in this section.} Fubini's Theorem; criterion: if the two iterated integrals over a rectangle exist but are unequal, then the function is not Lebesgue-integrable on that rectangle; comparisons with improper integrals.



\begin{problembox}[15.5: Non-Integrable Function]
\begin{problemstatement}
Let \( f(x, y) = (x^2 - y^2)/(x^2 + y^2)^2 \) for \( 0 \leq x \leq 1, 0 < y \leq 1 \), and let \( f(0, 0) = 0 \). Prove that both iterated integrals
\[
\int_0^1 \left[ \int_0^1 f(x, y) \, dy \right] \, dx \quad \text{and} \quad \int_0^1 \left[ \int_0^1 f(x, y) \, dx \right] \, dy
\]
exist but are not equal. This shows that \( f \) is not Lebesgue-integrable on \([0, 1] \times [0, 1]\).
\end{problemstatement}
\end{problembox}

\noindent\textbf{Strategy:} Compute both iterated integrals directly using partial fraction decomposition or substitution. Show they yield different values (\(\pi/4\) and \(-\pi/4\)), which by the criterion that unequal iterated integrals imply non-integrability, proves \(f \notin L([0,1]^2)\).

\bigskip\noindent\textbf{Solution:}
For fixed \(x>0\),
\[
\int_0^1 \frac{x^2-y^2}{(x^2+y^2)^2}\,dy = \frac{1}{1+x^2}.
\]
Thus \(\int_0^1[\int_0^1 f(x,y)\,dy]dx=\int_0^1\!\frac{dx}{1+x^2}=\pi/4\). For fixed \(y>0\), symmetry gives
\[
\int_0^1 \frac{x^2-y^2}{(x^2+y^2)^2}\,dx = -\frac{1}{1+y^2},
\]
so \(\int_0^1[\int_0^1 f(x,y)\,dx]dy=-\pi/4\). Since the iterated integrals exist but are unequal, \(f\notin L([0,1]^2)\).\qed


\begin{problembox}[15.6: Another Non-Integrable Function]
\begin{problemstatement}
Let \( I = [0, 1] \times [0, 1] \), let \( f(x, y) = (x - y)/(x + y)^3 \) if \( (x, y) \in I \), \( (x, y) \neq (0, 0) \), and let \( f(0, 0) = 0 \). Prove that \( f \notin L(I) \) by considering the iterated integrals
\[
\int_0^1 \left[ \int_0^1 f(x, y) \, dy \right] \, dx \quad \text{and} \quad \int_0^1 \left[ \int_0^1 f(x, y) \, dx \right] \, dy.
\]
\end{problemstatement}
\end{problembox}

\noindent\textbf{Strategy:} Similar to the previous problem, compute both iterated integrals directly. Use the symmetry of the function to show they give opposite values (\(1/2\) and \(-1/2\)), establishing non-integrability by the same criterion.

\bigskip\noindent\textbf{Solution:}
For fixed \(x\in[0,1]\),
\[
\int_0^1 \frac{x-y}{(x+y)^3}\,dy = \frac{1}{(1+x)^{2}},
\]
so the outer \(x\)-integral equals \(\int_0^1 (1+x)^{-2}dx=1/2\). For fixed \(y\in[0,1]\),
\[
\int_0^1 \frac{x-y}{(x+y)^3}\,dx = -\frac{1}{(1+y)^{2}},
\]
so the other iterated integral equals \(-1/2\). Hence \(f\notin L(I)\).\qed


\begin{problembox}[15.7: Non-Integrable Function on Infinite Interval]
\begin{problemstatement}
Let \( I = [0, 1] \times [1, +\infty) \) and let \( f(x, y) = e^{-xy} - 2e^{-2xy} \) if \( (x, y) \in I \). Prove that \( f \notin L(I) \) by considering the iterated integrals
\[
\int_0^1 \left[ \int_0^\infty f(x, y) \, dy \right] \, dx \quad \text{and} \quad \int_1^\infty \left[ \int_0^1 f(x, y) \, dx \right] \, dy.
\]
\end{problemstatement}
\end{problembox}

\noindent\textbf{Strategy:} Show that one iterated integral diverges while the other converges. The first integral diverges due to a singularity at \(x=0\), while the second converges absolutely. This demonstrates non-integrability since both iterated integrals must converge for Lebesgue integrability.

\bigskip\noindent\textbf{Solution:}
For \(x\in(0,1]\),
\[
\int_1^{\infty} \big(e^{-xy}-2e^{-2xy}\big)\,dy = \frac{e^{-x}-2e^{-2x}}{x},
\]
whose integral in \(x\in(0,1]\) diverges at 0, so \(\int_0^1[\int_1^{\infty} f\,dy]dx\) diverges. On the other hand, for fixed \(y\ge 1\),
\[
\int_0^1 \big(e^{-xy}-2e^{-2xy}\big)\,dx = \frac{e^{-2y}-e^{-y}}{y},
\]
and \(\int_1^{\infty} |e^{-2y}-e^{-y}|\,y^{-1} dy<\infty\). Thus one iterated integral converges while the other diverges, so \(f\notin L(I)\).\qed
\section{Change of Variables}

\noindent\textbf{Tools used in this section.} Change of Variables Theorem with Jacobian determinant; smooth one-to-one transformations between regions; polar, cylindrical, and spherical coordinate maps and their Jacobians.



\begin{problembox}[15.8: Transformation of Integrals]
\begin{problemstatement}
The following formulas for transforming double and triple integrals occur in elementary calculus. Obtain them as consequences of Theorem 15.11 and give restrictions on \( T \) and \( T' \) for validity of these formulas.
\begin{enumerate}[label=(\alph*)]
\item \[ \iint_T f(x, y) \, dx \, dy = \iint_{T'} f(r \cos \theta, r \sin \theta)r \, dr \, d\theta.\]
\item \[ \iiint_T f(x, y, z) \, dx \, dy \, dz = \iiint_{T'} f(r \cos \theta, r \sin \theta, z)r \, dr \, d\theta \, dz.\]
\item 
\begin{align*}
&\iiint_T f(x, y, z) \, dx \, dy \, dz \\
=& \iiint_{T'} f(\rho \cos \theta \sin \varphi, \rho \sin \theta \sin \varphi, \rho \cos \varphi) \rho^2 \sin \varphi \, d\rho \, d\theta \, d\varphi.
\end{align*}
\end{enumerate}
\end{problemstatement}
\end{problembox}

\noindent\textbf{Strategy:} Apply the Change of Variables Theorem (Theorem 15.11) to the coordinate transformations for polar, cylindrical, and spherical coordinates. Calculate the Jacobian determinants for each transformation and specify the conditions on \(T\) and \(T'\) for the theorem to apply.

\bigskip\noindent\textbf{Solution:}
These follow from the Change of Variables Theorem applied to the maps
\( (r,\theta)\mapsto (r\cos\theta, r\sin\theta) \),
\( (r,\theta,z)\mapsto (r\cos\theta, r\sin\theta, z) \), and
\( (\rho,\theta,\varphi)\mapsto (\rho\cos\theta\sin\varphi,\rho\sin\theta\sin\varphi,\rho\cos\varphi) \).
The determinants are respectively \(r\), \(r\), and \(\rho^2\sin\varphi\). Validity requires that \(T'\) be a measurable set on which the coordinate map is one-to-one (modulo negligible sets), with measurable inverse onto \(T\); typically take \(T'\) a rectangle in coordinates with appropriate ranges and \(T\) the corresponding polar/cylindrical/spherical image.\qed
\section{Gaussian Integrals}

\noindent\textbf{Tools used in this section.} Polar coordinates in \(\mathbb{R}^2\); product structure of Gaussian measures; the Gamma function and scaling.



\begin{problembox}[15.9: Gaussian Integrals]
\begin{problemstatement}
\begin{enumerate}[label=(\alph*)]
\item Prove that \(\int_{\mathbb{R}^2} e^{-(x^2 + y^2)} \, d(x, y) = \pi\) by transforming the integral to polar coordinates.
\item Use part (a) to prove that \(\int_{-\infty}^{\infty} e^{-x^2} \, dx = \sqrt{\pi}\).
\item Use part (b) to prove that \(\int_{\mathbb{R}^n} e^{-\|x\|^2} \, d(x_1, \ldots, x_n) = \pi^{n/2}\).
\item Use part (b) to calculate \(\int_{-\infty}^{\infty} e^{-tx^2} \, dx\) and \(\int_{-\infty}^{\infty} x^2 e^{-tx^2} \, dx, t > 0\).
\end{enumerate}
\end{problemstatement}
\end{problembox}

\noindent\textbf{Strategy:} Start with polar coordinates for the 2D integral, then use the product structure of Gaussian measures to extend to higher dimensions. For the parameterized integrals, use scaling and differentiation techniques to derive the results from the basic Gaussian integral.

\bigskip\noindent\textbf{Solution:}
\begin{enumerate}[label=(\alph*)]
\item In polar coordinates,
\(\int_{\mathbb{R}^2} e^{-(x^2+y^2)} d(x,y)=\int_0^{2\pi}\int_0^{\infty} e^{-r^2} r\,dr\,d\theta=2\pi\cdot\tfrac12=\pi\).
\item By symmetry, \(\left(\int_{-\infty}^{\infty} e^{-x^2}dx\right)^2=\int_{\mathbb{R}^2} e^{-(x^2+y^2)}d(x,y)=\pi\). Hence the one-dimensional integral equals \(\sqrt{\pi}\).
\item Using product structure, \(\int_{\mathbb{R}^n} e^{-\|x\|^2}dx=\big(\int_{-\infty}^{\infty} e^{-t^2}dt\big)^n=\pi^{n/2}\).
\item Scaling gives \(\int_{-\infty}^{\infty} e^{-t x^2} dx = t^{-1/2}\int e^{-u^2}du = \sqrt{\pi/t}\). Differentiating in \(t\) or using symmetry yields \(\int x^2 e^{-t x^2} dx = \tfrac{\sqrt{\pi}}{2} t^{-3/2}\).
\end{enumerate}\qed
\section{Volumes of n-Balls}

\noindent\textbf{Tools used in this section.} Polar/spherical coordinates; scaling under linear changes; Gamma-function identities and simple recursions.



\begin{problembox}[15.10: Volume of \( n \)-Ball]
\begin{problemstatement}
Let \( V_n(a) \) denote the \( n \)-measure of the \( n \)-ball \( B(0; a) \) of radius \( a \). This exercise outlines a proof of the formula
\[
V_n(a) = \frac{\pi^{n/2} a^n}{\Gamma( \frac{1}{2} n + 1 )}.
\]
\begin{enumerate}[label=(\alph*)]
\item Use a linear change of variable to prove that \( V_n(a) = a^n V_n(1) \).
\item Assume \( n \geq 3 \), express the integral for \( V_n(1) \) as the iteration of an \( (n - 2) \)-fold integral and a double integral, and use part (a) for an \( (n - 2) \)-ball to obtain the formula
\[
V_n(1) = V_{n-2}(1) \int_0^{2\pi} \left[ \int_0^1 (1 - r^2)^{n/2 - 1}r \, dr \right] d\theta = V_{n-2}(1) \frac{2\pi}{n}.
\]
\item From the recursion formula in (b) deduce that
\[
V_n(1) = \frac{\pi^{n/2}}{\Gamma(\frac{1}{2}n + 1)}.
\]
\end{enumerate}
\end{problemstatement}
\end{problembox}

\noindent\textbf{Strategy:} Use scaling properties for part (a). For part (b), decompose the \(n\)-ball into a product of a 2D disk and an \((n-2)\)-ball, then use polar coordinates for the 2D part. For part (c), use mathematical induction with the recursion formula and known values for low dimensions.

\bigskip\noindent\textbf{Solution:}
\begin{enumerate}[label=(\alph*)]
\item Under \(x\mapsto ax\), \(n\)-volume scales by \(a^n\), so \(V_n(a)=a^n V_n(1)\).
\item For \(n\ge 3\), write \(\|x\|^2=r^2+\rho^2\) with \(\rho\in\mathbb{R}^{n-2}\). Then
\[
V_n(1)=V_{n-2}(1)\int_0^{2\pi}\!\int_0^1 (1-r^2)^{\frac{n}{2}-1} r\,dr\,d\theta = V_{n-2}(1)\cdot \frac{2\pi}{n}.
\]
\item Induct on \(n\) using (b). With \(V_0(1)=1\), \(V_1(1)=2\), the recursion yields \(V_n(1)=\dfrac{\pi^{n/2}}{\Gamma(\tfrac{n}{2}+1)}\).
\end{enumerate}\qed


\begin{problembox}[15.11: Integral over \( n \)-Ball]
\begin{problemstatement}
Refer to Exercise 15.10 and prove that
\[
\int_{B(0;1)} x_k^2 \, d(x_1, \ldots, x_n) = \frac{V_n(1)}{n + 2}
\]
for each \( k = 1, 2, \ldots, n \).
\end{problemstatement}
\end{problembox}

\noindent\textbf{Strategy:} Use the symmetry of the ball to show all \(\int x_k^2\) are equal. Then use the identity \(\sum_{k=1}^n x_k^2 = \|x\|^2\) and spherical coordinates to compute \(\int \|x\|^2\) in terms of the surface area of the unit sphere.

\bigskip\noindent\textbf{Solution:}
By symmetry, \(\int_{B(0;1)} x_k^2\,dx\) is the same for each \(k\) and \(\sum_{k=1}^n x_k^2=\|x\|^2\). Hence
\[
\sum_{k=1}^n \int_{B(0;1)} x_k^2\,dx = \int_{B(0;1)} \|x\|^2\,dx.
\]
Using spherical coordinates, \(\int_{B(0;1)} \|x\|^2 dx = \omega_{n}\int_0^1 r^{n+1}dr=\omega_n/(n+2)\), where \(\omega_n= n V_n(1)\) is the surface area of the unit sphere. Therefore each \(\int x_k^2 = V_n(1)/(n+2)\).\qed


\begin{problembox}[15.12: Recursion Formula for \( n \)-Ball Volume]
\begin{problemstatement}
Refer to Exercise 15.10 and express the integral for \( V_n(1) \) as the iteration of an \( (n - 1) \)-fold integral and a one-dimensional integral, to obtain the recursion formula
\[
V_n(1) = 2V_{n-1}(1) \int_0^1 (1 - x^2)^{(n-1)/2} \, dx.
\]
Put \( x = \cos t \) in the integral, and use the formula of Exercise 15.10 to deduce that
\[
\int_0^{\pi/2} \cos^n t \, dt = \frac{\sqrt{\pi}}{2} \frac{\Gamma(\frac{1}{2}n + \frac{1}{2})}{\Gamma(\frac{1}{2}n + 1)}.
\]
\end{problemstatement}
\end{problembox}

\noindent\textbf{Strategy:} Decompose the \(n\)-ball into a product of a 1D interval and an \((n-1)\)-ball, then use the substitution \(x = \cos t\) to transform the integral. Combine this with the result from Exercise 15.10 to derive the cosine integral formula.

\bigskip\noindent\textbf{Solution:}
Writing \(V_n(1)=2V_{n-1}(1)\int_0^1 (1-x^2)^{(n-1)/2}dx\) and substituting \(x=\cos t\) gives
\[
\int_0^{\pi/2} \cos^n t\,dt = \frac{\sqrt{\pi}}{2}\,\frac{\Gamma(\tfrac{n+1}{2})}{\Gamma(\tfrac{n}{2}+1)}.
\]
Combining with Exercise 15.10 yields the stated recursion.\qed
\section{Volumes in Other Regions}

\noindent\textbf{Tools used in this section.} Linear changes and scaling; induction by slicing; simplex volumes via Beta integrals and the substitution \(y_i=x_i^2\) on the first quadrant.



\begin{problembox}[15.13: Volume of \( n \)-Dimensional Diamond]
\begin{problemstatement}
If \( a > 0 \), let \( S_n(a) = \{(x_1, \ldots, x_n): |x_1| + \cdots + |x_n| \leq a\} \), and let \( V_n(a) \) denote the \( n \)-measure of \( S_n(a) \). This exercise outlines a proof of the formula \( V_n(a) = 2^n a^n / n! \).
\begin{enumerate}[label=(\alph*)]
\item Use a linear change of variable to prove that \( V_n(a) = a^n V_n(1) \).
\item Assume \( n \geq 2 \), express the integral for \( V_n(1) \) as an iteration of a one-dimensional integral and an \( (n - 1) \)-fold integral, use (a) to show that
\[
V_n(1) = V_{n-1}(1) \int_{-1}^1 (1 - |x|)^{n-1} \, dx = 2V_{n-1}(1)/n,
\]
and deduce that \( V_n(1) = 2^n / n! \).
\end{enumerate}
\end{problemstatement}
\end{problembox}

\noindent\textbf{Strategy:} Use scaling for part (a). For part (b), slice the diamond by fixing one coordinate and use the fact that the cross-section is a scaled \((n-1)\)-dimensional diamond. Use mathematical induction to establish the factorial formula.

\bigskip\noindent\textbf{Solution:}
\begin{enumerate}[label=(\alph*)]
\item Scaling gives \(V_n(a)=a^n V_n(1)\).
\item Slice by \(x_1\) and use the \((n-1)\)-dimensional volume of the cross-section \(\{(x_2,\ldots,x_n): |x_2|+\cdots+|x_n|\le 1-|x_1|\}\). Induction yields
\(V_n(1)=2\,V_{n-1}(1)\int_0^1 (1-x)^{n-1}dx=2\,V_{n-1}(1)/n\), hence \(V_n(1)=2^n/n!\).
\end{enumerate}\qed


\begin{problembox}[15.14: Volume of Special \( n \)-Dimensional Set]
\begin{problemstatement}
If \( a > 0 \) and \( n \geq 2 \), let \( S_n(a) \) denote the following set in \( \mathbb{R}^n \):
\[
S_n(a) = \{(x_1, \ldots, x_n): |x_i| + |x_n| \leq a \quad \text{for each } i = 1, \ldots, n - 1\}.
\]
Let \( V_n(a) \) denote the \( n \)-measure of \( S_n(a) \). Use a method suggested by Exercise 15.13 to prove that \( V_n(a) = 2^n a^n / n \).
\end{problemstatement}
\end{problembox}

\noindent\textbf{Strategy:} Fix the last coordinate \(x_n\) and observe that the cross-section in the first \(n-1\) coordinates forms an \((n-1)\)-dimensional diamond. Use the result from Exercise 15.13 for the volume of this cross-section, then integrate over \(x_n\).

\bigskip\noindent\textbf{Solution:}
Fix \(x_n\in[-a,a]\). The cross-section in the first \(n-1\) coordinates is an \((n-1)\)-dimensional diamond of radius \(a-|x_n|\) with volume \(V_{n-1}(a-|x_n|)=2^{n-1}(a-|x_n|)^{n-1}/(n-1)!\). Integrating in \(x_n\) gives
\[
V_n(a)=\int_{-a}^{a} \frac{2^{n-1}}{(n-1)!} (a-|t|)^{n-1} dt = \frac{2^n a^n}{n}.
\]\qed


\begin{problembox}[15.15: Integral over First Quadrant of \( n \)-Ball]
\begin{problemstatement}
Let \( Q_n(a) \) denote the ``first quadrant'' of the \( n \)-ball \( B(0:a) \) given by
$Q_n(a) = \{(x_1, \ldots, x_n): \|x\| \leq a$ and $ 0 \leq x_i \leq a $ for each $ i = 1, 2, \ldots, n.$
Let \( f(x) = x_1 \cdots x_n \) and prove that
\[
\int_{Q_n(a)} f(x) \, dx = \frac{a^{2n}}{2^n n!}.
\]
\end{problemstatement}
\end{problembox}

\noindent\textbf{Strategy:} Use the change of variables \(y_i = x_i^2\) to transform the region into a simplex. The Jacobian simplifies the integrand, and the volume of the simplex can be computed using the formula for the volume of an \(n\)-dimensional simplex.

\bigskip\noindent\textbf{Solution:}
Let \(y_i=x_i^2\) for \(i=1,\ldots,n\). On the first quadrant, \(x_i\ge 0\), the region \(Q_n(a)\) maps to the simplex \(\{y\ge 0: y_1+\cdots+y_n\le a^2\}\). The Jacobian gives \(dx=\frac{1}{2^n}(y_1\cdots y_n)^{-1/2}dy\) and \(x_1\cdots x_n=\sqrt{y_1\cdots y_n}\). Therefore the integrand times \(dx\) is \(\frac{1}{2^n} dy\), and
\[
\int_{Q_n(a)} x_1\cdots x_n\,dx = \frac{1}{2^n}\,\mathrm{Vol}\,\{y\ge 0: y_1+\cdots+y_n\le a^2\} = \frac{a^{2n}}{2^n n!}.
\]

\section{Solving and Proving Techniques}

\subsection*{Working with Fubini's Theorem}
\begin{itemize}
\item Use Fubini's theorem to express double integrals as iterated integrals
\item Apply the fact that the order of integration can be interchanged for integrable functions
\item Use the fact that nonnegative measurable functions satisfy Tonelli's theorem
\item Apply the fact that absolute integrability ensures Fubini's theorem applies
\item Use the fact that indicator functions can be used to restrict integration to specific regions
\end{itemize}

\subsection*{Working with Lebesgue Integrals}
\begin{itemize}
\item Use the fact that Lebesgue integrals extend Riemann integrals
\item Apply the fact that Lebesgue integrals are linear and monotone
\item Use the fact that Lebesgue integrals can handle unbounded functions and infinite regions
\item Apply the fact that Lebesgue integrals are invariant under changes of variables
\item Use the fact that Lebesgue integrals can be computed as limits of simple functions
\end{itemize}

\subsection*{Applying Change of Variables}
\begin{itemize}
\item Use the fact that the Jacobian determinant gives the scaling factor for volume
\item Apply the fact that linear transformations scale volumes by the determinant
\item Use the fact that polar and spherical coordinates have known Jacobians
\item Apply the fact that changes of variables preserve measurability
\item Use the fact that the Jacobian can be computed from the derivative matrix
\end{itemize}

\subsection*{Working with Slicing}
\begin{itemize}
\item Use the fact that the measure of a set equals the integral of its cross-sectional measures
\item Apply the fact that slicing can be done in any coordinate direction
\item Use the fact that slicing preserves measurability
\item Apply the fact that slicing can be used to compute volumes of complex regions
\item Use the fact that slicing can be used to prove geometric formulas
\end{itemize}

\subsection*{Computing Volumes in Higher Dimensions}
\begin{itemize}
\item Use the fact that $n$-ball volumes can be computed using spherical coordinates
\item Apply the fact that volumes scale by the $n$th power of the scaling factor
\item Use the fact that volumes can be computed by slicing into lower-dimensional regions
\item Apply the fact that symmetry can be used to simplify volume calculations
\item Use the fact that volumes can be computed using recursion formulas
\end{itemize}

\subsection*{Working with Special Functions}
\begin{itemize}
\item Use the Gamma function: $\Gamma(n+1) = n!$ for positive integers
\item Apply the fact that $\Gamma(1/2) = \sqrt{\pi}$
\item Use the fact that Beta functions can be expressed in terms of Gamma functions
\item Apply the fact that special functions can be used to evaluate difficult integrals
\item Use the fact that recursion formulas can be used to compute special function values
\end{itemize}