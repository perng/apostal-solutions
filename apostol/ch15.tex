\chapter{Multiple Lebesgue Integrals}


\section{Fubini--Tonelli and Slicing}
 
\begin{problembox}[15.1: Integral over Triangular Region]
If \( f \in L(T) \), where \( T \) is the triangular region in \( \mathbb{R}^2 \) with vertices at \((0, 0)\), \((1, 0)\), and \((0, 1)\), prove that
\[
\int_T f(x, y) \, d(x, y) = \int_0^1 \left[ \int_0^x f(x, y) \, dy \right] \, dx = \int_0^1 \left[ \int_y^1 f(x, y) \, dx \right] \, dy.
\]
\end{problembox}

\begin{problembox}[15.2: Double Integral Calculation]
For fixed \( c, 0 < c < 1 \), define \( f \) on \( \mathbb{R}^2 \) as follows:
\[
f(x, y) = 
\begin{cases} 
(1 - y)^c / (x - y)^c & \text{if } 0 \leq y < x, 0 < x < 1, \\
0 & \text{otherwise}.
\end{cases}
\]
Prove that \( f \in L(\mathbb{R}^2) \) and calculate the double integral 
\[
\int_{\mathbb{R}^2} f(x, y) \, d(x, y).
\]
\end{problembox}

\begin{problembox}[15.3: Measure of a Subset]
Let \( S \) be a measurable subset of \( \mathbb{R}^2 \) with finite measure \( \mu(S) \). Using the notation of Definition 15.4, prove that
\[
\mu(S) = \int_{-\infty}^{\infty} \mu(S^x) \, dx = \int_{-\infty}^{\infty} \mu(S_y) \, dy.
\]
\end{problembox}

\begin{problembox}[15.4: Iterated Integrals vs Double Integral]
Let \( f(x, y) = e^{-xy} \sin x \sin y \) if \( x \geq 0, y \geq 0 \), and let \( f(x, y) = 0 \) otherwise. Prove that both iterated integrals
\[
\int_{\mathbb{R}^1} \left[ \int_{\mathbb{R}^1} f(x, y) \, dx \right] \, dy \quad \text{and} \quad \int_{\mathbb{R}^1} \left[ \int_{\mathbb{R}^1} f(x, y) \, dy \right] \, dx
\]
exist and are equal, but that the double integral of \( f \) over \( \mathbb{R}^2 \) does not exist. Also, explain why this does not contradict the Tonelli-Hobson test (Theorem 15.8).
\end{problembox}

\section{Non-Integrable Examples and Iterated Integrals}

\begin{problembox}[15.5: Non-Integrable Function]
Let \( f(x, y) = (x^2 - y^2)/(x^2 + y^2)^2 \) for \( 0 \leq x \leq 1, 0 < y \leq 1 \), and let \( f(0, 0) = 0 \). Prove that both iterated integrals
\[
\int_0^1 \left[ \int_0^1 f(x, y) \, dy \right] \, dx \quad \text{and} \quad \int_0^1 \left[ \int_0^1 f(x, y) \, dx \right] \, dy
\]
exist but are not equal. This shows that \( f \) is not Lebesgue-integrable on \([0, 1] \times [0, 1]\).
\end{problembox}

\begin{problembox}[15.6: Another Non-Integrable Function]
Let \( I = [0, 1] \times [0, 1] \), let \( f(x, y) = (x - y)/(x + y)^3 \) if \( (x, y) \in I \), \( (x, y) \neq (0, 0) \), and let \( f(0, 0) = 0 \). Prove that \( f \notin L(I) \) by considering the iterated integrals
\[
\int_0^1 \left[ \int_0^1 f(x, y) \, dy \right] \, dx \quad \text{and} \quad \int_0^1 \left[ \int_0^1 f(x, y) \, dx \right] \, dy.
\]
\end{problembox}

\begin{problembox}[15.7: Non-Integrable Function on Infinite Interval]
Let \( I = [0, 1] \times [1, +\infty) \) and let \( f(x, y) = e^{-xy} - 2e^{-2xy} \) if \( (x, y) \in I \). Prove that \( f \notin L(I) \) by considering the iterated integrals
\[
\int_0^1 \left[ \int_0^\infty f(x, y) \, dy \right] \, dx \quad \text{and} \quad \int_1^\infty \left[ \int_0^1 f(x, y) \, dx \right] \, dy.
\]
\end{problembox}

\section{Change of Variables}

\begin{problembox}[15.8: Transformation of Integrals]
The following formulas for transforming double and triple integrals occur in elementary calculus. Obtain them as consequences of Theorem 15.11 and give restrictions on \( T \) and \( T' \) for validity of these formulas.
\begin{enumerate}[label=(\alph*)]
\item \[ \iint_T f(x, y) \, dx \, dy = \iint_{T'} f(r \cos \theta, r \sin \theta)r \, dr \, d\theta.\]
\item \[ \iiint_T f(x, y, z) \, dx \, dy \, dz = \iiint_{T'} f(r \cos \theta, r \sin \theta, z)r \, dr \, d\theta \, dz.\]
\item \[ \iiint_T f(x, y, z) \, dx \, dy \, dz = \iiint_{T'} f(\rho \cos \theta \sin \varphi, \rho \sin \theta \sin \varphi, \rho \cos \varphi) \rho^2 \sin \varphi \, d\rho \, d\theta \, d\varphi.\]
\end{enumerate}
\end{problembox}

\section{Gaussian Integrals}

\begin{problembox}[15.9: Gaussian Integrals]
\begin{enumerate}[label=(\alph*)]
\item Prove that \(\int_{\mathbb{R}^2} e^{-(x^2 + y^2)} \, d(x, y) = \pi\) by transforming the integral to polar coordinates.
\item Use part (a) to prove that \(\int_{-\infty}^{\infty} e^{-x^2} \, dx = \sqrt{\pi}\).
\item Use part (b) to prove that \(\int_{\mathbb{R}^n} e^{-\|x\|^2} \, d(x_1, \ldots, x_n) = \pi^{n/2}\).
\item Use part (b) to calculate \(\int_{-\infty}^{\infty} e^{-tx^2} \, dx\) and \(\int_{-\infty}^{\infty} x^2 e^{-tx^2} \, dx, t > 0\).
\end{enumerate}
\end{problembox}

\section{Volumes of n-Balls}

\begin{problembox}[15.10: Volume of \( n \)-Ball]
Let \( V_n(a) \) denote the \( n \)-measure of the \( n \)-ball \( B(0; a) \) of radius \( a \). This exercise outlines a proof of the formula
\[
V_n(a) = \frac{\pi^{n/2} a^n}{\Gamma( \frac{1}{2} n + 1 )}.
\]
\begin{enumerate}[label=(\alph*)]
\item Use a linear change of variable to prove that \( V_n(a) = a^n V_n(1) \).
\item Assume \( n \geq 3 \), express the integral for \( V_n(1) \) as the iteration of an \( (n - 2) \)-fold integral and a double integral, and use part (a) for an \( (n - 2) \)-ball to obtain the formula
\[
V_n(1) = V_{n-2}(1) \int_0^{2\pi} \left[ \int_0^1 (1 - r^2)^{n/2 - 1}r \, dr \right] d\theta = V_{n-2}(1) \frac{2\pi}{n}.
\]
\item From the recursion formula in (b) deduce that
\[
V_n(1) = \frac{\pi^{n/2}}{\Gamma(\frac{1}{2}n + 1)}.
\]
\end{enumerate}
\end{problembox}

\begin{problembox}[15.11: Integral over \( n \)-Ball]
Refer to Exercise 15.10 and prove that
\[
\int_{B(0;1)} x_k^2 \, d(x_1, \ldots, x_n) = \frac{V_n(1)}{n + 2}
\]
for each \( k = 1, 2, \ldots, n \).
\end{problembox}

\begin{problembox}[15.12: Recursion Formula for \( n \)-Ball Volume]
Refer to Exercise 15.10 and express the integral for \( V_n(1) \) as the iteration of an \( (n - 1) \)-fold integral and a one-dimensional integral, to obtain the recursion formula
\[
V_n(1) = 2V_{n-1}(1) \int_0^1 (1 - x^2)^{(n-1)/2} \, dx.
\]
Put \( x = \cos t \) in the integral, and use the formula of Exercise 15.10 to deduce that
\[
\int_0^{\pi/2} \cos^n t \, dt = \frac{\sqrt{\pi}}{2} \frac{\Gamma(\frac{1}{2}n + \frac{1}{2})}{\Gamma(\frac{1}{2}n + 1)}.
\]
\end{problembox}

\section{Volumes in Other Regions}

\begin{problembox}[15.13: Volume of \( n \)-Dimensional Diamond]
If \( a > 0 \), let \( S_n(a) = \{(x_1, \ldots, x_n): |x_1| + \cdots + |x_n| \leq a\} \), and let \( V_n(a) \) denote the \( n \)-measure of \( S_n(a) \). This exercise outlines a proof of the formula \( V_n(a) = 2^n a^n / n! \).
\begin{enumerate}[label=(\alph*)]
\item Use a linear change of variable to prove that \( V_n(a) = a^n V_n(1) \).
\item Assume \( n \geq 2 \), express the integral for \( V_n(1) \) as an iteration of a one-dimensional integral and an \( (n - 1) \)-fold integral, use (a) to show that
\[
V_n(1) = V_{n-1}(1) \int_{-1}^1 (1 - |x|)^{n-1} \, dx = 2V_{n-1}(1)/n,
\]
and deduce that \( V_n(1) = 2^n / n! \).
\end{enumerate}
\end{problembox}

\begin{problembox}[15.14: Volume of Special \( n \)-Dimensional Set]
If \( a > 0 \) and \( n \geq 2 \), let \( S_n(a) \) denote the following set in \( \mathbb{R}^n \):
\[
S_n(a) = \{(x_1, \ldots, x_n): |x_i| + |x_n| \leq a \quad \text{for each } i = 1, \ldots, n - 1\}.
\]
Let \( V_n(a) \) denote the \( n \)-measure of \( S_n(a) \). Use a method suggested by Exercise 15.13 to prove that \( V_n(a) = 2^n a^n / n \).
\end{problembox}

\begin{problembox}[15.15: Integral over First Quadrant of \( n \)-Ball]
Let \( Q_n(a) \) denote the ``first quadrant'' of the \( n \)-ball \( B(0:a) \) given by
\[
Q_n(a) = \{(x_1, \ldots, x_n): \|x\| \leq a \quad \text{and } 0 \leq x_i \leq a \quad \text{for each } i = 1, 2, \ldots, n\}.
\]
Let \( f(x) = x_1 \cdots x_n \) and prove that
\[
\int_{Q_n(a)} f(x) \, dx = \frac{a^{2n}}{2^n n!}.
\]
\end{problembox}