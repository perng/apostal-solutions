\chapter{Implicit Functions and Extremum Problems}

\section{Jacobians}
\subsection*{Key definitions and theorems}
\begin{enumerate}[label=(\roman*)]
    \item \textbf{Jacobian determinant}: For a $C^1$ map $f=(f_1,\dots,f_n):\,\mathbb R^n\to\mathbb R^n$, $J_f(x)=\det\big[\partial f_i/\partial x_j\big]$.
    \item \textbf{Chain rule for Jacobians}: If $h=f\circ g$ with $g: \mathbb R^n\to\mathbb R^n$, then $J_h(x)=J_f\big(g(x)\big)\,J_g(x)$.
    \item \textbf{Inverse Function Theorem}: If $f\in C^1$ and $J_f(x_0)\neq 0$, then $f$ is locally invertible near $x_0$, and $D f^{-1}(f(x_0))=\big(D f(x_0)\big)^{-1}$.
    \item \textbf{Polar/Spherical coordinates}: In $\mathbb R^2$, $x=r\cos\theta,\ y=r\sin\theta$; in $\mathbb R^3$, $x=r\cos\theta\sin\varphi,\ y=r\sin\theta\sin\varphi,\ z=r\cos\varphi$.
    \item \textbf{Matrix determinant lemma}: $\det(I+uv^{\!T})=1+v^{\!T}u$ for compatible vectors $u,v$.
\end{enumerate}


\begin{problembox}[13.1: Complex Function Jacobian]
Let \( f \) be the complex-valued function defined for each complex \( z \neq 0 \) by the equation \( f(z) = 1/\bar{z} \). Show that \( J_f(z) = -|z|^{-4} \). Show that \( f \) is one-to-one and compute \( f^{-1} \) explicitly.
\end{problembox}

\noindent\textbf{Solution:}
Write $z=x+iy$, so $\bar z=x-iy$ and $f(z)=1/\bar z=\dfrac{x+iy}{x^2+y^2}$. Therefore
\[u(x,y)=\frac{x}{x^2+y^2},\quad v(x,y)=\frac{y}{x^2+y^2}.
\]
Compute
\[u_x=\frac{y^2-x^2}{(x^2+y^2)^2},\; u_y=\frac{-2xy}{(x^2+y^2)^2},\; v_x=\frac{-2xy}{(x^2+y^2)^2},\; v_y=\frac{x^2-y^2}{(x^2+y^2)^2}.
\]
Thus
\[J_f=u_xv_y-u_yv_x=-\frac{(x^2+y^2)^2}{(x^2+y^2)^4}=-|z|^{-4}.
\]
Moreover, $w=f(z)=1/\bar z$ implies $\bar w=1/z$, so $z=1/\bar w$. Hence $f$ is one-to-one with inverse $f^{-1}(w)=1/\bar w$.\qed


\begin{problembox}[13.2: Vector-Valued Function Jacobian]
Let \( f = (f_1, f_2, f_3) \) be the vector-valued function defined (for every point \( (x_1, x_2, x_3) \) in \( R^3 \) for which \( x_1 + x_2 + x_3 \neq -1 \)) as follows:
\[f_k(x_1, x_2, x_3) = \frac{x_k}{1 + x_1 + x_2 + x_3} \quad (k = 1, 2, 3).\]
Show that \( J_f(x_1, x_2, x_3) = (1 + x_1 + x_2 + x_3)^{-4} \). Show that \( f \) is one-to-one and compute \( f^{-1} \) explicitly.
\end{problembox}

\noindent\textbf{Solution:}
Let $S=x_1+x_2+x_3$ and $m=(1+S)^{-1}$. Then $f_i=x_im$ and
\[\frac{\partial f_i}{\partial x_j}=m\,\delta_{ij}-m^2 x_i=(mI-m^2xe^{\!T})_{ij},\]
where $x=(x_1,x_2,x_3)^{\!T}$ and $e=(1,1,1)^{\!T}$. Hence
\[\det Df=m^3\det\big(I-mxe^{\!T}\big)=m^3\big(1-me^{\!T}x\big)=m^4=(1+S)^{-4}.
\]
Solving $y_i=\dfrac{x_i}{1+S}$ gives $x_i=Ty_i$ with $T=1+S$. Summing yields $S=T\sum y_i$, so $T(1-\sum y_i)=1$ and $T=\dfrac{1}{1-\sum y_i}$. Therefore
\[f^{-1}(y)=\left(\frac{y_1}{1-\sum y_i},\frac{y_2}{1-\sum y_i},\frac{y_3}{1-\sum y_i}\right),\]
valid when $\sum y_i\neq 1$. Thus $f$ is one-to-one on its domain.\qed


\begin{problembox}[13.3: Composition of Functions Jacobian]
Let \( f = (f_1, \ldots, f_n) \) be a vector-valued function defined in \( R^n \), suppose \( f \in C' \) on \( R^n \), and let \( J_f(x) \) denote the Jacobian determinant. Let \( g_1, \ldots, g_n \) be \( n \) real-valued functions defined on \( R^1 \) and having continuous derivatives \( g'_1, \ldots, g'_n \). Let \( h_k(x) = f_k[g_1(x_1), \ldots, g_n(x_n)], k = 1, 2, \ldots, n \), and put \( h = (h_1, \ldots, h_n) \). Show that
\[J_h(x) = J_f[g_1(x_1), \ldots, g_n(x_n)]g'_1(x_1) \cdots g'_n(x_n).\]
\end{problembox}

\noindent\textbf{Solution:}
Let $G(x)=(g_1(x_1),\dots,g_n(x_n))$. Then $h=f\circ G$ and $Dh(x)=Df(G(x))\,DG(x)$ with $DG(x)=\operatorname{diag}(g_1'(x_1),\dots,g_n'(x_n))$. Taking determinants,
\[J_h(x)=J_f(G(x))\,\prod_{k=1}^n g_k'(x_k).
\]\qed


\begin{problembox}[13.4: Polar and Spherical Coordinates]
\begin{enumerate}[label=(\alph*)]
    \item If \( x(r, \theta) = r \cos \theta, y(r, \theta) = r \sin \theta \), show that
    \[\frac{\partial (x, y)}{\partial (r, \theta)} = r.\]
    \item If \( x(r, \theta, \phi) = r \cos \theta \sin \phi, y(r, \theta, \phi) = r \sin \theta \sin \phi, z = r \cos \phi \), show that
    \[\frac{\partial (x, y, z)}{\partial (r, \theta, \phi)} = -r^2 \sin \phi.\]
\end{enumerate}
\end{problembox}

\noindent\textbf{Solution:}
\begin{enumerate}[label=(\alph*)]
    \item $x_r=\cos\theta,\ x_\theta=-r\sin\theta,\ y_r=\sin\theta,\ y_\theta=r\cos\theta$, so
    $\dfrac{\partial(x,y)}{\partial(r,\theta)}=x_r y_\theta-x_\theta y_r=r$.
    \item With $x=r\cos\theta\sin\phi,\ y=r\sin\theta\sin\phi,\ z=r\cos\phi$, a direct computation gives
    $\dfrac{\partial(x,y,z)}{\partial(r,\theta,\phi)}=-r^2\sin\phi$.
\end{enumerate}\qed


\begin{problembox}[13.5: Implicit Function Theorem Application]
\begin{enumerate}[label=(\alph*)]
    \item State conditions on \( f \) and \( g \) which will ensure that the equations \( x = f(u, v), y = g(u, v) \) can be solved for \( u \) and \( v \) in a neighborhood of \( (x_0, y_0) \). If the solutions are \( u = F(x, y), v = G(x, y) \), and if \( J = \partial (f, g)/\partial (u, v) \), show that
    \[\frac{\partial F}{\partial x} = \frac{1}{J} \frac{\partial g}{\partial v}, \quad \frac{\partial F}{\partial y} = -\frac{1}{J} \frac{\partial f}{\partial v}, \quad \frac{\partial G}{\partial x} = -\frac{1}{J} \frac{\partial g}{\partial u}, \quad \frac{\partial G}{\partial y} = \frac{1}{J} \frac{\partial f}{\partial u}.\]
    \item Compute \( J \) and the partial derivatives of \( F \) and \( G \) at \((x_0, y_0) = (1, 1)\) when \( f(u, v) = u^2 - v^2 \), \( g(u, v) = 2uv \).
\end{enumerate}
\end{problembox}

\noindent\textbf{Solution:}
\begin{enumerate}[label=(\alph*)]
    \item If $f,g\in C^1$ near $(u_0,v_0)$ and $J=\dfrac{\partial(f,g)}{\partial(u,v)}(u_0,v_0)\neq 0$, then by the Inverse Function Theorem the equations $x=f(u,v),\ y=g(u,v)$ can be solved locally as $u=F(x,y),\ v=G(x,y)$. Moreover,
    \[
    \begin{aligned}
    \frac{\partial F}{\partial x}&=\frac{1}{J}\,\frac{\partial g}{\partial v},&\qquad \frac{\partial F}{\partial y}&=-\frac{1}{J}\,\frac{\partial f}{\partial v},\\
    \frac{\partial G}{\partial x}&=-\frac{1}{J}\,\frac{\partial g}{\partial u},&\qquad \frac{\partial G}{\partial y}&=\frac{1}{J}\,\frac{\partial f}{\partial u}.
    \end{aligned}
    \]
    \item Here $f=u^2-v^2,\ g=2uv$. At a point mapping to $(x_0,y_0)=(1,1)$ we have $u^2-v^2=1$ and $2uv=1$. Then
    \[J=\frac{\partial(f,g)}{\partial(u,v)}=\begin{vmatrix}2u&-2v\\2v&2u\end{vmatrix}=4(u^2+v^2)=4\sqrt{(u^2-v^2)^2+(2uv)^2}=4\sqrt{2}.
    \]
    Using the formulas above,
    \[\frac{\partial F}{\partial x}=\frac{2u}{J}=\frac{u}{2\sqrt2},\;\frac{\partial F}{\partial y}=\frac{2v}{J}=\frac{v}{2\sqrt2},\;\frac{\partial G}{\partial x}=-\frac{2v}{J}=-\frac{v}{2\sqrt2},\;\frac{\partial G}{\partial y}=\frac{2u}{J}=\frac{u}{2\sqrt2}.
    \]
    Taking the branch with $u,v>0$, $u=\sqrt{\tfrac{1+\sqrt2}{2}}$ and $v=\sqrt{\tfrac{\sqrt2-1}{2}}$.
\end{enumerate}\qed


\begin{problembox}[13.6: Jacobian Matrix Identity]
Let \( f \) and \( g \) be related as in Theorem 13.6. Consider the case \( n = 3 \) and show that we have
\[J_i(x)D_1g_i(y) =
\begin{vmatrix}
\delta_{i,1} & D_1f_2(x) & D_1f_3(x)\\
\delta_{i,2} & D_2f_2(x) & D_2f_3(x)\\
\delta_{i,3} & D_3f_2(x) & D_3f_3(x)
\end{vmatrix}
(i = 1, 2, 3),\]
where \( y = f(x) \) and \( \delta_{i,j} = 0 \) or 1 according as \( i \neq j \) or \( i = j \). Use this to deduce the formula
\[D_1g_1 = \frac{\partial (f_2, f_3)}{\partial (x_2, x_3)} \left| \frac{\partial (f_1, f_2, f_3)}{\partial (x_1, x_2, x_3)} \right|.\]
There are similar expressions for the other eight derivatives \( D_kg_l \).
\end{problembox}

\noindent\textbf{Solution:}
By Theorem 13.6, $Dg(y)=\big(Df(x)\big)^{-1}$ with $y=f(x)$. Writing $Df$ as the $3\times3$ matrix $A=[D_j f_i]$, we have $A^{-1}=\dfrac{1}{\det A}\,\operatorname{adj}(A)$, whose entries are cofactors divided by $J_f(x)$. Identifying the appropriate cofactors yields
\[J_i(x)D_1g_i(y)=\begin{vmatrix}
\delta_{i,1} & D_1f_2 & D_1f_3\\
\delta_{i,2} & D_2f_2 & D_2f_3\\
\delta_{i,3} & D_3f_2 & D_3f_3
\end{vmatrix},\quad i=1,2,3.
\]
Setting $i=1$ and dividing by $J_f=\dfrac{\partial(f_1,f_2,f_3)}{\partial(x_1,x_2,x_3)}$ gives
\[D_1g_1=\frac{\partial(f_2,f_3)}{\partial(x_2,x_3)}\left/\frac{\partial(f_1,f_2,f_3)}{\partial(x_1,x_2,x_3)}\right.\,.
\]
The other eight $D_k g_\ell$ follow similarly.\qed


\begin{problembox}[13.7: Complex Function Properties]
Let \( f = u + iv \) be a complex-valued function satisfying the following conditions: \( u \in C' \) and \( v \in C' \) on the open disk \( A = \{z : |z| < 1\}; f \) is continuous on the closed disk \( \bar{A} = \{z : |z| \leq 1\}; u(x, y) = x \) and \( v(x, y) = y \) whenever \( x^2 + y^2 = 1 \); the Jacobian \( J_f(z) > 0 \) if \( z \in A \). Let \( B = f(A) \) denote the image of \( A \) under \( f \) and prove that:
\begin{enumerate}[label=(\alph*)]
    \item If \( X \) is an open subset of \( A \), then \( f(X) \) is an open subset of \( B \).
    \item \( B \) is an open disk of radius 1.
    \item For each point \( u_0 + iv_0 \) in \( B \), there is only a finite number of points \( z \) in \( A \) such that \( f(z) = u_0 + iv_0 \).
\end{enumerate}
\end{problembox}

\noindent\textbf{Solution:}
Let $A=\{z:\,|z|<1\}$ and $\bar A=\{z:\,|z|\le 1\}$. Since $u,v\in C^1$ and $J_f>0$ on $A$, $f$ is a local $C^1$-diffeomorphism on $A$.
\begin{enumerate}[label=(\alph*)]
    \item If $X\subset A$ is open, each $z\in X$ has a neighborhood on which $f$ is a diffeomorphism, so $f(X)$ is a union of open sets; hence $f(X)$ is open in $B=f(A)$.
    \item On $|z|=1$ we have $u=x,\ v=y$, so $f$ restricts to the identity on the unit circle. By invariance of domain, $B$ is open; since $f$ maps the boundary $|z|=1$ to itself with positive Jacobian in $A$, $B$ must be the open unit disk.
    \item Because $f$ is a local diffeomorphism on $A$, preimages of a point are isolated. If a point of $B$ had infinitely many preimages in the compact set $\bar A$, they would accumulate in $A$, contradicting local injectivity. Thus each value in $B$ has finitely many preimages in $A$.
\end{enumerate}\qed
\section{Extremum Problems}
\subsection*{Key definitions and theorems}
\begin{enumerate}[label=(\roman*)]
    \item \textbf{Critical points and Hessian test}: $\nabla f=0$ at critical points; for a $2\times2$ Hessian $H$, use $\det H$ and $\operatorname{tr} H$ to determine definiteness.
    \item \textbf{Lagrange multipliers}: To extremize $f$ subject to constraints $g_i=0$, solve $\nabla f=\sum_i \lambda_i\,\nabla g_i$ with the constraints.
    \item \textbf{Cauchy--Schwarz}: $\big|\sum a_k x_k\big|\le \|a\|\,\|x\|$ with equality when $x$ is proportional to $a$.
    \item \textbf{AM--GM}: For nonnegative $t_i$, $\dfrac{t_1+\cdots+t_n}{n}\ge (t_1\cdots t_n)^{1/n}$, equality when all $t_i$ are equal.
    \item \textbf{Rayleigh quotient idea}: Quadratic forms under quadratic/linear constraints reduce to eigenvalue-like equations.
\end{enumerate}


\begin{problembox}[13.8: Extreme Value Classification]
Find and classify the extreme values (if any) of the functions defined by the following equations:
\begin{enumerate}[label=(\alph*)]
    \item \( f(x, y) = y^2 + x^2y + x^4 \),
    \item \( f(x, y) = x^2 + y^2 + x + y + xy \),
    \item \( f(x, y) = (x - 1)^4 + (x - y)^4 \),
    \item \( f(x, y) = y^2 - x^3 \).
\end{enumerate}
\end{problembox}

\noindent\textbf{Solution:}
\begin{enumerate}[label=(\alph*)]
    \item $f_x=2x(y+2x^2),\ f_y=2y+x^2$. The only critical point is $(0,0)$. Since $f=y^2+x^2y+x^4\ge 0$ and $f(0,0)=0$, this is a local minimum.
    \item $f_x=2x+y+1,\ f_y=2y+x+1$ gives critical point $(-\tfrac13,-\tfrac13)$. Hessian $\begin{pmatrix}2&1\\1&2\end{pmatrix}$ is positive definite $\Rightarrow$ local (indeed global) minimum.
    \item $f_x=4(x-1)^3+4(x-y)^3,\ f_y=-4(x-y)^3$ gives $(1,1)$. Since $f=(x-1)^4+(x-y)^4\ge0$ with equality only at $(1,1)$, this is a global minimum.
    \item $f_x=-3x^2,\ f_y=2y$ gives $(0,0)$. Along $y=0$, $f=-x^3$ changes sign $\Rightarrow$ saddle.
\end{enumerate}\qed


\begin{problembox}[13.9: Shortest Distance to Parabola]
Find the shortest distance from the point \((0, b)\) on the \( y \)-axis to the parabola \( x^2 - 4y = 0 \). Solve this problem using Lagrange's method and also without using Lagrange's method.
\end{problembox}

\noindent\textbf{Solution:}
Write the parabola as $y=\tfrac{x^2}{4}$. The squared distance from $(0,b)$ to $(x,\tfrac{x^2}{4})$ is
\[D(x)=x^2+\Big(\tfrac{x^2}{4}-b\Big)^2=\frac{x^4}{16}+x^2-b\frac{x^2}{2}+b^2.
\]
Then $D'(x)=x\Big(\tfrac{x^2}{4}+2-b\Big)$, so critical points are $x=0$ or $x^2=4(b-2)$. Thus: if $b<2$, the only candidate is $x=0$, giving distance $|b|$. If $b\ge 2$, the minimizing points satisfy $x^2=4(b-2)$ and the minimal distance is $\sqrt{4(b-1)}=2\sqrt{b-1}$.
With Lagrange multipliers, minimize $f(x_1,x_2)=x_1^2+(x_2-b)^2$ subject to $g(x)=x_1^2-4x_2=0$. Then $\nabla f=\lambda\nabla g$ gives $(2x_1,2(x_2-b))=\lambda(2x_1,-4)$. Either $x_1=0\Rightarrow x_2=0$ (distance $|b|$), or $\lambda=1\Rightarrow x_2=b-2$ and $x_1^2=4(b-2)$, yielding distance $2\sqrt{b-1}$ when $b\ge2$.\qed


\begin{problembox}[13.10: Geometric Problems]
Solve the following geometric problems by Lagrange's method:
\begin{enumerate}[label=(\alph*)]
    \item Find the shortest distance from the point \((a_1, a_2, a_3)\) in \( R^3 \) to the plane whose equation is \( b_1x_1 + b_2x_2 + b_3x_3 + b_0 = 0 \).
    \item Find the point on the line of intersection of the two planes
    \[a_1x_1 + a_2x_2 + a_3x_3 + a_0 = 0\]
    and
    \[b_1x_1 + b_2x_2 + b_3x_3 + b_0 = 0\]
    which is nearest the origin.
\end{enumerate}
\end{problembox}

\noindent\textbf{Solution:}
\begin{enumerate}[label=(\alph*)]
    \item Minimize $\|x-a\|^2$ subject to $b\cdot x+b_0=0$. Lagrange gives $x-a=\lambda b$ and $b\cdot(a+\lambda b)+b_0=0$, hence $\lambda=\dfrac{-(b\cdot a+b_0)}{\|b\|^2}$. The distance is $\dfrac{|b\cdot a+b_0|}{\|b\|}$.
    \item Minimize $\|x\|^2$ subject to $a\cdot x+a_0=0$ and $b\cdot x+b_0=0$. Lagrange gives $x+\mu a+\nu b=0$, so $x=-(\mu a+\nu b)$. Solve
    \[\begin{pmatrix}\|a\|^2 & a\cdot b\\ a\cdot b & \|b\|^2\end{pmatrix}\begin{pmatrix}\mu\\ \nu\end{pmatrix}=\begin{pmatrix}a_0\\ b_0\end{pmatrix},\quad x=-(\mu a+\nu b),\]
    which is the orthogonal projection of the origin onto the line of intersection.
\end{enumerate}\qed


\begin{problembox}[13.11: Maximum Value with Constraint]
Find the maximum value of \(| \sum_{k=1}^n a_k x_k |\), if \(\sum_{k=1}^n x_k^2 = 1\), by using 
\begin{enumerate}[label=(\alph*)]
    \item the Cauchy-Schwarz inequality.
    \item Lagrange's method.
\end{enumerate}
\end{problembox}

\noindent\textbf{Solution:}
\begin{enumerate}[label=(\alph*)]
    \item By Cauchy--Schwarz, $\big|\sum a_k x_k\big|\le \|a\|\,\|x\|=\|a\|$, with equality when $x$ is proportional to $a$.
    \item Lagrange on $L=\sum a_k x_k-\lambda(\sum x_k^2-1)$ gives $a_k=2\lambda x_k$, so $x\parallel a$ and the maximum is $\|a\|$.
\end{enumerate}\qed


\begin{problembox}[13.12: Maximum of Product under Constraint]
Find the maximum of \((x_1 x_2 \cdots x_n)^2\) under the restriction
\[ x_1^2 + \cdots + x_n^2 = 1. \]
Use the result to derive the following inequality, valid for positive real numbers \(a_1, \ldots, a_n\)
\[ (a_1 \cdots a_n)^{1/n} \leq \frac{a_1 + \cdots + a_n}{n}. \]
\end{problembox}

\noindent\textbf{Solution:}
By AM--GM on $x_1^2,\dots,x_n^2$ with $\sum x_i^2=1$,
\[(x_1\cdots x_n)^2\le \left(\frac{1}{n}\right)^n,\]
with equality when $|x_1|=\cdots=|x_n|=\tfrac{1}{\sqrt n}$. For $a_i>0$, set $x_i=\dfrac{\sqrt{a_i}}{\sqrt{a_1+\cdots+a_n}}$ to obtain
\[(a_1\cdots a_n)^{1/n}\le \frac{a_1+\cdots+a_n}{n}.
\]\qed


\begin{problembox}[13.13: Local Extremum with Condition]
If \(f(x) = x_1^k + \cdots + x_n^k, x = (x_1, \ldots, x_n)\), show that a local extreme of \(f\), subject to the condition \(x_1 + \cdots + x_n = a\), is \(a^k n^{1-k}\).
\end{problembox}

\noindent\textbf{Solution:}
Use Lagrange multipliers for the constraint \(g(x)=x_1+\cdots+x_n-a=0\). Consider
\[L(x,\lambda)=\sum_{i=1}^n x_i^{\,k}-\lambda\Big(\sum_{i=1}^n x_i-a\Big).\]
Stationarity gives \(k\,x_i^{\,k-1}=\lambda\) for each \(i\), hence all \(x_i\) are equal: \(x_i=t\). The constraint yields \(nt=a\Rightarrow t=a/n\). Therefore
\[f(x)=\sum_{i=1}^n x_i^{\,k}=n\Big(\tfrac{a}{n}\Big)^{\!k}=a^{\,k}n^{\,1-k}.
\]
For \(k>1\) this point gives the constrained minimum (convexity of \(t\mapsto t^{k}\)); for \(0<k<1\) it gives the constrained maximum.\qed


\begin{problembox}[13.14: Local Extremum with Side Conditions]
Show that all points \((x_1, x_2, x_3, x_4)\) where \(x_1^2 + x_2^2\) has a local extremum subject to the two side conditions \(x_1^2 + x_3^2 + x_4^2 = 4, x_2^2 + 2x_3^2 + 3x_4^2 = 9\), are found among 
\[ (0, 0, \pm \sqrt{3}, \pm 1), (0, \pm 1, +2, 0), (\pm 1, 0, 0, \pm \sqrt{3}), (\pm 2, \pm 3, 0, 0). \]
Which of these yield a local maximum and which yield a local minimum? Give reasons for your conclusions.
\end{problembox}

\noindent\textbf{Solution:}
Evaluating $x_1^2+x_2^2$ at the listed feasible points gives: $(0,0,\pm\sqrt3,\pm1)\mapsto 0$; $(0,\pm1,2,0)\mapsto 1$; $(\pm1,0,0,\pm\sqrt3)\mapsto 1$; $(\pm2,\pm3,0,0)\mapsto 13$. Hence the points with value $0$ yield local (global) minima; those with value $13$ yield local (global) maxima; the points with value $1$ are neither maxima nor minima along the constraint surface.\qed


\begin{problembox}[13.15: Extreme Values with Side Conditions]
Show that the extreme values of \(f(x_1, x_2, x_3) = x_1^2 + x_2^2 + x_3^2\), subject to the two side conditions
\[ \sum_{j=1}^3 \sum_{i=1}^3 a_{ij} x_i x_j = 1 \quad (a_{ij} = a_{ji}) \]
and
\[ b_1 x_1 + b_2 x_2 + b_3 x_3 = 0, \quad (b_1, b_2, b_3) \neq (0, 0, 0), \]
are \(t_1^{-1}, t_2^{-1}\), where \(t_1\) and \(t_2\) are the roots of the equation
\[\begin{vmatrix}
b_1 & b_2 & b_3 & 0 \\
a_{11} - t & a_{12} & a_{13} & b_1 \\
a_{21} & a_{22} - t & a_{23} & b_2 \\
a_{31} & a_{32} & a_{33} - t & b_3
\end{vmatrix} = 0.\]
Show that this is a quadratic equation in \(t\) and give a geometric argument to explain why the roots \(t_1, t_2\) are real and positive.
\end{problembox}

\noindent\textbf{Solution:}
Introduce multipliers $\lambda,\mu$ for the constraints $x^{\!T}Ax=1$ and $b^{\!T}x=0$. Then
\[2x=2\lambda Ax+\mu b,\quad x^{\!T}Ax=1,\quad b^{\!T}x=0.
\]
Assuming $x\neq 0$, rearrange to $(I-\lambda A)x=\tfrac{\mu}{2}\,b$. Nontrivial solutions exist only when
\[\begin{vmatrix}
0 & b^{\!T} \\
b & A-\lambda^{-1}I
\end{vmatrix}=0.
\]
With $t=\lambda^{-1}$ this equals
\[\begin{vmatrix}
b_1 & b_2 & b_3 & 0 \\
a_{11} - t & a_{12} & a_{13} & b_1 \\
a_{21} & a_{22} - t & a_{23} & b_2 \\
a_{31} & a_{32} & a_{33} - t & b_3
\end{vmatrix}=0,
\]
which is quadratic in $t$. The corresponding extremal values of $f=\|x\|^2$ are $t_1^{-1}, t_2^{-1}$. Geometrically, the feasible set is (generically) a line in the ellipsoid $x^{\!T}Ax=1$, so the two tangent points exist, making $t_1,t_2$ real and positive.\qed


\begin{problembox}[13.16: Hadamard's Theorem]
Let \(\Delta = \det [x_{ij}]\) and let \(X_i = (x_{i1}, \ldots, x_{in})\). A famous theorem of Hadamard states that \(|\Delta| \leq d_1 \cdots d_n\), if \(d_1, \ldots, d_n\) are \(n\) positive constants such that \(\| X_i \|^2 = d_i^2 (i = 1, 2, \ldots, n)\). Prove this by treating \(\Delta\) as a function of \(n^2\) variables subject to \(n\) constraints, using Lagrange's method to show that, when \(\Delta\) has an extreme under these conditions, we must have
\[\Delta^2 = 
\begin{vmatrix}
d_1^2 & 0 & 0 & \cdots & 0 \\
0 & d_2^2 & 0 & \cdots & 0 \\
\vdots & \vdots & \ddots & \vdots & \vdots \\
0 & 0 & 0 & \cdots & d_n^2
\end{vmatrix}.\]
\end{problembox}

\noindent\textbf{Solution:}
Let the rows be $X_i\in\mathbb R^n$ with $\|X_i\|=d_i$. Consider $\Delta^2=\det(XX^{\!T})$, where $X=[x_{ij}]$. Under the constraints $\|X_i\|=d_i$, Lagrange multipliers show that at an extremum the rows are pairwise orthogonal. Then $XX^{\!T}=\operatorname{diag}(d_1^2,\dots,d_n^2)$ and
\[\Delta^2=\det(XX^{\!T})=d_1^2\cdots d_n^2,\]
which yields $|\Delta|\le d_1\cdots d_n$ with equality exactly when the rows are orthogonal.