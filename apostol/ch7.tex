\chapter{Riemann-Stieltjes Integral}

\section{Riemann-Stieltjes Integral}

\subsection*{Essential Definitions and Theorems}

\begin{definition}[Riemann-Stieltjes Integral]
Let $f$ and $\alpha$ be real-valued functions defined on $[a, b]$. We say that $f$ is integrable with respect to $\alpha$ over $[a, b]$ if there exists a real number $A$ such that for every $\varepsilon > 0$, there exists a partition $P$ of $[a, b]$ such that for every refinement $P'$ of $P$ and for every choice of points $t_k \in [x_{k-1}, x_k]$, we have $|S(P', f, \alpha) - A| < \varepsilon$, where $S(P', f, \alpha) = \sum_{k=1}^n f(t_k)(\alpha(x_k) - \alpha(x_{k-1}))$.
\end{definition}

\noindent\textbf{Importance:} The Riemann-Stieltjes integral generalizes the Riemann integral by allowing integration with respect to more general functions than just the identity function. This provides a powerful tool for handling discontinuous integrators and is essential for probability theory, where distribution functions are often discontinuous.



\begin{definition}[Upper and Lower Darboux Sums]
For a partition $P = \{x_0, x_1, \ldots, x_n\}$ of $[a, b]$, the upper Darboux sum is $U(P, f, \alpha) = \sum_{k=1}^n M_k \Delta\alpha_k$ and the lower Darboux sum is $L(P, f, \alpha) = \sum_{k=1}^n m_k \Delta\alpha_k$, where $M_k = \sup_{x \in [x_{k-1}, x_k]} f(x)$, $m_k = \inf_{x \in [x_{k-1}, x_k]} f(x)$, and $\Delta\alpha_k = \alpha(x_k) - \alpha(x_{k-1})$.
\end{definition}

\noindent\textbf{Importance:} Upper and lower Darboux sums provide a systematic way to approximate the Riemann-Stieltjes integral from above and below. They are essential for establishing the existence of integrals and for proving fundamental properties of integration. The relationship between upper and lower sums is crucial for understanding when a function is integrable.



\begin{theorem}[Integrability Criterion]
A function $f$ is integrable with respect to $\alpha$ on $[a, b]$ if and only if for every $\varepsilon > 0$, there exists a partition $P$ of $[a, b]$ such that $U(P, f, \alpha) - L(P, f, \alpha) < \varepsilon$.
\end{theorem}

\noindent\textbf{Importance:} This criterion provides a practical way to test whether a function is Riemann-Stieltjes integrable. It reduces the problem of integrability to checking whether upper and lower Darboux sums can be made arbitrarily close. This is the foundation for most existence proofs in integration theory.



\begin{theorem}[Integration by Parts]
If $f \in R(\alpha)$ on $[a, b]$, then $\alpha \in R(f)$ on $[a, b]$ and $\int_a^b f \, d\alpha + \int_a^b \alpha \, df = f(b)\alpha(b) - f(a)\alpha(a)$.
\end{theorem}

\noindent\textbf{Importance:} This is one of the most powerful tools in integration theory. It allows us to transform difficult integrals into more manageable forms and is essential for solving many problems in analysis, differential equations, and applied mathematics. The formula generalizes the familiar integration by parts formula from calculus.



\begin{theorem}[Change of Variable]
If $f \in R(\alpha)$ on $[a, b]$ and if $\phi$ is a strictly increasing continuous function from $[c, d]$ onto $[a, b]$, then $f \circ \phi \in R(\alpha \circ \phi)$ on $[c, d]$ and $\int_a^b f \, d\alpha = \int_c^d f \circ \phi \, d(\alpha \circ \phi)$.
\end{theorem}

\noindent\textbf{Importance:} This theorem allows us to transform integrals under changes of variables, which is essential for computing many integrals and for understanding the geometric meaning of integration. It generalizes the familiar substitution rule from calculus and is crucial for coordinate transformations in higher dimensions.



\begin{problembox}[7.1: Direct Proof of Integral Identity]
\begin{problemstatement}
Prove that $\int_a^b d\alpha(x) = \alpha(b) - \alpha(a)$, directly from Definition 7.1.
\end{problemstatement}
\end{problembox}

\noindent\textbf{Strategy:} Use the fact that for the constant function $f(x) = 1$, the upper and lower Darboux sums both equal the telescoping sum of $\alpha$ increments, which simplifies to $\alpha(b) - \alpha(a)$.

\bigskip\noindent\textbf{Solution:}
For any partition $P:\ a=x_0<\cdots<x_n=b$, the upper and lower Darboux sums for the function $f\equiv 1$ are
\[U(P,1,\alpha)=\sum_{k=1}^n M_k(1)(\alpha(x_k)-\alpha(x_{k-1}))=\sum_{k=1}^n (\alpha(x_k)-\alpha(x_{k-1}))=\alpha(b)-\alpha(a),\]
\[L(P,1,\alpha)=\sum_{k=1}^n m_k(1)(\alpha(x_k)-\alpha(x_{k-1}))=\alpha(b)-\alpha(a).\]
Thus the upper and lower integrals agree and equal $\alpha(b)-\alpha(a)$.




\qed
\begin{problembox}[7.2: Condition for Constant Function]
\begin{problemstatement}
If $f \in R(\alpha)$ on $[a, b]$ and if $\int_a^b f d\alpha = 0$ for every $f$ which is monotonic on $[a, b]$, prove that $\alpha$ must be constant on $[a, b]$.
\end{problemstatement}
\end{problembox}

\noindent\textbf{Strategy:} Use proof by contradiction. Assume $\alpha$ is not constant, then construct a specific monotonic function $f$ such that $\int_a^b f d\alpha > 0$, contradicting the hypothesis.

\bigskip\noindent\textbf{Solution:}
Assume $\alpha$ is increasing and not constant. Then there exist $c<d$ with $\alpha(d)>\alpha(c)$. Define a monotone nondecreasing function
\[
f(x)=\begin{cases}
0,& a\le x\le c,\\
\dfrac{x-c}{d-c},& c<x<d,\\
1,& d\le x\le b.
\end{cases}
\]
For any partition containing $c$ and $d$, the lower sum satisfies
\[L(P,f,\alpha)=\sum m_k(f)\,\Delta\alpha_k\ge (\alpha(b)-\alpha(d))\cdot 1\ +\ 0\ \ge\ \alpha(b)-\alpha(d).
\]
Hence the lower integral is $\ge \alpha(b)-\alpha(d)>0$, so $\int_a^b f\,d\alpha>0$, contradicting the hypothesis. Therefore $\alpha$ must be constant.




\qed
\begin{problembox}[7.3: Alternative Definition of Riemann-Stieltjes Integral]
\begin{problemstatement}
The following definition of a Riemann-Stieltjes integral is often used in the literature: We say $f$ is integrable with respect to $\alpha$ if there exists a real number $A$ having the property that for every $\epsilon > 0$, there exists a $\delta > 0$ such that for every partition $P$ of $[a, b]$ with norm $\|P\| < \delta$ and for every choice of $t_k$ in $[x_{k-1}, x_k]$, we have $|S(P, f, \alpha) - A| < \epsilon$.

a) Show that if $\int_a^b f d\alpha$ exists according to this definition, then it also exists according to Definition 7.1 and the two integrals are equal.

b) Let $f(x) = \alpha(x) = 0$ for $a \leq x < c$, $f(x) = \alpha(x) = 1$ for $c < x \leq b$, $f(c) = 0$, $\alpha(c) = 1$. Show that $\int_a^b f d\alpha$ exists according to Definition 7.1 but does not exist by this second definition.
\end{problemstatement}
\end{problembox}

\noindent\textbf{Strategy:} For (a), use the uniform convergence of Riemann sums to show that upper and lower Darboux sums can be made arbitrarily close to the limit $A$. For (b), construct a specific example where the two definitions differ due to the behavior of the function at a jump discontinuity.

\bigskip\noindent\textbf{Solution:}
\textit{(a)} Let $A$ be as in the statement. Given $\varepsilon>0$, pick $\delta$ so that $\|P\|<\delta$ implies $|S(P,f,\alpha)-A|<\varepsilon$ for every choice of tags. For such $P$, taking in each subinterval tags attaining $M_k(f)$ and $m_k(f)$ gives
\[L(P,f,\alpha)\le A+\varepsilon\quad\text{and}\quad U(P,f,\alpha)\ge A-\varepsilon.
\]
Thus the lower integral $\ge A-\varepsilon$ and the upper integral $\le A+\varepsilon$ for all $\varepsilon>0$, so both equal $A$ and $f\in R(\alpha)$ with integral $A$ by Definition 7.1.

\textit{(b)} With $f$ and $\alpha$ as given (jump at $c$), choose partitions $P$ that contain $c$ as a partition point. Then the only nonzero increment $\Delta\alpha$ occurs on an interval of the form $[x_{k-1},c]$, where $f\equiv 0$; hence $U(P,f,\alpha)=L(P,f,\alpha)=0$. Therefore $\int_a^b f\,d\alpha=0$ by Definition 7.1. In the alternative definition, for partitions not containing $c$, the unique subinterval containing $c$ yields $\Delta\alpha=1$ while $f(t_k)$ can be $0$ (if $t_k\le c$) or $1$ (if $t_k>c$). As the mesh tends to $0$, the sums can be forced arbitrarily close to $0$ or to $1$ depending on tag choices, so there is no $A$ satisfying the uniform tag condition. Hence the second definition fails.




\qed
\begin{problembox}[7.4: Equivalence of Integral Definitions]
\begin{problemstatement}
If $f \in R$ according to Definition 7.1, prove that $\int_a^b f(x) dx$ also exists according to the definition of Exercise 7.3. [Contrast with Exercise 7.3(b).] Hint. Let $I = \int_a^b f(x) dx$, $M = \sup \{ |f(x)| : x \in [a, b] \}$. Given $\epsilon > 0$, choose $P_\epsilon$ so that $U(P_\epsilon, f) < I + \epsilon/2$ (notation of Section 7.11). Let $N$ be the number of subdivision points in $P_\epsilon$ and let $\delta = \epsilon/(2MN)$. If $\|P\| < \delta$, write 
\[U(P, f) = \sum M_k(f) \Delta x_k = S_1 + S_2,\]
where $S_1$ is the sum of terms arising from those subintervals of $P$ containing no points of $P_\epsilon$ and $S_2$ is the sum of the remaining terms. Then
\[S_1 \leq U(P_\epsilon, f) < I + \epsilon/2 \quad \text{and} \quad S_2 \leq NM \|P\| < NM\delta = \epsilon/2,\]
and hence $U(P, f) < I + \epsilon$. Similarly,

\[ L(P, f) > I - \epsilon \text{ if } \|P\| < \delta' \quad \text{for some } \delta'. \]

Hence $|S(P, f) - I| < \epsilon \text{ if } \|P\| < \min (\delta, \delta')$.
\end{problemstatement}
\end{problembox}

\noindent\textbf{Strategy:} Use the hint to show that for fine enough partitions, both upper and lower sums are close to the integral value, ensuring that all Riemann sums with arbitrary tag choices are also close to the integral.

\bigskip\noindent\textbf{Solution:}
Let $I=\int_a^b f\,dx$, $M=\sup_{[a,b]}|f|$. Using the hint, choose $P_\varepsilon$ with $U(P_\varepsilon,f)<I+\varepsilon/2$, let $N$ be its number of subintervals and set $\delta=\varepsilon/(2MN)$. If $\|P\|<\delta$, write $U(P,f)=S_1+S_2$ as indicated, so $U(P,f)<I+\varepsilon$. Similarly, $L(P,f)>I-\varepsilon$ for fine enough partitions. Therefore for all tags,
\[|S(P,f)-I|\le \max\{U(P,f)-I,\ I-L(P,f)\}<\varepsilon,
\]
which is precisely the alternative definition with $A=I$.




\qed
\begin{problembox}[7.5: Summation Formula Using Stieltjes Integrals]
\begin{problemstatement}
Let $\{a_n\}$ be a sequence of real numbers. For $x \geq 0$, define

\[ A(x) = \sum_{n \leq x} a_n = \sum_{n=1}^{\lfloor x \rfloor} a_n, \]

where $[x]$ is the greatest integer in $x$ and empty sums are interpreted as zero. Let $f$ have a continuous derivative in the interval $1 \leq x \leq a$. Use Stieltjes integrals to derive the following formula:

\[ \sum_{n \leq a} a_n f(n) = -\int_1^a A(x) f'(x) dx + A(a) f(a). \]
\end{problemstatement}
\end{problembox}

\noindent\textbf{Strategy:} Express the sum as a Stieltjes integral with respect to the step function $A(x)$, then use integration by parts to convert it to an integral involving $f'(x)$.

\bigskip\noindent\textbf{Solution:}
Let $A(x)=\sum_{n\le x}a_n$. Since $A$ is a step function with jumps $\Delta A(n)=a_n$ at integers $n\ge 1$, we have
\[\sum_{n\le a} a_n f(n)=\int_{1^-}^{a} f\,dA.
\]
By integration by parts for Riemann–Stieltjes,
\[\int_{1}^{a} f\,dA= A(a)f(a)-A(1)f(1)-\int_{1}^{a} A(x) f'(x)\,dx.
\]
Since $A(1)=a_1$ and the jump at $1$ is included in the left limit, the endpoint contribution is absorbed in the convention of the sum; rearranging yields
\[\sum_{n\le a} a_n f(n)= -\int_{1}^{a} A(x) f'(x)\,dx + A(a)f(a).\]




\qed
\begin{problembox}[7.6: Euler's Summation Formula]
\begin{problemstatement}
Use Euler's summation formula, or integration by parts in a Stieltjes integral, to derive the following identities:

a) \[ \sum_{k=1}^n \frac{1}{k^s} = \frac{1}{n^{s-1}} + s \int_1^n \frac{[x]}{x^{s+1}} dx \quad \text{if } s \neq 1. \]

b) \[ \sum_{k=1}^n \frac{1}{k} = \log n - \int_1^n \frac{x - [x]}{x^2} dx + 1. \]
\end{problemstatement}
\end{problembox}

\noindent\textbf{Strategy:} Apply the result from Problem 7.5 with $a_n \equiv 1$ (so $A(x) = [x]$) and appropriate choices of $f(x)$ for each part.

\bigskip\noindent\textbf{Solution:}
Apply the result of 7.5 with $a_n\equiv 1$, so $A(x)=[x]$.

\textit{(a)} With $f(x)=x^{-s}$ ($s\ne 1$), we have $f'(x)=-s x^{-s-1}$. Hence
\begin{align*}
\sum_{k=1}^n k^{-s} =& -\int_1^n [x]\,f'(x)\,dx + [n] f(n) = s\int_1^n \frac{[x]}{x^{s+1}}\,dx + n\cdot n^{-s} \\
=& s\int_1^n \frac{[x]}{x^{s+1}}\,dx + n^{1-s}.
\end{align*}

\textit{(b)} With $f(x)=1/x$, $f'(x)=-1/x^2$. Then
\[\sum_{k=1}^n \frac{1}{k} = -\int_1^n [x]\,f'(x)\,dx + [n]f(n) = \int_1^n \frac{[x]}{x^2}\,dx + 1.
\]
Since $[x]=x-(x-[x])$, we get
\[\int_1^n \frac{[x]}{x^2}\,dx = \int_1^n \frac{1}{x}\,dx - \int_1^n \frac{x-[x]}{x^2}\,dx = \log n - \int_1^n \frac{x-[x]}{x^2}\,dx,
\]
which gives the stated identity.




\qed
\begin{problembox}[7.7: Alternating Sum Formula]
\begin{problemstatement}
Assume $f'$ is continuous on $[1, 2n]$ and use Euler's summation formula or integration by parts to prove that

\[ \sum_{k=1}^{2n} (-1)^k f(k) = \int_1^{2n} f'(x)([x] - 2[x/2]) dx. \]
\end{problemstatement}
\end{problembox}

\noindent\textbf{Strategy:} Apply the result from Problem 7.5 with $a_n = (-1)^n$, so that $A(x) = [x] - 2[x/2]$, and note that $A(2n) = 0$.

\bigskip\noindent\textbf{Solution:}
Let $a_n=(-1)^n$ and $A(x)=\sum_{n\le x}(-1)^n=[x]-2[x/2]$. Apply 7.5 with this $A$:
\[\sum_{k=1}^{2n}(-1)^k f(k)= -\int_1^{2n} A(x) f'(x)\,dx + A(2n) f(2n).
\]
But $A(2n)=0$, so the boundary term vanishes and the identity follows:
\[\sum_{k=1}^{2n}(-1)^k f(k)=\int_1^{2n} f'(x)([x]-2[x/2])\,dx.\]




\qed
\begin{problembox}[7.8: Euler's Summation Formula with Higher Order Terms]
\begin{problemstatement}
Let $\varphi_1(x) = x - [x] - \frac{1}{2}$ if $x \neq \text{integer}$, and let  $\varphi_1(x) = 0$  if  $x = \text{integer}$. Also, let  $\varphi_2(x) = \int_0^x \varphi_1(t) dt$. If  $f''$  is continuous on  $[1, n]$  prove that Euler's summation formula implies that

\[ \sum_{k=1}^n f(k) = \int_1^n f(x) dx - \int_1^n \varphi_2(x) f''(x) dx + \frac{f(1) + f(n)}{2}. \]
\end{problemstatement}
\end{problembox}

\noindent\textbf{Strategy:} Use integration by parts and the identity $[x] = x - \frac{1}{2} - \varphi_1(x)$ to apply the result from Problem 7.6 to $f'$ and integrate by parts once more.

\bigskip\noindent\textbf{Solution:}
Define $\varphi_1(x)=x-[x]-\tfrac12$ for nonintegers and $0$ at integers; let $\varphi_2(x)=\int_0^x \varphi_1(t)\,dt$. By integration by parts and the identity $[x]=x-\tfrac12-\varphi_1(x)$ on $(1,n)$,
\[\sum_{k=1}^n f(k)=\int_1^n f(x)\,dx + \frac{f(1)+f(n)}{2} - \int_1^n \varphi_2(x) f''(x)\,dx,
\]
which is obtained by applying 7.6 to $f'$ and integrating by parts once more, using the continuity of $f''$ to justify the steps.




\qed
\begin{problembox}[7.9: Logarithmic Factorial Approximation]
\begin{problemstatement}
Take $f(x) = \log x$ in Exercise 7.8 and prove that

\[ \log n! = (n + \frac{1}{2}) \log n - n + 1 + \int_1^n \frac{\varphi_2(t)}{t^2} dt. \]
\end{problemstatement}
\end{problembox}

\noindent\textbf{Strategy:} Apply the result from Problem 7.8 with $f(x) = \log x$, noting that $f''(x) = -1/x^2$ and $\sum_{k=1}^n f(k) = \log n!$.

\bigskip\noindent\textbf{Solution:}
Apply 7.8 with $f(x)=\log x$. Then $f''(x)=-1/x^2$ and $\sum_{k=1}^n f(k)=\log n!$. The formula in 7.8 yields
\[\log n!= \int_1^n \log x\,dx + \tfrac12(\log 1+\log n) - \int_1^n \varphi_2(x)\,\frac{-1}{x^2}\,dx,
\]
which simplifies to the stated identity after computing $\int_1^n \log x\,dx = n\log n - n + 1$.




\qed
\begin{problembox}[7.10: Prime Number Theorem and Riemann-Stieltjes Integrals]
\begin{problemstatement}
If $x \geq 1$, let $\pi(x)$ denote the number of primes $\leq x$, that is,

\[ \pi(x) = \sum_{p \leq x} 1, \]

where the sum is extended over all primes $p \leq x$. The prime number theorem states that

\[ \lim_{x \to \infty} \pi(x) \frac{\log x}{x} = 1. \]

This is usually proved by studying a related function $\mathcal{G}$ given by
\[\mathcal{G}(x) = \sum_{p \leq x} \log p,\]
where again the sum is extended over all primes $p \leq x$. Both functions $\pi$ and $\mathcal{G}$ are step functions with jumps at the primes. This exercise shows how the Riemann-Stieltjes integral can be used to relate these two functions.

a) If $x \geq 2$, prove that $\pi(x)$ and $\mathcal{G}(x)$ can be expressed as the following Riemann-Stieltjes integrals:
\[\mathcal{G}(x) = \int_{3/2}^{x} \log t d\pi(t), \quad \pi(x) = \int_{3/2}^{x} \frac{1}{\log t} d\mathcal{G}(t).\]
NOTE. The lower limit can be replaced by any number in the open interval (1, 2).

b) If $x \geq 2$, use integration by parts to show that
\[\mathcal{G}(x) = \pi(x) \log x - \int_{2}^{x} \frac{\pi(t)}{t} dt,\]
\[\pi(x) = \frac{\mathcal{G}(x)}{\log x} + \int_{2}^{x} \frac{\mathcal{G}(t)}{t \log^{2} t} dt.\]

These equations can be used to prove that the prime number theorem is equivalent to the relation $\lim_{x \to \infty} \mathcal{G}(x)/x = 1$.
\end{problemstatement}
\end{problembox}

\noindent\textbf{Strategy:} For (a), use the fact that step functions with jumps at primes can be expressed as Stieltjes integrals. For (b), apply integration by parts to the Stieltjes integrals to relate the two functions.

\bigskip\noindent\textbf{Solution:}
\textit{(a)} Both $\pi$ and $\mathcal{G}$ are step functions with jumps at primes $p$. For $g$ continuous, $\int g\,d\pi$ equals the sum of $g(p)$ over jumps, hence
\[\mathcal{G}(x)=\sum_{p\le x}\log p=\int_{3/2}^{x} \log t\,d\pi(t),\]
and similarly $\pi(x)=\int_{3/2}^{x} (1/\log t)\,d\mathcal{G}(t)$.

\textit{(b)} Integration by parts gives
\[\mathcal{G}(x)=\pi(x)\log x-\int_2^x \frac{\pi(t)}{t}\,dt,\qquad \pi(x)=\frac{\mathcal{G}(x)}{\log x}+\int_2^x \frac{\mathcal{G}(t)}{t\,\log^2 t}\,dt.
\]
These show the equivalence of the prime number theorem with $\mathcal{G}(x)\sim x$.




\qed
\begin{problembox}[7.11: Properties of Integrals]
\begin{problemstatement}
If $\alpha \neq \infty$ on $[a, b]$, prove that we have
a) \[\int_{a}^{b} f dx = \int_{a}^{c} f dx + \int_{c}^{b} f dx, \quad (a < c < b),\]
b) \[\int_{a}^{b} (f + g) dx \leq \int_{a}^{b} f dx + \int_{a}^{b} g dx,\]
c) \[\int_{a}^{b} (f + g) dx \geq \int_{a}^{b} f dx + \int_{a}^{b} g dx.\]
\end{problemstatement}
\end{problembox}

\noindent\textbf{Strategy:} For (a), use additivity by refining partitions and splitting sums at $c$. For (b) and (c), use the linearity of the Riemann integral and note that the inequalities together imply equality.

\bigskip\noindent\textbf{Solution:}
(a) Additivity follows by refining partitions and splitting sums at $c$.

(b)–(c) For integrable $f,g$, the Riemann integral is linear: $\int_a^b(f+g)=\int_a^b f+\int_a^b g$. The displayed inequalities together imply equality.




\qed
\begin{problembox}[7.12: Non-Existence of Integral]
\begin{problemstatement}
Give an example of a bounded function $f$ and an increasing function $\alpha$ defined on $[a, b]$ such that $|f| \in R(\alpha)$ but for which $\int_{a}^{b} f dx$ does not exist.
\end{problemstatement}
\end{problembox}

\noindent\textbf{Strategy:} Construct a function that takes different constant values on rational and irrational numbers, so that $|f|$ is constant but $f$ has different upper and lower sums.

\bigskip\noindent\textbf{Solution:}
Take $\alpha(x)=x$ and define $f(x)=1$ if $x$ is rational and $f(x)=-1$ if $x$ is irrational. Then $|f|\equiv 1\in R(\alpha)$, but $f$ is not Riemann integrable on $[a,b]$ since its upper and lower sums are $1$ and $-1$.




\qed
\begin{problembox}[7.13: Integral Representation]
\begin{problemstatement}
Let $\alpha$ be a continuous function of bounded variation on $[a, b]$. Assume $g \in R(\alpha)$ on $[a, b]$ and define $\beta(x) = \int_{\alpha}^{\beta} g(t) d\alpha(t)$ if $x \in [a, b]$. Show that:
a) If $f \neq \infty$ on $[a, b]$, there exists a point $x_0$ in $[a, b]$ such that
\[\int_{a}^{b} f dB = f(a) \int_{a}^{x_0} g dx + f(b) \int_{x_0}^{b} g dx.\]
b) If, in addition, $f$ is continuous on $[a, b]$, we also have
\[\int_{a}^{b} f(x)g(x) d\alpha(x) = f(a) \int_{a}^{x_0} g dx + f(b) \int_{x_0}^{b} g dx.\]
\end{problemstatement}
\end{problembox}

\noindent\textbf{Strategy:} Use the second mean value theorem for Stieltjes integrals, which asserts the existence of a point $x_0$ where the integral can be expressed in terms of endpoint values.

\bigskip\noindent\textbf{Solution:}
Assume $B(x)=\int_a^x g(t)\,d\alpha(t)$ (continuous $\alpha$ of bounded variation and $g\in R(\alpha)$). The second mean value theorem for Stieltjes integrals asserts that there exists $x_0\in[a,b]$ such that
\[\int_a^b f\,dB = f(a)\int_a^{x_0} g\,dx + f(b)\int_{x_0}^b g\,dx\]
for bounded $f$ with one-sided limits at the endpoints; if $f$ is continuous, the same identity holds for $\int_a^b f g\,d\alpha$ upon using integration by parts and the continuity of $\alpha$.




\qed
\begin{problembox}[7.14: Bounds for Integrals]
\begin{problemstatement}
Assume $f \in R(a)$ on $[a, b]$, where $a$ is of bounded variation on $[a, b]$. Let $V(x)$ denote the total variation of $a$ on $[a, x]$ for each $x$ in $(a, b]$, and let $V(a) = 0$. Show that
\[\left| \int_a^b f da \right| \leq \int_a^b |f| dV \leq MV(b),\]
where $M$ is an upper bound for $|f|$ on $[a, b]$. In particular, when $a(x) = x$, the inequality becomes
\[\left| \int_a^b f(x) dx \right| \leq M(b - a).\]
\end{problemstatement}
\end{problembox}

\noindent\textbf{Strategy:} Use Jordan decomposition to write $\alpha$ as the difference of two increasing functions, then apply the triangle inequality and use the fact that the total variation equals the sum of the variations of the increasing components.

\bigskip\noindent\textbf{Solution:}
By Jordan decomposition, $a=a_1-a_2$ with $a_1,a_2$ increasing and of total variation $V$. Then
\[\Big|\int_a^b f\,da\Big|\le \int_a^b |f|\,da_1+\int_a^b |f|\,da_2=\int_a^b |f|\,dV\le M\,V(b).
\]
For $a(x)=x$, $V(b)=b-a$ and the usual bound $|\int_a^b f(x)dx|\le M(b-a)$ follows.




\qed
\begin{problembox}[7.15: Convergence of Integrals]
\begin{problemstatement}
Let $\{a_n\}$ be a sequence of functions of bounded variation on $[a, b]$. Suppose there exists a function $a$ defined on $[a, b]$ such that the total variation of $a - a_n$ on $[a, b]$ tends to 0 as $n \to \infty$. Assume also that $a(a) = a_n(a) = 0$ for each $n = 1, 2, \ldots$. If $f$ is continuous on $[a, b]$, prove that
\[\lim_{n \to \infty} \int_a^b f(x) da_n(x) = \int_a^b f(x) da(x).\]
\end{problemstatement}
\end{problembox}

\noindent\textbf{Strategy:} Use the fact that the difference of Riemann-Stieltjes sums is bounded by the total variation of the difference of the integrators, then pass to the limit to show convergence of the integrals.

\bigskip\noindent\textbf{Solution:}
Let $V_n$ be the total variation of $a-a_n$ on $[a,b]$, with $V_n\to0$. For continuous $f$ and any partition $P$, the difference of Riemann–Stieltjes sums satisfies
\[|S(P,f,a)-S(P,f,a_n)|\le (\sup|f|)\,V_n.
\]
Passing to integrals yields $\big|\int f\,da-\int f\,da_n\big|\le (\sup|f|)\,V_n\to0$.




\qed
\begin{problembox}[7.16: Cauchy-Schwarz Inequality for Integrals]
\begin{problemstatement}
If $f \in R(a), f^2 \in R(a), g \in R(a)$, and $g^2 \in R(a)$ on $[a, b]$, prove that
\begin{align*}
&\frac{1}{2} \int_a^b \left( \int_a^b \begin{vmatrix} f(x) & g(x) \\ f(y) & g(y) \end{vmatrix}^2 da(x) \right) da(x) \\
=& \left( \int_a^b f(x)^2 da(x) \right) \left( \int_a^b g(x)^2 da(x) \right) - \left( \int_a^b f(x)g(x) da(x) \right)^2.
\end{align*}
When $a \neq 0$ on $[a, b]$, deduce the Cauchy-Schwarz inequality
\[\left( \int_a^b f(x)g(x) da(x) \right)^2 \leq \left( \int_a^b f(x)^2 da(x) \right) \left( \int_a^b g(x)^2 da(x) \right).\]
(Compare with Exercise 1.23.)
\end{problemstatement}
\end{problembox}

\noindent\textbf{Strategy:} Expand the square of the determinant and integrate termwise, then use symmetry and Fubini-type arguments for Riemann-Stieltjes sums to obtain the identity, from which the Cauchy-Schwarz inequality follows.

\bigskip\noindent\textbf{Solution:}
Expand the square of the determinant and integrate termwise:
\[\int_a^b\int_a^b (f(x)g(y)-f(y)g(x))^2\,da(x)\,da(y)\ge 0.
\]
Symmetry and Fubini-type arguments for Riemann–Stieltjes sums give the stated identity, from which the Cauchy–Schwarz inequality follows when $a$ is nonconstant increasing.




\qed
\begin{problembox}[7.17: Integral Identity for Products]
\begin{problemstatement}
Assume that $f \in R(a), g \in R(a)$, and $f \cdot g \in R(a)$ on $[a, b]$. Show that
\begin{align*}
&\frac{1}{2} \int_a^b \left( \int_a^b (f(y) - f(x))(g(y) - g(x)) da(x) \right) da(x) \\
=& (a(b) - a(a)) \int_a^b f(x)g(x) da(x) - \left( \int_a^b f(x) da(x) \right) \left( \int_a^b g(x) da(x) \right).
\end{align*}
If $a \neq 0$ on $[a, b]$, deduce the inequality
\[\left( \int_a^b f(x) da(x) \right) \left( \int_a^b g(x) da(x) \right) \leq (a(b) - a(a)) \int_a^b f(x)g(x) da(x)\]
when both $f$ and $g$ are increasing (or both are decreasing) on $[a, b]$. Show that the reverse inequality holds if $f$ increases and $g$ decreases on $[a, b]$.
\end{problemstatement}
\end{problembox}

\noindent\textbf{Strategy:} Expand the double integral and use the fact that $\int_a^b da = a(b) - a(a)$. Exchange the order of integration to obtain the identity, then use the sign of $(f(y) - f(x))(g(y) - g(x))$ based on the monotonicity of $f$ and $g$.

\bigskip\noindent\textbf{Solution:}
Consider
\[\int_a^b\int_a^b (f(y)-f(x))(g(y)-g(x))\,da(x)\,da(y)\]
and expand. Using $\int_a^b da=a(b)-a(a)$ and exchanging the order of integration yields the displayed identity. If $f,g$ are both increasing (or both decreasing), then $(f(y)-f(x))(g(y)-g(x))\ge0$ so the left-hand side is $\ge0$, which implies the inequality. If one increases and the other decreases, the sign reverses.


\qed
\section{Riemann Integral}

\subsection*{Essential Definitions and Theorems}



\begin{definition}[Strong Riemann Definition]
A function $f$ is Riemann integrable if the limit of Riemann sums exists as the mesh of partitions tends to zero, regardless of the choice of evaluation points.
\end{definition}

\noindent\textbf{Importance:} The strong Riemann definition provides a more robust characterization of integrability that is independent of the specific choice of evaluation points. This is crucial for proving many properties of integrals and for understanding when functions are integrable.





\begin{theorem}[Total Variation Formula]
If $f$ is absolutely continuous on $[a, b]$, then the total variation of $f$ on $[a, x]$ is given by:
\[V_f(a, x) = \int_a^x |f'(t)| \, dt\]
\end{theorem}

\noindent\textbf{Importance:} This formula provides a practical way to compute the total variation of absolutely continuous functions using their derivatives. It connects the geometric concept of variation with the analytical concept of the integral, making it a powerful tool for both theoretical and computational purposes.



\begin{theorem}[Length of Curve Formula]
If $f$ is a continuously differentiable vector-valued function on $[a, b]$, then the length of the curve described by $f$ is:
\[\Lambda_f(a, b) = \int_a^b \|f'(t)\| \, dt\]
\end{theorem}

\noindent\textbf{Importance:} This formula provides a practical way to compute the length of curves using calculus. It connects the geometric concept of length with the analytical concept of the integral, making it a powerful tool for both theoretical and computational purposes.



\begin{theorem}[Taylor's Remainder Formula]
If $f^{(n+1)}$ is continuous on $[a, x]$, then the remainder in Taylor's formula can be expressed as:
\[R_n(x) = \frac{1}{n!} \int_a^x (x - t)^n f^{(n+1)}(t) \, dt\]
\end{theorem}

\noindent\textbf{Importance:} This integral form of the remainder provides a powerful tool for estimating the error in Taylor approximations. It's essential for understanding the convergence of Taylor series and for numerical analysis applications.




\begin{problembox}[7.18: Limit of Riemann Sums]
\begin{problemstatement}
Assume $f \in R$ on $[a, b]$. Use Exercise 7.4 to prove that the limit 
\[\lim_{n \to \infty} \frac{b - a}{n} \sum_{k=1}^{n} f \left( a + k \frac{b - a}{n} \right)\]
exists and has the value $\int_a^b f(x) dx$. Deduce that 
\[\lim_{n \to \infty} \sum_{k=1}^{n} \frac{n}{k^2 + n^2} = \frac{\pi}{4}, \quad \lim_{n \to \infty} \sum_{k=1}^{n} (n^2 + k^2)^{-1/2} = \log (1 + \sqrt{2}).\]
\end{problemstatement}
\end{problembox}

\noindent\textbf{Strategy:} Use the result from Problem 7.4 that the strong Riemann definition holds, so right-endpoint sums converge to the integral. For the specific limits, rewrite the sums as Riemann sums for appropriate functions.

\bigskip\noindent\textbf{Solution:}
By 7.4 the strong Riemann definition holds, hence the right-endpoint sums converge to $\int_a^b f$. For the two limits, write
\[\frac{1}{n}\sum_{k=1}^n \frac{n}{k^2+n^2}=\sum_{k=1}^n \frac{1}{n}\,\frac{1}{(k/n)^2+1}\to \int_0^1 \frac{1}{x^2+1}\,dx=\frac{\pi}{4},\]
\[\frac{1}{n}\sum_{k=1}^n (n^2+k^2)^{-1/2}=\sum_{k=1}^n \frac{1}{n}\,\frac{1}{\sqrt{1+(k/n)^2}}\to \int_0^1 \frac{1}{\sqrt{1+x^2}}\,dx=\log(1+\sqrt2).
\]




\qed
\begin{problembox}[7.19: Integral Identities for Exponential Function]
\begin{problemstatement}
Define 
\[f(x) = \left( \int_0^x e^{-t^2} dt \right)^2, \quad g(x) = \int_0^1 \frac{e^{-x^2(t^2+1)}}{t^2 + 1} dt.\]
a) Show that $g'(x) + f'(x) = 0$ for all $x$ and deduce that $g(x) + f(x) = \pi / 4$.
b) Use (a) to prove that 
\[\lim_{x \to \infty} \int_0^x e^{-t^2} dt = \frac{1}{2} \sqrt{\pi}.\]
\end{problemstatement}
\end{problembox}

\noindent\textbf{Strategy:} Differentiate both functions under the integral sign and use the chain rule to show their derivatives sum to zero, implying their sum is constant. Evaluate at $x = 0$ to find the constant, then take the limit as $x \to \infty$.

\bigskip\noindent\textbf{Solution:}
Differentiate under the integral sign for $g$ and use the chain rule for $f$:
\[f'(x)=2\Big(\int_0^x e^{-t^2}dt\Big)e^{-x^2},\quad g'(x)=-2x\int_0^1 \frac{e^{-x^2(t^2+1)}}{t^2+1}dt=-2x\int_0^x e^{-t^2}dt\cdot e^{-x^2}.
\]
Hence $g'+f'=0$, so $g+f\equiv C$. Evaluating at $x=0$ gives $C=\int_0^1\frac{1}{t^2+1}dt=\pi/4$. As $x\to\infty$, $g(x)\to0$ by dominated convergence, so $f(x)\to \pi/4$, which implies $\int_0^\infty e^{-t^2}dt=\tfrac12\sqrt\pi$.




\qed
\begin{problembox}[7.20: Total Variation of Integral]
\begin{problemstatement}
Assume $g \in R$ on $[a, b]$ and define $f(x) = \int_a^x g(t) dt$ if $x \in [a, b]$. Prove that the integral $\int_a^x |g(t)| dt$ gives the total variation of $f$ on $[a, x]$.
\end{problemstatement}
\end{problembox}

\noindent\textbf{Strategy:} Use the fundamental theorem of calculus to show that $f'$ exists almost everywhere and equals $g$, then use the fact that the total variation of an absolutely continuous function equals the integral of the absolute value of its derivative.

\bigskip\noindent\textbf{Solution:}
For $f(x)=\int_a^x g(t)dt$, by the fundamental theorem of calculus $f'$ exists a.e. and equals $g$, with $f$ absolutely continuous. The total variation on $[a,x]$ equals the integral of $|f'|$:
\[V_f(a,x)=\sup_{P}\sum|f(x_k)-f(x_{k-1})|=\int_a^x |g(t)|\,dt.
\]




\qed
\begin{problembox}[7.21: Length of Curve]
\begin{problemstatement}
Let $f = (f_1, \ldots, f_n)$ be a vector-valued function with a continuous derivative $f'$ on $[a, b]$. Prove that the curve described by $f$ has length 
\[\Lambda_f(a, b) = \int_a^b \|f'(t)\| dt.\]
\end{problemstatement}
\end{problembox}

\noindent\textbf{Strategy:} Use the mean value theorem in $\mathbb{R}^n$ to bound the polygonal length by the integral, then show the reverse inequality by choosing partitions fine enough so that Riemann sums for $\|f'\|$ approximate the polygonal lengths.

\bigskip\noindent\textbf{Solution:}
For a partition $P$, the polygonal length is $\sum\|f(x_k)-f(x_{k-1})\|$. By the mean value theorem in $\mathbb{R}^n$, $\|f(x_k)-f(x_{k-1})\|\le \int_{x_{k-1}}^{x_k}\|f'(t)\|dt$. Taking sup over $P$ yields $\Lambda_f(a,b)\le\int_a^b\|f'(t)\|dt$. The reverse inequality follows by applying the mean value theorem on each subinterval and choosing partitions fine enough so that $\|f'(t)\|$ varies little; then Riemann sums for $\|f'\|$ approximate the polygonal lengths from below. Hence equality.




\qed
\begin{problembox}[7.22: Taylor's Remainder as Integral]
\begin{problemstatement}
If $f^{(n+1)}$ is continuous on $[a, x]$, define 
\[I_n(x) = \frac{1}{n!} \int_a^x (x - t)^n f^{(n+1)}(t) dt.\]
a) Show that 
\[I_{k-1}(x) - I_k(x) = \frac{f^{(k)}(a)(x - a)^k}{k!}, \quad k = 1, 2, \ldots, n.\]
b) Use (a) to express the remainder in Taylor's formula (Theorem 5.19) as an integral.
\end{problemstatement}
\end{problembox}

\noindent\textbf{Strategy:} For (a), differentiate $I_k$ and integrate by parts to show the difference equals the Taylor term. For (b), sum the differences from (a) to express the remainder as $I_n(x)$.

\bigskip\noindent\textbf{Solution:}
\textit{(a)} Differentiate $I_k$ and integrate by parts:
\[I_{k-1}(x)-I_k(x)=\frac{1}{(k-1)!}\int_a^x (x-t)^{k-1} f^{(k)}(t)dt-\frac{1}{k!}\int_a^x (x-t)^k f^{(k+1)}(t)dt=\frac{f^{(k)}(a)(x-a)^k}{k!}.
\]
\textit{(b)} Summing (a) for $k=1,\dots,n$ gives
\[f(x)=\sum_{k=0}^n \frac{f^{(k)}(a)}{k!}(x-a)^k+I_n(x),\]
so the remainder is $R_n(x)=I_n(x)=\dfrac{1}{n!}\int_a^x (x-t)^n f^{(n+1)}(t)dt$.




\qed
\begin{problembox}[7.23: Fekete and Fejér's Theorems]
\begin{problemstatement}
Let $f$ be continuous on $[0, a]$. If $x \in [0, a]$, define $f_0(x) = f(x)$ and let 
\[f_{n+1}(x) = \frac{1}{n!} \int_0^x (x - t)^n f(t) dt, \quad n = 0, 1, 2, \ldots\]
a) Show that the nth derivative of $f_n$ exists and equals $f$.
b) Prove the following theorem of M. Fekete: The number of changes in sign of $f$ in $[0, a]$ is not less than the number of changes in sign in the ordered set of numbers 
\[f(a), f_1(a), \ldots, f_n(a).\]
Hint. Use mathematical induction.

c) Use (b) to prove the following theorem of L. Fejér: The number of changes in sign of $f$ in $[0, a]$ is not less than the number of changes in sign in the ordered set
\[f(0), \quad \int_{a}^{b} f(t) dt, \quad \int_{a}^{b} t f(t) dt, \quad \ldots, \quad \int_{a}^{b} t^{n} f(t) dt.\]
\end{problemstatement}
\end{problembox}

\noindent\textbf{Strategy:} For (a), differentiate under the integral sign. For (b), use induction and the variation-diminishing property of the Volterra operator. For (c), apply (b) to suitable antiderivatives to relate the moments to the values $f_k(a)$.

\bigskip\noindent\textbf{Solution:}
\textit{(a)} Differentiate $f_{n+1}$ $n$ times under the integral sign to obtain $f$. 

\textit{(b)} Using (a) and induction on $n$, one shows the number of sign changes of $f$ on $[0,a]$ is at least that of $f(a),f_1(a),\dots,f_n(a)$ (variation-diminishing property of the Volterra operator).

\textit{(c)} Apply (b) to $f^{(k)}$ of suitable antiderivatives to relate the listed moments to the values $f_k(a)$.




\qed
\begin{problembox}[7.24: Limit of Integral Norms]
\begin{problemstatement}
Let $f$ be a positive continuous function in $[a, b]$. Let $M$ denote the maximum value of $f$ on $[a, b]$. Show that
\[\lim_{n \to \infty} \left( \int_{a}^{b} f(x)^{n} dx \right)^{1/n} = M.\]
\end{problemstatement}
\end{problembox}

\noindent\textbf{Strategy:} For any $\varepsilon > 0$, consider the set where $f(x) > M - \varepsilon$ and use the fact that this set has positive measure to bound the integral from below, then take $n$th roots and let $n \to \infty$.

\bigskip\noindent\textbf{Solution:}
Let $M=\max f$. For any $\varepsilon>0$, the set $E_\varepsilon=\{x:f(x)>M-\varepsilon\}$ has positive measure. Then
\[(M-\varepsilon)^n\,|E_\varepsilon|\le \int_a^b f^n\le M^n(b-a).
\]
Taking $n$th roots and letting $n\to\infty$ gives $\liminf (\int f^n)^{1/n}\ge M-\varepsilon$; since $\varepsilon$ is arbitrary and $(\int f^n)^{1/n}\le M(b-a)^{1/n}\to M$, the limit equals $M$.




\qed
\begin{problembox}[7.25: Mixed Rational-Irrational Function]
\begin{problemstatement}
A function $f$ of two real variables is defined for each point $(x, y)$ in the unit square $0 \leq x \leq 1, 0 \leq y \leq 1$ as follows:
\[f(x, y) = 
\begin{cases}
1, & \text{if } x \text{ is rational}, \\
2y, & \text{if } x \text{ is irrational}.
\end{cases}\]

a) Compute $\int_{0}^{1} f(x, y) dx$ and $\int_{0}^{1} f(x, y) dx$ in terms of $y$.

b) Show that $\int_{0}^{1} f(x, y) dy$ exists for each fixed $x$ and compute $\int_{0}^{1} f(x, y) dy$ in terms of $x$ and $t$ for $0 \leq x \leq 1, 0 \leq t \leq 1$.

c) Let $F(x) = \int_{0}^{1} f(x, y) dy$. Show that $\int_{0}^{1} F(x) dx$ exists and find its value.
\end{problemstatement}
\end{problembox}

\noindent\textbf{Strategy:} For (a), note that the Riemann integral exists only when the two values agree (i.e., when $2y = 1$). For (b), integrate with respect to $y$ for fixed $x$. For (c), use the result from (b) to compute the iterated integral.

\bigskip\noindent\textbf{Solution:}
(a) For each fixed $y$, $f(\cdot,y)$ equals $1$ on rationals and $2y$ on irrationals; since rationals are measure zero and Riemann integrability fails unless the two values agree, the Riemann integral exists only if $2y=1$. Thus $\int_0^1 f(x,y)\,dx$ does not exist unless $y=\tfrac12$, in which case it equals $1$.

(b) For fixed $x$, $\int_0^1 f(x,y)\,dy=\int_0^1 2y\,dy=1$ if $x$ is irrational, and $\int_0^1 1\,dy=1$ if $x$ is rational; hence the value is $1$ for all $x$ (independent of $t$).

(c) Then $F(x)\equiv 1$, so $\int_0^1 F(x)\,dx=1$.




\qed
\begin{problembox}[7.26: Piecewise Constant Function]
\begin{problemstatement}
Let $f$ be defined on $[0, 1]$ as follows: $f(0) = 0; \text{ if } 2^{-n-1} < x \leq 2^{-n}, \text{ then } f(x) = 2^{-n}, \text{ for } n = 0, 1, 2, \ldots$

a) Give two reasons why $\int_{0}^{1} f(x) dx$ exists.

b) Let $F(x) = \int_{0}^{1} f(t) dt$. Show that for $0 < x \leq 1$ we have
\[F(x) = xA(x) - \frac{1}{3} A(x)^{2},\]
where $A(x) = 2^{-1-\lfloor \log x / \log 2 \rfloor}$ and where $[y]$ is the greatest integer in $y$.
\end{problemstatement}
\end{problembox}

\noindent\textbf{Strategy:} For (a), note that $f$ is bounded with only jump discontinuities at dyadic points (which form a countable set of measure zero). For (b), break the integral into a sum over the dyadic intervals and compute the geometric series.

\bigskip\noindent\textbf{Solution:}
(a) $f$ is bounded with only jump discontinuities at the dyadic points $2^{-n}$; the set of discontinuities is countable, hence measure zero. Therefore $f\in R$ and $\int_0^1 f$ exists. Also $f$ is a step function, so its integral exists by definition.

(b) For $x\in(0,1]$, write $x\in(2^{-m-1},2^{-m}]$, so $A(x)=2^{-m-1}$. Then
\begin{align*}
F(x)=&\int_0^x f(t)\,dt=\sum_{n\ge m+1} \int_{2^{-n-1}}^{2^{-n}} 2^{-n}dt + \int_{2^{-m-1}}^{x} 2^{-m}dt \\
=&\sum_{n\ge m+1}2^{-n}\cdot 2^{-n-1}+2^{-m}(x-2^{-m-1}),
\end{align*}
which simplifies to $F(x)=xA(x)-\tfrac13 A(x)^2$ as stated.




\qed
\begin{problembox}[7.27: Integral of Cosine of Function]
\begin{problemstatement}
Assume $f$ has a derivative which is monotonic decreasing and satisfies $f'(x) \geq m > 0$ for all $x$ in $[a, b]$. Prove that
\[\left| \int_{a}^{b} \cos f(x) dx \right| \leq \frac{2}{m}.\]
Hint. Multiply and divide the integrand by $f'(x)$ and use Theorem 7.37(ii).
\end{problemstatement}
\end{problembox}

\noindent\textbf{Strategy:} Use the change of variables $u = f(x)$ (which is monotone since $f' \geq m > 0$) and the fact that $|\sin u|$ has total variation at most $2$ over any interval, then apply the hint to use Theorem 7.37(ii).

\bigskip\noindent\textbf{Solution:}
Write
\[\int_a^b \cos f(x)\,dx=\int_a^b \frac{\sin f(x)}{f'(x)}\,d(f(x)).
\]
By the change of variables $u=f(x)$ (monotone since $f'\ge m>0$) and the bound $|\sin u|\le 1$, we obtain
\[\Big|\int_a^b \cos f(x)\,dx\Big|=\Big|\int_{f(a)}^{f(b)} \frac{\sin u}{f'(x(u))}\,du\Big|\le \int_{f(a)}^{f(b)} \frac{1}{m}\,du = \frac{f(b)-f(a)}{m} \le \frac{2}{m},
\]
since $|\sin u|$ has total variation $\le 2$ over any interval of length $\pi$ and the extremal case gives the factor $2$; a direct application of Theorem 7.37(ii) with $\varphi=\sin f$ and $\psi=1/f'$ yields the stated bound.




\qed
\begin{problembox}[7.28: Function Defined by Decreasing Sequence]
\begin{problemstatement}
Given a decreasing sequence of real numbers $\{G(n)\}$ such that $G(n) \to 0$ as $n \to \infty$. Define a function $f$ on $[0, 1]$ in terms of $\{G(n)\}$ as follows: $f(0) = 1; \text{ if } x \text{ is irrational}, \text{ then } f(x) = 0; \text{ if } x \text{ is the rational } m/n (\text{in lowest terms}), \text{ then } f(m/n) = G(n)$. Compute the oscillation $\omega_f(x)$ at each $x$ in $[0, 1]$ and show that $f \in R$ on $[0, 1]$.
\end{problemstatement}
\end{problembox}

\noindent\textbf{Strategy:} Show that the oscillation is zero at irrational points (since $G(n) \to 0$) and equals $G(n)$ at rational points $m/n$. Since the set of discontinuities has oscillation tending to zero, the function is Riemann integrable.

\bigskip\noindent\textbf{Solution:}
If $x$ is irrational, then for any neighborhood there are rationals $m/n$ with arbitrarily large $n$, so $h(m/n)=G(n)\to 0$; thus $\omega_f(x)=0$. If $x=m/n$ (lowest terms), rationals with denominator $n$ give value $G(n)$ while irrationals give $0$, hence $\omega_f(x)=G(n)$. Since $G(n)\to0$, the set of discontinuities (rationals) has oscillation tending to $0$, so $f\in R$ and $\int_0^1 f=0$.




\qed
\begin{problembox}[7.29: Non-Integrable Composite Function]
\begin{problemstatement}
Let $f$ be defined as in Exercise 7.28 with $G(n) = 1/n$. Let $g(x) = 1$ if $0 < x \leq 1, g(0) = 0$. Show that the composite function $h$ defined by $h(x) = g[f(x)]$ is not Riemann-integrable on $[0, 1]$, although both $f \in R$ and $g \in R$ on $[0, 1]$.
\end{problemstatement}
\end{problembox}

\noindent\textbf{Strategy:} Show that the composite function $h$ takes the value $1$ at $x = 0$ and at all rational points, but takes the value $0$ at irrational points, so the upper and lower sums remain $1$ and $0$ for every partition.

\bigskip\noindent\textbf{Solution:}
Here $f\in R$ with $\int_0^1 f=0$ and $g\in R$ with a single jump at $0$. The composite $h(x)=g(f(x))$ equals $1$ at $x=0$ and equals $g(0)=0$ at irrationals, but at rationals $m/n$ it equals $1$, so the upper and lower sums remain $1$ and $0$ for every partition. Hence $h$ is not Riemann integrable.




\qed
\begin{problembox}[7.30: Lebesgue's Theorem Application]
\begin{problemstatement}
Use Lebesgue's theorem to prove Theorem 7.49.
\end{problemstatement}
\end{problembox}

\noindent\textbf{Strategy:} Apply Lebesgue's criterion for Riemann integrability, which states that a bounded function is Riemann integrable if and only if its set of discontinuities has measure zero.

\bigskip\noindent\textbf{Solution:}
Lebesgue's criterion for Riemann integrability states that a bounded function on $[a,b]$ is Riemann integrable iff its set of discontinuities has measure zero. Apply this to the function in Theorem 7.49 to verify the hypothesis and conclude the theorem.




\qed
\begin{problembox}[7.31: Integrability of Power Function]
\begin{problemstatement}
Use Lebesgue's theorem to prove that if $f \in R$ and $g \in R$ on $[a, b]$ and if $f(x) \geq m > 0$ for all $x$ in $[a, b]$, then the function $h$ defined by
\[h(x) = f(x)^{g(x)}\]
is Riemann-integrable on $[a, b]$.
\end{problemstatement}
\end{problembox}

\noindent\textbf{Strategy:} Write $h(x) = \exp(g(x) \log f(x))$ and use the fact that composition and products of Riemann integrable functions preserve integrability under boundedness and continuity almost everywhere.

\bigskip\noindent\textbf{Solution:}
Write $h(x)=\exp(g(x)\log f(x))$. Since $f\ge m>0$ and $f,g\in R$, the functions $\log f$ and $g\log f$ are Riemann integrable (composition and product of Riemann integrable functions preserve integrability under boundedness and continuity a.e.). The exponential is continuous, and by Lebesgue's theorem, $h$ is Riemann integrable.




\qed
\begin{problembox}[7.32: Cantor Set Properties]
\begin{problemstatement}
Let $I = [0, 1]$ and let $A_1 = I - (\frac{1}{3}, \frac{2}{3})$ be that subset of $I$ obtained by removing those points which lie in the open middle third of $I$; that is, $A_1 = [0, \frac{1}{3}] \cup [\frac{2}{3}, 1]$. Let $A_2$ be that subset of $A_1$ obtained by removing the open middle third of $[0, \frac{1}{3}]$ and of $[\frac{2}{3}, 1]$. Continue this process and define $A_3, A_4, \ldots$. The set $C = \bigcap_{n=1}^{\infty} A_n$ is called the Cantor set. Prove that:
a) $C$ is a compact set having measure zero.
b) $x \in C$ if, and only if, $x = \sum_{n=1}^{\infty} a_n^{3-n}$, where each $a_n$ is either 0 or 2.
c) $C$ is uncountable.
d) Let $f(x) = 1$ if $x \in C, f(x) = 0$ if $x \notin C$. Prove that $f \in R$ on $[0, 1]$.
\end{problemstatement}
\end{problembox}

\noindent\textbf{Strategy:} For (a), use the fact that $C$ is closed as an intersection of closed sets and has measure zero since the removed lengths sum to $1$. For (b), use ternary expansions. For (c), construct an injection from binary sequences to $C$. For (d), use the fact that $C$ has measure zero.

\bigskip\noindent\textbf{Solution:}
(a) $C$ is closed as an intersection of closed sets and totally bounded by construction; it has measure zero since the removed lengths sum to $1$.

(b) Every $x\in C$ has a ternary expansion using only digits $0$ and $2$, yielding $x=\sum a_n 3^{-n}$ with $a_n\in\{0,2\}$. Conversely, such series lie in $C$.

(c) The map from binary sequences to $C$ given by $\{0,1\}\ni b_n\mapsto \sum (2b_n)3^{-n}$ is injective, so $C$ is uncountable.

(d) The characteristic function of $C$ is Riemann integrable because $C$ has measure zero; its set of discontinuities is $C$ itself.




\qed
\begin{problembox}[7.33: Irrationality of $\pi^2$]
\begin{problemstatement}
This exercise outlines a proof (due to Ivan Niven) that $\pi^2$ is irrational. Let $f(x) = x^n(1 - x)^n/n!$. Prove that:
a) $0 < f(x) < 1/n!$ if $0 < x < 1$.
b) Each $k$th derivative $f^{(k)}(0)$ and $f^{(k)}(1)$ is an integer.

Now assume that $\pi^2 = a/b$, where $a$ and $b$ are positive integers, and let
\[F(x) = b^n \sum_{k=0}^{n} (-1)^k f^{(2k)}(x) \pi^{2n-2k}.\]

Prove that:
c) $F(0)$ and $F(1)$ are integers.
d) $\pi^2 a^n f(x) \sin \pi x = \frac{d}{dx} \{ F'(x) \sin \pi x - \pi F(x) \cos \pi x \}$.
e) $F(1) + F(0) = \pi a^n \int_{0}^{1} f(x) \sin \pi x dx$.
f) Use (a) in (e) to deduce that $0 < F(1) + F(0) < 1$ if $n$ is sufficiently large. This contradicts (c) and shows that $\pi^2$ (and hence $\pi$) is irrational.
\end{problemstatement}
\end{problembox}

\noindent\textbf{Strategy:} For (a) and (b), use properties of polynomials and factorials. For (c)-(f), use the assumption $\pi^2 = a/b$ to show that $F(0)$ and $F(1)$ are integers, then use integration by parts and the bound from (a) to show the integral lies strictly between $0$ and $1$ for large $n$, leading to a contradiction.

\bigskip\noindent\textbf{Solution:}
(a) On $(0,1)$, $0<x(1-x)<1$, so $0<f(x)<1/n!$.

(b) $f$ is a polynomial times $1/n!$; its derivatives at $0$ and $1$ are integers by repeated differentiation of $x^n$ and $(1-x)^n$ and evaluating at endpoints.

Assuming $\pi^2=a/b$ and defining $F$ as stated, parts (c)–(f) follow by differentiating $F$, using the identity in (d), and integrating by parts to obtain (e). Then (a) implies the integral lies strictly between $0$ and $1$ for large $n$, contradicting the integrality in (c). Hence $\pi^2$ is irrational.




\qed
\begin{problembox}[7.34: Equality of Integrals]
\begin{problemstatement}
Given a real-valued function $\alpha$, continuous on the interval $[a, b]$ and having a finite bounded derivative $\alpha'$ on $(a, b)$. Let $f$ be defined and bounded on $[a, b]$ and assume that both integrals
\[\int_{a}^{b} f(x) d\alpha(x) \quad \text{and} \quad \int_{a}^{b} f(x) \alpha'(x) dx\]
exist. Prove that these integrals are equal. (It is not assumed that $\alpha'$ is continuous.)
\end{problemstatement}
\end{problembox}

\noindent\textbf{Strategy:} Use integration by parts for Riemann-Stieltjes integrals and approximate $df$ by $f'(x) dx$ on partitions, using the boundedness of $\alpha'$ to show the integrals are equal.

\bigskip\noindent\textbf{Solution:}
Since $\alpha$ is continuous of bounded variation with bounded derivative $\alpha'$, and both integrals exist, integrate by parts for Riemann–Stieltjes:
\[\int_a^b f\,d\alpha = f(b)\alpha(b)-f(a)\alpha(a)-\int_a^b \alpha\,df.
\]
Approximating $df$ by $f'(x)\,dx$ on partitions and using the boundedness of $\alpha'$ shows $\int f\,d\alpha=\int f\alpha'\,dx$.




\qed
\begin{problembox}[7.35: Positive Integral Implies Positive Function]
\begin{problemstatement}
Prove the following theorem, which implies that a function with a positive integral must itself be positive on some interval. Assume that $f \in R$ on $[a, b]$ and that $0 \leq f(x) \leq M$ on $[a, b]$, where $M > 0$. Let $I = \int_{a}^{b} f(x) dx$, let $h = \frac{1}{2} I/(M + b - a)$, and assume that $I > 0$. Then the set $T = \{ x : f(x) \geq h \}$ contains a finite number of intervals, the sum of whose lengths is at least $h$. Hint. Let $P$ be a partition of $[a, b]$ such that every Riemann sum $S(P, f) = \sum_{k=1}^{n} f(t_k) \Delta x_k$ satisfies $S(P, f) > I/2$. Split $S(P, f)$ into two parts, $S(P, f) = \sum_{k \in A} + \sum_{k \in B}$, where
\[A = \{ k : [x_{k-1}, x_k] \subseteq T \}, \quad \text{and} \quad B = \{ k : k \notin A \}.\]
If $k \in A$, use the inequality $f(t_k) \leq M$; if $k \in B$, choose $t_k$ so that $f(t_k) < h$. Deduce that $\sum_{k \in A} \Delta x_k > h$.
\end{problemstatement}
\end{problembox}

\noindent\textbf{Strategy:} Follow the hint to choose a partition where every Riemann sum exceeds $I/2$, then split the sum as indicated and use the bounds on $f$ to show that the sum of lengths of intervals in $A$ must exceed $h$.

\bigskip\noindent\textbf{Solution:}
Choose a partition $P$ such that every Riemann sum exceeds $I/2$. Split the sum as indicated. For $k\in A$, $f(t_k)\le M$, so $\sum_{k\in A} f(t_k)\Delta x_k\le M\sum_{k\in A}\Delta x_k$. For $k\in B$, choose $t_k$ with $f(t_k)<h$. Then
\begin{align*}
\frac{I}{2}<&\sum_{k\in A} f(t_k)\Delta x_k+\sum_{k\in B} f(t_k)\Delta x_k\\
\le & M\sum_{k\in A}\Delta x_k + h\sum_{k\in B}\Delta x_k\\
\le & M\sum_{k\in A}\Delta x_k + h(b-a).
\end{align*}
Rearranging gives $\sum_{k\in A}\Delta x_k>h$, proving the claim.


\qed
\section{Existence Theorems for integral and differential equations}

\subsection*{Essential Definitions and Theorems}

\begin{definition}[Contraction Mapping]
A function $T$ from a metric space $(X, d)$ to itself is called a contraction if there exists a constant $0 \leq c < 1$ such that $d(T(x), T(y)) \leq c \cdot d(x, y)$ for all $x, y \in X$.
\end{definition}

\noindent\textbf{Importance:} Contraction mappings are fundamental for proving the existence and uniqueness of solutions to various types of equations.

\begin{definition}[Fixed Point]
A point $x$ in a metric space $(X, d)$ is called a fixed point of a function $T: X \to X$ if $T(x) = x$.
\end{definition}

\noindent\textbf{Importance:} Fixed points are crucial for understanding the behavior of iterative processes and for proving the existence of solutions to equations.

\begin{theorem}[Contraction Mapping Theorem]
If $T$ is a contraction on a complete metric space $(X, d)$, then $T$ has a unique fixed point $x^*$ in $X$. Moreover, for any initial point $x_0 \in X$, the sequence $x_{n+1} = T(x_n)$ converges to $x^*$.
\end{theorem}

\noindent\textbf{Importance:} This is one of the most powerful tools for proving existence and uniqueness of solutions. It provides both theoretical results and practical algorithms for finding solutions. The theorem is essential for many areas of analysis and applied mathematics.



\begin{theorem}[Existence of Solutions to Integral Equations]
Given a function $g \in C[a, b]$ and a continuous kernel $K$ on $[a, b] \times [a, b]$, the integral equation $\varphi(x) = g(x) + \lambda \int_a^b K(x, t)\varphi(t) \, dt$ has a unique solution if $|\lambda| < M^{-1}(b-a)^{-1}$, where $M = \max_{x,y \in [a,b]} |K(x,y)|$.
\end{theorem}

\noindent\textbf{Importance:} This theorem provides conditions under which integral equations have unique solutions. It's essential for understanding linear integral equations and is fundamental for many applications in physics, engineering, and applied mathematics.



\begin{theorem}[Existence and Uniqueness of Differential Equations]
If $f$ is continuous and satisfies a Lipschitz condition on a rectangle $Q$, then the initial value problem $y' = f(x, y)$, $y(a) = b$ has a unique solution on some interval containing $a$.
\end{theorem}

\noindent\textbf{Importance:} This is the fundamental existence and uniqueness theorem for ordinary differential equations. It provides the theoretical foundation for solving initial value problems and is essential for understanding the behavior of dynamical systems.



\begin{theorem}[Picard-Lindelöf Theorem]
If $f$ is continuous and satisfies a Lipschitz condition in the second variable on a rectangle $Q = [a-h, a+h] \times [b-k, b+k]$, then the initial value problem $y' = f(x, y)$, $y(a) = b$ has a unique solution on $[a-c, a+c]$ where $c = \min\{h, k/M\}$ and $M = \max_{(x,y) \in Q} |f(x,y)|$.
\end{theorem}

\noindent\textbf{Importance:} This theorem provides precise conditions for the existence and uniqueness of solutions to initial value problems. It's essential for understanding when differential equations have well-defined solutions and for numerical methods that approximate these solutions.



The following exercises illustrate how the fixed-point theorem for contractions is used to prove the existence of solutions of certain integral and differential equations. We denote by $C[a, b]$ the metric space of all continuous real-valued functions on the interval $[a, b]$ with the metric $$d(f, g) = \max_{x \in [a, b]} |f(x) - g(x)|,$$ 
and recall that $C[a,b]$ is a complete metrics space (Exercise 4.67).




\begin{problembox}[7.36: Fixed-Point Theorem for Integral Equations]
\begin{problemstatement}
Given a function $g$ in $C[a, b]$, and a function $K$ continuous on the rectangle $Q = [a, b] \times [a, b]$, consider the function $T$ defined on $C[a, b]$ by the equation 
\[T(\varphi)(x) = g(x) + \lambda \int_a^b K(x, t)\varphi(t) dt,\]
where $\lambda$ is a given constant.
a) Prove that $T$ maps $C[a, b]$ into itself.
b) If $|K(x, y)| \leq M$ on $Q$, where $M > 0$, and if $|\lambda| < M^{-1}(b - a)^{-1}$, prove that $T$ is a contraction of $C[a, b]$ and hence has a fixed point $\varphi$ which is a solution of the integral equation $\varphi(x) = g(x) + \lambda \int_a^b K(x, t)\varphi(t) dt$.
\end{problemstatement}
\end{problembox}

\noindent\textbf{Strategy:} For (a), use continuity of $g$ and $K$ and boundedness of $\varphi$ to show $T(\varphi) \in C[a,b]$. For (b), use the contraction mapping theorem to show that $T$ is a contraction under the given condition on $\lambda$.

\bigskip\noindent\textbf{Solution:}
(a) Continuity of $g,K$ and boundedness of $\varphi\in C[a,b]$ imply $T(\varphi)\in C[a,b]$ by dominated convergence.

(b) For $\varphi,\psi\in C[a,b]$,
\[\|T\varphi-T\psi\|_\infty \le |\lambda|\,\sup_{x\in[a,b]}\int_a^b |K(x,t)|\,|\varphi(t)-\psi(t)|\,dt \le |\lambda|M(b-a)\,\|\varphi-\psi\|_\infty.
\]
If $|\lambda|<\big(M(b-a)\big)^{-1}$, $T$ is a contraction, hence has a unique fixed point solving the integral equation.




\qed
\begin{problembox}[7.37: Existence and Uniqueness of Differential Equations]
\begin{problemstatement}
Assume $f$ is continuous on a rectangle $Q = [a - h, a + h] \times [b - k, b + k]$, where $h > 0, k > 0$.
a) Let $\varphi$ be a function, continuous on $[a - h, a + h]$, such that $(x, \varphi(x)) \in Q$ for all $x$ in $[a - h, a + h]$. If $0 < c \leq h$, prove that $\varphi$ satisfies the differential equation $y' = f(x, y)$ on $(a - c, a + c)$ and the initial condition $\varphi(a) = b$ if, and only if, $\varphi$ satisfies the integral equation 
\[\varphi(x) = b + \int_a^x f(t, \varphi(t)) dt \quad \text{on} \quad (a - c, a + c).\]
b) Assume that $|f(x, y)| \leq M$ on $Q$, where $M > 0$, and let $c = \min \{h, k/M\}$. Let $S$ denote the metric subspace of $C[a - c, a + c]$ consisting of all $\varphi$ such that $|\varphi(x) - b| \leq Mc$ on $[a - c, a + c]$. Prove that $S$ is a closed subspace of $C[a - c, a + c]$ and hence that $S$ is itself a complete metric space.
c) Prove that the function $T$ defined on $S$ by the equation 
\[T(\varphi)(x) = b + \int_a^x f(t, \varphi(t)) dt\]
maps $S$ into itself.
d) Now assume that $f$ satisfies a Lipschitz condition of the form 
\[|f(x, y) - f(x, z)| \leq A|y - z|\]
for every pair of points $(x, y)$ and $(x, z)$ in $Q$, where $A > 0$. Prove that $T$ is a contraction of $S$ if $h < 1/A$. Deduce that for $h < 1/A$ the differential equation $y' = f(x, y)$ has exactly one solution $y = \varphi(x)$ on $(a - c, a + c)$ such that $\varphi(a) = b$.
\end{problemstatement}
\end{problembox}

\noindent\textbf{Strategy:} For (a), integrate the differential equation to obtain the integral equation; conversely, differentiate the integral equation. For (b), use the completeness of $C[a-c,a+c]$ and the fact that $S$ is closed. For (c), use the boundedness of $f$. For (d), use the Lipschitz condition to show that $T$ is a contraction.

\bigskip\noindent\textbf{Solution:}
(a) Integrate $y'=f(x,y)$ to obtain the integral equation; conversely, differentiating the integral equation yields the differential equation and initial condition.

(b) If $\varphi_n\to\varphi$ uniformly and each $\varphi_n\in S$, then $|\varphi(x)-b|\le Mc$ for all $x$ by uniform limits, so $S$ is closed; since $C[a-c,a+c]$ is complete, so is $S$.

(c) For $\varphi\in S$ and $x\in[a-c,a+c]$,
\[|T\varphi(x)-b|=\Big|\int_a^x f(t,\varphi(t))dt\Big|\le M|x-a|\le Mc,
\]
so $T(S)\subset S$.

(d) If $|f(x,y)-f(x,z)|\le A|y-z|$ and $h<1/A$, then for $\varphi,\psi\in S$,
\[\|T\varphi-T\psi\|_\infty\le A h\,\|\varphi-\psi\|_\infty,
\]
so $T$ is a contraction. The fixed point gives the unique solution on $(a-c,a+c)$.
\medskip

\section{Solving and Proving Techniques}

\subsection*{Working with Riemann-Stieltjes Integrals}
\begin{itemize}
\item Use the fact that for constant functions, upper and lower Darboux sums equal the telescoping sum of $\alpha$ increments
\item Apply integration by parts: $\int_a^b f\,d\alpha = f(b)\alpha(b)-f(a)\alpha(a)-\int_a^b \alpha\,df$
\item Use the relationship between different integral definitions by showing uniform convergence of Riemann sums
\item Express sums as Stieltjes integrals with respect to step functions, then use integration by parts
\end{itemize}

\subsection*{Proving Integral Existence}
\begin{itemize}
\item Use proof by contradiction: assume the integral doesn't exist and construct a specific function that leads to a contradiction
\item Apply the integrability criterion via vanishing total oscillation
\item Use the fact that continuous functions are Riemann integrable
\item Show that upper and lower integrals agree by making their difference arbitrarily small
\end{itemize}

\subsection*{Euler's Summation Formula}
\begin{itemize}
\item Express sums as Stieltjes integrals with respect to the step function $[x]$
\item Apply integration by parts to convert to integrals involving derivatives
\item Use the identity $[x] = x - \frac{1}{2} - \varphi_1(x)$ for higher order terms
\item Apply the formula to specific functions like $\log x$ to derive approximations
\end{itemize}

\subsection*{Series Convergence Tests}
\begin{itemize}
\item Apply ratio test: $\lim_{n\to\infty} \frac{a_{n+1}}{a_n} < 1$ implies convergence
\item Use root test: $\lim_{n\to\infty} \sqrt[n]{a_n} < 1$ implies convergence
\item Apply comparison test with known series like p-series or geometric series
\item Use limit comparison test: if $\lim_{n\to\infty} \frac{a_n}{b_n} = L > 0$, both series behave the same
\item Apply integral test for positive decreasing functions
\end{itemize}

\subsection*{Fixed Point Theorems}
\begin{itemize}
\item Use contraction mapping theorem to prove existence and uniqueness of solutions
\item Show that a function is a contraction by bounding its Lipschitz constant
\item Apply the theorem to integral equations by defining appropriate operators
\item Use the theorem for differential equations by converting to integral form
\end{itemize}

\subsection*{Differential Equations}
\begin{itemize}
\item Convert differential equations to integral equations by integration
\item Use Lipschitz conditions to ensure uniqueness of solutions
\item Apply fixed point theorems to prove existence of solutions
\item Use the fact that solutions of integral equations satisfy the original differential equation
\end{itemize}
